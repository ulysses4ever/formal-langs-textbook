\documentclass[a4paper,twoside]{scrbook}

\usepackage[T2A]{fontenc}
\usepackage[utf8]{inputenc}
\usepackage[russian]{babel}
\usepackage{amsmath}
\usepackage{amsfonts}
\usepackage{paratype}  % шрифты фирмы Паратайп
\usepackage{multirow}
\usepackage[hidelinks]{hyperref}
\usepackage{tikz}
\usepackage{tabularx}
\usepackage{subcaption}
\usetikzlibrary{automata,shapes,arrows,positioning,calc,chains, backgrounds}
\usepackage{calc}
\usepackage[many]{tcolorbox} % рамки

\frenchspacing % Отключение лишних отступов после точек
\KOMAoptions{%
%    footinclude=true, % по умолчанию false
%    headinclude=true, % по умолчанию false
    BCOR=3em            % величина отступа под переплёт?
}
\let\cleardoublepage\clearpage % отключаем правило: глава с нечётной страницы

% абзацный отступ
\usepackage{indentfirst}
\newlength\MyIndent
\setlength\MyIndent{1cm}
\setlength{\parindent}{\MyIndent}

% Списки
\usepackage {enumitem}

\setlist[itemize,1]{label=-}
\setlist %
{ %
  leftmargin = \parindent, itemsep=0ex, topsep=1ex
} %
\setenumerate[1]{label=\textup{\arabic*)}}

\makeatletter
\AddEnumerateCounter{\Asbuk}{\@Asbuk}{Ы}
\AddEnumerateCounter{\asbuk}{\@asbuk}{ы}
\makeatother

% Списки (i), (ii), ...
%\usepackage{paralist}

% Подписи к таблицам и рисункам
%\usepackage{caption}
\renewcommand*{\captionformat}{~}
%\renewcommand{\thetable}{\thechapter.\arabic{table}}
%\captionsetup[table]{labelsep=space}

% Колонтитулы
\usepackage{scrpage2}
\pagestyle{scrheadings}
\clearscrheadfoot
\chead{\headmark}
\ohead{\pagemark}
\renewcommand{\headfont}{\small\itshape}
% линейка для верхнего колонтитула
\setheadsepline{.4pt}

% Заголовок главы
\newcommand{\SuperFont}{\Large\normalfont\sffamily}
\newcommand{\CentSuperFont}{\centering\SuperFont}

\KOMAoptions{%
    headings=normal,    % размеры заголовков поменьше стандартных
    chapterprefix=true, % печатать слово Глава
    appendixprefix=false, % печатать слово Глава
    numbers=endperiod  % если хочется точек после номеров разделов
}

\usepackage{fncychap}
\ChNameVar{\SuperFont}
\ChNumVar{\CentSuperFont}
\ChTitleVar{\CentSuperFont}
\ChNameUpperCase
\ChTitleUpperCase

% заголовок (под)раздела с абзацного отступа
\addtokomafont{sectioning}{\hspace{\MyIndent}\Large} % хак для старой Комы
% новый вариант:
% \RedeclareSectionCommands[indent=\the\parindent]{section,subsection}
% применить, когда все  обновят Кому

% а также с символом параграфа
\renewcommand{\othersectionlevelsformat}[3]{%
\S\ #3\autodot\enskip}

\renewcommand{\baselinestretch}{1.35} % Интервал полуторный

% Оглавление (в т.ч расхлёбываем проблемы хака выше)
\usepackage{tocloft}

\renewcommand{\cfttoctitlefont}{\hfil\SuperFont\MakeUppercase}

\renewcommand{\cftchappresnum}{\normalfont{}Глава }
\addtolength{\cftchapnumwidth}{\widthof{\normalfont{}Глава }}
\renewcommand{\cftchapfont}{\bfseries}%\sffamily}
\renewcommand{\cftchappagefont}{}%\bfseries\sffamily}
\renewcommand{\cftchapaftersnum}{.}
%\renewcommand*{\cftappendixname}{\appendixname~}
\renewcommand{\cftsecaftersnum}{.}
\renewcommand{\cftsecpresnum}{\S\ }
\addtolength{\cftsecnumwidth}{\widthof{\S\ }}
\setlength{\cftsecindent}{\cftsecindent - .125cm}

% нумерация уравнений: глава.раздел.номер
\numberwithin{equation}{section}

% списки в несколько столбцов
\usepackage{multicol}

% красивые таблицы
\usepackage{booktabs}

% ----------------------------------------------------------------
% Настройка переносов и разрывов страниц

\binoppenalty = 10000      % Запрет переносов строк в формулах
\relpenalty = 10000        %

\sloppy                    % Не выходить за границы бокса
%\tolerance = 400          % или более точно
\clubpenalty = 10000       % Запрет разрывов страниц после первой
\widowpenalty = 10000      % и перед предпоследней строкой абзаца

% ----------------------------

\KOMAoptions{DIV=14}    % Пересчёт геометрии


%%%%%%%%%%%%%%%%%%%%  Ссылки %%%%%%%%%%%%%%%%%%%%%%%%%%%%%%%%%%%%%%%%%
% 1) Ссылка на методичку: https://yadi.sk/d/r4xjaDWBeuoaK
% 2) Нарисовать нужный символ и узнать код:
%     http://detexify.kirelabs.org/classify.html
% 3) Список символов:
%     http://tug.ctan.org/info/symbols/comprehensive/symbols-a4.pdf

%%%%%%%%%%%%%%%%%%%%% Важные моменты %%%%%%%%%%%%%%%%%%%%%%%%%%%%%%%
% 0) Все латинские (и греческие) буквочки считаются
%     математическими символами и потому пишутся в
%     «математической моде», то есть между $...$ для
%     внутристрочных формул и \[...\] для выключных (не $$...$$, как
%     часто пишут).
% 1) Не забывайте про знаки препинания в формулах (точки, запятые).
% 2) Правильно: L^+, a^*, f\colon A \to B,  неправильно: L+, a*, f: A -> B.
% 3) Обращайте внимание на спелл-чекер (красное подчёркивание слов
%     с ошибками и неизвестных слов).
% 4) Пустое слово обозначается символом $\eps$, а не $e$.
% 5) В записи множеств наподобие $\{ h(w) \mid w \in L \}$
%     следует использовать команду \mid, а не |.
%     Обратите также внимание, что для печати символов { и }
%     необходимы ‘\’, иначе скобки понимаются как средство
%     группировки, например: a_{n-1} (n-1 это нижний индекс).

%%%%%%%%%%%%%%%%%%%%% Часто используемые символы %%%%%%%%%%%%%%%%
% Большинство значков используется в математическом режиме:
%     внутри $...$ или \[...\].
% - символ  |- это  \vdash.
% - греческие буквы: \alpha, \beta, \gamma, \eps,...
%     или заглавные: \Alpha, \Sigma, \Omega,...
% - принадлежит (элемент множеству): \in
% - индексы: a^n, L_i, L^+, L^*
% - объединение/пересечение:
%     двух множеств: A \cup B, A \cap B,
%     семейства множеств: \bigcup_{n \ge 0} L^n
% - нестрогие неравенства: \le, \ge,
% - пустое множество: \es
% - кавычки в тексте: <<текст>>
% - троеточие: \ldots, а не ...! Заполнение точками: \dotfill
% - угловые скобки: \langle \rangle
% - Для тире следует использовать --- а не один -
% - Для длины слова: | а не \mid.
% - Текст внутри формулы: \text.
% - Кавычки: не ", а <<ёлочки>> для русских слов и дважды ' ' для английских.
%   образец кавычек-ёлочек для русских слов «праволинейный»
% - Не пишите слова типа begin и liFe внутри $$: у них будут неверные пробелы между буквами,
%    используйте обычный курсив: \emph{здесь курсив}.
% - Для горизонтального отступа в формуле вместо mspace: \quad или \qquad (поменьше и побольше).
% - Следите за длинными формулами. Если формулы две, например, h(0)=a, h(1)=b, то нужно
%    ставить их в две разные пары долларов: $h(0)=a$, $h(1)=b$, а не в одну. Если формулы
%    длинная и плохо помещается на строке (выходит на поля), то делайте её выключной (т.е. \[...\]).
% - Следите за знаками препинания. В и~т.~д. желательно ставить вот такие ~ (неразрывный пробел).
% - разбивайте строки с помощью Enter на более короткие, иначе под Git'ом неудобно diff'ы смотреть.
%%%%%%%%%%%%%%%%%%%%% Крупные элементы %%%%%%%%%%%%%%%%%%%%%%%%%%%%
% Системы уравнений:
%\begin{equation}% нумерованное уравнение
%    \begin{cases}
%        X_1 = 1X_1 + 0X_2 + \es X_3; \\
%        X_2 = 1X_1 + \es X_2 + 0X_3; \\
%        X_3 = 0X_1 + \es X_2 + 1X_3.
%    \end{cases}
%\end{equation}
% Если нужна текстовая вставка внутри любой формулы
% (например, системы уравнений), она делается с помощью
% команды \text{текст}.
%
% Продукции:
%\begin{equation}
%\begin{array}{l}
%    S \to 1A \mid 2S; \\
%    A \to 0B \mid 0S \mid 1A; \\
%    B \to 1C \mid 2C; \\
%    C \to \eps \mid 1S \mid 2A.
%\end{array}
%\end{equation}
%
% Матрицы (с любыми скобками) и другие элементы:
% http://en.wikibooks.org/wiki/LaTeX/Mathematics#Matrices_and_arrays
%
% Нумерованные списки:
%\begin{enumerate}
%    \item пункт
%    \item и ещё…
%\end{enumerate}

% почта/скайп: ulysses4ever, +79612902878 Telegram

% вопросы, которые не решились, ставьте %TODO: ....
% в конце прогнать весь документ через CTRL+F и все 0 проверить на \es
%%%%%%%%%%%%%%%%% Окружения типа Теорема %%%%%%%%%%%%%%%%%%%%%%%%%%%%%
% Оформление теорем (ntheorem)

\usepackage{MnSymbol}
\usepackage [thmmarks, amsmath] {ntheorem}
\theorempreskipamount 0.3cm

\theoremstyle {plain} %
\theoremheaderfont {\normalfont \bfseries} %
\theorembodyfont {\slshape} %
\theoremsymbol {\ensuremath {_\square}} %
\theoremseparator {.} %
\newtheorem {mystatement} {Утверждение} [section] %
\newtheorem {mylemma} {Лемма} [section] %
\newtheorem {mytheorem} {Теорема} [section] %
\newtheorem {mycorollary} {Следствие} [section] %

\theoremstyle {nonumberplain} %
\theoremseparator {.} %
\theoremsymbol {\ensuremath {_\diamondsuit}} %
\newtheorem {mydefinition} {Определение} %

\theoremstyle {plain} %
\theoremheaderfont {\normalfont \bfseries}
\theorembodyfont {\normalfont}
%\theoremsymbol {\ensuremath {_\Box}} %
\theoremseparator {.} %
\newtheorem {mytask} {Задача} [section]%
\renewcommand{\themytask}{\arabic{mytask}}

\theorembodyfont {\upshape} %
\theoremseparator {.} %

\theoremsymbol {\ensuremath{\blacksquare}} %
\newtheorem {myexample} {Пример}[section] %
\newtheorem {myexamples}[section] {Примеры} %

\theoremsymbol {\ensuremath{\square}} %
\newtheorem {myproblem} {Упражнение}[section] %

\theorempreskipamount 0cm

\theoremstyle {nonumberplain} %
\newtheorem {myremark} {Замечание} %

\theoremheaderfont {\itshape} %
\theorembodyfont {\upshape} %
\theoremsymbol {\rule {1ex} {1ex}} %
\newtheorem {myproof} {Доказательство} %

\theorempreskipamount 0.3cm

\theoremseparator {:} %
\theoremsymbol {\ensuremath {_\triangle}} %
\theoremstyle {nonumberbreak} %
\newtheorem {myremarks} {Замечания} %

%%%%%%%  Алгоритмы

\usepackage{clrscode}
\DeclareMathOperator{\Gen}{Gen}
\DeclareMathOperator{\Reach}{Reach}

% Линейка
\newcommand*\varhrulefill[1][0.4pt]{\leavevmode\leaders\hrule height#1\hfill\kern0pt}


% Список для описания алгоритма
\newlist{algoenum}{enumerate}{3}
\setlist[algoenum]{%
     label=\textit{Шаг~\arabic*.}%
    %,before=\raggedright%{}
    ,leftmargin=\parindent+\widthof{\textit{Шаг~9.}}+\labelsep%{}
    ,topsep=0pt
}

\usepackage{etoolbox}
\usepackage{float}
\floatstyle{ruled}

 \makeatletter
\patchcmd{\fs@ruled}
 {\def\@fs@post{\kern2pt\hrule\relax}}
 {\def\@fs@post{\kern2pt\hrule height 0pt depth .8pt\relax}}
 {}{}

\renewcommand\floatc@ruled[2]{{\@fs@cfont #1.} #2\par}
\makeatother

\floatname{AlgoEnv}{Алгоритм}
\newfloat{AlgoEnv}{htbp}{loa}[section]



\newcommand{\AlgoPre}[3]{%
\begin{description}
    \item [Вход:]
#1

    \item [Выход:]
#2

    \item [Метод:]
#3
\end{description}%
}

\newcommand{\AlgoPreNoMeth}[2]{%
\begin{description}
    \item [Вход:]
#1

    \item [Выход:]
#2
\end{description}%
}

\newcommand{\AlgoBody}[1]{%
\begin{algoenum}
#1
\end{algoenum}
}

% #1 - placement (optional), #2 - name, #3-5 - preamble (In/Out/Method), 6 - body
\newcommand{\Algo}[6][t]{%
\begin{AlgoEnv}[#1]
\caption{#2}
\AlgoPre{#3}{#4}{#5}
\vspace{-5mm}
\varhrulefill[.2pt]
\AlgoBody{#6}
\end{AlgoEnv}
}

% #1 - placement (optional), #2 - name, #3-4 - preamble (In/Out), 5 - body
\newcommand{\AlgoNoMeth}[5][t]{%
\begin{AlgoEnv}[#1]
\caption{#2}
\AlgoPreNoMeth{#3}{#4}
\vspace{-5mm}
\varhrulefill[.2pt]
\AlgoBody{#5}
\end{AlgoEnv}
}

% #1 - placement (optional), #2 - name, #3-4 - preamble (In/Out), 5 - body
\newcommand{\AlgoPseudoCode}[5][t]{%
\begin{AlgoEnv}[#1]
\caption{#2}
\AlgoPreNoMeth{#3}{#4}
\vspace{-5mm}
\varhrulefill[.2pt]
\begin{codebox}
#5
\end{codebox}
\end{AlgoEnv}
}


% {mystatement} {Утверждение}
% {mylemma} {Лемма}
% {mycorollary} {Следствие}
% {myproof} {Доказательство}
% {myremark} {Замечание}
% {mytask} {Задача}
% {myexample} {Пример}
% {mydefinition} {Определение}
% {mytheorem} {Теорема}
% {myproblem} {Упражнение}
% Пример использования (с myexample) см. в начале chapter-1.tex
\newcommand {\la} {\langle}
\newcommand {\ra} {\rangle}
\renewcommand {\le} {\leqslant}
\renewcommand {\ge} {\geqslant}
\renewcommand {\leq} {\leqslant}
\renewcommand {\geq} {\geqslant}
\renewcommand {\emptyset} {\varnothing}
\newcommand {\es} {\varnothing}
\newcommand {\eps} {\varepsilon}
\newcommand {\mydef}[1]{\emph{#1}}
\renewcommand{\To}{\Rightarrow}

\newcommand{\Sig}{\ensuremath{\Sigma}}
\newcommand{\N}{\ensuremath{\mathbb N}}
\newcommand{\NO}{\ensuremath{\mathbb N_0}}
\DeclareMathOperator{\RE}{RE}

% Неразрывный дефис, который допускает перенос внутри слов,
% типа жёлто-синий: нужно писать жёлто"/синий.
\makeatletter
    \defineshorthand[russian]{"/}{\mbox{-}\bbl@allowhyphens}
\makeatother


\begin{document}
\tableofcontents
\large

\addchap{Введение}
\label{Intro}
В настоящее время в связи с бурным развитием информационных технологий и расширения сферы их применения в теоретической информатике возникают новые и развиваются старые направления. Всё это находит своё отражение в формировании учебных дисциплин по направлению <<Фундаментальная информатика и информационные технологии>>. 

Целью изучения дисциплины <<Теория конечных автоматов и формальных языков>> является выработка у студентов компетенций, связанных с теоретическими понятиями, идеями, методами, моделями, алгоритмами, применяемыми при разработке и сопровождению проблемно-ориентированных языков. Эта дисциплина содержит сведения, необходимые для научно-исследовательской и практической работы в области системного программирования, использования и развития языков программирования.

К основным задачам дисциплины относится следующее: 
\begin{itemize}
\item освоение математических основ теории формальных языков; изучение классов формальных языков и конечных способов задания потенциально бесконечных множеств; овладение методами разбора слов, принадлежащим языку, определения эквивалентности выводов, установление однозначности грамматик;
\item освоение математических основ теории регулярных языков и выражений; изучение свойств регулярных выражений; \item овладение методами построения регулярных выражений для языков, заданных неформально и в виде формальных грамматик; освоение методов перехода от грамматики к регулярному выражению и наоборот;
\item освоение математических основ теории детерминированных и недетерминированных конечных автоматов; изучение общей схемы автомата-распознавателя; освоение метода редукции недетерминированного автомата к недетерминированному; овладение методами построение конечного автомата по регулярной грамматике; изучение свойств праволинейных, регулярных и автоматных языков;
\item освоение понятия конечного автомата со спонтанными переходами; овладение методами построения конечного автомата по праволинейной грамматике и вычисления языка конечного автомата с помощью последовательного исключения состояний; изучение понятий отношения эквивалентности на множестве конечных автоматов, недостижимых состояний автомата;
\item изучение задачи минимизации детерминированного конечного автомата; освоение понятий различимых и неразличимых состояний и метода построения системы отношений неразличимости состояний конечного автомата; 
\item овладение методами определения регулярности языка; изучение алгоритмических проблем регулярных языков;
\item освоение математических основ теории контекстно-свободных (КС) языков; изучение понятий дерева вывода в КС-грамматике; овладение методами проверки КС-грамматики на непустоту, построения по КС-грамматике стабилизационного множества, построения приведенной КС-грамматики; изучение нормальных форм Хомского и Грейбах КС-языков; освоение метода определения принадлежности цепочки языку, заданного КС-грамматикой;
\item освоение математических основ теории автоматов с магазинной памятью (МП-автоматов); изучение понятия недетерминизма для МП-автоматов, расширенной формы МП-автомата (РМП-автомат), автомата, допускающего язык опустошением магазина (МП$\eps$-автомат); овладение методами перехода от МП-автомата к МП$\eps$-автомату и наоборот; 
\item овладение методом построения МП$\eps$-автомата по КС-грамматике; овладение методом перехода от МП$\eps$-автомата к КС-грамматике. 
\end{itemize}

Исчерпывающее изложение элементов теории формальных языков и конечных автоматов имеется в ставшей редкостью замечательной монографии А.Ахо и Дж.Ульмана~\cite{AU}, материалы которой в современной форме представлены в~\cite{Hop}. Настоящий учебник содержит систематическое изложение значительной части материала курса <<Теория конечных автоматов и формальных языков>> в объёме, достаточном для успешного его освоения. При изложении материала мы пытались придерживаться канонов монографии А.Ахо и Дж.Ульмана~\cite{AU}, опуская, однако, более глубокие аспекты рассматриваемой теории, на которые не достает времени в рамках данного курса. По курсу <<Теория автоматов и формальных языков>> можно порекомендовать ряд хороших книг, названия некоторых содержатся в списке литературы. 

Учебник состоит из введения, восьми глав, списка литературы и четырёх приложений. В главе~\ref{Chapter1} идет речь о способах задания и распознавания формальных языков; в главе~\ref{Chapter2} исследуются регулярные языки; в главах~\ref{Chapter3} и~\ref{Chapter4} изучаются разные типы конечных автоматов и доказывается совпадение классов конечно-автоматных, регулярных и праволинейных языков; в главе~\ref{Chapter5} изучается булева алгебра регулярных языков, свойства замкнутости операций над регулярными множествами и алгоритмические проблемы регулярных языков; в главах~\ref{cfg-intro} и~\ref{normal-cfg} исследуются контекстно-свободные грамматики и языки; в главе~\ref{Chapter8FSMSM} устанавливается связь между контекстно"/свободными языками и конечными автоматами с магазинной памятью. Каждая глава снабжена набором упражнений для лучшего понимания и усвоения материала. В учебнике имеется четыре приложения, которые содержат алгоритмы для контекстно"/свободных грамматик, задания к курсовой работе и варианты к ним, пример выполнения одного варианта курсовой работы.

Отметим, что у читателя предполагается знакомство с некоторыми темами стандартного курса <<Дискретной математики>>, в остальном же изложение замкнуто. Полезным, однако, является хорошее освоение материала курсов <<Математическая логика>> и <<Теория алгоритмов>>.

Нумерация всех утверждений имеет вид: $\alpha.\beta.\gamma$, где $\alpha$ --- номер главы, $\beta$ --- номер раздела главы, $\gamma$ --- номер утверждения в разделе. Формулировки теорем, лемм и утверждений заканчиваются символом $_\square$, а доказательства заканчиваются символом $\blacksquare$. Формулировки упражнений заканчиваются символом $\square$, а окончания примеров помечаются символом $\blacksquare$.

В тексте не выделяются и не нумеруются определения, однако некоторые важные понятия, появляющиеся по ходу изложения, набраны \mydef{курсивом}.

\chapter{Способы задания и распознавания формальных языков}
\label{Chapter1}
\section{Алфавит и слова}
\label{Chapter1Alphabet}
Алфавитом будем называть любое множество символов. Предполагается,
что термин <<символ>> имеет достаточно ясный интуитивный смысл и не
нуждается в дальнейшем пояснении. Алфавит, вообще говоря, не обязан
быть ни конечным, ни даже счетным, но везде далее мы будем
предполагать его конечным. Термины <<буква>> и <<знак>> используются
как синонимы термина <<символ>> для обозначения элемента алфавита.
Термин <<буква>> будет для нас основным.

Если написать последовательность букв алфавита $\Sigma$, располагая их одну за другой, то получится <<слово>>. Термины <<цепочка>>, <<строка>> и даже <<предложение>> часто используются как синонимы термина <<слово>>. Термин <<слово>> будет для нас основным.

Слово называется пустым, если оно не содержит ни одной буквы. Это слово обозначается символом $\eps$.

\begin{myexample}
Латинский алфавит: множество, состоящее из 26 прописных и 26 строчных латинских букв; \emph{begin} --- слово в этом алфавите.
\end{myexample}

\begin{myexample}
Бинарный (или двоичный) алфавит: множество $(0;1)$; $001001$ --- слово в этом алфавите.
\end{myexample}

Определим две операции над словами:

\begin{enumerate}%
    \item
Если $\alpha$ и $\beta$ --- слова, то слово $\alpha\beta$ называется \mydef{конкатенацией} (\mydef{сцеплением}) $\alpha$ и $\beta$. Например, если $\alpha= \text{вино}$, $\beta=\text{град}$, то $\alpha\beta=\text{виноград}$. Для любого слова $\alpha$ всегда $\alpha\eps=\eps\alpha=\alpha$.

    \item
\mydef{Обращением} слова $\alpha$ называется слово $\alpha^R$, которое отличается от $\alpha$ только порядком следования входящих в него букв, т.е. если
 $a_1$, $a_2$, \ldots , $a_n$ --- буквы и $\alpha=a_1 \ldots a_n$, то $\alpha^R=a_n\ldots a_1$. Ясно, что
$\eps^R=\eps$. Например, если $\alpha= \text{нос}$, то $\alpha^R=\text{сон}$.
\end{enumerate}


Пусть $\alpha$, $\beta$ и $\gamma$ --- произвольные слова в некотором алфавите $\Sigma$. Назовем $\alpha$ \mydef{префиксом} слова $\alpha\beta$, а $\beta$ --- \mydef{суффиксом} слова $\alpha\beta$. Слово $\beta$ назовем подсловом слова $\alpha\beta\gamma$. Префикс и суффикс являются подсловами. Заметим, что пустое слово является префиксом, суффиксом и подсловом любого слова. Если $\alpha\neq\beta$ и $\alpha$ --- префикс слова $\beta$, то $\alpha$ называется собственным префиксом слова $\beta$; аналогично определяются собственные суффиксы и подслова.

Слова вида $aa$\ldots$a$ ($n$ букв) будем записывать короче: $a^n$.
Например:
\[aabbba=a^2b^3a, \qquad \eps=a^0.\]

\mydef{Длиной} слова будем называть число букв в нем. Длину слова $\alpha$ будем обозначать $|\alpha|$. Например, $|aab|=3,$ $|\eps|=0$.

\begin{myproblem}
Выясните, сколько букв в русском алфавите.
\end{myproblem}

\begin{myproblem}
Найдите все префиксы, суффиксы и подслова слова \emph{арбуз}.
\end{myproblem}

\section{Языки и операции над языками}
\label{Chapter1LangsOps}

\mydef{Языком} над алфавитом $\Sigma$ называется произвольное множество слов, записанных буквами из $\Sigma$. Под это определение, конечно, подходит почти любое известное понятие языка. Например, английский и русский языки, различные алгоритмические языки и т.~д.

Рассмотрим простейшие примеры языков над некоторым алфавитом $\Sigma$:
\begin{itemize}
    \item пустое множество $\es$;
    \item множество $\{\eps\}$, состоящее из одного пустого слова;
    \item множество $\{a\}$, где $a\in\Sigma$, состоящее из одного однобуквенного слова.
\end{itemize}
Отметим, что $\es$ и $\{\eps\}$ --- два различных языка.

Обозначим через $\Sigma^*$ множество, содержащее все слова в алфавите $\Sigma$, включая $\eps$. Пусть $\Sigma^+=\Sigma^* - \{\eps\}$. Например, если $\Sigma$ --- бинарный алфавит $\{0,1\}$, то
\[
    \Sigma^* = \{ \eps; 0; 1; 00; 01; 10; 11; 000; 001; \ldots \}.
\]
Каждый язык $L$ в алфавите $\Sigma$ является подмножеством множества $\Sigma^*$ и содержит язык $\es$. Отметим, что слово
``\emph{liFe}'', составленное из английских букв, не является словом английского языка, а слово <<жызнь>>, составленное из русских букв, не является словом русского языка.

\begin{myexample}
Рассмотрим язык $L_{1}$, содержащий все слова из нуля или более букв $a$. Тогда $L_1=\{a^i\mid i\ge0\}=\{a\}^*$.
\end{myexample}

В тех случаях, когда это не может привести к путанице, мы будем обозначать множество, состоящее из одного элемента, самим элементом. В соответствии с этим соглашением $a^*=\{a\}^*$.

Если язык $L$ таков, что произвольное слово из $L$ не может являться собственным префиксом (суффиксом) никакого другого слова из $L$, то говорят, что $L$ обладает префиксным (суффиксным) свойством. Например, язык $a^*$ не обладает префиксным свойством, а язык $\{a^ib \mid i \ge 0\}$ этим свойством обладает. В некотором смысле слова из языка, обладающего префиксным свойством, не продолжимы вправо, а слова из языка, обладающего суффиксным свойством, не продолжимы влево.

Рассмотрим некоторые операции над языками. Из того, что язык является множеством, вытекает, что операции объединения, пересечения и разности применимы к произвольным языкам. Операцию конкатенации можно применять к языкам так же, как и к словам: а именно, если $L_1$ --- язык в алфавите $\Sigma_1$, а $L_2$ --- язык в алфавите $\Sigma_2$, то язык $L_1L_2$, называемый конкатенацией (а также сцеплением или произведением) языков $L_1$, и $L_2$. определяется равенством:
\[
	L_1L_2 = \{xy \mid x\in L_1, y\in L_2\} \quad  (\subset\{\Sigma_1\cup\Sigma_2\}^*).
\]

Итерации языка 2 определяются следующим образом:
\begin{enumerate}
\item $L^0 = \{\eps\}$,
\item $L^n = LL^{n-1}$ для $n\ge1$,
\item $L^* = \bigcup_{n \ge 0} L^n$ --- полная итерация,
\item $L^+ = \bigcup_{n \ge 1}L^{n}$ --- позитивная итерация.
\end{enumerate}

Отметим, что
\[
	L^+ = LL^* = L^*L, \qquad  L^* = L^+\cup\{\eps\}.
\]

Пусть $\Sigma_1$ и $\Sigma_2$ --- алфавиты. Рассмотрим произвольное
отображение $h\colon \Sigma_1 \to \Sigma_2^*$ и расширим его до
отображения $\Sigma_1^*$ в $\Sigma_2^*$, полагая
\[
    h(\eps)=\eps, \quad h(xa)=h(x)(a)
\]
для всех $x\in\Sigma_1^*$ и $a\in\Sigma_1$. Легко
показать, что новое отображение определено корректно. Для него мы
сохраним символ $h$. Отображение $h\colon  \Sigma_1^*\to\Sigma_2^*$
называется \mydef{гомоморфизмом}.

Применяя гомоморфизм $h$ к языку $L$, мы получаем новый язык $h(L)$, который представляет собой множество слов $\{h(\omega)\mid\omega\in L\}$.

\begin{myexample}
Рассмотрим алфавиты $\Sigma_1=\{0;1\}$ и $\Sigma_2=\{a;b\}$. Пусть
\[
    L=\{0^n1^n\mid n\ge1 \}.
\]
Предположим, мы хотим заменить каждое вхождение буквы $0$ в словах из языка $L$ на букву $a$, а каждое вхождение буквы $1$ --- на $bb$. Тогда можно определить гомоморфизм $h$ так, что $h(0)=a, h(1)=bb$. В этом случае $h(L)=\{a^nb^{2n}\mid n\ge1 \}$.
\end{myexample}

Пусть $\Sigma_1$ и $\Sigma_2$ --- алфавиты, $L$ --- язык над алфавитом $\Sigma_2$. Рассмотрим произвольный гомоморфизм $h\colon \Sigma_1^*\to\Sigma_2^*$ . \mydef{Прообразом} языка $L$ называется язык
\[h^{-1}(L) = \bigcup_{y\in L}h^{-1}(y) = \{x\mid h(x)\in L\}\  (\subset\Sigma_1^*).\]

\begin{myexample}
Пусть $h\colon \{0;1\}^*\to\{a;b\}^*$ --- гомоморфизм, для которого
\[h(0)=h(1)=a.\] Тогда $h^{-1}(a)=\{0;1\},~h^{-1}(b)=\es$. Пусть
$L_1=\{b\}^*, L_2=\{a\}^*$. Тогда $h^{-1}(L_1)=\{\eps\}$, $h^{-1}(L_2)=\{0;1\}^*$.
\end{myexample}

\begin{myexample}
Пусть $h\colon \{0;1\}^*\to\{a;b\}^*$ --- гомоморфизм, для которого
$h(0)=a$, $h(1)=\eps$. Тогда $h^{-1}(\eps)=1^*,~h^{-1}(a)=\{1^i01^j\mid i,j\ge0\}$.
\end{myexample}

\begin{myproblem}
Верно ли, что $L^+=L^*-\{\eps\}$.
\end{myproblem}

\begin{myproblem}
Какие из следующих языков обладают префиксным (суффиксным) свойством?
\begin{itemize}
    \item $\es$;
    \item $\{\eps\}$;
    \item $\{a^nb^n\mid n\ge1 \}$;
    \item $L^*$, если $L$ обладает префиксным (суффиксным) свойством;
    \item $\{\omega\mid \omega\in\{a,b\}^*$ и число символов $a$ в $\omega$ равно числу символов $b\}$.
\end{itemize}
\end{myproblem}

\begin{myproblem}
Пусть $h\colon \{0;1;2\}^*\to\{a;b\}^*$ --- гомоморфизм, определённый равенствами $h(0)=a$, $h(1)=bb$ и $h(2)=\eps$. Опишите языки $h(\{012\}^*)$, $h(\{0;1;2\}^*)$, $h^{-1}(\{ab\}^*)$.
\end{myproblem}


\begin{myproblem}
Докажите или опровергните следующие утверждения:
\begin{gather*}
h^{-1}(h(L))=L, \\
h(h^{-1}(L))=L.
\end{gather*}
\end{myproblem}

\section{Грамматики}
\label{Chapter1Grammars}

\mydef{Грамматика} --- это математическая система, определяющая язык.
В грамматике $G$, определяющей язык $L$, используются два конечных непересекающихся множества символов: множество нетерминальных символов, которое часто будет обозначаться буквой $N$, и множество терминальных символов, обозначаемое обычно $\Sigma$. Из терминальных символов образуются слова языка $L$, а нетерминальные символы играют вспомогательную роль. Ядром грамматики является конечное множество $P$ правил образования, которые описывают процесс порождения слов языка.

Дадим теперь точное определение грамматики. \mydef{Грамматикой} называется четверка $G=(N,\Sigma,P,S)$, где $N$ --- конечное множество нетерминальных символов (алфавит нетерминалов), $\Sigma$ --- не пересекающееся с $N$ конечное множество терминальных символов (алфавит терминалов), $P$ --- конечное подмножество множества

\[
	(N\cup\Sigma)^*N(N\cup\Sigma)^* \times (N\cup\Sigma)^*.
\]
Пара $(\alpha,\beta)\in P$, где $\alpha\in(N\cup\Sigma)^*N(N\cup\Sigma)^*$,$\beta\in(N\cup\Sigma)^*$, называется \mydef{правилом} или \mydef{продукцией} и записывается в виде $\alpha\to\beta$. $S$ --- выделенный символ из $N$, называемый \mydef{начальным} или
\mydef{стартовым} символом.

\begin{myexample}
\label{example11}
Рассмотрим грамматику $G=(\{A;S\},\{0;1\},P,S)$, где $P$ состоит из продукций
\begin{equation}
\begin{array}{l}
	S \to 0A1,  \\
	S \to 01A01,\\
	S \to 00A1, \\
	A \to \eps.
\end{array}
\end{equation}
Нетерминальными символами в этой грамматике являются буквы $A$ и $S$, терминальными --- 0 и 1, а $S$ --- начальный символ.
\end{myexample}

Продукции с одинаковыми правыми частями будем иногда записывать в одну строчку через символ |, например, две первых продукции из примера~\ref{example11} будем записывать так:
\begin{equation}
\begin{array}{l}
	S \to 0A1 \mid 01A01.
\end{array}
\end{equation}

Грамматика определяет язык рекурсивным образом. Рекурсивность проявляется в определении особого рода слов, называемых выводимыми словами грамматики $G=(N,\Sigma,P,S)$: $S$ --- выводимое слово; если $\alpha\beta\gamma$ --- выводимое слово и $(\beta\to\delta)\in P$, то $\alpha\beta\gamma$ --- тоже выводимое слово; никакие другие слова нe являются выводимыми. Выводимое слово грамматики $G$. не содержащее нетерминальных символов, называется терминальным словом, порождаемым грамматикой $G$.

Язык $L(G)$, порождаемый грамматикой $G$, определяется как множество всех терминальных слов, порождаемых грамматикой $G$.

Пусть $G=(N,\Sigma,P,S)$ --- некоторая грамматика. Если $\alpha\beta\gamma\in(N\cup\Sigma)^*$ и $(\beta\to\delta)\in P$, то будем говорить, что слово $\alpha\delta\gamma$ непосредственно выводимо из $\alpha\beta\gamma$ и записывать это так: $\alpha\beta\gamma\to_G\alpha\delta\gamma$. В тех случаях, когда из контекста ясно, о какой грамматике идет речь, нижний индекс $G$ будем опускать. Через $\to^k$ будем обозначать $k$-ю степень отношения $\to$, т.е. будем записывать $\alpha\to^k\beta$, если существует последовательность $k+1$ слов $\alpha_0$, $\alpha_1$, $\ldots$ , $\alpha_k$, для которых $\alpha=\alpha_0$, $\alpha_{i-1}\to\alpha_i$ при $1\le i\le k$ и $\alpha_k=\beta$. Эта последовательность слов называется выводом длины $k$ слова $\beta$ из слова $\alpha$ в грамматике $G$. Транзитивное замыкание отношения $\to$ обозначим через $\to^+$; $\varphi\to^+\psi$  читается так: слово $\psi$ выводимо из $\varphi$ нетривиальным образом. Рефлексивное и транзитивное замыкание отношения $\to$ обозначим через $\to^*$; $\varphi\to^*\psi$ читается так: $\psi$ выводимо из $\varphi$. Отметим, что $\alpha\to^+\beta$ тогда и только тогда, когда $\alpha\to^i\beta$ для некоторого $i\ge 1$, а $\alpha\to^*\beta$ тогда и только тогда, когда $\alpha\to^i\beta$ для некоторого $i\ge 0$.
Таким образом, $L(G) = \{\omega\mid\omega\in\Sigma^*, S\to^*\omega\}$.

\begin{myproblem}
Рассмотрим грамматику $G=(\{A;S\},\{0;1\},P,S)$, где $P$ состоит из продукций:
\begin{equation}
\begin{array}{l}
    S  \to 0A1,\\
    0A \to 00A1,\\
    A  \to \eps.
\end{array}
\end{equation}
Доказать, что $L(G)=\{0^n1^n\mid n\ge 1\}$.
\end{myproblem}

Приведем существенные для дальнейшего примеры грамматик.
\begin{myexample}
\label{exampleDigitsGrammar}
Пусть $G=(\{\emph{Ц}\}$,
$\{0;1;\ldots;9\}$,
${\emph{Ц} \to 0\mid1\mid\dots\mid9}$,
$\emph{Ц})$.
В грамматике $G$ единственный нетерминальный символ. Ясно, что $L(G)=\{0;1;\ldots;9\}$.
\end{myexample}

\begin{myexample}
\label{exampleArithmGrammar}
Пусть $G_0=(\{E;T;F\},~\{a;+;*;(;)\},P,E)$, где $P$ состоит из продукций:
\begin{equation}
\begin{array}{l}
	E  \to E+T \mid T,\\
	T  \to T*F \mid F,\\
	F  \to (E) \mid a.
\end{array}
\end{equation}
Пример вывода в этой грамматике:
\begin{multline*}
	E \To
    E+T \To 
    T+T \To \\
    F+T \To 
    a+T \To 
    a+T*F \To 
    a+F*F \To \\
    a+a*F \To 
    a+a*a.
\end{multline*}
Можно доказать, что язык $L(G_0)$ --- это множество арифметических выражений, построенных из пяти символов $+$, $*$, $a$, $($, $)$.
\end{myexample}

\begin{myexample}
\label{exampleAnBnCnGrammar}
Рассмотрим грамматику $G=(\{B;C;S\},\{a,b,c\},P,S)$, где $P$ состоит из продукций:
\begin{equation}
\begin{array}{l}
	S  \to aSBC \mid abC,\\
	CB \to BC,\\
	bB \to bb,\\
	bC \to bc,\\
	cC \to cc.
\end{array}
\end{equation}
Пример вывода в этой грамматике:
\begin{*ne}
	S \To 
    aSBC \To
    aabCBC \To 
    aabBCC \To \\
    aabbCC \To 
    aabbcC \To
    aabbcc.
\end{multline*}
Можно доказать, что эта грамматика порождает язык $\{a^nb^nc^n\mid n \ge1\}$.
\end{myexample}

\begin{myexample}
\label{exampleFreeGrammar}
Пусть $G=(\{A;B;C;D;S\},\{a;b\},P,S)$, где $P$ состоит из продукций:
\begin{align*}
	S  &\to CD, &
    Ab &\to bA, \\
	C  &\to aCA \mid bCB \mid \eps, &
    Ba &\to aB, \\
	AD &\to aD, &
    Bb &\to bB, \\
	BD &\to bD, &
    Aa &\to aA,\\
	D  &\to \eps.
\end{align*}
Пример вывода в этой грамматике:
\begin{multline*}
	S \To
    CD \To 
    aCAD \To
    abCBAD \To \\
    abBAD \To 
    abBaD \To 
    abaBD \To 
    ababD \To 
    abab.
\end{multline*}
Покажем, что $L(G)=\{\omega\omega\mid\omega\in\{a,b\}^*\}$, т.е. язык $L(G)$ состоит из слов четной длины, составленных из букв $a$ и $b$, причем первая половина каждого слова совпадает со второй половиной.

Сначала докажем, что $\{\omega\omega\mid\omega\in\{a,b\}^*\}\subseteq L(G)$. Для этого надо проверить, что каждое слово вида $\omega\omega$ можно вывести из $S$. Непосредственно проверяется, что в $G$ возможны следующие выводы:
\begin{enumerate}
	\item $S \To CD$
	\item $C \To^n~c_1c_2\ldots c_nCX_nX_{n-1}\ldots X_1 \To c_1c_2\ldots c_nX_nX_{n-1}\ldots X_1$,  где для всех $i$ $c_i=a$ тогда и только тогда, когда $X_i=A$, и $c_i=b$ тогда и только тогда, когда $X_i=B$;
	\item
\begin{multline*}
    X_n\ldots X_2X_1D \To X_n\ldots X_2c_1D \To^{n-1} c_1X_n \ldots X_2D \To \\
     c_1X_n\ldots X_3c_2D \To^{n-2} c_1c_2X_n \ldots X_3D \To \dots \To \\
     c_1c_2\ldots c_{n-1}X_nD \To c_1c_2\ldots c_{n-1}c_nD \To c_1c_2\ldots c_{n-1}c_n.
\end{multline*}
\end{enumerate}

В 2) из $C$ выводится слово, составленное из букв $a$ и $b$, за которым следует его зеркальное отражение, составленное соответственно из букв $A$ и $B$.

В 3) нетерминалы $A$ и $B$ перемещаются к правому концу слова, где $A$ становится терминалом $a$, а $B$ становится терминалом $b$, вступая в контакт с нетерминалом $D$. Нетерминалы $A$ и $B$ могут превратиться в терминалы единственным способом --- только передвинувшись к правому концу слова. При этом слово, составленное из букв $A$ и $B$, будет обращено и совпадет, таким образом, со cловом с буквами $a$ и $b$, выведенным из $C$ в выводе 2).

Комбинируя выводы 1), 2) и 3), получаем для $n\ge 0$
\[
	S \To^*~c_1c_2\ldots c_nc_1c_2\ldots c_n,
\]
где $c_i\in\{a,b\}$ для $1\le i\le n$. Итак, $\{\omega\omega\mid\omega\in\{a,b\}^*\}\subseteq L(G)$.

Теперь докажем, что $L(G)\subseteq\{\omega\omega\mid\omega\in\{a,b\}^*\}$. Для этого надо проверить, что из $S$ выводятся только те терминальные слова, которые имеют вид $\omega\omega$. Зададим два гомоморфизма:

\[
	g\colon\{a;b;A;B\}^* \to \{a;b;A;B\}^*, \qquad
    h\colon\{a;b;A;B\}^* \to \{a;b;A;B\}^*,
\]
удовлетворяющие условиям:
\begin{align*}
    g(a)&=a, &
    g(b)&=b,  &
    g(A)&=g(B)=\eps, \\
    h(A)&=A,  &
    h(B)&=B,  &
    h(a)&=h(b)=\eps.
\end{align*}
Применяя метод математической индукции по параметру $m$ и анализируя множество продукций $P$, легко доказать следующее вспомогательное утверждение: если $S \To^m \alpha$, то $\alpha$ представимо в виде
\[
    c_1c_2\ldots c_nU\beta V,
\]
где $c_i\in\{a;b\}$, $U\in\{C;\eps\}$, $V\in\{D;\eps\}$, а $\beta$ такое слово длины $n$ из языка $\{a;b;A;B\}^*$, что
\[
    g(\beta)=c_1c_2\ldots c_i, \qquad
    h(\beta)=X_nX_{n-1}\ldots X_{i+1},
\]
где
\[
    X_j =
        \begin{cases}
            A, \text{ если $c_{j}=a$;} \\
            B, \text{ если $c_{j}=b$.}
        \end{cases}
\]
Из этого утверждения вытекает, что все слова из $L(G)$ имеют вид
\[
    c_1c_2\ldots c_nc_1c_2\ldots c_n,
\]
где $c_i\in\{a,b\}$, поэтому $L(G)\subseteq\{\omega\omega\mid\omega\in\{a,b\}^*\}$.

Итак, равенство $L(G)=\{\omega\omega\mid\omega\in\{a,b\}^*\}$ доказано.
\end{myexample}

\begin{myproblem}
Докажите, что язык $L(G_0)$ из примера~\ref{exampleArithmGrammar} совпадает со множеством всех арифметических выражений, построенных из пяти символов:
\[
    +, \quad *, \quad a, \quad (, \quad ).
\]
\end{myproblem}

\begin{myproblem}
Докажите, что грамматика из примера~\ref{exampleAnBnCnGrammar} порождает язык \[\{a^nb^nc^n\mid n\ge 1\}.\]
\end{myproblem}

\section{Классификация грамматик}
\label{Chapter1GrammarsClasses}

В соответствии с подходом Хомского грамматики классифицируют по структуре их продукций. Грамматика $G=(N,\Sigma,P,S)$ называется:
\begin{enumerate}
    \item \mydef{праволинейной}, если каждая продукция из $P$
    имеет вид
    $A \to xB$ или $A \to x$, где $A,B \in N$, $x\in\Sigma^*$;

    \item \mydef{контекстно-свободной} (или \mydef{бесконтекстной}),
    если каждая продукция из
    $P$ имеет вид $A \to \alpha$, где $A \in N$,
    $\alpha\in(N\cup\Sigma)^*$;

    \item \mydef{контекстно-зависимой} (или \mydef{неукорачивающейся}),
    если каждая продукция из $P$ имеет вид
    $\alpha\to\beta$, где $|\alpha| \le |\beta|$.
\end{enumerate}

Грамматика, не удовлетворяющая ни одному из указанных ограничений,
называется грамматикой общего вида (или без ограничений). Далее мы
будем пользоваться сокращениями ПЛ, КС и КЗ для терминов
«праволинейный», «кон\-текс\-тно"/свободный» и «контекстно"/зависимый»
соответственно.

Грамматика примера~\ref{exampleDigitsGrammar} --- праволинейная. Такой же является и грамматика $G=(\{S\},\{0;1\},\{S\to0S\mid1S\mid\eps\},S)$ с языком $\{0,1\}^*$.

Примером КС-грамматики служит грамматика из примера~\ref{exampleArithmGrammar}. Ясно, что
каждая ПЛ-грамматика является контекстно-свободной.

В примере ~\ref{exampleAnBnCnGrammar} грамматика является контекстно-зависимой.

КЗ-грамматика не допускает продукций вида $A\to\eps$, называемых
$\eps$"/продукциями. Поэтому КС-грамматика, содержащая
$\eps$"/продукции, не является КЗ-грамматикой.

Если язык $L$ порождается грамматикой типа $i$, то $L$ называется
языком типа $i$. Таким образом, язык $L(G)$ примера~\ref{exampleDigitsGrammar} ---
праволинейный, язык $L(G_0)$ примера~\ref{exampleArithmGrammar} --- контекстно-свободный, а
язык $L(G)$ примера~\ref{exampleAnBnCnGrammar} --- контекстно-зависимый. Язык, порожденный
грамматикой примера~\ref{exampleFreeGrammar}, --- это язык без ограничений.

Если язык задан какой-то грамматикой, это еще не значит, что его нельзя породить менее мощной грамматикой. Например, КС-грамматика
\[
	G=(\{A;S\}, \{0;1\}, \{S\to AS \mid \eps; A \to 0 \mid 1 \}, S)
\]
порождает язык $\{0;1\}^*$, который, как отмечено выше, можно
получить и с помощью ПЛ"/грамматики.

Определенные выше четыре типа грамматик и языков называют
\mydef{иерархией Хомского}.

\begin{myproblem}
Постройте ПЛ-грамматику для языка, состоящего из идентификаторов, которые могут быть произвольной длины, но должны начинаться с буквы (как в Алголе).
\end{myproblem}

\begin{myproblem}
Постройте ПЛ-грамматику для языка, состоящего из идентификаторов, которые должны содержать от одного до шести символов и начинаться с буквы $I$, $J$, $K$, $L$, $M$ или $N$(как идентификаторы целых переменных в Фортране).
\end{myproblem}
\begin{myproblem}
Постройте КС-грамматику, порождающую язык
\[
	(a_1a_2 \ldots a_na_n \ldots a_2a_1 \mid
    	a_i\in\{0,1\}, 1\le i\le n\}.
\]
\end{myproblem}

\section{Распознаватели}
\label{Chapter1Parsers}

В~\ref{Chapter1Grammars} было отмечено, что грамматика --- это математическая
система, определяющая язык. Второй метод, обеспечивающий
задание языка конечными средствами, состоит в использовании
\mydef{распознавателей}. Распознаватель это очень схематизированный
алгоритм, определяющий некоторое множество. В нём можно
выделить три основные части: \mydef{входную ленту} ВЛ,
\mydef{управляющее устройство} с конечной памятью УУ, \mydef{вспомогательную} или \mydef{рабочую память} РП.

Входная лента (ВЛ) --- это линейная последовательность ячеек,
причем каждая
ячейка содержит точно одну букву из некоторого конечного входного
алфавита. Самую левую и самую правую ячейки могут занимать особые
концевые маркеры; причем маркер может стоять только на правом конце
ленты, или маркеров может не быть совсем.

Входная головка в каждый данный момент читает одну входную ячейку.
За один шаг работы распознавателя входная головка может
сдвинуться на одну
ячейку влево, остаться неподвижной или сдвинуться на одну ячейку
вправо. Распознаватель, который никогда не передвигает свою входную
головку влево, называется  односторонним. Обычно предполагается, что
входная головка только читает, т.е. в ходе работы распознавателя
буквы на входной ленте не меняются. Но можно рассматривать и такие
распознаватели, входная головка которых и читает, и пишет.

Рабочей памятью (РП) распознавателя может быть любого типа хранилище
информации. Предполагается, что алфавит памяти конечен и хранящаяся в
памяти информация образована только из букв этого алфавита.
Предполагается также, что в любой момент времени можно конечными
средствами описать содержимое и структуру памяти, хотя с течением
времени объем памяти может становиться сколь угодно большим.

Ядром распознавателя является управляющее устройство с конечной памятью
(УУ), под которым можно понимать программу, управляющую поведением
распознавателя. УУ представляет собой конечное множество состояний
вместе с функцией, которое описывает, как меняются состояния в
соответствии с текущим входным символом и текущей информацией,
извлеченной из памяти. УУ определяет также, в каком направлении
сдвинуть входную головку и какую информацию поместить в память.

Распознаватель работает, проделывая некоторую последовательность
\mydef{шагов} (или \mydef{тактов}). В начале такта читается
текущий входной символ и исследуется память. После этого с
распознавателем происходит следующее:
\begin{enumerate}
    \item входная головка сдвигается на одну ячейку влево, одну ячейку вправо
или сохраняется в исходном положении;

    \item в память помещается некоторая информация;

    \item изменяется состояние управляющего устройства.
\end{enumerate}

Поведение распознавателя удобно описывать в терминах
\mydef{конфигураций}
распознавателя, которые описывают состояние УУ, содержимое ВЛ
вместе с положением входной головки и, наконец, содержимое
РП. Во множестве всех состояний выделяют начальное состояние
и множество заключительных
состояний. Конфигурация называется \mydef{начальной},
если УУ находится в начальном состоянии, входная головка
читает самую левую
букву на ВЛ, и РП имеет заранее установленное начальное
содержимое. Конфигурация называется \mydef{заключительной}, если УУ находится в
заключительном состоянии, а входная головка читает правый концевой
маркер или, если маркер отсутствует, сошла с правого конца входной
ленты. (Иногда требуют, чтобы РП в заключительной конфигурации тоже
удовлетворяла некоторым условиям.)

Говорят, что распознаватель \mydef{допускает} входное слово $\omega$,
если, начиная с начальной конфигурации, в которой слово $\omega$
записано на входной ленте, распознаватель может проделать
последовательность шагов, заканчивающуюся заключительной
конфигурацией. Язык, определяемый распознавателем, --- это множество
входных слов, которые он допускает.

Для каждого класса грамматик из иерархии Хомского существует
естественный класс распознавателей, определяющий тот же класс языков.
Некоторые из таких распознавателей будут изучаться далее.

\section{Упражнения}
\label{Chapter1Exs}

Для каждой грамматики, встречающейся в заданиях, следует указать её тип (в
иерархии Хомского). Написать грамматику, порождающую:
\begin{enumerate}
 \item язык $\Sig^{\ast}$, где (a) $\Sig = \{0, 1\}$; (b) \Sig{}~— произвольный
 (конечный) алфавит;

 \item некоторый конечный язык $L = \{\omega_i\}^n_{i=1}$;

 \item $\{a^{+}b^{+}\}, \{a^nb^n \mid n \in \N\}, \{a^nb^na^m \mid m, n \in
     \N\}$; $\{ a^nb^nc^n \mid n \in \N \}$;

 \item множество правильных скобочных последовательностей («язык Дика») с
     одним типом скобок;
 \item множество правильных скобочных последовательностей («язык Дика») с
     двумя типами скобок;
  \item арифметическую прогрессию $\{a + nd \mid n \in \NO\}$, $d > 0$, $0
    \leqslant a < d$ (имея в виду изоморфизм моноидов $(\NO, +) \cong
    (\{|\}^{\ast}, {\cdot})$, где ${\cdot}$ означает операцию конкатенации);
   \item язык, являющийся объединением конечного числа арифметических прогрессий.
\end{enumerate}

\chapter{Регулярные языки}
\label{Chapter2}
\section{Регулярные множества и регулярные выражения}
\label{Chapter2RegExprs}

Пусть $\Sigma$ --- конечный алфавит. \mydef{Регулярное множество} над алфавитом
$\Sigma$ определяется рекурсивно следующим образом: множества $\es$,
$\{\eps\}$ и $\{a\}$ для каждого $a$ из $\Sigma$ --- регулярные; если $P$
и $Q$ --- регулярные множества, то таковы же и множества $P\cup Q$,
$PQ$, $P^*$; ничто другое не является регулярным множеством. Другими
словами, множество регулярно над алфавитом $\Sigma$ тогда и только
тогда, когда оно либо $\es$, либо $\{\eps\}$, либо $\{a\}$ для
некоторого $a\in\Sigma$. либо его можно получить из этих множеств
применением конечного числа операций объединения, конкатенации и
итерации.

Регулярное множество над алфавитом $\Sigma$ иначе называется регулярным языком над этим алфавитом

Для обозначения регулярных множеств используют регулярные выражения.
Регулярные выражения над алфавитом $\Sigma$ и регулярные множества,
которые они обозначают, определяются рекурсивно следующим образом:
$\es$ --- регулярное выражение, обозначающее регулярное множество
$\es$; $\eps$ --- регулярное выражение, обозначающее регулярное
множество $\{\eps\}$ если $a\in\Sigma$, то $a$ --- регулярное
выражение, обозначающее регулярное множество $\{a\}$; если $p$ и $q$ ---
регулярные выражения, обозначающие регулярные множества $P$ и $Q$
соответственно, то $(p+q)$, $(pq)$ и $(p)^*$ --- регулярные выражения,
обозначающие $P\cup Q$, $PQ$ и $p^*$ соответственно; ничто другое не
является регулярным выражением.

Для сокращения будем далее вместо $pp^*$ писать $p^+$ и устранять из регулярных выражений лишние скобки там, где это не приводит к недоразумениям. При этом будем предполагать, что наивысшим приоритетом обладает итерация, затем конкатенация и, наконец, операция $+$. Например, вместо $(01(1^*))+(1(0^*)))$ будем писать $01^++10^*$.

\begin{myexample}
Приведем несколько регулярных выражений и соответствующих им множеств:
\begin{enumerate}
	\item $(0+1)^*$ обозначает $\{0;1\}^*$;
	\item $(0+1)^*0101$ обозначает множество всех слов над алфавитом $\{0;1\}$ с суффиксом $0101$;
	\item $(a+b+c+d)(a+b+c+d+0+1)^*$ обозначает множество всех слов над алфавитом $\{0;1;a;b;c;d\}$ с префиксами $a, b, c$ или $d$;
	\item $(00+11)^*((01+10)(00+11)^*(01+10)(01+11)^*)^* $обозначает множество всех слов над алфавитом $\{0;1\},$ содержащих четное число нулей и четное число единиц.
\end{enumerate}

Последний пример представляется весьма поучительным, и хотелось бы порекомендовать читателю поразмышлять над ним до тех пор, пока он не станет «очевидным».
\end{myexample}

Для каждого регулярного множества можно найти по крайней мере одно регулярное выражение, которое его обозначает. Обратно, для каждого регулярного выражения можно построить регулярное множество, обозначаемое этим выражением. Но для каждого регулярного множества существует бесконечно много обозначающих его регулярных выражений. Будем говорить, что два регулярных выражения равны $(=)$, если они обозначают одно и то же множество.

\begin{mylemma}
\label{LemmaRegExprFeat}
Пусть $\alpha$, $\beta$, $\gamma$ --- регулярные выражения. Тогда
\begin{multicols}{2}
\begin{enumerate}
	\item $\alpha+\es=\alpha$,
	\item $\alpha+\alpha=\alpha$,
	\item $\alpha+\beta=\beta+\alpha$,
	\item $\alpha+(\beta+\gamma)=(\alpha+\beta)+\gamma$,
	\item $\alpha(\beta+\gamma)=\alpha\beta+\alpha\gamma$,
	\item $(\alpha+\beta)\gamma=\alpha\gamma+\beta\gamma$,
	\item $\alpha\eps=\eps\alpha=\alpha$,
	\item $\es\alpha=\alpha\es=\es$,
	\item $\alpha(\beta\gamma)=(\alpha\beta)\gamma$,
	\item $\es^*=\eps$,
	\item $(\alpha^*)^*=\alpha^*$,
	\item $\alpha^*=\alpha+\alpha^*$,
	\item $(\alpha+\beta)^*=(\alpha^*\beta^*)^*$,
	\item $\alpha\alpha^*+\eps=\alpha^*$.
\end{enumerate}
\end{multicols}
\end{mylemma}

\begin{myproof}
Каждое из равенств легко проверить, используя свойства множеств и приведенные выше определения. Рассмотрим, для примера, лишь некоторые из приведенных равенств.

Равенство 4). Пусть $\alpha$, $\beta$ и $\gamma$ --- регулярные выражения, обозначающие регулярные множества $A$, $B$ и $\Gamma$ соответственно. Тогда $\beta+\gamma$ обозначает $B \cup \Gamma$, а $\alpha+(\beta+\gamma)$ обозначает $A \cup (B \cup \Gamma)$; аналогично, $(\alpha+\beta)+\gamma$ обозначает $(A \cup B)\cup\Gamma$. Но $A \cup (B \cup \Gamma) = (A \cup B) \cup \Gamma$, поэтому $\alpha+(\beta+\gamma)=(\alpha+\beta)+\gamma$.

Равенство 12). Пусть $\alpha$ --- регулярное выражение, обозначающее регулярное множество $A$. Ясно, что множество $A^*= \bigcup_{n\ge 0}A^n$ содержит $A$, поэтому $A^*\cup A=A^*$ и, следовательно, $\alpha^*+\alpha=\alpha^*$.
\end{myproof}

\begin{myproblem}
Приведите доказательства всех равенств из леммы~\ref{LemmaRegExprFeat} Постарайтесь обнаружить новые равенства и доказать их (см. \cite{Sal}).
\end{myproblem}

В дальнейшем, если это не будет приводить к недоразумению, мы не будем различать регулярное выражение и обозначаемое им регулярное множество.

\section{Уравнения и системы уравнений с регулярными коэффициентами}
\label{Chapter2Systems}

Рассмотрим уравнение
\begin{equation}\label{lin-reg-eq}
    X=\alpha X+\beta,
\end{equation}
где $\alpha$ и $\beta$ --- регулярные выражения над некоторым
алфавитом $\Sigma$. Прямой подстановкой проверим, что
$X=\alpha^*\beta$~--- решение уравнения~\eqref{lin-reg-eq}:
%TODO вот такой вот "список" здесь, лучшее что я придумал это так:

(?)	$\quad \alpha^*\beta=\alpha(\alpha^*\beta)+\beta,$

(?) $\quad \alpha^*\beta=(\alpha\alpha^*)\beta+\eps\beta,$

(?)	$\quad \alpha^*\beta=(\alpha\alpha^*+\eps)\beta,$

(!)	$\quad \alpha^*\beta=\alpha^*\beta.$

Решение уравнения с регулярными коэффициентами может быть не единственным. Например, если в уравнении~\eqref{lin-reg-eq} потребовать, чтобы $\eps\in\alpha$, то
\[
    X=\alpha^*(\beta\cup L)
\]
будет решением этого уравнения для любого (не обязательно регулярного) языка $L$. Действительно,

(?) $\quad \alpha^*(\beta\cup L)=\alpha(\alpha^*(\beta\cup L))\cup\beta,$

(?)	$\quad \alpha^*(\beta\cup L)=(\alpha\alpha^*)(\beta\cup L))\cup\beta,$

(?)	$\quad \alpha^*(\beta\cup L)=\alpha^*(\beta\cup L)\cup\beta,$

(!)	$\quad \alpha^*(\beta\cup L)=\alpha^*(\beta\cup L).$

Таким образом, уравнение~\eqref{lin-reg-eq} при $\eps \in \alpha$
% TODO: подумать. Написано, вроде, \eps \in \alpha, но
% кажется, что это неверно.
имеет бесконечно много решений.

В теории формальных языков представляет интерес, главным образом, наименьшее решение, которое называют наименьшей неподвижной точкой. (В математике решение уравнения вида $x=f(x)$ принято называть неподвижной точкой отображения $f$, поэтому термин <<неподвижная точка>> используется и здесь~\cite{Bau}.)

\begin{mytheorem}
\label{Theorem-NNT-EQ}
Пусть $\alpha$ и $\beta$ --- регулярные выражения над алфавитом $\Sigma$ и $X\notin\Sigma$. Рассмотрим уравнение~\eqref{lin-reg-eq}. Тогда
\begin{enumerate}
    \item $X=\alpha^*\beta$ --- наименьшая неподвижная точка уравнения;

    \item если $\eps\notin\alpha$, то $X=\alpha^*\beta$ --- единственное
    решение уравнения;

    \item если $\eps\in\alpha$, то для произвольного (не обязательно
    регулярного) языка $L$ язык $\alpha^*(\beta\cup L)$ является
    решением уравнения, и наоборот, всякое решение уравнения
    может быть записано в таком виде.
\end{enumerate}
\end{mytheorem}

\begin{myproof}
\hfill \break
1) Ранее мы проверили, что $X=\alpha^*\beta$ --- решение уравнения~\eqref{lin-reg-eq}. Пусть $\Omega$ --- произвольное решение этого уравнения; покажем, что $\alpha^*\beta\subset\Omega$. Предположим, что $\omega\in\alpha^*\beta$. Тогда $\omega=\omega_k\omega_{k-1}\ldots\omega_1b$, где $\omega_i\in\alpha$, $b\in\beta$. Из того, что $\Omega$ --- решение уравнения~\eqref{lin-reg-eq}, вытекает, что
\[
	\Omega = \alpha\Omega + \beta
\]
и, следовательно, $\beta\subset\Omega$, $\alpha\Omega\subset\Omega.$ Тогда
\[
b\in\Omega, \omega_1b\in\Omega, \omega_2\omega_1b\in\Omega, \ldots , \omega_k\omega_{k-1}\ldots\omega_1b\in\Omega.
\]
Таким образом, $\alpha^*\beta\subset\Omega$. Это означает, что $X=\alpha^*\beta$ - наименьшая неподвижная точка.

2)	Пусть $\eps\notin\alpha$. Предположим, что $\Omega$ --- произвольное решение уравнения~\eqref{lin-reg-eq}, и покажем, что $\Omega=\alpha^*\beta$. Пусть $\omega=a_1a_2\ldots a_m\in\Omega$ где $a_i\in\Sigma$. Из того, что $\Omega=\alpha\Omega+\beta$, вытекает: либо $\omega\in\beta\subset\alpha^*\beta$, либо $\omega\in\alpha\Omega$. Если $\omega\in\alpha\Omega=\alpha\alpha\Omega+\alpha\beta$, то либо $\omega\in\alpha\beta\subset\alpha^*\beta$, либо $\omega\in\alpha\alpha\Omega$. Продолжим этот процесс до тех пор, пока альтернатива $\omega\in\alpha\alpha\ldots\alpha\Omega$ окажется невозможной (в силу того, что $\omega$ имеет $m$ букв, а $\eps\notin\alpha$, это обязательно произойдет по крайней мере при $k>m$). В итоге получим, что $\omega\in\alpha^*\beta$.

3) Пусть $\eps\in\alpha$. Ранее мы проверили, что множество $\alpha^*(\beta\cup L)$, где $L$ --- произвольный язык, является решением уравнения~\eqref{lin-reg-eq}. Пусть теперь $\Omega$ --- какое-нибудь решение этого уравнения; покажем, что $\Omega$ можно записать в виде $\alpha^*(\beta\cup L)$, где $L$ --- некоторый язык. По условию
\[
\Omega = \alpha\Omega + \beta.
\]
Отсюда вытекает, что
\[
\alpha\Omega \subseteq \Omega, \beta \subseteq \Omega.
\]
С другой стороны, из условия $\eps\in\alpha$ следует, что
\[
\Omega \subseteq \alpha\Omega.
\]
Получаем равенство:
\[
\Omega = \alpha\Omega.
\]
Его непосредственным следствием является равенство:
\[
\Omega = \alpha^*\Omega.
\]
Тогда в силу соотношения $\beta\subseteq\Omega$ получаем:
\[
\Omega = \alpha^*(\Omega \cup \beta),
\]
так как $\Omega=\Omega\cup\beta$. Итак, действительно показано, что если множество $\Omega$ --- решение уравнения~\eqref{lin-reg-eq}, то оно имеет такой вид:
\[
\Omega = \alpha^*(L \cup \beta),
\]
где $L$ --- некоторый язык.
\end{myproof}

Систему уравнений с регулярными коэффициентами над некоторым алфавитом $\Sigma$ назовем стандартной системой со множеством неизвестных
\[
	\Delta = \{ X_1;X_2;\ldots ;X_n \},
\]
если она имеет вид
\begin{equation}
\label{GeneralSysWRegExprK}
\begin{cases}
X_{1} = \alpha_{10} + \alpha_{11}X_1 + \alpha_{12}X_2 + \ldots  + \alpha_{1n}X_n\\
X_{2} = \alpha_{20} + \alpha_{21}X_1 + \alpha_{22}X_2 + \ldots  + \alpha_{2n}X_n\\
\dotfill\\
X_n = \alpha_{n0} + \alpha_{n1}X_1 + \alpha_{n2}X_2 + \ldots  + \alpha_{nn}X_n
\end{cases}
\end{equation}
где $\alpha_{ij}$ --- регулярные выражения над $\Sigma$ и $\Sigma\cap\Delta=\es$.

Коэффициентами уравнений являются регулярные выражения $\alpha_{ij}$. Если $\alpha_{ij}=\es$, то в уравнении для $X_i$ нет компонента, содержащего $X_j$. Аналогично, если $\alpha_{ij}=\eps$, то в уравнении для $X_i$ компонент, содержащий $X_j$, совпадает с $X_j$. Иными словами, $\es$ играет роль коэффициента $0$, а $\eps$ - роль коэффициента $1$ в обычных линейных алгебраических уравнениях.

Пусть $Q$ --- стандартная система уравнения со множеством неизвестных $\Delta$ и с коэффициентами над алфавитом $\Sigma$. Отображение $f$ множества $\Delta$ во множество языков над $\Sigma$ называют решением системы $Q$, если после подстановки $f(X)$ вместо $X(\in\Delta)$ в каждое уравнение системы все уравнения становятся равенствами. Отображение $f\colon \Delta\to P(\Sigma^*)$ называют наименьшей неподвижной точкой системы $Q$, если, во-первых, $f$ --- решение и, во-вторых, $f(X)\subseteq g(X)$ для любого другого решения $g$ и всех $X\in\Delta$.

\begin{mytheorem}
\label{theorem-NNTSysReg}
Пусть $Q$ --- стандартная система уравнений, а $Y_i$ --- множество всех слов $\omega\ldots\omega_m$, где $m\ge 1$ и $\omega_m\in\alpha_{j_m0}$, $\omega_k\in\alpha_{j_kj_{k+1}}$ для некоторой последовательности $j_i=j_1, \ldots , j_m (\in\{1;2;\dots ;n\})$. Отображение $f$ множества $\Delta=\{X_1;X_2;\ldots ;X_n\}$ во множество языков над $\Sigma$ определим равенством: $f(X_i)=Y_i$, где $i\in\{1;2;\ldots ;n\}$. Тогда $f$ --- наименьшая неподвижная точка системы $Q$.
\end{mytheorem}

\begin{myproof}
Путем непосредственной подстановки проверяется, что для всех $i$
\[
f(X_{i}) = \alpha_{i0} \cup \alpha_{i1}f(X_1) \cup \dots \cup \alpha_{in}f(X_n).
\]
Таким образом, $f$ --- решение.

Покажем, что $f$ --- наименьшая неподвижная точка. Пусть $g$ --- какое-нибудь решение системы и $i\in\{1;2;\ldots ;n\}$. Рассмотрим произвольное слово $\omega$ из $f(X_i)$. Тогда $\omega=\omega_1\ldots\omega_m (m\ge 1)$, где для некоторой последовательности чисел $j_1, \ldots , j_m (\in\{1;2;\ldots ;n\}$) выполнены условия:
\[
	\omega_m\in\alpha_{j_m0}, \quad \omega_k\in\alpha_{j_kj_{k+1}}
\]
Так как $g$ - решение, то
\[
	g(X_i) = \alpha_{i0} \cup \alpha_{i1}g(X_{1}) \cup \dots  \cup \alpha_{in}g(X_n).
\]
В частности,
\[
	\alpha_{i0} \subseteq g(X_i), 	\alpha_{ik}g(X_k) \subseteq g(X_i)
\]
для всех $i$ и $k$. Тогда
\begin{align*}
    \omega_m & \in \alpha_{j_m0} \subseteq g(X_{j_m}),\\
    \omega_{m-1}\omega_m & \in \alpha_{j_{m-1}j_m}g(X_{j_m})
        \subseteq g(X_{j_{m-1}}),\\
    &\ldots \\
    \omega=\omega_1\omega_2\ldots\omega_m & \in
        \alpha_{j_1j_2}g(X_{j_2}) \subseteq g(X_{j_1})=g(X_i).
\end{align*}

Таким образом, $f(X_i)\subseteq g(X_i)$. Отсюда следует, что $f$ --- наименьшая неподвижная точка системы $Q$.
\end{myproof}

В этой теореме получено описание наименьшей неподвижной точки стандартной системы с регулярными коэффициентами, но, во-первых, нет гарантий, что наименьшая неподвижная точка является регулярным множеством, и, во"/вторых, неясно, как эту наименьшую неподвижную точку найти. Эти вопросы будут рассмотрены в следующем пункте.

\begin{myproblem}
Проверьте (любым способом), что наименьшая неподвижная точка системы
\begin{equation}
\begin{cases}
X_1 = a_{10} + a_{11}X_1 + a_{12}X_2\\
X_2 = a_{20} + a_{21}X_1 + a_{22}X_2
\end{cases}
\end{equation}
имеет вид
\begin{equation}
\begin{cases}
f(X_1) = (a_{11}+a_{12}a^*_{22}a_{21})^*(a_{10}+a_{12}a^*_{22}a_{20}), \\
f(X_2) = (a_{22}+a_{21}a^*_{11}a_{12})^*(a_{20}+a_{21}a^*_{11}a_{10})
\end{cases}
\end{equation}
\end{myproblem}

\begin{myproblem}
Сформулируйте и докажите аналог теоремы~\ref{Theorem-NNT-EQ} для систем.
\end{myproblem}

\section[Алгоритм решения систем с регулярными коэффициентами]{Алгоритм решения систем уравнений с регулярными коэффициентами}
\label{Chapter2SysSolverAlg}

Рассмотрим стандартную систему уравнений с регулярными коэффициентами вида~\eqref{GeneralSysWRegExprK}.

%Алгоритм 2.3.1.%TODO - определить алгоритм, и как делать вот эти штуки: Вход, Выход, Метод, Шаг 1, и т.д. Страница 20 в методичке

\Algo{Решение систем уравнений с регулярными коэффициентами}%
{\label{Algo-SysEq-Solver} Стандартная система уравнения с регулярными коэффициентами над алфавитом $\Sigma$ и множеством неизвестных $\Delta=\{X_1,\ldots ,X_n\}$.}%
{Решение системы в виде $X_1=\alpha_1,\ldots, X_n=\alpha_n$, где $\alpha_i$ --- регулярные выражения над алфавитом $\Sigma$.}%
{Напоминает обычный метод Гаусса решения систем линейных уравнения и основан на систематическом использовании формулы $X=\alpha^*\beta$ для наименьшей неподвижной точки уравнения~\eqref{lin-reg-eq}.}%
{
    \item Положим $i=1$.
		
    \item Если $i=n$, перейти к шагу 4. В противном случае с помощью тождеств из леммы~\ref{LemmaRegExprFeat} записать уравнение для $X_i$ в виде $X_i=\alpha X_i+\beta$, где $\alpha$ и $\beta$ --- регулярные выражения над $\Sigma$. Затем в правых частях уравнений для $X_{i+1}, /ldots , X_n$ заменить $X_i$ регулярным выражением $\alpha^*\beta$.

    \item Увеличить $i$ на 1 и вернуться к шагу 2.

    \item Записать уравнение для $X_n$ в виде $X_n=\alpha X_n+\beta$ где $\alpha$ и $\beta$ --- регулярные выражения над $\Sigma$. (После выполнения шага 2 для каждого $i$ при $l<n$ в правой части уравнения для $X_i$ не будет неизвестных $X_1, \ldots,X_{i-1}$. В частности, на шаге 4 этим свойством будет обладать и уравнение для $X_n$.) Перейти к шагу 5 (при этом $i=n$).

%    \item Уравнение для $X_i$ имеет вид $X_i=\alpha X_i+\beta$, где $\alpha$ и $\beta$ --- регулярные выражения над $\Sigma$. Записать $X_i=\alpha^*\beta$ и в уравнения для $X_{i-1}, \ldots , X_1$ подставить $\alpha^*\beta$ вместо $X_i$.

%    \item Если $i=1$, то остановиться. В противном случае уменьшить $i$ на 1 и вернуться к шагу 5.
}

\begin{mytheorem}
\label{Theorem-Algo-SysEq-Solver-RegLang}
Алгоритм~\ref{Algo-SysEq-Solver} строит наименьшую неподвижную точку системы уравнений как регулярный язык.
\end{mytheorem}

Доказательству теоремы~\ref{Theorem-Algo-SysEq-Solver-RegLang} предпошлём две леммы.

\begin{mylemma}
\label{RegSysSolver-lemma1}
Пусть $Q_1$ и $Q_2$ --- системы уравнений с регулярными коэффициентами до и после одного применения шага 2 алгоритма~\ref{Algo-SysEq-Solver}. Тогда эти системы имеют одну и ту же наименьшую неподвижную точку.
\end{mylemma}

\begin{myproof}
Отметим, что в ходе работы алгоритма для $1\le l<i$ коэффициент при $A_l$ в уравнении для $A_i$ оказывается равным $\es$. Предположим, что на шаге 2 рассматривается уравнение
\[
A_i = \alpha_{i0} + \alpha{il}A_i + \alpha_{i,i+1}A_{i+1} + \ldots + \alpha_{in}A_n
\]
системы $Q_1$.

В системах $Q_1$ и $Q_2$ уравнения для $A_j$ при $j\le i$ совпадают, а при $j>i$ эти уравнения совпадать не обязаны.

Пусть $j>i$ и
\begin{equation}
\label{eq231}
A_j = \alpha_{j0} + \underset{h=i}{\overset{n}{\sum}} \alpha_{jk}A_h	%(2.3.1)
\end{equation}
это уравнение для $A_j$ в системе $Q_1$. Тогда в силу того, что
\begin{equation*}
	A_i = \alpha^*_{ii}\alpha_{i0} + \underset{h=i+1}{\overset{n}{\sum}} \alpha^*_{ii}\alpha_{ih}A_h,
\end{equation*}
уравнение для $A_j$ в системе $Q_2$ имеет вид
\begin{equation}
\label{eq232}
A_j = (\alpha_{j0}+\alpha_{ji}\alpha^*_{ii}\alpha_{i0}) + \underset{h=i}{\overset{n}{\sum}} (\alpha_{jk}+\alpha_{ji}\alpha*_{ii}\alpha_{ih})A_h.
\end{equation}

Пусть
\begin{equation*}	\beta_{jk}=\alpha_{jh}+\alpha_{ji}\alpha*_{ii}\alpha_{jk} \quad	(k\in\{0;1+1;\ldots ;n\}.
\end{equation*}
Через $f_1$ и $f_2$ обозначим наименьшие неподвижные точки систем $Q_1$ и $Q_2$ соответственно. Далее будем пользоваться тем описанием наименьших неподвижных точек систем, которое дано в теореме~\ref{theorem-NNTSysReg}. В частности, $f_2(A_j)$ --- множество всех слов, представимых в виде $\omega_1\ldots\omega_m$, где $m\ge 1$ и $\omega_m\in\beta_{j_m0}$, $\omega_k\in\beta_{j_{k}j_{k+1}}$ для некоторой последовательности $j_1=j, \dots , j_{m} (\in\{1;2;\ldots ;n\})$.

Сравнение формул~\eqref{eq231} и~\eqref{eq232} показывает, что
\[
	f_2(A_j)\subseteq f_1(A_j).
\]
Действительно, любое слово $x$ из $\alpha_{ji}\alpha^*_{ii}\alpha_{jk}$ можно представить в виде $x_1x_2\ldots x_r$, где $\omega_1\in\alpha_{ji}, \omega_r\in\alpha_{jk}$ и $\omega_2,\ldots ,\omega_{r-1}\in\alpha_{ii}$. Таким образом, слово $\omega$ является конкатенацией слов, каждое из которых принадлежит множеству, обозначаемому некоторым коэффициентом системы $Q_1$, и индексы этих коэффициентов удовлетворяют необходимым условиям.

	Докажем обратное включение
\[
	f_2(A_j)\supseteq f_1(A_j).
\]
Предположим, что $\omega\in f_1(A_j)$. Тогда $\omega=\omega_1\ldots\omega_m$, где для некоторой последовательности ненулевых индексов $l_1=f,\ldots ,l_m$ выполнены условия $\omega_m\in\alpha_{l_m0}$ и $\omega_p\in\alpha_{l_{p}l_{p+1}}$ для $1\le p<m$. Сгруппируем слова $\omega_p$ так, чтобы имело место равенство $\omega=y_1\ldots y_r$, где $y_p=\omega_t\ldots\omega_s$, и
\begin{enumerate}
	\item если $l_t\le i$, то $s=t+1$,
	\item если $l_t>i$, то $s$ таково, что $l_{t+1}=\ldots =l_s=i$, $l_{s+1}\neq i$.
\end{enumerate}
Отсюда следует, что в любом случае $y_p$ --- коэффициент при $A_{j_{s+1}}$ в уравнении для $A_{l_t}$ системы $Q_2$, и, значит, $\omega\in f_2(A_j)$. В итоге получаем, что $f_1(A_j)=f_2(A_j)$ для всех $j$.
\end{myproof}

\begin{mylemma}
Пусть $Q_1$ и $Q_2$ --- системы уравнений с регулярными коэффициентами до и после одного применения шага 5 алгоритма~\ref{Algo-SysEq-Solver}. Тогда эти системы имеют одну и ту же наименьшую неподвижную точку.
\end{mylemma}

\begin{myproof}
аналогично доказательству леммы~\ref{RegSysSolver-lemma1} и здесь не приводится.
\end{myproof}

\begin{myproof}
теоремы~\ref{Theorem-Algo-SysEq-Solver-RegLang}. После того как шаг 5 алгоритма~\ref{Algo-SysEq-Solver} применен для всех $i$, мы получим новую систему, у которой все уравнения имеют вид $X_i=\alpha_i$, где $\alpha_i$ --- регулярное выражение над алфавитом $\Sigma$. Согласно теореме~\ref{theorem-NNTSysReg} отображение $f$, для которого $f(X_{i})=\alpha_{i}$, является наименьшей неподвижной точкой этой новой системы. В силу лемм~\ref{RegSysSolver-lemma1} и~\ref{RegSysSolver-lemma2} в ходе работы алгоритма наименьшая неподвижная точка не менялась. Таким образом, показано, что алгоритм~\ref{Algo-SysEq-Solver} строит наименьшую неподвижную точку системы уравнений как регулярный язык.
\end{myproof}

\begin{myproblem}
Найдите наименьшую неподвижную точку системы~\eqref{GeneralSysWRegExprK} при условии, что $\alpha_{i0}=\es$ для всех $i$.
\end{myproblem}

\section{Совпадение классов регулярных и ПЛ-языков}
\label{Chapter2MatchRegRL}

\begin{mylemma}
\label{lemma-PL-elem-lang}
Пусть $\Sigma$ --- конечный алфавит. Множества $\es$, $\{\eps\}$ и $\{a\}$ для всех $a\in\Sigma$ являются ПЛ"/языками.
\end{mylemma}

\begin{myproof}
Чтобы доказать лемму, достаточно для каждого из трех множеств построить такую ПЛ-грамматику $G$,  чтобы язык $L(G)$ совпадал с этим множеством:
\begin{enumerate}[itemsep=1ex]
\item $G_\es=(\{S\},\Sigma,\es,S)$ --- ПЛ-грамматика и $L(G_{\es})=\es;$
\item $G_\eps=(\{S\},\Sigma,\{S\to\eps\},S)$ --- ПЛ-грамматика и $L(G_\eps)=\{\eps\};$
\item $G_a=(\{S\}, \Sigma, \{S\to a\},S)$ --- ПЛ-грамматика и $L(G_a)=\{a\}$.
\end{enumerate}
\end{myproof}

\begin{mylemma}
\label{lemma-PL-op-lang}
Пусть $\Sigma$ --- конечный алфавит. Если $L_1$, $L_2$ --- ПЛ-языки над $\Sigma$, то и языки $L_3=L_1\cup L_2$, $L_4=L_1L_2$, $L_5=L^*_1$ тоже являются ПЛ-языками над $\Sigma$.
\end{mylemma}

\begin{myproof}
Пусть $G_1=(N_1,\Sigma,P_1,S_1)$, $G_2=(N_2,\Sigma,P_2,S_2)$ --- такие ПЛ"/грамматики, для которых $L(G_1)=L_1$, $L(G_2)=L_2$. Не теряя общности, будем предполагать, что алфавиты $N_1$ и $N_2$ не пересекаются. Теперь, как и в доказательстве леммы~\ref{lemma-PL-elem-lang}, для каждого из трех новых языков $L_i$ построим такую ПЛ-грамматику $G_i$, чтобы $L(G_i)=L_i$, $i\in\{3;4;5\}$.
\begin{enumerate}
\item Пусть $G_3 = (N_1\cup N_2\cup\{S_3\}$, $\Sigma, P_1\cup P_2\cup\{S_3\to S_1\mid S_2\}$, $S_3)$, где $S_3$ --- новый нетерминальный символ, не принадлежащий ни $N_1$, ни $N_2$. Ясно, что $G_3$ --- ПЛ-грамматика, причем $L(G_3)=L_1\cup L_2$, так как для каждого вывода $S_3\To_{G_3}^+\omega$ существует либо вывод $S_1\To_{G_1}^+\omega$, либо вывод $S_2\To_{G_2}^+\omega$, и обратно. Таким образом, $L_3=L_1\cup L_2$ --- ПЛ-язык.

\item Пусть $G_4=(N_1 \cup N_2 \}, \Sigma, P_4, S_1,)$, а множество продукций $P_4$ определяется так:

\begin{enumerate}[label=(\emph{\roman*})]
	\item если $A,B\in N$, $x\in\Sigma^*$ и $(A\to xB)\in P_1$, то $A\to xB)\in P_4$;
	\item если $A\in N$, $x\in\Sigma^*$ и $(A\to x)\in P_1$, то $(A\to xS_2)\in P_4$;
	\item $P_2\subset P_4$.
\end{enumerate}
Ясно, что $G_4$ --- ПЛ-грамматика. Если $S_1\To_{G_1}^+\omega$, то $S_1\To_{G_4}^+\omega S_2$, а если $S_2\To_{G_2}^+x$, то $S_2\To_{G_4}^+x$. Таким образом, $L(G_1)\cup L(G_2)\subseteq L(G_4)$.

Проверим теперь противоположное вложение. Пусть $S_1\To_{G_4}^+\omega$. Так как в $P_4$ нет продукций вида $A\to x$, которые попали бы туда из $P_1$, то этот вывод можно записать в виде $S_1\To_{G_4}^+xS_2\To_{G_4}^+xy$, где $\omega=xy$ и все продукции, используемые в выводе $S_1\To_{G_4}^+xS_2$, попали в $P_4$ с помощью $(i)$ и $(ii)$. Следовательно, должны быть выводы $S_1\To_{G_1}^+x$ и $S_2\To_{G_2}^+y$. Отсюда вытекает вложение $L(G_4)\subseteq L(G_1)L(G_2)$.

Итак, $L(G_4)=L(G_1)L(G_2)$.

\item Пусть грамматика $G_5=(N_1\cup \{S_5\},\Sigma,P_5,S_5)$ такова, что $S_5\nsubseteq N_1$, а $P_5$ строится следующим образом:
\begin{enumerate}[label=(\emph{\roman*})]
	\item если $A,B\in N$, $x\in\Sigma^*$ и $(A\to xB)\in P_1$, то $(A\to xB)\in P_5$;

	\item если $A\in N$, $x\in\Sigma^*$ и $(A\to x)\in P_1$, то $\{A\to xS_5;A\to x\}\in P_5$;

	\item $\{S_5\to S_1\mid\eps\}\subset P_5$.
\end{enumerate}
Равенство $L(G_5)=(L(G_1))^*$ является следствием того, что вывод
\[
S_5 \To_{G_5}^+ x_1S_5 \To_{G_5}^+ x_1x_2S_5 \To_{G_5}^+ x_1x_2\ldots x_{n-1}S_5 \To_{G_5}^+ x_1x_2\ldots x_{n-1}x_n
\]
реализуем тогда и только тогда, когда реализуема последовательность выводов
\[
S_1 \To_{G_1}^+ x_1,~ S_1 \To_{G_1}^+x_2,~\ldots ~, S_1 \To_{G_1}^+ x_n.
\]
\end{enumerate}
\end{myproof}

\begin{mylemma}
\label{lemma-PL-reg-lang}
Рассмотрим ПЛ-грамматику $G=(N,\Sigma,P,S)$, где
\[
    N=\{A_1=S,\ldots ,A_1\},
\]
и построим стандартную систему уравнений с регулярными коэффициентами, неизвестными которой служат нетерминалы из $N$, а уравнение для $A_i$ определяется равенством
\[
A_i=\alpha_{i0}+\alpha_{i1}A_1+ \ldots +\alpha_{in}A_n,
\]
где
\begin{enumerate}
    \item $\alpha_{i0} = x_1 + \ldots + x_k$, если $A_i \to x_1 \mid \ldots
    \mid x_k$~--- все продукции с левой частью $A_i$ и правой частью,
    состоящей только из терминалов (при $k=0$ полагаем
    $\alpha_{i0}=\es$);

    \item для $j>0$ $\alpha_{ij} = x_1 + \ldots + x_m$, если
    $A_i \to x_1A_j \mid \ldots  \mid x_mA_j$~--- все продукции с
    левой частью $A_i$ и правой частью, содержащей $A_j$
    (при $m=0$ полагаем $\alpha_{ij}=\es$).
\end{enumerate}

Пусть $f$ --- наименьшая неподвижная точка построенной системы уравнений. Тогда $L(G)=f(S)$ --- регулярный язык.
\end{mylemma}

\begin{myproof}
Пусть $\omega\in f(S)$. Ввиду теоремы~\ref{theorem-NNTSysReg} слово $\omega$ можно записать в виде: $\omega=\omega_1\ldots \omega_m$, где $m\ge l$ и $\omega_m\in\alpha_{j_m0}$, $\omega_k\in\alpha_{j_kj_{k+1}}$ для некоторой последовательности $j_1=1, \ldots , j_m (\in\{1;2;\ldots ;n\})$. Заметим, что коэффициенты $\alpha_{ij}$ определены таким образом, что в грамматике $G$ возможен вывод:
\begin{multline}
    S \To_G 
    \omega_1A_{j_2} \To_G 
    \omega_1\omega_2A_{j_3} \To_G 
    \ldots \To_G \\
    \omega_1\omega_2\ldots \omega_{m-1}A_{j_m} \To_G 
    \omega_1\omega_2\ldots \omega_{m-1}\omega_m.
\end{multline}
Это означает, что $f(S)\subseteq L(G)$. С помощью похожих рассуждений проверяется обратное вложение $L(G)\subseteq f(S)$. Напомним теперь, что алгоритм~\ref{Algo-SysEq-Solver} строит $f(S)$ как регулярный язык. Таким образом, язык $L(G)=f(S)$ --- регулярный.
\end{myproof}

\begin{mytheorem}
\label{threorem-RegExp-PL241}
Пусть $\Sigma$ --- конечный алфавит. Для того, чтобы язык $L$ над алфавитом $\Sigma$ был регулярным, необходимо и достаточно, чтобы он был праволинейным.
\end{mytheorem}

\begin{myproof}
Достаточность вытекает из леммы~\ref{lemma-PL-reg-lang}, проверим необходимость. Пусть $L$ --- регулярное множество над $\Sigma$. Напомним, что множество регулярно над алфавитом $\Sigma$ тогда и только тогда, когда оно либо $\es$, либо $\{\eps\}$, либо $\{a\}$ для некоторого $a\in\Sigma$, либо его можно получить из этих множеств применением конечного числа операций объединения, конкатенации и итерации. По лемме~\ref{lemma-PL-elem-lang}1 $\es$, $\{\eps\}$ и $<a>$ --- ПЛ-языки. В силу леммы~\ref{lemma-PL-op-lang} для завершения доказательства теперь можно воспользоваться индукцией по числу шагов построения регулярного множества, где один шаг соответствует применению одного из правил, определяющих регулярное множество.
\end{myproof}

\begin{myexample}
\label{example-FindRegLang}
Пусть $G=(\{A;B;S\},\{0;1\},P,S)$, где $P$ состоит из продукций
\begin{align*}
	S &\to 0A \mid 1S \mid \eps\\
    A &\to 0B \mid 1A\\
    B &\to 0S \mid 1B.
\end{align*}
Построим язык $L(G)$. Применим конструкцию из леммы~\ref{lemma-PL-reg-lang} и составим систему уравнений
\begin{equation*}
\begin{cases}
	X_1 = 0X_2 + 1X_1 + \eps \\
    X_2 = 0X_3 + 1X_2 \\
    X_3 = 0X_1 + 1X_3
\end{cases}
\end{equation*}
где $X_1=S$, $X_2=A$, $X_3=B$. Для решения системы воспользуемся алгоритмом~\ref{Algo-SysEq-Solver} и на его выходе получим:
\begin{equation*}
\begin{cases}
	X_3~=~(01^*01^*0+1)^*01^* \\
    X_2~=~1^*0(01^*01^*0+1)^*01^* \\
    X_1~=~1^*(01^*0(01^*01^*0+1)^*01^*+\eps
\end{cases}
\end{equation*}
Из теоремы~\ref{theorem-NNTSysReg} вытекает, что $L(G)=f(X_1)$, где $f$ --- наименьшая неподвижная точка системы уравнений. Таким образом,
\[
	L(G)~=~1^*(01^*0(01^*01^*0+1)^*01^*+\eps).
\]
\end{myexample}

\begin{myproblem}
Проверьте, что язык $L(G)$ из примера~\ref{example-FindRegLang} состоит из множества всех таких слов над алфавитом $\{0;1\}$, в которых число нулей кратно трем.
\end{myproblem}

\section{Упражнения}
\label{Chapter2Exs}

\subsection*{Составление регулярных выражений}

Написать регулярное выражение для
\begin{enumerate}
    \item языка над $\{a, b, c\}$ из всех слов, содержащих хотя бы один символ
    $a$;
    \item языка над $\{a, b, c\}$ из всех слов, содержащих хотя бы один символ
    $a$ и хотя бы один символ $b$;
    \item языка над $\{0, 1\}$ из всех слов, в которых третий с правого края
    символ равен $1$;
    \item языка над $\{0, 1\}$ из всех слов, в которых нет двух подряд идущих единиц;
    \item языка над $\{0, 1\}$ из всех слов, в которых любая пара смежных нулей,
     расположена левее любой пары смежных единиц;
    \item языка над $\{0, 1\}$ из всех слов с чередующимися нулями и единицами.
\end{enumerate}

\subsection*{Системы линейных уравнений с регулярными коэффициентами}

Решить системы уравнений:
\begin{equation}
    \begin{cases}
    X_1 = aX_1 + aX_2;\\
    X_2 = bX_2 + b.
    \end{cases}
\end{equation}

\begin{equation}
    \begin{cases}
        X_1 = 1X_1 + 0X_2 + \es X_3; \\
        X_2 = 1X_1 + \es X_2 + 0X_3; \\
        X_3 = 0X_1 + \es X_2 + 1X_3.
    \end{cases}
\end{equation}

\begin{equation}
    \begin{cases}
        X_1 = a^{\ast}X_1 + (a+b)^{\ast}X_2;\\
        X_2 = (a+b^{\ast})X_1 + aX_2 + b^\ast.
    \end{cases}
\end{equation}

\begin{equation}
    \begin{cases}
        X_1 = (a+b)X_1 + \es X_2 + a^{\ast}X_3; \\
        X_2 = \es X_1 + aX_2 + a^{\ast}; \\
        X_3 = b^{\ast}X_1 + \es X_2 + a^{\ast}X_3.
    \end{cases}
\end{equation}

Написать регулярные выражения для языков, заданных грамматиками cо
следующими продукциями:
\begin{equation}
\begin{array}{l}
S \to 1A \mid 2S; \\
A \to 0B \mid 0S \mid 1A; \\
B \to 1C \mid 2C; \\
C \to \varepsilon \mid 1S \mid 2A.
\end{array}
\end{equation}

\begin{equation}
\begin{array}{l}
S \to 0A \mid 1S \mid \varepsilon ;\\
A \to 0B \mid 1A;\\
B \to 0S \mid 1B.
\end{array}
\end{equation}

\chapter{Конечные автоматы}
\label{Chapter3}
\section{Определения и примеры}
\label{Chapter3Defines}
\mydef{Конечный автомат} является одним из простейших
распознавателей (см.~\ref{Chapter1Parsers}). Он состоит из входной ленты и
управляющего устройства с конечной памятью. Управляющее
устройство может быть недетерминированным. Входную
головку далее будем полагать односторонней, более того,
фактически мы потребуем, чтобы входная головка сдвигалась
вправо на каждом такте.

\mydef{Недетерминированным конечным} автоматом называется пятерка
\[
    M = (Q,\Sigma, \delta, q_0, F),
\]
где $Q$ --- конечное множество состояний, $\Sigma$ --- конечное
множество допустимых входных символов, $\delta$ --- отображение
множества $\Sigma$ во множество $P(Q)$ всех подмножеств множества $Q$,
$q_0$ ($\in Q)$ --- начальное состояние управляющего устройства,
$F$ ($\subseteq Q)$ --- множество заключительных состояний.

Функцию $\delta$ называют \mydef{функцией переходов}, именно она
фактически определяет поведение управляющего устройства. Функция
переходов по данному <<текущему>> состоянию и <<текущему>> входному
символу указывает набор всех возможных следующих состояний. Нестрого
говоря, автомат дублирует себя так, что в каждом из возможных следующих
состояний находится как бы один из равноправных экземпляров этого
устройства; при этом недетерминированный автомат допускает входное
слово в том случае, если какой"/нибудь из его параллельно работающих
экземпляров достигает допускающего состояния.

<<Недетерминизм>> конечного автомата не следует смешивать со
<<случайностью>>, при которой автомат может случайно выбрать одно из
следующих состояний с фиксированными вероятностями, но этот автомат
всегда имеется только в одном экземпляре. Такой автомат называется
<<вероятностным>>, и мы его изучать не будем.

Детерминированный конечный автомат определим как частный случай
недетерминированного. Именно, пусть $M = (Q,\Sigma,
\delta, q_0, F)$ --- недетерминированный конечный автомат. Автомат $M$
назовем \mydef{детерминированным}, если множество $\delta(q,a)$
содержит не
более одного состояния для любых $q\in Q$ и $a\in\Sigma$. Если
$\delta(q,a)$ содержит точно одно состояние, то автомат $M$ назовем
\mydef{полностью определенным}.

Работа конечного автомата представляет собой некоторую последовательность тактов. Такт определяется текущим состоянием управляющего устройства и входным символом, считываемым в данный момент входной головкой. Сам такт состоит из изменения состояния и сдвига входной головки на одну ячейку вправо. Для того чтобы определить будущее поведение конечного автомата, нужно знать лишь текущее состоятие управляющего устройства и слово на входной ленте, состоящее из буквы под головкой и всех букв, расположенных вправо от неё. Эги два элемента информации дают мгновенное описание конечного автомата, которое называют конфигурацией. Другими словами, если $M = (Q,\\ \Sigma, \delta, q_0, F)$ --- конечный автомат, то пара $(q,\omega)\in Q*\Sigma^*$ называется конфигурацией автомата $M$. Конфигурация $(q_0\omega)$ называется начальной, а конфигурация $(q,\eps)$, где $q\in F$, называется заключительной.

Состояние $p$ называется достижимым, если существует такое слово $\omega$, что $(q_0,\omega)\vdash^*(p,\eps)$.

Такт автомата $M$ представляется бинарным отношением $\vdash_M$ (или, короче, $\vdash$), определенным на конфигурациях: если $q'\in\delta(q,a)$, то $(q,a\omega)\vdash(q',\omega)$ для всех $\omega\in\Sigma^*$. Таким образом, если $M$ находится в состоянии $q$ и входная головка обозревает входной символ $a$, то автомат $M$ может делать такт, за который он переходит в состояние $q'$ сдвигает головку на одну ячейку вправо. Так как автомат $M$, вообще говоря, может быть недетерминированным, то могут быть и другие состояния, в которые он тоже может перейти за один такт.

Запись $C\vdash_M^0 C'$ означает, что $C=C'$, а $C_0\vdash_M^kC_k$ для $k\ge 1$ --- что существуют такие конфигурации $C_i, \ldots ,C_{k-1}$, что $C_i\vdash_MC_{i+1}$ $0\le i<k$. Запись $C\vdash_M^+C'$ означает, что $C\vdash_M^kC'$ для некоторого $k\ge 1$, а $C\vdash_M^*C'$ --- что $C\vdash_M^k$ для некоторого $k\ge 0$. Будем опускать нижний индекс $M$ там, где это не приведет к недоразумениям.

Говорят, что автомат $M$ допускает слово $\omega$, если $(q_0,\omega)\vdash^*(q,\eps)$ для некоторого $q\in F$. Языком, допускаемым (определяемым, распознаваемым) автоматом $M$, называется множество $L(M)$ всех входных слов, допускаемых автоматом $M$, т. е.
\[
	L(M)=\{\omega(\in\Sigma^*)\mid~\exists q\in F\colon (q_0,\omega)\vdash^*(q,\eps)\}.
\]

Рассмотрим два конечных автомата, один из которых является детерминированным (пример~\ref{example-chap3-DKA1}), а второй --- недетерминированным (пример~\ref{aut-312}).

\begin{myexample}
\label{example-chap3-DKA1}
Пусть $M=(\{p;q;r\},\{0;1\},\delta,p,\{r\})$ --- конечный автомат, где функция переходов $\delta$ задаётся таблицей~\ref{tab1}. 
\begin{table}
\centering
\begin{tabular}{cccc}
\toprule
%
\multicolumn{2}{c}{\multirow{2}{*}{\Large $\delta$}}
	& \multicolumn{2}{c}{\text{Вход}} \\
%
\cmidrule(lr){3-4}
%
\multicolumn{2}{c}{}
	& 0  & 1                          \\
%
\midrule
%
\multirow{3}{*}{\text{Состояние}}%
    &  $p$ & $\{q\}$ & $\{p\}$		  \\
    &  $q$ & $\{r\}$ & $\{p\}$		  \\
    &  $r$ & $\{r\}$ & $\{r\}$		  \\
\bottomrule
\end{tabular}
\caption{Функция перехода $\delta$ для автомата из примера~\ref{example-chap3-DKA1}.}
\label{tab1}
\end{table}

Этот автомат допускает все слова из нулей и единиц, содержащие два стоящих рядом нуля. Начальное состояние $p$ можно интерпретировать так: <<два стоящих рядом нуля еще не появились, и предыдущая буква не нуль>>. Состояние $q$ означает, что <<два стоящих рядом нуля еще не появились, но предыдущая буква --- нуль>>. Состояние $r$ означает, что <<Два стоящих рядом нуля уже появились>>. Попав в состояние $r$, автомат остается в этом состоянии навсегда.

Например, если на вход автомата поступило слово $\omega=1010011$, то единственно возможной последовательностью конфигураций, начинающейся с конфигурации $(p,1010011)$, будет следующая последовательность:
\begin{multline*}
(p,1010011)
    \vdash\\(p,010011)
    \vdash(q,10011)
    \vdash(p,0011)
    \vdash(q,011)
    \vdash(r,11)
    \vdash(r,1)
    \vdash\\(r,\eps).
\end{multline*}
Таким образом, $1010011\in L(M)$.
\end{myexample}

Далее в случае детерминированных конечных автоматов мы будем писать, как правило, $\delta(q,a)=p$ вместо $\delta(q,a)=\{p\}$. Если $\delta(q,a)=\es$, то мы часто будем говорить, что $\delta(q,a)$ не определено.

\begin{myproblem}
Построите конечный автомат $M=(Q,\Sigma,\delta,q_0,F)$, допускающий те и только те слова над алфавитом $\Sigma=\{0;1\}$, у которых встречается три стоящих рядом нуля, но после этого подслова нет двух стоящих рядом единиц.
\end{myproblem}

\begin{myproblem}
Покажите, что конечный автомат из примера~\ref{example-chap3-DKA1} допускает язык $(0+1)^*00(0+1)^*$.
\end{myproblem}

\begin{myexample}\label{aut-312}
Построим недетерминированный конечный автомат
\[M=(Q.\Sigma,\delta, q_0, F),
\]
допускающий те и только те слова над алфавитом $\Sigma=\{1;2;3\}$, у которых последняя буква уже появлялась раньше. Прежде всего введем так называемое нейтральное состояние $q_0$, в котором автомат не пытается ничего распознать. Введем также состояния $q_1$, $q_2$ и $q_3$, в каждом из которых $M$ делает предположение о том, что последняя буква слова совпадает с индексом состояния. Пусть $q_f$ --- одно заключительное состояние. Находясь в состоянии $q_0$, один из экземпляров автомата <<бездумно>> остаётся в нем и дальше, а другой экземпляр переходит в состояние $q_i$, если $i$ --- очередной символ. Находясь в состоянии $q_i$, один из экземпляров автомата, независимо от считываемой буквы, остаётся в нем и <<размышляет>> о том, что букву $i$ в слове он уже обнаружил, а именно она и может совпасть с последней буквой в слове. Другой же экземпляр, если видит букву $i$, сразу без всяких сомнений переходит в финальное состояние $q_f$. Из состояния $q_f$ автомат никуда не переходит, так как вопрос о том, допускается ли входное слово, решается только один раз, когда автомат сочтет входную букву последней. Итак, формально автомат определяется как пятерка
\[
	M=(\{q_0;q_1;q_2;q_3;q_f\},\{1;2;3\},\delta,q,\{q_f\}),
\]
где функция переходов $\delta$ задается таблицей~\ref{tab2}.

\begin{figure}
\centering
\begin{tabular}{lllll}
\toprule
%
\multicolumn{2}{c}{\multirow{2}{*}{\Large $\delta$}}
	& \multicolumn{3}{c}{\text{Вход}} \\
%
\cmidrule(rl){3-5}
%
\multicolumn{2}{c}{}
	& \multicolumn{1}{c}{1}
    & \multicolumn{1}{c}{2}
    & \multicolumn{1}{c}{3} \\
%
\midrule
%
\multirow{5}{*}{Состояние}
    & $q_0$ & \{$q_0;q_1$\} & \{$q_0;q_2$\} & \{$q_0;q_3$\} \\
    %
    & $q_1$ & \{$q_1;q_f$\} & \{$q_1$\}     & \{$q_1$\}     \\
    %
    & $q_2$ & \{$q_2$\}     & \{$q_2;q_f$\} & \{$q_2$\}     \\
    %
    & $q_3$ & \{$q_3$\}     & \{$q_3$\}     & \{$q_3;q_f$\} \\
    %
    & $q_f$ & $\es$         & $\es$         & $\es$         \\
\bottomrule
\end{tabular}
\caption{Функция перехода $\delta$ для автомата из примера~\ref{aut-312}.}
\label{tab2}
\end{figure}


Работа этого автомата при чтении слова $\omega=12321$ проиллюстрирована схемой на рисунке~\ref{aut-312-sch}.

\begin{figure}
\begin{tikzpicture}
    \matrix (m1) [row sep=2.5mm, column sep=5mm]
    {
        \node (q012321) {$(q_0, 12321)$}; &
        \node (q02321)  {$(q_0, 2321)$};  &
        \node (q0321)   {$(q_0, 321)$};   &
        \node (q021)    {$(q_0, 21)$};    &
        \node (q01)     {$(q_0, 1)$};     &
        \node (q0eps)   {$(q_0, \eps)$};
        \\
        & & & & &
        \node (q1eps)   {$(q_1, \eps)$};
        \\
        & & & &
        \node (q21)     {$(q_2, 1)$};     &
        \node (q2eps)   {$(q_2, \eps)$};
        \\
        & & &
        \node (q321)    {$(q_3, 21)$};    &
        \node (q31)     {$(q_3, 1)$};     &
        \node (q3eps)   {$(q_3, \eps)$};
        \\
        & &
        \node (q2321)   {$(q_2, 321)$};   &
        \node (q221)    {$(q_2, 21)$};    &
        \node (q21X)    {$(q_2, 1)$};     &
        \node (q2epsX)  {$(q_2, \eps)$};
        \\
        & & & &
        \node (qf1)     {$(q_f, 1)$};     &
        \\
        &
        \node (q12321)  {$(q_1, 2321)$};  &
        \node (q1321)   {$(q_1, 321)$};   &
        \node (q121)    {$(q_1, 21)$};    &
        \node (q11)     {$(q_1, 1)$};     &
        \node (q1epsX)  {$(q_1, \eps)$};
        \\
        & & & & &
        \node (qfeps)     {$(q_f, \eps)$};
        \\
    };

    \draw (q012321)      -- (q02321) -- (q02321)
                          -- (q0321) -- (q021) -- (q01)  -- (q0eps);
    \draw (q01.south)    |- (q1eps);
    \draw (q021.south)   |- (q21)    -- (q2eps);
    \draw (q0321.south)  |- (q321)   -- (q31)  -- (q3eps);
    \draw (q02321.south) |- (q2321)  -- (q221) -- (q21X) -- (q2epsX);
    \draw (q221.south)   |- (qf1);
    \draw (q012321)      |- (q12321)
                          -- (q1321)  -- (q121) -- (q11)  -- (q1epsX);
    \draw (q11.south)    |- (qfeps);
\end{tikzpicture}
\caption{Схема работы автомата из примера~\ref{aut-312} на слове
$\omega=12321$.}
\label{aut-312-sch}
\end{figure}


Таким образом, возможна последовательность конфигураций:
\[
(q_0,12321)\vdash(q_1,2321)\vdash(q_1,321)\vdash(q_1,21)\vdash(q_1,1)\vdash(q_f,\eps).
\]
Это означает, что $12321\in L(M)$.
\end{myexample}

\begin{myproblem}
Докажите, что конечный автомат из примера~\ref{aut-312} допускает язык $\{\omega a\mid a\in\{1,2,3\}$ и $a$ входит в $\omega\}$. Постройте регулярное выражение для этого языка.
\end{myproblem}

\section[Редукция НКА к ДКА]{Редукция недетерминированных конечных автоматов к детерминированным}
\label{Chapter3Reduct}

В этом разделе будет доказано, что класс языков, определяемых недетерминизированными конечными автоматами, совпадает с классом языков, определяемых полностью определенными детерминизированными конечными автоматами. В силу этого произвольный язык из этого класса мы будем называть далее конечно"/автоматным.

\begin{mytheorem}
\label{theorem-reduction-NKAtoDKA}
Пусть $M=(Q,\Sigma,\delta,q_0,F)$ --- недетерминированный конечный автомат. Тогда существует такой детерминированый конечный автомат $M'$ что $L(M) = L(M')$.
\end{mytheorem}

\begin{myproof}
По данному недетерминированному автомату $M$ построим новый автомат $M=(Q',\Sigma,\delta ',q_0',F')$ следующим образом:
\begin{enumerate}
\item $Q'=P(Q)$, т.е. состояниями автомата $M'$ являются всевозможные подмножеста множества состояний автомата $M$;
\item Если $S\in Q'$, $a\in\Sigma$, то определим $\delta '(S,a)$ равенством
\[
	\delta '(S,a) = \bigcup_{q\in S} \delta(q,a);
\]

\item $q'_0=\{q_0\}$;

\item $F'$ состоит из всех таких подмножеств $S$ множества состояний $Q$, что \[
	S\cap F\neq\es.
\]
\end{enumerate}
Применяя метод математической индукции по $i$, нетрудно проверить, что $(S,\omega)\vdash_{M'}^i(S',\eps)$ в том и только в том случае, когда
\[
S'=\{p\mid\exists q\in S\colon (q,\omega)\vdash_M^i(p,\eps)\}
\]
Отсюда, в частности, следует, что $(\{q_0\},\omega)\vdash_{M'}^i(S',\eps)$ для некоторого $S'\in F'$ тогда и только тогда, когда $(q_0,\omega)\vdash_M^i(p,\eps)$ для некоторого $p\in F$. Итак, $L(M')=L(M)$.
\end{myproof}

\begin{myexample}
\label{example-NKAtoDKA-321}\label{aut-313}
Пусть
\[
M=(\{q_0;q_1;q_2;q_3;q_f\},\{1;2;3\},\delta,q,\{q_f\})
\]
недетерминированный конечный автомат, определенный в примере~\ref{aut-312}. Воспользуемся конструкцией из доказательства теоремы~\ref{theorem-reduction-NKAtoDKA} и построим полностью определенный детерминированный конечный автомат $M'=(Q',\{1;2;3\},\delta',\{q_0\},F')$, допускающий язык $L(M)$.

Так как автомат $M$ имеет пять состояний, то новый автомат $M'$ может иметь $32~(=2^5)$ состояния, однако не все они достижимы из начального. (Напомним, что состояние $p$ называется достижимым, если существует такое слово $\omega$, что $(q_0,\omega)\vdash^*(p,\eps)$.) Мы будем строить, естественно, только достижимые состояния.

Сделаем несколько полезных наблюдений.
\begin{enumerate}
    \item Очевидно, что состояние $\{q_0\}$ достижимо.

    \item Достижимыми являются состояния $\{q_0;q_a\}$ для
    $a\in \{1;2;3\}$; действительно,
    \[
        \delta'(\{q_0\}a)=\{q_0;q_a\}.
    \]

    \item Достижимыми являются состояния $\{q_0;q_a;q_f\}$ для $a\in\{1;2;3\}$; действительно, $\delta'(\{q_0;q_a\},a)=\{q_0;q_a;q_f\}$. Аналогично проверяется, что состояния $\{q_0;q_1;q_2\}$, $\{q_0;q_1;q_3\}$, $\{q_0;q_2;q_3\}$ --- достижимы.

    \item Из предыдущего факта легко выводится достижимость состояний
    \[
        \{q_0;q_1;q_2;q_3\}, \quad \{q_0;q_1;q_2;q_f\}, \quad
        \{q_0;q_1;q_3;q_f\}, \quad \{q_0;q_2;q_3;q_f\},
    \]
    а отсюда --- достижимость состояния $\{q_0;q_1;q_2;q_3;q_f\}$.
\end{enumerate}

Всего мы выявили 16 достижимых состояний. Анализ показывает, что подмножество $D$ множества состояний автомата $M$, т.е. состояние автомата $M'$, достижимо тогда и только тогда, когда выполняется условия: 1) $D$ содержит $q_0$, 2) если $D$ содержит $q_f$, то оно содержит также и $q_1$, $q_2$ или $q_3$. Легко видеть, что состояния, отличные от выделенных, не достижимы.

Итак, $Q'$ --- множество всех достижимых состояний автомата $M'$ --- состоит из 15 элементов и вместе с функцией переходов $\delta'$ представлено в следующей таблице \ref{tab3}, где для простоты вместо состояний вида $\{q_i;q_j;\ldots ;q_k\}$ помещены списки $(i;j;\ldots ;k)$.

\begin{table}[H]
\centering
\begin{tabular}{llll}
\toprule
\multicolumn{1}{c}{\multirow{2}{*}{\Large $\delta^\prime$}}
	& \multicolumn{3}{c}{Вход} \\
\cmidrule(rl){2-4}
	& \multicolumn{1}{c}{1}
    & \multicolumn{1}{c}{2}
    & \multicolumn{1}{c}{3} \\
\midrule
$\{0\}$         & $\{0;1\}$       & $\{0;2\}$       & $\{0;3\}$       \\
$\{0;1\}$       & $\{0;1;f\}$     & $\{0;1;2\}$     & $\{0;1;3\}$     \\
$\{0;2\}$       & $\{0;1;2\}$     & $\{0;2;f\}$     & $\{0;2;3\}$     \\
$\{0;3\}$       & $\{0;1;3\}$     & $\{0;2;3\}$     & $\{0;3;f\}$     \\
$\{0;1;2\}$     & $\{0;1;2;f\}$   & $\{0;1;2;f\}$   & $\{0;1;2;3\}$   \\
$\{0;1;3\}$     & $\{0;1;3;f\}$   & $\{0;1;2;3\}$   & $\{0;1;3;f\}$   \\
$\{0;1;f\}$     & $\{0;1;f\}$     & $\{0;1;2\}$     & $\{0;1;3\}$     \\
$\{0;2;3\}$     & $\{0;1;2;3\}$   & $\{0;2;3;f\}$   & $\{0;2;3;f\}$   \\
$\{0;2;f\}$     & $\{0;1;2\}$     & $\{0;2;f\}$     & $\{0;2;3\}$     \\
$\{0;3;f\}$     & $\{0;1;3\}$     & $\{0;2;3\}$     & $\{0;3;f\}$     \\
$\{0;1;2;f\}$   & $\{0;1;2;f\}$   & $\{0;1;2;f\}$   & $\{0;1;2;3\}$   \\
$\{0;1;3;f\}$   & $\{0;1;3;f\}$   & $\{0;1;2;3\}$   & $\{0;1;3;f\}$   \\
$\{0;2;3;f\}$   & $\{0;1;2;3\}$   & $\{0;2;3;f\}$   & $\{0;2;3;f\}$   \\
$\{0;1;2;3\}$   & $\{0;1;2;3;f\}$ & $\{0;1;2;3;f\}$ & $\{0;1;2;3;f\}$ \\
$\{0;1;2;3;f\}$ & $\{0;1;2;3;f\}$ & $\{0;1;2;3;f\}$ & $\{0;1;2;3;f\}$ \\ \bottomrule
\end{tabular}
\caption{Функция перехода $\delta'$ для автомата из примера~\ref{aut-313}}\label{tab3}
\end{table}


Начальным состоянием автомата $M'$ является $\{q_0\}\in Q'$, а множество заключительных состояний $F'$ имеет вид:
\[\begin{array}{rllll}
F' = \{
	& \{q_0,q_1,q_f\};     & \{q_0,q_2,q_f\};     & \{q_0,q_3,q_f\};&\\
	& \{q_0,q_1,q_2,q_f\}; & \{q_0,q_1,q_3,q_f\}; & \{q_0,q_2,q_3,q_f\};&\\
	& & & \{q_0,q_1,q_2,q_3,q_f \} & \}.
\end{array}\]
\end{myexample}

\section{Граф переходов}
\label{Chapter3Graph}
Пусть $M=(Q,\Sigma,\delta,q_0,F)$ --- недетерминированный конечный автомат. Графом переходов (или диаграммой) автомата $M$ называют неупорядоченый помеченный граф, вершины которого помечены именами состояний; в графе есть дуга $(p,q)$, если существует такая буква $a\in\Sigma$, что $q\in\delta(p,a)$; кроме того, дуга $(p,q)$ помечается списком, состоящим из всех таких $a$, что $q\in\delta(p,a)$.

При изображении графа переходов условимся начальное состояние указывать направленной в него стрелкой, помеченной словом начало, а заключительные состояния обводить двойным кружком.

\begin{myexample}
Рассмотрим детерминированный конечный автомат из примера~\ref{example-chap3-DKA1}. Его граф переходов изображен на рисунке~\ref{aut-graph-1}.
\begin{figure}[H]
\centering
\begin{tikzpicture}[node distance=3cm,>=stealth',auto,every state/.style={thick}]
	\node (init) {};
	\node[state] (p) [right=.7cm of init] {$P$};
	\node[state] (q) [right of=p] {$q$};
	\node[state,accepting] (r) [right of=q] {$r$};

	\path[->]
    	(init) edge (p)
		(p) edge [loop above] node {$1$} (p)
		(p) edge node {$0$} (q)
		(q) edge [bend right] node[above] {$1$} (p)
		(q) edge node {$0$} (r)
        (r) edge [loop above] node {$0, 1$} (r);
\end{tikzpicture}
\caption{Граф переходов автомата из примера~\ref{example-chap3-DKA1}}
\label{aut-graph-1}
\end{figure}
\end{myexample}

\begin{myexample}
Рассмотрим недетерминированный конечный автомат из примера~\ref{aut-312}. Его граф переходов изображен на рисунке~\ref{aut-graph-2}.
\end{myexample}

\begin{figure}
\centering
\begin{tikzpicture}[auto,>=stealth', node distance=4cm,auto,every state/.style={thick}]
	\node (init) {};
	\node[state] (q0) [right=.7cm of init] {$q_0$};
	\node[state] (q1) [above right of=q0] {$q_1$};
    \node[state] (q2) [right of=q0] {$q_2$};
    \node[state] (q3) [below right of=q0] {$q_3$};
	\node[state,accepting] (qf) [right of=q2, above right of=q3] {$q_f$};

	\path[->]
    	(init) edge (q0)
		(q0) edge [loop above] node {$1, 2, 3$} (q0)
		(q0) edge node {$1$} (q1)
        (q0) edge node {$2$} (q2)
        (q0) edge node {$3$} (q3)
        (q1) edge [loop above] node[right] {$1, 2, 3$} (q1)
        (q1) edge node {$1$} (qf)
        (q2) edge [loop above] node[right] {$1, 2, 3$} (q2)
        (q2) edge node {$2$} (qf)
        (q3) edge [loop above] node[right] {$1, 2, 3$} (q3)
        (q3) edge node {$3$} (qf);
\end{tikzpicture}
\caption{Граф переходов автомата из примера 3.1.2.}
\label{aut-graph-2}
\end{figure}


\begin{myproblem}
Изобразите граф переходов детерминированного конечного автомата из примера~\ref{example-NKAtoDKA-321} (с.~\pageref{example-NKAtoDKA-321}).
\end{myproblem}

\section[Совпадение классов КА-, регулярных и ПЛ-языков]{Совпадение классов конечно"/автоматных, регулярных и ПЛ-языков}
\label{Chapter3MathesFARL}

\begin{mylemma} 
\label{lemma-dka-to-pl}
Пусть $M=(Q,\Sigma,\delta,q_{0},F)$ --- детерминированный конечный автомат. Тогда существует такая ПЛ-грамматика $G$, что $L(M)=L(G)$.
\end{mylemma}

\begin{myproof}
Рассмотрим грамматику $G=(Q,\Sigma,P,q_{0})$, где множество продукций $P$ определяется по правилам:
\begin{enumerate}
\item если $q,r\in Q$, $a\in\Sigma$ и $\delta(q,a)=r$, то $(q\to ar)\in P$.
\item если $p\in F$, то $(p\to\eps)\in P$.
\end{enumerate}
Грамматика $G$ определена таким образом, что каждый шаг вывода в ней имитирует такт автомата $M$.

Применяя метод математической индукции по параметру $i$, докажем следующее вспомогательное

\begin{mystatement}
$q\To^{i+1}\omega$ для $q\in Q$ тогда и только тогда, когда $(q,\omega)\vdash^i(r,\eps)$ для $r\in F$.
\end{mystatement}

Если $i=0$, то это утверждение приобретает вид:	$q\To\eps$ тогда и только тогда, когда $(q,\eps)\vdash^0(q,\eps)$ для $q\in F$, и является очевидным.

Предположим теперь, что утверждение верно для $i=k$, и проверим его для $i=k+1$. Пусть $\omega=ax$, где $|x|=k$, --- слово над алфавитом $\Sigma$. Тогда вывод $q\To^{k+1}\omega$ равносилен выводу $q\To as\To^kax$ для некоторого $s\in Q$, а вывод $q\To as$ равносилен тому, что $\delta(q,a)=s$. По предположению индукции $s\To^kx$ тогда и только тогда, когда $(s,x)\vdash^{k-1}(r,\eps)$ для некоторого $r\in F$. Следовательно, вывод $q\To^{k+1}\omega$ равносилен тому, что $(q,\omega)\vdash^k(r,\eps)$ для некоторого $r\in F$.

Итак, вспомогательное утверждение верно.

Подставляя в доказанное утверждение $q_0$ вместо $q$, получаем: $q_0\To^+\omega$ тогда и только тогда, когда $(q_0,\omega)\vdash^*(r,\eps)$ для некоторого $r\in F$. Таким образом, $L(G)=L(M)$.
\end{myproof}

Отметим, что в лемме~\ref{lemma-dka-to-pl} не просто доказывается существование нужной грамматики $G$, а указывается ее конструкция.

\begin{mylemma}
\label{lemma-FA-of-ElemLangs}
Пусть $\Sigma$ --- конечный алфавит. Множества $\es$, $\{\eps\}$ и $\{a\}$ для всех $a\in\Sigma$ являются конечно"/автоматными языками.
\end{mylemma}

\begin{myproof}
Доказательство. Любой конечный автомат с пустым множеством заключительных состояний допускает $\es$.

Пусть $M_\eps=(\{q_0\},\Sigma,\delta,q_0,\{q_0\})$, где $\delta(q_0,a)$ не определяется ни для каких $a\in\Sigma$. Тогда $L(M_\eps)=\{\eps\}$.

Пусть $M_a=(\{q_0,q_a\},\Sigma,\delta,q_0,\{q_a\})$, где $\delta(q_0,a)=q_a$, а в остальных случаях функция $\delta$ не определена. Тогда, очевидно, $L(M_a)=\{a\}$
\end{myproof}

\begin{mylemma}
\label{lemma-FA-of-OpLangs}
Пусть $M_1=(Q_1,\Sigma,\delta_1,q_1,F_1)$. $M_2(Q_2,\Sigma,\delta_2,q_2,F_2)$ --- конечные автоматы; $L_i=L(M_i)$. Тогда множества $L_1\cup L_2$, $L_1L_2$ и $L_1^*$ являются конечно"/автоматными языками.
\end{mylemma}

\begin{myproof}
Не теряя общности, можно считать, что $Q_1\cap Q_2=\es$, так как в противном случае состояния можно было бы переименовать. Чтобы доказать лемму, достаточно для каждого из трех множеств построить такой конечный автомат $M$, чтобы язык $L(M)$ совпадал с этим множеством.
\begin{enumerate}
\item Построим автомат $M=(Q_1\cup Q_2\cap\{q_0\},\Sigma,\delta,q_0,F)$ для языка $L_1\cup L_2$, полагая, что $q_0$ --- новое состояние, $F=F_1\cup F_2$ при $\eps\notin L_1\cup L_2$, и $F=F_1\cup F_2\cup\{q_0\}$ при $\eps\in L_1\cup L_2$ а функция переходов $\delta$ определяется так:
\begin{enumerate}[label=(\emph{\roman*})]
\item $\delta(q_0,a)=\delta_1(q_1,a)\cup\delta(q_2,a)$ для всех $a\in\Sigma$,
\item $\delta(q,a)=\delta_1(q,a)$ для всех $q\in Q_1, a\in\Sigma$,
\item $\delta(q,a)=\delta_2(q,a)$ для всех $q\in Q_2, a\in\Sigma$.
\end{enumerate}
Таким образом, сначала автомат $M$ как бы решает, какой из автоматов $M1$, $M2$ ему моделировать, но так как $M$ --- недетерминированный автомат, то фактически он моделирует и тот и другой.

Используя непосредственный анализ функции переходов $\delta$ легко проверить, что $(q_0,\omega)\vdash_M^i(q,\eps)$ тогда и только тогда, когда $q\in Q_1$ и $(q_1,\omega)\vdash_{M_1}^i(q,\eps)$ или $q\in Q_2$ и $(q_2,\omega)\vdash_{M_2}^i(q,e)$.

Из этого утверждения и определения множества $F$ вытекает, что $L_1\cup L_2=L(M_1)\cup L(M_2)=L(M)$ --- конечно"/автоматный язык.

\item Построим автомат $M=(Q_1\cup Q_2,\Sigma,\delta,q_1,F)$ для $L_1L_2$, полагая

\begin{equation*}
F =
\begin{cases}
	F_2, & \text{если $q_2\notin F_2$,} \\
	F_1\cup F_2, & \text{если $q_2\in F_2$,}
\end{cases}
\end{equation*}
а функцию переходов $\delta$ определяя равенствами
\begin{enumerate}[label=(\emph{\roman*})]
\item $\delta(q,a)=\delta_1(q,a)$ для всех $q\in Q_1\backslash F_1$, $a\in\Sigma$,

\item $\delta(q,a)=\delta_1(q,a)\cup\delta_2(q,a)$ для всех $q\in F$, $a\in\Sigma$,

\item $\delta(q,a)=\delta_2(q,a)$ для всех $q\in Q_2$, $a\in\Sigma$.
\end{enumerate}

Ясно, что когда $M$ начинает работать, то он моделирует автомат $M_1$. Когда же $M$ достигает заключительного состояния $M_1$, он может (если пожелает!) предположить, что оказался в начальном состоянии автомата $M_2$ и начать моделировать $M_2$, а может (допускается и это!) продолжать функционировать в режиме $M_1$.

Проверим, что $L_1L_2\subset L(M)$.

Пусть $x\in L$, $y\in L$. Тогда $(q_1,xy)\vdash_M^*(q,y)$ для некоторого $q\in F_1$, причем $q=q_1$ в случае $x=\eps$. Пусть $y=\eps$. Тогда $q_2\in F_2$ и, следовательно, $F=F_1\cup F_2$. Последняя конфигурация $(q,\eps)$, где $q\in F_1\subset F$, означает, что слово $xy=x\eps$ автомат $M$ считал.

Пусть теперь $y\neq\eps$. Применяя один раз $(ii)$ и нуль или более раз $(iii)$, получаем: $(q,y)\vdash_M^+(r,e)$ для некоторого $r$ из $F_2\subset F$. Отсюда вытекает, что $xy\in L(M)$, и, следовательно, $L_1L_2\subset L(M)$.

Проверим, что $L(M)\subset L_1L_2$.

Пусть $\omega\in L(M)$. Тогда $(q_1,\omega)\vdash_M^*(q,\eps)$ для некоторого $q\in F$. Рассмотрим отдельно два случая: $q\in F_2$ и $q\in F_1(\subset F)$. Если $q\in F_2$, то $\omega=xay$ для некоторого $a\in\Sigma$, удовлетворяющего условиям
\[
(q_1,xay)\vdash_M^*(r,ay)\vdash_M(s,y)\vdash_M^*(q,\eps),
\]
где $r\in F_1$, $s\in Q_2$ и $s\in\delta(r,a)$. Следовательно, $x\in L_1$, $ay\in L_2$ и, таким образом, $\omega=xay\in L_1L_2$. Если же $q\in F_1(\subset F)$, то $q_2\in F_2$, $\eps\in L_2$ и, таким образом, $\omega\in L_1$, $\omega\eps\in L_1L_2$. Отсюда вытекает: $L(M)\subset L_1L_2$.

В итоге получаем, что $L_1L_2=L(M)$ --- конечно"/автоматный язык.

\item Построим автомат $M=(Q_1\cup\{q'\},\Sigma,\delta,q',F\cup\{q'\})$ для языка $L_1^*$, полагая, что $q'$ --- новое состояние, не принадлежащее $Q_1$, а функцию переходов $\delta$ определяя равенствами
\begin{enumerate}[label=(\emph{\roman*})]
\item $\delta(q',a)=\delta_1(q_1,a)$ для всех $a\in\Sigma$,

\item $\delta(q,a)=\delta_1(q,a)$, для всех $q\in Q\backslash F_1$, $a\in\Sigma$,

\item $\delta(q,a)=\delta_1(q,a)\cup\delta_1(q_1,a)$, для всех $q\in F_1$, $a\in\Sigma$.
\end{enumerate}

Ясно, что когда $M$ начинает работать, то он моделирует автомат $M_1$. Когда же $M$ достигает заключительного состояния $M_1$, он может предположить, что вновь оказался в начальном состоянии автомата $M_1$ и снова начать моделировать $M_1$, а может и продолжать функционировать в режиме $M_1$. Доказательство того, что $L(M)=L_1^*$, аналогично доказательству из 2) и мы его не приводим. Отметим лишь, что $\eps\in L(M)$, так как $q$ --- заключительное состояние автомата $M$.
\end{enumerate}
\end{myproof}

\begin{mytheorem}
\label{theorem-Kleene}
Пусть $\Sigma$ --- конечный алфавит, $L$ --- язык над этим алфавитом. Следующие утверждения эквивалентны:
\begin{enumerate}
\item $L$ --- регулярный язык;

\item $L$ --- праволинейный язык;

\item $L$ --- конечно"/автоматный язык.
\end{enumerate}
\end{mytheorem}
\begin{myproof}
Эквивалентность утверждений 1) и 2) доказана в теореме~\ref{threorem-RegExp-PL241}. То, что утверждение 2) является следствием утверждения 3) доказано в лемме~\ref{lemma-dka-to-pl}. Из лемм~\ref{lemma-FA-of-ElemLangs} и~\ref{lemma-FA-of-OpLangs} вытекает, что утверждение 3) --- следствие утверждения 1). Таким образом, теорема доказана.
\end{myproof}

\section{Лемма о разрастании для регулярных языков}
\label{Chapter3Pumplemma}
Лемма о разрастании (или накачке) позволяет решать конкретную задачу: опровергать принадлежность заданного языка классу регулярных языков. Зачем может понадобиться такое опровержение? Если считать класс регулярных языков в определенном смысле (а именно, с точки зрения иерархии Хомского) самым простым классом языков, то «непринадлежность» языка этому классу может говорить о «сложности» этого языка (в том же специальном смысле). Стоит обратить внимание на указание относительности понятий «простой» и «сложный»: язык, который с обыденной точки зрения может
казаться «простым», вполне может не принадлежать классу регулярных языков.

\begin{mytheorem}
\label{theorem-PumpingLemma}
\textup{\textbf{(«Лемма о накачке» или «Лемма о разрастании»)}}
Пусть $L$ — регулярный язык.
Тогда существует такая константа $n\in \mathbb N$, что для любого слова $w \in L$,
такого что $|w|\geqslant n$, существует разбиение $w=xyz$ слова $w$ со следующими свойствами:
\begin{enumerate}
  \item\label{ynotempty} $y \neq \varepsilon$;
  \item\label{shortprefix} $|xy| \leqslant n$;
  \item\label{pumping} $\{ xy^kz \mid k \geqslant 0\} \subset L$.
\end{enumerate}
\end{mytheorem}
\begin{myproof}
Так как $L$ --- регулярный язык, то существует детерминированный конечный автомат
$\mathcal A=(Q, \Sigma, \delta, q_0, F)$ с $|Q|=N$ состояниями, распознающий
язык $L$. Пусть $w \in L$ и $|w|=N+1$. Подадим на вход автомату $\mathcal A$
слово $w$. Очевидно, существует состояние $q \in Q$, в котором автомат окажется
дважды, читая это слово (принцип Дирихле / принцип голубятни). Разобьём слово
$w$ на три части $w=xyz$, так что:
\[
    (q_0, xyz) \vdash^*  (q, yz) \stackrel{\bigtriangleup}{\vdash^*} (q, z)
    \vdash^* (q_F, \varepsilon),
\]
где $q_F \in F$. Покажем, что для любого целого $k \geqslant 0$ автомат
распознает слово $xy^kz$. Действительно, последовательность переходов при чтении
цепочек $x$ и $z$ остаётся такой же, как для слова $w$. Часть $y^k$ читается
$k$"/кратным повтором последовательности переходов $\bigtriangleup$. Таким
образом, $\{ xy^kz \mid k \geqslant 0\} \subset L$ и выполнено
условие~\ref{pumping}.

Части $x$ и $y$ слова $w$ удовлетворяют
условиям~\ref{ynotempty}–\ref{shortprefix} по построению. Полагая
$n=N+1$, получаем выполненными все требования теоремы.
\end{myproof}

Заметим, что лемма о накачке формулирует необходимое свойство регулярного языка. То есть, если язык не обладает этим свойством, то он точно не является регулярным. На этом основано применение леммы на практике: вначале доказывается, что для рассматриваемого языка нарушается условие леммы, на основе этого делается вывод о нерегулярности. Такое доказательство обычно проводится от противного.

В случае, когда нарушение свойства, описываемого леммой о накачке, доказать не удается, ничего определенного о классе языка без дополнительных рассуждений сказать нельзя.

\begin{myexample}
Рассмотрим язык $L= \{ 0^m1^m \mid m \in \N\}$. Предположим, что $L$ регулярный,
а значит, существует константа $n$, о которой идет речь в лемме.
Рассмотрим слово $0^n1^n$ ($\in L$). Пусть $xyz = 0^n1^n$~--- разбиение,
о котором говорится в лемме. Заметим, что в силу условия $|xy| \le n$
из леммы подстрока $y$ целиком состоит из нулей и при этом не пуста.
Очевидно, что для любого $k > 1$ слово $xy^kz$ уже не будет принадлежать
языку $L$, так как в этом слове нулей станет больше, чем единиц. Это свидетельствует о нарушении условия леммы о накачке, а значит, о том, что язык $L$
не является регулярным.
\end{myexample}

\section{Упражнения}
\label{Chapter3Exs}

\subsection*{Построение недетерминированных автоматов}

Построить автомат, распознающий
\begin{enumerate}
  \item язык над $\{0,1\}$ из слов, заканчивающихся на $01$;
  \item язык, представляющий собой десятичную запись чисел, делящихся на $4$;
  \item язык над $\{a, b\}$, заданный регулярным выражением $(ab + aba)^\ast$;
  \item язык над $\{0, 1, \ldots , 9\}$ из слов, в которых последняя цифра
  встречается ещё где-то в них;
  \item язык над $\{0, 1, \ldots , 9\}$ из слов, в которых последняя цифра
  больше нигде в них не встречается;
  \item язык над $\{0,1\}$ из слов, в которых содержится два $0$, разделённых
  символами, количество которых кратно $4$ (нуль символов также считать
  кратными четырём).
\end{enumerate}

\subsection*{Детерминизация конечных автоматов}

Провести детерминизацию следующих НКА:
\begin{enumerate}
  \item распознающего язык над $\{0,1\}$ из слов, заканчивающихся на $01$;

  \item распознающего язык, заданный регулярным выражением $(ab + aba)^\ast$;
\end{enumerate}
\begin{multicols}{3}
\begin{enumerate}
\setcounter{enumi}{2}
  \item
     \begin{tabular}{rll}
     \toprule
     \multirow{2}{*}{\Large $\delta$}
      & \multicolumn{2}{c}{\text{Вход}} \\
    \cmidrule(rl){2-3}
        & \multicolumn{1}{c}{0}
        &\multicolumn{1}{c}{1}\\
     \midrule
     ${}\to p$ & $\{p, q\}$ & $\{p\}$\\
     $q$ & $\{r\}$ & $\{r\}$\\
     $r$ & $\{s\}$ & $\emptyset$\\
     \boxed{s} & $\{s\}$ & $\{s\}$\\
     \bottomrule
    \end{tabular}

  \item
     \begin{tabular}{rll}
     \toprule
     \multirow{2}{*}{\Large $\delta$}
      & \multicolumn{2}{c}{\text{Вход}} \\
     \cmidrule(rl){2-3}
        & \multicolumn{1}{c}{0}
        &\multicolumn{1}{c}{1}\\
     \midrule
     ${}\to p$ & $\{q, s\}$ & $\{q\}$\\
     \boxed{q} & $\{r\}$ & $\{q, r\}$\\
     $r$ & $\{s\}$ & $\{p\}$\\
     \boxed{s} & $\emptyset$ & $\{p\}$\\
     \bottomrule
    \end{tabular}

  \item
     \begin{tabular}{rll}
     \toprule
     \multirow{2}{*}{\Large $\delta$}
      & \multicolumn{2}{c}{\text{Вход}} \\
     \cmidrule(rl){2-3}
        & \multicolumn{1}{c}{0}
        &\multicolumn{1}{c}{1}\\
     \midrule
     ${}\to p$ & $\{p, q\}$ & $\{p\}$\\
     $q$ & $\{r, s\}$ & $\{t\}$\\
     $r$ & $\{p, r\}$ & $\{t\}$\\
     \boxed{s} & $\emptyset$ & $\emptyset$\\
     \boxed{t} & $\emptyset$ & $\emptyset$\\
     \bottomrule
    \end{tabular}
\end{enumerate}
\end{multicols}

\subsection*{Доказательство нерегулярности языков}

С помощью леммы о накачке доказать, что следующие языки нерегулярны:
\begin{multicols}{2}
\begin{enumerate}
  \item $\{ w \in \{0, 1\}^* \mid
  |w|_0 = |w|_1\}$;
  \item $\{w w \mid w \in \{0, 1\}^\ast \}$;
  \item $\{0^n1^m \mid n \leqslant m \}$;
  \item $\{1^p \mid \text{$p$ — простое число} \}$;
  \item $\{0^n10^n \mid n \in \mathbb N\}$;
  \item $\{1^{n^2} | n \in \mathbb N\}$;
  \item $\{1^{n!} | n \in \mathbb N\}$.
\end{enumerate}
\end{multicols}
\noindent Запись вида $|w|_z$  (см.~язык 1) означает количество вхождений символа $z$ в строку~$w$.


\chapter{Конечные автоматы со спонтанными переходами}
\label{Chapter4}
\section{Определения и примеры}
\label{Chapter4Defines}
\mydef{Конечный автомат со спонтанными переходами (или $\eps$-переходами)} --- это одно из обобщений конечных автоматов.У конечного автомата появляется новое свойство --- возможность совершать переходы по $\eps$ (пустой цепочке), т. е. спонтанно, не получая на вход никакого символа. Эта новая возможность, как и недетерминизм, не расширяет класс языков, допустимых конечными автоматами, но дает некоторое дополнительноеогда  <<удобство программирования>>.

пусть $M = (Q,\Sigma, \delta, q_0, F)$ --- недетерминированный конечный автомат. Автомат $M$
назовем \mydef{$\eps$-НКА}, если
\[
    \delta \colon Q \times (\Sigma \cup \{ \eps \} ) \to P(Q)
\]


\begin{myexample}
Рассмотрим $\eps$-НКА, распознающий ключевые слова \emph{web} и \emph{day} в последовательности символов $\{ a..z \}$. Граф переходов этого автомата представлен на рисунке~\ref{ch4-aut-graph-3}.
\begin{figure}[H]
\centering
\begin{tikzpicture}[node distance=3cm,>=stealth',auto,every state/.style={thick}]
	\node (init) {};
	\node[state] (8) [right=.7cm of init] {$8$};
    \node[state] (0) [below right of=8] {$0$};
	\node[state] (1) [above right of=8] {$1$};
    \node[state] (5) [right of=0] {$5$};
    \node[state] (2) [right of=1] {$2$};
    \node[state] (6) [right of=5] {$6$};
    \node[state] (3) [right of=2] {$3$};
	\node[state,accepting] (4) [right of=3] {$4$};
    \node[state,accepting] (7) [right of=6] {$7$};

	\path[->]
    	(init) edge (8)
		(8) edge [loop above] node {$a..z$} (8)
		(8) edge node {$\eps$} (1)
        (1) edge node {$w$} (2)
        (2) edge node {$e$} (3)
        (3) edge node {$b$} (4)
        (8) edge node {$\eps$} (0)
        (0) edge node {$d$} (5)
        (5) edge node {$a$} (6)
        (6) edge node {$y$} (7);
\end{tikzpicture}
\caption{Использование спонтанных переходов для распознавания ключевых слов}
\label{ch4-aut-graph-3}
\end{figure}

Для каждого ключевого слова строится полная последоватльность состояний, как если бы это было единственное слово, которое автомат должен распознать. Затем добавляется новое начальное состояние (состояние 8 на рисунке~\ref{ch4-aut-graph-3}) с $\eps$-переходами в начальные состояния автоматов для каждого из ключевых слов.
\end{myexample}

\begin{myexample}
Рассмотрим $\eps$-НКА (рисунок~\ref{aut-graph-4}), допускающий запись десятичных чисел из следующих элементов $\colon$
\begin{enumerate}
   \item Необязательный знак + или $-$.
   \item Цепочка цифр.
   \item Разделяющая десятичная точка.
   \item Еще одна цепочка цифр. Эта цепочка, как и цепочка (2), может быть пустой, но хотя бы одна из них должна быть непустой.
\end{enumerate}
\begin{figure}[h]
\centering
\begin{tikzpicture}[node distance=3cm,>=stealth',auto,every state/.style={thick}]
	\node (init) {};
	\node[state] (q_0) [right=.7cm of init] {$q_0$};
    \node[state] (q_1) [right of=q_0] {$q_1$};
	\node[state] (q_2) [right of=q_1] {$q_2$};
    \node[state] (q_3) [right of=q_2] {$q_3$};
    \node[state] (q_4) [below of=q_2] {$q_4$};
    \node[state,accepting] (q_5) [right of=q_3] {$q_5$};

	\path[->]
    	(init) edge (q_0)
        (q_0) edge node {$\eps, +, -$} (q_1)
		(q_1) edge [loop above] node {$0,1,..,9$} (q_1)
        (q_1) edge node {$.$} (q_2)
        (q_1) edge node {$0,1,..,9$} (q_4)
        (q_2) edge node {$0,1,..,9$} (q_3)
        (q_4) edge node {$.$} (q_3)
        (q_3) edge [loop above] node {$0,1,..,9$} (q_3)
        (q_3) edge node {$\eps$} (q_5);
\end{tikzpicture}
\caption{$\eps$-НКА, допускающий десятичные числа}
\label{aut-graph-4}
\end{figure}


Поскольку переход из состояния $q_0$ в $q_1$ может произойти по любому из символов $+, -, \eps$, то состояние $q_1$ моделирует ситуацию, когд а прочитан знак числа (если есть), он не прочитана ни одна из цифр, ни десятичная точка. Состояние $q_2$ соответствует ситуации, когда только что прочитана десятичная точка, а цифры целой части числа либо уже прочитаны, либо нет. В состоянии $q_4$ уже наверняка прочитана хотя бы одна цифра, но еще не прочитана десятичная точка. Состояние $q_3$ определяет ситуацию, когда  прочитаны десятичная точка и хотя бы одна цифра слева или справа от нее. Автомат может оставаться в состоянии $q_3$, продолжая читать цифры, а может и <<предположить>>, что цепочка цифр закончена, и спонтанно перейти в допускающее состояние $q_5$.
$\delta$-функция переходов автомата представлена в таблице~\ref{tab4}.
\begin{table}[H]
\centering
\begin{tabular}{llllll}
\toprule
%
\multicolumn{2}{c}{\multirow{2}{*}{\Large $\delta$}}
	& \multicolumn{4}{c}{\text{Вход}} \\
%
\cmidrule(rl){3-6}
%
\multicolumn{2}{c}{}
	& \multicolumn{1}{c}{$\eps$}
    & \multicolumn{1}{c}{\{+;-\}}
    & \multicolumn{1}{c}{\{.\}}
    & \multicolumn{1}{c}{\{0,1,..,9\}} \\
%
\midrule
%
\multirow{5}{*}{Состояние}
    & $\to q_0$ & \{$q_1$\} 	& \{$q_1$\} 	& $\es$ 		& $\es$ \\
    %
    & $q_1$ & $\es$ 		& $\es$ 		& \{$q_2$\}     & \{$q_1;q_4$\}     \\
    %
    & $q_2$ & $\es$ 		& $\es$     	& $\es$ 		& \{$q_3$\}     \\
    %
    & $q_3$ & \{$q_3$\} 	& $\es$     	& $\es$     	& \{$q_3$\} \\
    %
    & $q_4$ & $\es$			& $\es$         & \{$q_3$\}     & $\es$         \\
    %
    & $\boxed{q_5}$ & $\es$ 		& $\es$         & $\es$         & $\es$         \\
\bottomrule
\end{tabular}
\caption{Функция перехода $\delta$ для автомата, распознающего десятичные числа}
\label{tab4}
\end{table}

\end{myexample}

Пусть $M = (Q,\Sigma, \delta, q_0, F)$ --- $\eps$-НКА. Определим \mydef{$\eps$-замыкание} состояния $q \in \Sigma$ рекурсивно следующим образом:
\begin{enumerate}
   \item $E(q) \subseteq P(q)$.
   \item $q \in E(q)$.
   \item Если $p \in E(q)$ и $r \in \delta(p, \eps)$, то $r \in E(q)$.
\end{enumerate}

\begin{myexample}
\label{example-413}
Построим $\eps$-замыкание для состояния $q$ из $\eps$-НКА, заданного графом переходов (рисунок~\ref{aut-graph-5}).
\begin{figure}[h]
\centering
\begin{tikzpicture}[node distance=3cm,>=stealth',auto,every state/.style={thick}]
	\node (init) {};
	\node[state] (q_1) [right=.7cm of init] {$q_1$};
    \node[state] (q_2) [right of=q_1] {$q_2$};
	\node[state] (q_4) [below of=q_2] {$q_4$};
    \node[state] (q_5) [right of=q_4] {$q_5$};
    \node[state, accepting] (q_7) [right of=q_5] {$q_7$};
    \node[state] (q_3) [right of=q_2] {$q_3$};
    \node[state] (q_6) [right of=q_3] {$q_6$};

	\path[->]
    	(init) edge (q_1)
        (q_1) edge node {$\eps$} (q_2)
		(q_1) edge node {$\eps$} (q_4)
        (q_2) edge node {$\eps$} (q_3)
        (q_3) edge node {$\eps$} (q_6)
        (q_4) edge node {$a$} (q_5)
        (q_5) edge node {$b$} (q_6)
        (q_5) edge node {$\eps$} (q_7);
\end{tikzpicture}
\caption{$\eps$-НКА из примера~\ref{example-413}}
\label{aut-graph-5}
\end{figure}

\[
	  E^0(q_1) = \{q_1\}; \\
    E^1(q_1) = \{ q_1; q_2; q_4 \}; \\
    E^2(q_1) = \{ q_1; q_2; q_4; q_3 \}; \\
    E^3(q_1) = \{ q_1; q_2; q_4; q_3; q_6 \}. \\
\]
Таким образом, $E(q_1) = \{ q_1; q_2; q_4; q_3; q_6 \} $.
\end{myexample}

Пусть $M = (Q,\Sigma, \delta, q_0, F)$ --- $\eps$-НКА и $S \subseteq Q$ --- произвольное подмножество множества $Q$. Назовем множестов $S$ \mydef{$\eps$-замкнутым}, если $S = \{ q \mid \text{если } p \in \delta(q, \eps), \text{то } p \in S \}$. Отметим, что $\es$ --- $\eps$-замкнутое множество.

\section{Редукция $\eps$-НКА к ДКА}
\label{Chapter4Reduct}
Для всякого $\eps$-НКА $E$ можно найти ДКА $D$, допускающий тот же язык, что и $E$.

\begin{mylemma}
\label{lemma-ENKAtoDKA}
Пусть $M_E = (Q_E,\Sigma, \delta_E, q_E, F_E)$ --- $\eps$-НКА. Тогда найдется такой ДКА $M_D$, что $L(M_E) = L(M_D)$.
\end{mylemma}
\begin{myproof}
По данному $\eps$-НКА построим новый ДКА $M_D = (Q_D, \Sigma, \delta_D, q_D, F_D)$ следующим образом:
\begin{enumerate}
   \item $Q_d = P(Q_E)$. Причем, достижимыми состояниями автомата $M_D$ будут только такие $S \in Q_D$, что $S$ является $\eps$-замкнутым множеством.
   \item $q_D = E(q_E)$.
   \item $F_D = \{ S \in Q_D \colon S \cap F_E \neq \es \}$.
   \item Функция переходов $\delta_D$ определим следующим образом: \newline
   если $S \in Q_D$ и $a \in \Sigma$, то $\delta_d(S,a)$ строится по правилам:
   \begin{enumerate}
   	\item пусть $\{ q_1; q_2;...;q_n \}$ --- множество состояний $S$;
    \item $\{ r_1; r_2;...;r_m \} = \bigcup_{i\in 1}^{n}\delta_E(q_i, a)$;
    \item $\delta_D(S,a) = \bigcup_{j\in 1}^{m}E(r_j)$.
   \end{enumerate}
\end{enumerate}
Доказательство корректности конструкций леммы основывается на стягивании вершин автомата по $\eps$-переходам.
\end{myproof}
\begin{myexample}
\label{example-reduceENKAtoDKA}
Пусть $M_E = (\{ p; q; r \},\{ a; b; c \}, \delta_E, p, \{ r \})$ --- $\eps$-НКА. $\delta$-функция переходов автомата задана таблицей~\ref{tab5}. 
\centering
\begin{tabular}{llllll}
\toprule
%
\multicolumn{2}{c}{\multirow{2}{*}{\Large $\delta$}}
	& \multicolumn{4}{c}{\text{Вход}} \\
%
\cmidrule(rl){3-6}
%
\multicolumn{2}{c}{}
	& \multicolumn{1}{c}{$\eps$}
    & \multicolumn{1}{c}{\{a\}}
    & \multicolumn{1}{c}{\{b\}}
    & \multicolumn{1}{c}{\{c\}} \\
%
\midrule
%
\multirow{5}{*}{Состояние}
    & $\to p$ & $\es$  		& \{$p$\} 		& \{$q$\} 	& \{$r$\} 	\\
    %
    & $q$ & \{$p$\} 	& \{$q$\} 		& \{$r$\} 	& $\es$     \\
    %
    & $\boxed{r}$ & \{$q$\} 	& \{$r$\} 		& $\es$		& \{$p$\} 	\\
\bottomrule
\end{tabular}
\caption{Функция переходов}
\label{tab5}

Граф переходов автомата представлен на рисунке~\ref{aut-graph-6}.
\centering
\begin{tikzpicture}[node distance=3cm,>=stealth',auto,every state/.style={thick}]
	\node (init) {};
	\node[state] (p) [right=.7cm of init] {$p$};
    \node[state] (q) [right of=p] {$q$};
	\node[state, accepting] (r) [below of=p] {$r$};

	\path[->]
    	(init) edge (p)
		(p) edge [loop above] node {$a$} (p)
		(p) edge [bend left] node[right] {$b$} (q)
        (q) edge [loop above] node {$a$} (q)
        (q) edge node {$\eps$} (p)
        (q) edge [bend left] node[right] {$b$} (r)
        (r) edge node {$\eps$} (q)
        (r) edge node {$c$} (p)
        (p) edge [bend right] node[left] {$c$} (r);
\end{tikzpicture}
\caption{Граф переходов}
\label{aut-graph-6}

Пользуясь конструкциями из леммы~\ref{lemma-ENKAtoDKA}, по данному автомату построим ДКА $M_D = (Q_D, \Sigma, \delta_D, q_D, F_D)$.
Начальным состоянием ДКА является $\eps$-замыкание начального состояния исходного автомата.
\[
	q_D = E(p).
\]
Построим $\eps$-замыкание:
\[
	E^0(p) = \{ p \};
	E^1(p) = \{ p \}.
\]
Таким образом $q_D = \{ p \}$.

Аналогично начальному состоянию, построим $\eps$-замыкания для остальных состояний исходного автомата. $\delta$-функция переходов автомата $M_D$ представлена в таблице~\ref{tab6}.
\begin{table}[H]
\centering
\begin{tabular}{llllll}
\toprule
%
\multicolumn{2}{c}{\multirow{2}{*}{\Large $\delta$}}
	& \multicolumn{3}{c}{\text{Вход}} \\
%
\cmidrule(rl){3-5}
%
\multicolumn{2}{c}{}
    & \multicolumn{1}{c}{\{a\}}
    & \multicolumn{1}{c}{\{b\}}
    & \multicolumn{1}{c}{\{c\}} \\
%
\midrule
%
\multirow{5}{*}{Состояние}
    & $\to \{p\}$ & $\{p\}$  		& \{$q;p$\} 		& \{$r;q;p$\}  	\\
    %
    & $\{q;p\}$ & \{$q;p$\} 	& \{$r;q;p$\} 		& \{$r;q;p$\} 	     \\
    %
    & $\boxed{\{r;q;p\}}$ & \{$r;q;p$\} 	& \{$r;q;p$\} 		& \{$r;q;p$\} 	\\
\bottomrule
\end{tabular}
\caption{Функция перехода $\delta$ для ДКА из примера~\ref{example-reduceENKAtoDKA}}
\label{tab6}
\end{table}

Граф переходов автомата $M_D$  представлен на рисунке~\ref{aut-graph-7}.
\begin{figure}[t]
\centering
\begin{tikzpicture}[node distance=3cm,>=stealth',auto,every state/.style={thick}]
	\node (init) {};
	\node[state] (p) [right=.7cm of init] {$\{p\}$};
	\node[state, accepting] (rqp) [right of=p] {$\{r;q;p\}$};
    \node[state] (qp) [right of=rqp] {$\{q;p\}$};

	\path[->]
    	(init) edge (p)
		(p) edge [loop above] node[right] {$a$} (p)
		(p) edge [bend left=50] node[above] {$b$} (qp)
        (qp) edge [loop above] node {$a$} (qp)
        (qp) edge node {$\{b;c\}$} (rqp)
        (rqp) edge [loop below] node[right] {$\{a;b;c\}$} (rqp)
        (p) edge node {$c$} (rqp);
\end{tikzpicture}
\caption{Граф переходов ДКА из примера~\ref{example-reduceENKAtoDKA}}
\label{aut-graph-7}
\end{figure}


Отметим, что множество конечных состояний $F_D$ содержит единственное состояние $\{ r; q; p \}$, поскольку состояние $\{ r \}$ было конечным состоянием исходного автомата.
\end{myexample}

\section{Преобразование регулярного выражения в автомат}
\label{Chapter4RegtoFA}
Для любого регулярного языка $L$, заданного регулярным выражением $R$, может быть построен $\eps$-НКА, распознающий этот язык. Эта возможность заложена в определение регулярного языка и задающего его регулярного выражения (см. раздел~\ref{Chapter2RegExprs}). Регулярными являются элементарные множества над алфавитом $\Sigma$ : пустое множество (ему соответствует регулярное выражение $\es$), множество из одной пустой цепочке (регулярное выражение $\eps$) и множество из одного однобуквенного слова из любой буквы $a$ множества $\Sigma$ (регулярное выражение $a$). Также из определения следует, что результаты объединения, конкатенации и итерации регулярных языков являются регулярными языками. Далее мы построим автоматы для элементарных языков и опишем процесс построения более сложных автоматов, допускающих объединение, конкатенацию или итерацию языков, распознаваемых более простыми автоматами.

Для связывания простых автоматов в сложные конструкции удобно использовать $\eps$-переходы. Условимся, что каждый автомат любой степени сложности будет иметь ровно один вход и один выход. Начальное состояние у автомата всегда единственное. Одно конечное состояние можем получить, если ввести новое конечное состояние и связать его $\eps$-переходами со старыми конечными состояниями.

Таким образом все конструируемые на основе регулярных выражений автоматы будут представлять собой $\eps$-НКА с одним допускающим состоянием.
\begin{mytheorem}
Любой язык, определяемый регулярным выражением, можно задать некоторым конечным автоматом.
\end{mytheorem}
\begin{myproof}
Предположим, что $L = L(R)$ для регулярного выражения $R$. Покажем, что  $L = L(E)$ для некоторого $\eps$-НКА $E$, обладающего следующими свойствами:
\begin{enumerate}
	\item Автомат имеет ровно одно допускающее состояние.
	\item У автомата нет дуг, ведущих в начальное состояние.
	\item У автомата нет дуг, выходящих из допускающего состояния.
\end{enumerate}
Для доказательства теоремы применим структурную индукцию по выражению $R$, следуя рекурсивному определению регулярных выражений из раздела~\ref{Chapter2RegExprs}.

В доказательстве леммы~\ref{lemma-FA-of-ElemLangs} построены конечные автоматы, распознающие элементарные языки. На рисунке~\ref{ra-struct-1} приведены схемы автоматов, распознающих пустой язык (\textsl{a}), язык из одного символа $eps$ (\textsl{b}) и язык из одного однобуквенного слова (\textsl{c}). 

\begin{figure}[H]
\centering
\begin{minipage}{0.3\textwidth} 
\centering
\begin{tikzpicture}[node distance=2cm,>=stealth',auto,every state/.style={thick}] 
\begin{scope}
	\node (init) {};
	\node[state] (p) [right=.7cm of init] {};
    \node[state, accepting] (f) [right of=p] {};
	\path[->]
    	(init) edge (p);
			\begin{pgfonlayer}{background} 
 \draw[rounded corners, thick] ($(p.south west)+(-2ex,-2ex)$) rectangle ($(f.north east)+(2ex,2ex)$)
             coordinate  [pos=0.5] (ce) 
             coordinate  [pos=1] (ne) 
             (ce |- ne)  coordinate (no) ;  
\end{pgfonlayer}
\end{scope}	
\end{tikzpicture}
\subcaption{}
\end{minipage}
\hfill
\begin{minipage}{0.3\textwidth} 
\centering
\begin{tikzpicture}[node distance=2cm,>=stealth',auto,every state/.style={thick}] 
\begin{scope}
	\node (init) {};
	\node[state] (p) [right=.7cm of init] {};
    \node[state, accepting] (f) [right of=p] {};
	\path[->]
    	(init) edge (p)
      (p) edge node {$\eps$} (f);
			
			\begin{pgfonlayer}{background} 
 \draw[rounded corners, thick] ($(p.south west)+(-2ex,-2ex)$) rectangle ($(f.north east)+(2ex,2ex)$)
             coordinate  [pos=0.5] (ce) 
             coordinate  [pos=1] (ne) 
             (ce |- ne)  coordinate (no) ;  
\end{pgfonlayer}
\end{scope}	
\end{tikzpicture}
\subcaption{}
\end{minipage}
\hfill
\begin{minipage}{0.3\textwidth}
\centering 
\begin{tikzpicture}[node distance=2cm,>=stealth',auto,every state/.style={thick}] 
\begin{scope}
	\node (init) {};
	\node[state] (p) [right=.7cm of init] {};
    \node[state, accepting] (f) [right of=p] {};
	\path[->]
    	(init) edge (p)
      (p) edge node {$a$} (f);
			
			\begin{pgfonlayer}{background} 
 \draw[rounded corners, thick] ($(p.south west)+(-2ex,-2ex)$) rectangle ($(f.north east)+(2ex,2ex)$)
             coordinate  [pos=0.5] (ce) 
             coordinate  [pos=1] (ne) 
             (ce |- ne)  coordinate (no) ;  
\end{pgfonlayer}
\end{scope}	
\end{tikzpicture}
\subcaption{}
\end{minipage}
\caption{Схемы автоматов, распознающих элементарные языки}
\label{ra-struct-1}
\end{figure}

Все эти автоматы удовлетворяют условиям (1), (2), (3) индуктивной гипотезы.

Предположим, что утверждение теоремы истинно для непосредственных подвыражений данного регулярного выражения, т. е. языки этих подвыражений являются также языками $\eps$-НКА с единственным допускающим состоянием. Возможны четыре случая.
\begin{enumerate}
	\item Данное выражение имеет вид $R + S$ для некоторых подвыражений $R$ и $S$. Тогда ему соответствует автомат $M_U$, представленный на рисунке~\ref{ra-struct-2_a}.
	
	\begin{figure}[t]
\centering
\begin{tikzpicture}[node distance=2cm,>=stealth',auto,every state/.style={thick}] 
	\node (init) {};
	\node[state] (p) [right=.7cm of init] {};
	\begin{scope}
  \node[state] (p1) [below right=1.5cm of p] {};
	\node[state] (q1) [right of=p1] {};
	\begin{pgfonlayer}{background} 
 \draw[rounded corners, thick] ($(p1.south west)+(-2ex,-2ex)$) rectangle ($(q1.north east)+(2ex,2ex)$) node[pos=.6] {$S$}
             coordinate  [pos=0.5] (ce) 
             coordinate  [pos=1] (ne) 
             (ce |- ne)  coordinate (no) ;  
\end{pgfonlayer}
\end{scope}
	\node[state, accepting] (f) [above right=1.5cm of q1] {};
	\begin{scope}
	\node[state] (p2) [above right=1.5cm of p] {};
	\node[state] (q2) [right of=p2] {};
  			\begin{pgfonlayer}{background} 
 \draw[rounded corners, thick] ($(p2.south west)+(-2ex,-2ex)$) rectangle ($(q2.north east)+(2ex,2ex)$) node[pos=.6] {$R$} 
             coordinate  [pos=0.5] (ce) 
             coordinate  [pos=1] (ne) 
             (ce |- ne)  coordinate (no) ;
\end{pgfonlayer}
\end{scope}
	
	\path[->]
    	(init) edge (p)
			(p) edge node {$\eps$} (p1)
			(p) edge node {$\eps$} (p2)
			(q1) edge node {$\eps$} (f)
			(q2) edge node {$\eps$} (f);	
\end{tikzpicture}
\caption{Схема автомата, распознающего объединение двух языков}
\label{ra-struct-2_a}
\end{figure}


В этот автомат добавлено новое начальное состояние, из которого можно перейти в начальное состояние автомата для выражения $R$ или $S$ и продолжать работу, моделируя выбранный автомат. Попав в допускающее состояние автомата для $R$ или $S$ (распознав цепочку из языка $L(R)$ или $L(S)$ соответственно), новый автомат может по одному из $\eps$-путей перейти в свое допускающее состояние. Таким образом,  $L(M_U) = L(R) \cup L(S)$.
	\item Выражение имеет вид $RS$ для некоторых подвыражений $R$ и $S$. Автомат $M_C$ для распознавания конкатенации представлен на рисунке~\ref{ra-struct-2_b}.
	
\begin{figure}[t]
\centering
\begin{tikzpicture}[node distance=2cm,>=stealth',auto,every state/.style={thick}] 
\begin{scope}
	\node (init) {};
	\node[state] (p) [right=.7cm of init] {};
    \node[state] (p1) [right of=p] {};
		\begin{pgfonlayer}{background} 
 \draw[rounded corners, thick] ($(p.south west)+(-2ex,-2ex)$) rectangle ($(p1.north east)+(2ex,2ex)$) node[pos=.6] {$R$}
             coordinate  [pos=0.5] (ce) 
             coordinate  [pos=1] (ne) 
             (ce |- ne)  coordinate (no) ;  
\end{pgfonlayer}
\end{scope}
\begin{scope}
    \node[state] (p2) [right=1.9cm of p1] {};
    \node[state, accepting] (f) [right of=p2] {};
		\begin{pgfonlayer}{background} 
 \draw[rounded corners, thick] ($(p2.south west)+(-2ex,-2ex)$) rectangle ($(f.north east)+(2ex,2ex)$) node[pos=.6] {$S$}
             coordinate  [pos=0.5] (ce) 
             coordinate  [pos=1] (ne) 
             (ce |- ne)  coordinate (no) ;  
\end{pgfonlayer}
\end{scope}
	\path[->]
    	(init) edge (p)
      (p1) edge node {$\eps$} (p2);
\end{tikzpicture}
\caption{Схема автомата для конкатенации двух языков}
\label{ra-struct-2_b}
\end{figure}

	
Начальное состояние первого автомата становится начальным для всего автомата $M_C$, представляющего конкатенацию, а допускающим для него будет допускающее состояние второго автомата. Вначале автомат $M_C$ моделирует поведение автомата для $R$ (распознает цепочку из языка $L(R)$), потом из допускающего состояния первого автомата он переходит в начальное состояние автомата для $S$ и моделирует его поведение (распознает цепочку из языка $L(S)$). Таким образом, $L(M_C) = L(R)L(S)$.
	\item Выражение имеет вид $R^*$ для некоторого подвыражения $R$. Рассмотрим автомат $M_I$, представленный на рисунке~\ref{ra-struct-2_c}. 
	\begin{figure}[H]
\centering
\begin{tikzpicture}[node distance=2cm,>=stealth',auto,every state/.style={thick}] 
	\node (init) {};
	\node[state] (p) [right=.7cm of init] {};
	\begin{scope}
	\node[state] (p1) [right of=p] {};
	\node[state] (p2) [right of=p1] {};
	\begin{pgfonlayer}{background} 
 \draw[rounded corners, thick] ($(p1.south west)+(-2ex,-2ex)$) rectangle ($(p2.north east)+(2ex,2ex)$) node[pos=.6] {$R$}
             coordinate  [pos=0.5] (ce) 
             coordinate  [pos=1] (ne) 
             (ce |- ne)  coordinate (no) ;  
\end{pgfonlayer}
\end{scope}	
    \node[state, accepting] (f) [right of=p2] {};
	\path[->]
    	(init) edge (p)
      (p) edge node {$\eps$} (p1)
      (p2) edge node {$\eps$} (f)
      (p) edge [bend left=40] node [above] {$\eps$} (f)
      (p2) edge [bend left] node [above] {$\eps$} (p1);
\end{tikzpicture}
\caption{Схема автомата, распознающего итерацию языка}
\label{ra-struct-2_c}
\end{figure}
Возможные случаи распознавания:
	\begin{enumerate}
		\item из начального состояния автомат  $M_I$ сразу переходит в допускающее состояние по символу $\eps$. В этом случае допускается цепочка $\eps$, которая принадлежит $L(R^*)$ независимо от выражения $R$;
		\item из начального состояния автомат  $M_I$ переходит в начальное состояние автомата для $R$, моделирует поведение этого автомата и попадает в его допускающее состояние, из которого может начать заново моделировать автомат для $R$ или перейти в свое допускающее состояние. Такое поведение автомата дает возможность распознавать цепочки, принадлежащие языкам $L(R)$, $L(R)L(R)$, $L(R)L(R)L(R)$ ... $(R^*)$, за исключением, возможно, цепочки $\eps$. Но возможность ее распознавания показана в предыдущем пункте. Таким образом, $L(M_I) = L(R^*)$.
	\end{enumerate}
	\item Выражение имеет вид $(R)$ для некоторого подвыражения $R$. Автомат для $R$ может быть автоматом и для $(R)$, поскольку скобки не влияют на язык, задаваемый выражением.
\end{enumerate}
Построенные автоматы удовлетворяют всем трем условиям индуктивной гипотезы: одно допускающее состояние, отсутствие дуг, ведущих в начальное состояние, и дуг, выходящих из допускающего состояния.
\end{myproof}
\begin{myexample}
Преобразуем регулярное выражение $(0 + 1)^*1(0 + 1)$ в $\eps$-НКА. Вначале построим автомат для выражения $0 + 1$. Для этого используем два автомата, построенные по схеме на рисунке~\ref{ra-struct-1} (\textsl{с}): один автомат с меткой $0$ на дуге, другой --- с меткой $1$. Эти автоматы соединим с помощью конструкции объединения по схеме~\ref{ra-struct-2_a}. Полученный результат представлен на рисунке~\ref{ra-struct-example-1_a}. 
\begin{figure}[H]
\centering
\begin{tikzpicture}[node distance=2cm,>=stealth',auto,every state/.style={thick}] 
	\node (init) {};
	\node[state] (p) [right=.7cm of init] {};
  \node[state] (p1) [above right of=p] {};
	\node[state] (q1) [below right of=p] {};
	\node[state] (f) [right=3.9cm of p] {};
	\node[state] (p2) [above left of=f] {};
	\node[state] (q2) [below left of=f] {};
	\path[->]
    	%(init) edge (p)
			(p) edge node {$\eps$} (p1)
			(p) edge node {$\eps$} (q1)
			(p1) edge node {$0$} (p2)
			(q1) edge node {$1$} (q2)
			(p2) edge node {$\eps$} (f)
			(q2) edge node {$\eps$} (f);	
\end{tikzpicture}
\caption{Автомат для регулярного выражения $0 + 1$}
\label{ra-struct-example-1_a}
\end{figure}
После этого применим к полученному автомату конструкцию итерации по схеме~\ref{ra-struct-2_c}. Получим автомат, представленный на рисунке~\ref{ra-struct-example-1_b}. 
\begin{figure}[t]
\centering
\begin{tikzpicture}[node distance=1.5cm,>=stealth',auto,every state/.style={thick}] 
	\node (init) {};
	\node[state] (p_ext) [right=.7cm of init] {};
	\node[state] (p) [right of=p_ext] {};
  \node[state] (p1) [above right of=p] {};
	\node[state] (q1) [below right of=p] {};
	\node[state] (f) [right=3.9cm of p] {};
	\node[state] (f_ext) [right of=f] {};
	\node[state] (p2) [above left of=f] {};
	\node[state] (q2) [below left of=f] {};
	\path[->]
    	%(init) edge (p)
			(p_ext) edge node {$\eps$} (p)
			(p) edge node {$\eps$} (p1)
			(p) edge node {$\eps$} (q1)
			(p1) edge node {$0$} (p2)
			(q1) edge node {$1$} (q2)
			(p2) edge node {$\eps$} (f)
			(q2) edge node {$\eps$} (f)
			(f) edge node {$\eps$} (f_ext)	
			(p_ext) edge [bend left=60,in=130,out=50] node [above] {$\eps$} (f_ext)
			(f) edge [bend left=160,in=75,out=100] node [below] {$\eps$} (p);	
\end{tikzpicture}
\caption{Автомат для регулярного выражения $(0 + 1)^*$}
\label{ra-struct-example-1_b}
\end{figure}

Осталось к полученному автомату применить конструкции конкатенации по схеме~\ref{ra-struct-2_b}. Сначала автомат, представленный на рисунке~\ref{ra-struct-example-1_b}, соединяется с автоматом, допускающим только цепочку $1$ (нужно еще раз применить конструкцию по схеме~\ref{ra-struct-1} (\textsl{с}) с меткой $1$ на дуге). Последним автоматом в конкатенации будет еще один автомат для выражения $0 + 1$, построенный по тем же принципам, что и автомат на рисунке~\ref{ra-struct-example-1_b}.
Полный автомат для выражения $(0 + 1)^*1(0 + 1)$ представлен на рисунке~\ref{ra-struct-example-1_c}.
\begin{figure}[t]
\centering
\begin{tikzpicture}[node distance=2.5cm,>=stealth',auto,every state/.style={thick},every node/.style={inner sep=-.4cm,minimum size=-1cm,scale=0.7},scale=0.7] 
	\node (init) {};
	\node[state] (p_ext) [right=.7cm of init] {};
	\node[state] (p) [right of=p_ext] {};
  \node[state] (p1) [above right of=p] {};
	\node[state] (q1) [below right of=p] {};
	\node[state] (f) [right=3.9cm of p] {};
	\node[state] (f_ext) [right of=f] {};
	\node[state] (p2) [above left of=f] {};
	\node[state] (q2) [below left of=f] {};
	\node[state] (p3) [below of=f_ext] {};
	\node[state] (p4) [below of=p3] {};
	\node[state] (p_) [right of=p4] {};
  \node[state] (p1_) [above right of=p_] {};
	\node[state] (q1_) [below right of=p_] {};
	\node[state, accepting] (f_) [right=3.9cm of p_] {};
	\node[state] (p2_) [above left of=f_] {};
	\node[state] (q2_) [below left of=f_] {};
	
	\path[->]
    	(init) edge (p_ext)
			(p_ext) edge node {$\eps$} (p)
			(p) edge node {$\eps$} (p1)
			(p) edge node {$\eps$} (q1)
			(p1) edge node {$0$} (p2)
			(q1) edge node {$1$} (q2)
			(p2) edge node {$\eps$} (f)
			(q2) edge node {$\eps$} (f)
			(f) edge node {$\eps$} (f_ext)	
			(p_ext) edge [bend left=60,in=130,out=50] node [above] {$\eps$} (f_ext)
			(f) edge [in=-90,out=-90, looseness=1.5] node [below] {$\eps$} (p)
			(f_ext) edge node {$\eps$} (p3)	
			(p3) edge node {$1$} (p4)	
			(p4) edge node {$\eps$} (p_)	
			(p_) edge node {$\eps$} (p1_)
			(p_) edge node {$\eps$} (q1_)
			(p1_) edge node {$0$} (p2_)
			(q1_) edge node {$1$} (q2_)
			(p2_) edge node {$\eps$} (f_)
			(q2_) edge node {$\eps$} (f_);
\end{tikzpicture}
\caption{Автомат для регулярного выражения $(0 + 1)^*1(0 + 1)$}
\label{ra-struct-example-1_c}
\end{figure}

\end{myexample}

\section{Построение $\eps$-НКА по ПЛ-грамматике}
\label{Chapter4GramtoFA}
В разделе $1.4$ приведена классификация Хомского формальных грамматик. По этой классификации ПЛ-грамматикой являются такая грамматика, все правила которой имеют вид:
\[
	\begin{array}{l}
	A \to xB \\
	A \to x \\
	\end{array}
\], где $A,B \in N$, $x\in\Sigma^*$.

Из леммы~\ref{lemma-dka-to-pl} известно, что для любого конечного автомата можно построить ПЛ-грамматику, порождающую тот же язык, что распознает исходный автомат. При этом получается, строго говоря, не ПЛ-грамматика, а автоматная грамматика, все правила которой имеют вид:
\[
 \begin{array}{l}
	A \to xB \\
	A \to x \\
	\end{array}
\], где $A,B \in N, x\in\Sigma$.

Фактически автоматная грамматика --- это ПЛ-грамматика, в правилах которой могут встречаться только одиночные терминальные буквы.
Автомат работает следующим образом: по текущему состоянию и текущему входному символу автомат переходит в следующее состояние. В формальной грамматике нетерминальные символы соответствуют состояниям, а терминальные символы - входным символом автомата. За один раз автомат может прочитать только один входной символ, который в принципе может состоять из нескольких букв, но автомат считает этот символ неделимым. С точки зрения же ПЛ-грамматики последовательность терминальных букв в правилах является частью выводимого слова.
\begin{mylemma}
\label{lemma-pl-to-nka}
Пусть задана ПЛ-грамматика $G = (\Sigma, N, \mathcal P, S \in N)$. Тогда $\exists$ такой $\eps$-НКА $M$, что $L(M) = L(G)$.
\end{mylemma}
\begin{myproof}
Построим $M = (Q, \Sigma, \delta, q_0, F)$ следующим образом:
\begin{enumerate}
	\item $Q = N \cup \{ f \}$, где $f \notin N$.
	\item $F = \{ f \}$.
	\item $q_0 = S$.
	\item Для определения $\delta$-функции введем вспомогательную $\hat{\delta}$-функцию по аналогии с конструкциями из леммы~\ref{lemma-dka-to-pl}:
		\begin{enumerate}
			\item Если $(A \to \alpha B)\in \mathcal P$, где $A, B \in N, \alpha \in \Sigma^*$, то $\hat{\delta}(A, \alpha) = B$.
			\item Если $(A \to \alpha)\in \mathcal P$, где $A \in N, \alpha \in \Sigma^*$, то $\hat{\delta}(A, \alpha) = f$.
		\end{enumerate}
		$\hat{\delta}$-функция в качестве второго входного параметра принимает цепочку (возможно, пустую) символов входного алфавита, т. е.
		\[
			\hat{\delta} \colon Q \times (\Sigma^*) \to P(Q).
		\]
		Поскольку автомат за один такт может прочитать не более одного символа входного алфавита, необходимо перейти от $\hat{\delta}$-функции к $\delta$-функции автомата $M$:
		\[
			\delta \colon Q \times (\Sigma \cup \{ \eps \} ) \to P(Q).
		\]
		Пусть $\omega \in L(M)$, тогда $(q_0,\omega)\vdash_M^*(f,\eps)$.
		Представим $\omega = \alpha_1\alpha_2...\alpha_n$. Тогда
		\[
			(q_0, \alpha_1\alpha_2...\alpha_n)\vdash_M(q_1, \alpha_2...\alpha_n)\vdash_M^*(q_{n-1}, \alpha_n)\vdash_M(f, \eps).
		\]
Дополним множество $\mathcal P$ правил грамматики $G$ правилом $S \to \alpha_1q_1$. Этому правилу будет соответствовать $\delta(q_0, \alpha_1) = \{q_1\}$.
		Повторяя эти действия для всех $\alpha_i$, получим:
		\[
			S \To^* \alpha_1\alpha_2...\alpha_{n-1}q_{n-1} \To \alpha_1...\alpha_n = \omega.
		\]
		Если $\hat{\delta}(q_1, \omega = \alpha_1\alpha_2...\alpha_k) = \{q_2\}$, то
		\[
			\begin{array}{l}
			\delta(q_1, \alpha_1) = \{q_1^{(1)}\};  \\
			\delta(q_1^{(1)}, \alpha_2) = \{q_1^{(2)}\}; \\
			... \\
			\delta(q_1^{(k-2)}, \alpha_{k-2}) = \{q_1^{(k-1)}\};\\
			\delta(q_1^{(k-1)}, \alpha_k) = \{q_2\}.
			\end{array}
		\]
	\end{enumerate}
	Таким образом, каждый такт автомата $M$ соответствует одному правилу грамматики $G$, и $L(M) = L(G)$.
\end{myproof}
\begin{myexample}
\label{ex-pl-to-nka}
	Рассмотрим грамматику $G = (\{S; T\}, \{0; 1\}, \mathcal P, S)$, где множество $\mathcal P$ состоит из следующих продукций:
	\[
		\begin{array}{l}
			S \to 01T \mid 1S \mid \eps; \\
			T \to 11S \mid 000T \mid 01 \mid \eps.
		\end{array}
	\]
Построим $\eps$-НКА $M$, применяя конструкции из леммы~\ref{lemma-pl-to-nka}.
	\begin{figure}
\centering
\begin{tikzpicture}[node distance=3cm,>=stealth',auto,every state/.style={thick}]
	\node (init) {};
	\node[state] (S) [right=.7cm of init] {$S$};
    \node[state] (T) [right of=S] {$T$};
	\node[state, accepting] (f) [right of=T] {$f$};

	\path[->]
    	(init) edge (S)
		(S) edge [loop above] node[right] {$1$} (S)
		(S) edge [bend left] node[above] {$01$} (T)
    (T) edge [loop above] node {$000$} (T)
    (T) edge [bend left] node[above] {$01$} (f)
    (T) edge [left] node[below] {$\eps$} (f)
    (T) edge [left] node[below] {$11$} (S)
    (S) edge [bend right] node[below] {$\eps$} (f);
\end{tikzpicture}
\caption{Граф переходов $\hat{\delta}$-функции из примера~\ref{ex-pl-to-nka}}
\label{aut-graph-8}
\end{figure}


\[
		\begin{array}{l}
			Q = \{ S; T \} \cup \{ f \}; \\
			\Sigma = \{ 0; 1 \}; \\
			q_0 = S;\\
			F = \{ f \}.\\
		\end{array}
\]
		По правилам грамматики $G$ определим $\hat{\delta}$-функцию:
		\[
			\begin{array}{l}
				\hat{\delta}(S, 01) = T; \\
				\hat{\delta}(S, 1) = S; \\
				\hat{\delta}(S, \eps) = f; \\
				\hat{\delta}(T, 11) = S; \\
				\hat{\delta}(T, 000) = T; \\
				\hat{\delta}(T, 01) = f; \\
				\hat{\delta}(T, \eps) = f.
		\end{array}
	\]
	Граф переходов $\hat{\delta}$-функции представлен на рисунке~\ref{aut-graph-8}.
	Для построения $\delta$-функции автомата $M$ определим множество <<промежуточных>> состояний, изменив множество $\mathcal P$ правил грамматики $G$ следующим образом:
	\[
			\begin{array}{l}
				S \to 0S_1 \mid 1S \mid \eps; \\
				S_1 \to 1T; \\
				T \to 1T_1 \mid 0T_2 \mid 0T_3 \mid 0T_4 \mid \eps; \\
				T_1 \to 1S; \\
				T_2 \to 0T_3; \\
				T_3 \to 0T; \\
				T_4 \to 1.
			\end{array}
	\]
	По модифицированным правилам грамматики $G$ определим ${\delta}$-функцию:
	\[
			\begin{array}{l}
				\delta(S, 0) = S_1; \\
				\delta(S_1, 1) = T; \\
				\delta(S, 1) = S; \\
				\delta(S, \eps) = f; \\
				\delta(T, 1) = T_1; \\
				\delta(T, 0) = T_2; \\
				\delta(T, 0) = T_3; \\
				\delta(T, 0) = T_4; \\
				\delta(T, \eps) = f; \\
				\delta(T_1, 1) = S; \\
				\delta(T_2, 0) = T_3; \\
				\delta(T_3, 0) = T; \\
				\delta(T_4, 1) = f.
		\end{array}
	\]
	Множество состояний $Q$ дополняется состояниями, полученными при расширении правил грамматики $G$:
	\[
	Q = \{ S; T; F \} \cup \{ S_1; T_1; T_2; T_3; T_4 \}
	\]
	Граф искомого автомата $M = (Q, \Sigma, q_0, \delta, F)$ представлен на рисунке~\ref{aut-graph-9}.
	\begin{figure}
\centering
\begin{tikzpicture}[node distance=3cm,>=stealth',auto,every state/.style={thick}]
	\node (init) {};
	\node[state] (S) [right=.7cm of init] {$S$};
    \node[state] (S_1) [right of=S] {$S_1$};
    \node[state] (T) [right of=S_1] {$T$};
    \node[state] (T_2) [right of=T] {$T_2$};
    \node[state] (T_3) [above of=T] {$T_3$};
    \node[state] (T_1) [above of=S_1] {$T_1$};
	\node[state, accepting] (f) [right of=T_2] {$f$};

	\path[->]
    	(init) edge (S)
		(S) edge [loop above] node[right] {$1$} (S)
		(S) edge node[above] {$0$} (S_1)
		(S_1) edge node[above] {$1$} (T)
		(T) edge node[above] {$0$} (T_2)
		(T_2) edge node[above] {$1$} (f)
		(T) edge[bend right] node[below] {$\eps$} (f)
    (T_2) edge node[right] {$0$} (T_3)
    (T_3) edge  node {$0$} (T)
    (T) edge node[right] {$1$} (T_1)
    (T_1) edge node[left] {$1$} (S)
    (S) edge [bend right, in=230] node[below] {$\eps$} (f);
\end{tikzpicture}
\caption{Граф переходов $\eps$-НКА из примера~\ref{ex-pl-to-nka}}
\label{aut-graph-9}
\end{figure}


\end{myexample}
\section{Вычисление языка $\eps$-НКА}
\label{Chapter4FALang}
По автомату, допускающему некоторый язык, можно построить регулярное выражение, задающее этот язык. Для этого нужно построить выражения, описывающие множества цепочек, которыми помечены определенные пути на графе переходов автомата. Эти пути могут проходить только через ограниченное подмножество состояний. При индуктивном определении таких выражений нужно начинать с самых простых выражений, описывающих пути, которые не проходят ни через одно состояние (т. е. являются отдельными вершинами или дугами). После этого индуктивно строятся выражения, которые позволяют этим путям проходить через постепенно расширяющиеся множества состояний. В конце этой процедуры будут получены пути, которые могут проходить через любые состояния, т. е. генерируются выражения, представляющие все возможные пути. Подробно эти идеи излагаются в~\cite{Hop} (раздел 3.2.1).
В данной работе будет рассмотрен менее трудоёмкий способ вычисления регулярного выражения по конечному автомату.
\subsection*{Метод последовательного исключения состояний}
Метод вычисления регулярного выражения, рассматриваемый в данном разделе, предполагает исключение состояний конечного автомата. Если исключить некоторое состояние $r$, то все пути автомата, проходящие через это состояние, исчезают. Чтобы язык, допускаемый автоматом, не изменился, необходимо написать на дуге, ведущей непосредственно из некоторого состояния $q$ в состояние $p$, метки всех тех путей, которые вели из состояния $q$ в состояние $p$, проходя через состояние $r$. Поскольку теперь метка такой дуги будет содержать цепочки, а не отдельные символы, и таких цепочек может быть даже бесконечно много, то нельзя использовать список этих цепочек в качестве метки. Необходимо использовать конечный способ представления всех подобных цепочек, т. е., использовать регулярные выражения.

Таким образом, мы можем рассматривать автоматы, у которых метками являются регулярные выражения. Язык такого автомата представляет собой объединение по всем путям, ведущим от начального к заключительному состоянию, языков, образуемых с помощью конкатенации языков регулярных выражений, расположенных вдоль этих путей.

Опишем процедуру исключения состояния.
Пусть $M = (Q, \Sigma, \delta, q_0, F)$ --- конечный автомат. Рассмотрим подграф графа переходов автомата $M$ (см. рис.~\ref{m_del_1}). Введем операцию $DEL(p, r, q)$, где $r$ --- вершина, подлежащая удалению, следующим образом:
\begin{enumerate}
	\item Если из вершины $r$ ведет дуга с меткой $\gamma$ в эту же вершину (петля), то после удаления вершины $r$ этой дуге будет соответствовать метка $\gamma^*$ (соответствие операции итерация).
	\item Если $\exists$ путь из вершины $p$ в вершину $q$, и этот путь проходит по дугам, помеченным $\alpha$, $\alpha_1$,..., $\beta$, то после удаления вершины $r$ дуга $(p, q)$ будет помечена меткой $\alpha\alpha_1,...,\beta$, составленной из меток всех дуг, образующих этот путь (соответствие операции конкатенация).
	\item Если $\exists$ другие пути, ведущие из вершины $p$ в вершину $q$, то их метки объединяются (соответствие операции объединение).
\end{enumerate}
Результат применения операции $DEL(p, r, q)$ к подграфу из рисунка~\ref{m_del_1} представлен на рисунке~\ref{m_del_2}.
\begin{figure}
\centering
\begin{tikzpicture}[node distance=3cm,>=stealth',auto,every state/.style={thick}]
	\node (init) {};
	\node[state] (p) [right=.7cm of init] {$p$};
    \node[state] (r) [right of=p] {$r$};
	\node[state] (q) [right of=r] {$q$};

	\path[->]
    	%(init) edge (p)
		(p) edge node[above] {$\alpha$} (r)
		(p) edge [bend right] node[below] {$\xi$} (q)
    (r) edge [loop above] node[above] {$\gamma$} (r)
    (r) edge node[above] {$\beta$} (q);
\end{tikzpicture}
\caption{Подграф графа конечного автомата $M$ до удаления вершины $r$}
\label{m_del_1}
\end{figure}


\begin{figure}
\centering
\begin{tikzpicture}[node distance=3cm,>=stealth',auto,every state/.style={thick}]
	\node (init) {};
	\node[state] (p) [right=.7cm of init] {$p$};
	\node[state] (q) [right of=r] {$q$};

	\path[->]
    	%(init) edge (p)
		(p) edge node[above] {$\xi + \alpha\gamma^*\beta$} (q);
\end{tikzpicture}
\caption{Подграф графа конечного автомата $M$ после удаления вершины $r$}
\label{m_del_2}
\end{figure}


\Algo{Алгоритм исключения вершины из графа конечного автомата}
{
	$\Gamma = (V, E, \phi)$ --- граф автомата $M$, где $\phi$ --- функция разметки; \\
	$r \in V$ --- вершина, которая исключается из графа автомата.
}
{$\Gamma' = (V', E', \phi)$ --- граф автомата $M$ без вершины $r$.}
{ }
{
\item Для $p \in V$, $p \neq r$ и $(p, r) \in E$ выполнить: \\
			Для $q \in V$, $q \neq r$ и $(r, q) \in E$ выполнить: \\
			$\phi(p, q) = \phi(p, q) \cup \phi(p, r)(\phi(r, r))^*\phi(r, q)$
\item  Исключить из графа $\Gamma$ вершину $q$ и инцидентные ей дуги.
}

Опишем стратегию построения регулярного выражения по конечному автомату.
\begin{enumerate}
	\item Если начальное состояние совпадает с допускающим, нужно ввести новое допускающее состояние и добавить $\eps$-переход из начального состояния в добавленное. После этого начальное состояние перестает быть допускающим.
	\item Если автомат содержит более одного допускающего состояния, то нужно ввести новое допускающее состояние и добавить $\eps$-переходы из старых допускающих состояний в добавленное состояние. Новое состояние становится единственным допускающим состоянием автомата.
	\item С помощью алгоритма~$4.5.1$ исключить все состояния, кроме начального ($q_0$) и конечного ($f$).
	\item После шагов 1-3 должен остаться автомат с двумя состояниями, подобный автомату на рисунке~\ref{excl-aut-1}. Регулярное выражение по этому автомату может быть выписано разными способами.

Рассмотрим выражение $(R + SU^*T)^*SU^*$ как один из возможных вариантов записи регулярного выражения по данному автомату. В автомате есть возможность переходить из начального состояния в него же любое количество раз, проходя по путям, метки которых принадлежат либо $L(R)$, либо $L(SU^*T)$. Выражение $SU^*T$ представляет пути, которые ведут в допускающее состояние по пути с меткой из языка $L(S)$, затем, возможно, несколько раз проходят через допускающее состояние, используя пути с метками из $L(U)$, и наконец возвращаются в начальное состояние, следуя по пути с меткой из $L(T)$. Отсюда нужно перейти в допускающее состояние, уже никогда не возвращаясь в начальное, вдоль пути с меткой из $L(S)$. Находясь в допускающем состоянии, можно произвольное количество раз вернуться в него по пути с меткой из $L(U)$.
\begin{figure}
\centering
\begin{tikzpicture}[node distance=3cm,>=stealth',auto,every state/.style={thick}]
	\node (init) {};
	\node[state] (p) [right=.7cm of init] {$q_0$};
	\node[state, accepting] (q) [right of=r] {$f$};

	\path[->]
    	(init) edge (p)
			(p) edge[loop above] node[above] {$R$} (p)
			(q) edge[loop above] node[above] {$U$} (q)
		(p) edge[bend left] node[above] {$S$} (q)
		(q) edge[bend left] node[below] {$T$} (p);
\end{tikzpicture}
\caption{Обобщенный автомат с двумя состояниями}
\label{excl-aut-1}
\end{figure}


\item Искомое выражение представляет собой регулярное выражение, являющееся меткой дуги из начального состояния в конечное сокращенного автомата.
\end{enumerate}

\section{Задача минимизации конечного автомата}
\label{Chapter4FALMin}
Конечный автомат распознаёт регулярный язык, для которого может существовать несколько вариантов записи в виде регулярных выражений. В разделе $4.3$ показано, что любое регулярное выражение можно преобразовать в конечный автомат, распознающий тот же язык. Таким образом один и тот же язык может быть распознан различными конечными автоматами, причём эти автоматы будут эквивалентны между собой.

В решении практических задач удобнее пользоваться конечным автоматом с меньшим числом состояний, поэтому возникает задача выбора конечного автомата с меньшим числом состояний из множества эквивалентных ему автоматов. Далее будет показано, что для любого конечного автомата можно построить эквивалентный ему автомат с меньшим числом состояний или установить минимальность исходного автомата.

\subsection*{Отношение эквивалентности на множестве конечных автоматов}
Напомним, что отношение эквивалентности на множестве $X$ --- это бинарное отношение, для которого выполняются следующие условия:
\begin{enumerate}
\item Рефлексивность: $a \sim a$ для $\forall a \in X$.
\item Симметричность: $a \sim b$, то $b \sim a$ для $\forall a, b \in X$.
\item Транзитивность: $a \sim b$ и $b \sim c$, то $a \sim c$ для $\forall a, b, c \in X$.
\end{enumerate}

\begin{mylemma}
Конечные автоматы ($\eps$-НКА, НКА, ДКА) $M_1$ и $M_2$ над одним и тем же конечным алфавитом $\Sigma$ называются эквивалентными, если $L(M_1) = L(M_2)$.
\end{mylemma}
\begin{myproof}
Покажем, что для введённого отношения эквивалентности на конечных автоматах выполняются свойства рефлексивности, симметричности и транзитивности.
Пусть $M_{\Sigma}$ --- множество всех конечных автоматов ($\eps$-НКА, НКА, ДКА) над входным алфавитом  $\Sigma$.
\begin{enumerate}
\item Рефлексивность: $\forall M \in M_{\Sigma}$ $M \sim M \Leftrightarrow L(M) = L(M)$.
\item Симметричность: $\forall M_1, M_2 \in M_{\Sigma}$ если $M_1 \sim M_2 \Leftrightarrow M_2 \sim M_1$, то $L(M_1) = L(M_2) \Leftrightarrow L(M_2) = L(M_1)$.
\item Транзитивность: $\forall M_1, M_2, M_3 \in M_{\Sigma}$ $M_1 \sim M_2$ и  $M_2 \sim M_3 \Rightarrow M_1 \sim M_3$, т.~к. $ L(M_1) = L(M_2)$ и $L(M_2) = L(M_3) \Rightarrow L(M_1) = L(M_3)$.
\end{enumerate}
\end{myproof}
Итак, два конечных автомата ($\eps$-НКА, НКА, ДКА) считаются эквивалентными между собой, если они допускают один и тот же язык. Нам известно, что автомат любого типа ($\eps-НКА$, НКА) может быть сведён к ДКА. Тогда сформулируем задачу минимизации конечного автомата следующим образом: \\
\textit{Задача:} Для произвольного ДКА найти эквивалентный ему ДКА с минимальным числом состояний.
Минимальные ДКА всегда одинаковые. Если даны два минимальных ДКА для одного и того же языка, то всегда можно переименовать состояния так, что данные ДКА будут одинаковыми.
Для того, чтобы построить минимальный ДКА, необходимо выделить состояния, которые можно удалить, не нарушая при этом эквивалентность исходного и модифицированного ДКА.

\subsection*{Недостижимые состояния конечного автомата}
Безболезненно из конечного автомата можно удалить только те состояния, которые не участвуют в распознавании языка.

\textbf{\textit{Определение:}} Состояние $q \in Q$ конечного автомата $M = (Q,\Sigma, \delta, q_0, F)$ называется недостижимым, если $\nexists \omega \in \Sigma^* \mid (q_0, \omega) \vdash_M^* (q, \eps)$.\\
Из определения следует, что нельзя подобрать никакую цепочку, по которой можно было бы пройти от стартового состояния до недостижимого, следовательно, это состояние не используется при распознавании любых цепочек языка и может быть удалено вместе с соответствующими переходами.

\Algo{Устранение недостижимых состояний конечного автомата}
{
	Конечный автомат $M = (Q,\Sigma, \delta, q_0, F)$.
}
{
	Конечный автомат $M' = (Q',\Sigma, \delta', q_0, F')$ --- КА без недостижимых состояний, такой что $L(M) = L(M')$.
}
{
 Рекурсивное построение расширяющейся последовательности подмножеств специального вида специального вида множества $Q$.
}
{
\item Положить $Q_{D_0}=\{q_0\}$, $i=1$.
\item Положить $Q_{D_i}=\{p \mid  \forall q \in Q_{D_{i-1}} \exists \delta(q, t) = \{p\}, t \in \Sigma  \}\cup Q_{D_{i-1}}$.
\item Если $Q_{D_i}\neq Q_{D_{i-1}}$, то положить $i=i+1$ перейти к шагу 2, в противном случае положить $Q_D=Q_{D_i}$.
\item Вернуть КА $M'=(Q',\Sigma, \delta', q_0, F')$, где $Q'=Q_D\cap Q$, $F'=Q_D\cap F$, $\delta' = \delta \setminus \{ \delta(q, t) = p \mid q \in (Q \setminus Q_D), t \in \Sigma \}$.
}

\begin{myexample}
Рассмотрим детерминированный конечный автомат $M = (Q,\Sigma, \delta, q_0, F)$, заданный графом переходов на рисунке~\ref{aut-graph-10}. По алгоритму $4.6.1$ построим множество достижимых состояний $Q_D$: \\
$Q_{D_0} = \{ q_0 \}$ \\
Из вершины $q_0$ можно перейти в вершины $q_1, q_2$. Таким образом \\
$Q_{D_1} = \{ q_1, q_2 \} \cup \{ q_0 \} $ \\
Из вершин $q_1, q_2$ можно попасть в вершины $q_3, q_4$ соответственно. Таким образом \\
$Q_{D_2} = \{ q_3, q_4 \} \cup \{ q_0, q_1, q_2 \} $ \\
Из вершины $q_3$ можно попасть в вершины $q_2, q_4$, а из $q_4$ --- в вершины $q_1, q_3$. Тогда \\
$Q_{D_3} = \{ q_2, q_4, q_1, q_3 \} \cup \{ q_0, q_1, q_2, q_3, q_4 \} $, следовательно $Q_{D_2} = Q_{D_3}$ и искомое множество достижимых состояний $Q_D = \{ q_0, q_1, q_2, q_3, q_4 \}$.
Конечный автомат без недостижимых состояний, эквивалентный исходному, представлен на рисунке ~\ref{aut-graph-11}.
\end{myexample}

\begin{figure}
\centering
\begin{tikzpicture}[node distance=3cm,>=stealth',auto,every state/.style={thick}]
	\node (init) {};
	\node[state] (q_0) [right=.7cm of init] {$q_0$};
    \node[state] (q_2) [right of=q_0] {$q_2$};
    \node[state] (q_1) [above of=q_2] {$q_1$};

    \node[state, accepting] (q_3) [right of=q_1] {$q_3$};
    \node[state, accepting] (q_4) [right of=q_2] {$q_4$};
    \node[state] (q_5) [right of=q_3] {$q_5$};
	\node[state] (q_6) [right of=q_4] {$q_6$};

	\path[->]
    	(init) edge (q_0)
		(q_0) edge node[right] {$a$} (q_1)
		(q_0) edge node[right] {$b$} (q_2)
		(q_1) edge node[right] {$b$} (q_3)
		(q_2) edge node[right] {$b$} (q_4)
		(q_3) edge node[below] {$a$} (q_2)
		(q_3) edge[bend left] node[below] {$b$} (q_4)
    (q_4) edge node[left] {$a$} (q_1)
    (q_4) edge node[above] {$b$} (q_3)
    (q_5) edge node[left] {$a$} (q_3)
    (q_5) edge node[below] {$b$} (q_6)
    (q_6) edge node[left] {$b$} (q_4)
    (q_6) edge[bend left] node[above] {$a$} (q_5)
    ;
\end{tikzpicture}
\caption{Граф переходов ДКА с недостижимыми состояниями}
\label{aut-graph-10}
\end{figure}


\begin{figure}
\centering
\begin{tikzpicture}[node distance=3cm,>=stealth',auto,every state/.style={thick}]
	\node (init) {};
	\node[state] (q_0) [right=.7cm of init] {$q_0$};
    \node[state] (q_2) [right of=q_0] {$q_2$};
    \node[state] (q_1) [above of=q_2] {$q_1$};

    \node[state, accepting] (q_3) [right of=q_1] {$q_3$};
    \node[state, accepting] (q_4) [right of=q_2] {$q_4$};

	\path[->]
    	(init) edge (q_0)
		(q_0) edge node[right] {$a$} (q_1)
		(q_0) edge node[right] {$b$} (q_2)
		(q_1) edge node[right] {$b$} (q_3)
		(q_2) edge node[right] {$b$} (q_4)
		(q_3) edge node[below] {$a$} (q_2)
		(q_3) edge[bend left] node[below] {$b$} (q_4)
    (q_4) edge node[left] {$a$} (q_1)
    (q_4) edge node[above] {$b$} (q_3);
\end{tikzpicture}
\caption{Граф переходов ДКА без недостижимых состояний}
\label{aut-graph-11}
\end{figure}



\subsection*{Неразличимые состояния конечного автомата}
\textit{\textbf{Определение:}} Пусть $M = (Q,\Sigma, \delta, q_0, F)$ --- детерминированный конечный автомат. Два состояния $p, q \in Q$ называются \textit{неразличимыми}, если для $\forall \omega \in \Sigma^*$ выполняется \[ (p, \omega) \vdash_M^* (f_1, \eps), (q, \omega) \vdash_M^* (f_2, \eps), \] при этом либо $f_1, f_2 \in F$, либо $f_1, f_2 \notin F$. $f_1, f_2$ могут быть двумя разными или одним состоянием.

Иными словами, состояния невозможно различить, если просто проверить, допускает ли данный КА входную цепочку, начиная работу в одном (произвольно) из этих состояний.

\textit{\textbf{Определение:}} Пусть $M = (Q,\Sigma, \delta, q_0, F)$ --- детерминированный конечный автомат. Два состояния $p, q \in Q$ называются \textit{различимыми}, если $\exists \omega \in \Sigma^*$ такая, что \[ (p, \omega) \vdash_M^* (f_1, \eps), (q, \omega) \vdash_M^* (f_2, \eps), \] при этом либо $f_1 \in F$ и $f_2 \notin F$, либо наоборот. Очевидно, что $f_1 \neq f_2$ в обязательном порядке.

Слово $\omega$ в этом случае называется различающим словом состояний $p, q$. Любая пара состояний конечного автомата либо различима, либо нет.

Рассмотрим лемму об отношении неразличимости.
\begin{mylemma}
Пусть $M = (Q,\Sigma, \delta, q_0, F)$ --- ДКА. Неразличимость является отношением эквивалентности на множестве $Q$ и разбивает $Q$ в объединение непересекающихся классов эквивалентности, каждый из которых состоит из попарно непересекающихся состояний, но при этом любая пара состояний из разных классов эквивалентности будет различимой.
\end{mylemma}
\begin{myproof}
Покажем, что для введённого отношения неразличимости на множестве состояний конечного автомата выполняются свойства рефлексивности, симметричности и транзитивности.
\begin{enumerate}
\item Рефлексивность: $\forall p \in Q$ $p \sim p \Leftrightarrow \forall \omega \in \Sigma^*$ $ (p, \omega) \vdash^* (f_1, \eps), (p, \omega) \vdash^* (f_1, \eps) $, так как $f_1 = f_1$, то $(p, p)$ --- неразличимы, следовательно $p \sim p$.
\item Симметричность: $\forall p, q \in Q$ если $p \sim q$, то $q \sim p$: \\
$\forall \omega \in \Sigma^* (p, \omega) \vdash^* (f_1, \eps), (q, \omega) \vdash^* (f_2, \eps) $. Так как $p \sim q$, то либо $f_1, f_2 \in F$, либо $f_1, f_2 \notin F$.
\item Транзитивность: $\forall p, q, r$ если $p \sim q, q \sim r$, то $p \sim r$.\\ $\forall \omega \in \Sigma^* (p, \omega) \vdash^* (f_1, \eps), (q, \omega) \vdash^* (f_2, \eps) $.
Так как $p \sim q$, то либо $f_1, f_2 \in F$, либо $f_1, f_2 \notin F$. \\
$ (q, \omega) \vdash^* (f_2, \eps), (r, \omega) \vdash^* (f_3, \eps) $.
Так как $q \sim r$, то либо $f_2, f_3 \in F$, либо $f_2, f_3 \notin F$. \\
$ (p, \omega) \vdash^* (f_1, \eps), (q, \omega) \vdash^* (f_2, \eps), (r, \omega) \vdash^* (f_3, \eps) $
и либо $f_1, f_2, f_3 \in F$, либо $f_1, f_2, f_3 \notin F$.
\end{enumerate}
\end{myproof}

\textit{\textbf{Определение:}} Пусть $M = (Q,\Sigma, \delta, q_0, F)$ --- ДКА. $q_1, q_2$ --- различимые состояния $M$.

Будем говорить, что $q_1, q_2$ --- $k$-неразличимы $q_1 \sim^k q_2$, если не существует такой цепочки $\omega$, различающей $q_1, q_2$, длина которой меньше или равна $k$.

Состояния $q_1, q_2$ неразличимы ($q_1 \sim q_2$), если они $k$-неразличимы при любом $k \geq 0$.

\begin{mylemma}
Пусть $M = (Q,\Sigma, \delta, q_0, F)$ --- ДКА c $n$ состояниями. Состояния $q_1, q_2$ неразличимы $\Leftrightarrow$ они $(n-2)$ неразличимы.
\end{mylemma}
\begin{myproof}
\textit{Необходимость} условия тривиальна. Для того, чтобы была возможность различения состояний, в конечном автомате должны быть минимум одно обычное и одно конечное состояние. Количество остальных состояний не превышает $n-2$.

\textit{Достаточность} тривиальна в тех случаях, когда множество $F$ содержит $0$ или $n$ элементов. Рассмотрим случай, когда число элементов множества $F$ отличается от $0$ или $n$.

Покажем, что
\[ \sim \subseteq \sim^{n-2} \subseteq \sim^{n-3} \subseteq \ldots \subseteq \sim^{2} \subseteq \sim^{1}  \subseteq \sim^{0} \]
Заметим, что для любых состояний $q_1, q_2$ выполняются следующие свойства:
\begin{enumerate}
\item $q_1 \sim^0 q_2 \Leftrightarrow q_1, q_2 \in F$ или $q_1, q_2 \notin F$.
\item $q_1 \sim^k q_2 \Leftrightarrow q_1 \sim^{k-1} q_2$ и $ \forall x \in \Sigma \delta(q_1, x) \sim^{k-1} \delta(q_2, x)$.
\end{enumerate}
Отношение $\sim^0$ самое грубое. Оно разбивает множество $Q$ на два класса: $F$ и $Q \setminus F$. Если $\sim^{k+1} \neq \sim^k$, то отношение $\sim^{k+1}$ содержит по крайней мере на один класс эквивалентности больше, чем $\sim^k$, т. е. оно тоньше. Поскольку каждое множество из $F$ и $Q \setminus F$ содержит не более $n-1$ элементов, можно получить не более $n-2$ последовательных утончений отношения $\sim^0$. Если $\sim^{k+1} = \sim^{k}$ для некоторого $k$, то в силу свойства $2$ $\sim^{k+1} = \sim^{k+2} = \ldots$. Таким образом, $\sim$ --- это первое из отношений $\sim^{k}$, для которых $\sim^{k+1} = \sim^{k}$.
\end{myproof}
\textit{Вывод}: если два состояния можно различить, то их можно различить с помощью входной цепочки, длина которой меньше числа состояний конечного автомата. Таким образом процесс различения любой пары состояний конечен.

\subsection*{Построение минимального конечного автомата}
Для любого конечного автомата можно найти эквивалентный ему минимальный конечный автомат. Для этого нужно убрать из исходного автомата недостижимые и неразличимые состояния. Поскольку неразличимые состояния не участвуют в распознавании цепочек, а пара неразличимых состояний не влияет на результат распознавания, то удаление этих состояний не приведёт к изменению распознаваемого языка.

\textbf{\textit{Определение:}} Полностью определённый детерминированный конечный автомат называется каноническим (приведённым), если он не содержит недостижимых состояний и любая пара состояний этого автомата различима.

\Algo{Построение канонического автомата (Минимизация ДКА)}
{
	Полностью определённый ДКА $M = (Q,\Sigma, \delta, q_0, F)$.
}
{
	Приведённый ДКА $M' = (Q',\Sigma, \delta', q_0, F')$, такой что $L(M') = L(M)$.
}
{
	Поиск и удаление недостижимых и неразличимых состояний.
}
{
\item Применить к конечному автомату $M$ алгоритм поиска недостижимых состояний и построить конечный автомат $M_1$ без недостижимых состояний, такой что $L(M_1) = L(M)$.
\item Строить отношения эквивалентности $\sim^0, \sim^1, \ldots $ по описанию в лемме $4.6.3$ до тех пор, пока это будет возможно, т. е. $\sim^{k+1} = \sim^{k}$. Взять в качестве $\sim$ отношение $\sim^k$.
\item Построить множество $Q'$ как множество классов эквивалентности отношения $\sim$. Через $[p]$ будем обозначать класс эквивалентности отношения $\sim$, содержащий состояние $p$.
\item Построить $\delta'([p], a) = [q]$, если $delta(p,a) = q$.
\item Обозначить $q'_0$ как $q_0$.
\item Обозначить $F'$ как $\{ [q], q \in F \}$.
\item Вернуть  $M' = (Q',\Sigma, \delta', q_0, F')$.
}

\begin{mytheorem}
Автомат $M'$, который строится алгоритмом $4.6.2$, содержит наименьшее число состояний среди всех эквивалентных ему конечных автоматов.
\end{mytheorem}
\begin{myproof}
Пусть $M' = (Q',\Sigma, \delta', q_0, F')$ --- приведённый конечный автомат. Предположим, что существует такой КА $M_m = (Q_m,\Sigma, \delta_m, q_{0_m}, F_m)$, что $\backslash Q_m \backslash < \backslash Q' \backslash$ и $L(M_m) = L(M')$.

В силу шага 1 алгоритма все состояния автомата $M'$ достижимы. Так как $M_m$ имеет меньше состояний, то найдутся цепочки $\omega, x$, переводящие состояние $q_0$ в разные состояния, а  $q_{0_m}$ --- в одно и то же: $(q_{0_m}, \omega) \vdash_{M_m}^* (q, \eps)$ и $(q_{0_m}, x) \vdash_{M_m}^* (q, \eps)$. Следовательно, $\omega, x$ переводят автомат $M'$  в различимые состояния, например в $p, r$. Следовательно, существует такая цепочка $y$, что точно одна из цепочек $\omega y, xy$ принадлежит $L(M')$. Но  $\omega y, xy$ должны переводить $M_m$ в одно и то же состояние $s$, для которого $(q, y) \vdash_{M_m}^* (s, \eps)$. Таким образом, точно одна из цепочек $\omega y, xy$ не может принадлежать $L(M_m)$, а это противречит предположению о том, что $L(M_m) = L(M')$.
\end{myproof}
\begin{table}
\centering
\begin{tabular}{cccc}
\toprule
%
\multicolumn{2}{c}{\multirow{2}{*}{\Large $\delta$}}
	& \multicolumn{2}{c}{\text{Вход}} \\
%
\cmidrule(lr){3-4}
%
\multicolumn{2}{c}{}
	& a  & b                          \\
%
\midrule
%
\multirow{6}{*}{\text{Состояние}}%
    &  $\boxed{{}\to A}$ & $\{F\}$ & $\{B\}$		  \\
    &  $B$ & $\{E\}$ & $\{D\}$		  \\
    &  $C$ & $\{C\}$ & $\{F\}$		  \\
    &  $D$ & $\{D\}$ & $\{A\}$		  \\
    &  $E$ & $\{B\}$ & $\{C\}$		  \\
    &  $\boxed{F}$ & $\{F\}$ & $\{E\}$		  \\
\bottomrule
\end{tabular}
\caption{Функция перехода $\delta$ для автомата из примера 4.6.2.}
\label{tab7}
\end{table}

\begin{figure}
\centering
\begin{tabular}{cccc}
\toprule
%
\multicolumn{2}{c}{\multirow{2}{*}{\Large $\delta$}}
	& \multicolumn{2}{c}{\text{Вход}} \\
%
\cmidrule(lr){3-4}
%
\multicolumn{2}{c}{}
	& a  & b                          \\
%
\midrule
%
\multirow{3}{*}{\text{Состояние}}%
    &  ${}\to X$ & $\{X\}$ & $\{Y\}$		  \\
    &  $Y$ & $\{Y\}$ & $\{Z\}$		  \\
    &  $\boxed{Z}$ & $\{Z\}$ & $\{X\}$		  \\
\bottomrule
\end{tabular}
\caption{Функция перехода $\delta$ для минимального автомата из примера 4.6.2.}
\label{tab8}
\end{figure}

\begin{myexample}
Пусть $M=(\{A;B;C;D;E;F\},\{a;b\},\delta,A,\{A;F\})$ --- детерминированный полностью определённый конечный автомат, где функция переходов $\delta$ задаётся таблицей~\ref{tab7}. Применим к автомату $M$ алгоритм $4.6.2$ минимизации ДКА и построим приведённый автомат $M'$, эквивалентный исходному.

Вначале убедимся, что все состояния автомата $M$ достижимы. Для этого проследим переходы из начального автомата во все остальные: из начального состояния $A$ достижимы состояния $F, B$. Из пары $F, B$ можно перейти в состояния $F, E, D$. Из состояний $E, D$ возможны переходы в состояния $B, C, D, A$. Таким образом, все состояния автомата $M$ достижимы.

Начнём разбиение множества $Q$ на классы неразличимости:
\begin{itemize}
\item \textit{Неразличимость цепочками длины $0 (\sim^0)$.} На этом этапе множество состояний разбивается на два класса финальных и обычных состояний. Чтобы отличить финальное состояние от нефинального, цепочка не требуется. В результате получим два непересекающихся подмножества множества $Q$: $[A;F], [B;C;D;E]$.
\item \textit{Неразличимость цепочками длины $1 (\sim^1)$.} Здесь будем работать с каждым классом, выделенным на предыдущем шаге. Неразличимость цепочками длины $1$ означает, что переход по одной букве из проверяемой пары состояний приводит в одинаковые по типу состояния. Проверим пару состояний $[A;F]$. Из состояния $A$ по букве $a$ попадаем в финальное состояние $F$, по букве $b$ --- в обычное состояние $B$. Из состояния $F$ по букве $a$ попадаем в финальное состояние $F$, по букве $b$ --- в обычное состояние $E$. Таким образом, переход по одной букве (любой из алфавита $\Sigma$ не различает состояния, множество неразличимый состояний стабилизировалось. Мы нашли первую группу неразличимых состояний $[A;F]$. Проверим пары состояний из множества $[B;C;D;E]$. Для установления неразличимости цепочками длины 1 достаточно посмотреть на правую часть таблицы переходов. Если в паре строк для проверяемых состояний справа финальные и нефинальные состояния расположены одинаково, то эти состояния находятся в классе неразличимости $\sim^1$. В данном примере будут выделены следующие классы неразличимости цепочками длины $1$: $[B;E]$ и $[C;D]$.
\item \textit{Неразличимость цепочками длины $0 (\sim^2)$.} На этом шаге нам осталось проверить, различимы ли пары  $[B;E]$ и $[C;D]$ цепочками терминалов длины $2$. Для этого из каждой пары состояний нужно сделать два перехода по всем комбинациям цепочек длины $2$ и сравнить пары состояний, в которых автомат остановился. Проверим пару состояний  $[B;E]$:
\begin{tabular}{rlll}
		 \midrule
		 $$ & $\sim^1$ & $$ & $\sim^2$ \\
     \midrule
     $B$ & $\{E,D\}$ & $E$ & $\{B,C\}$ \\
     $E$ & $\{B,C\}$ & $D$ & $\{D,\textbf{A}\}$ \\
     $$ & $$        & $B$ & $\{E,D\}$ \\
     $$  & $$        & $C$ & $\{C,\textbf{F}\}$ \\
     \bottomrule
    \end{tabular}

Жирным в таблице выделены финальные состояния. Заметим, что расположение финальных и обычных состояний в таблице одинаково, следовательно рассматриваемая пара состояний неразличима цепочками длины $2$.

Проверим пару состояний  $[C;D]$:
\begin{tabular}{rlll}
		 \midrule
		 $$ & $\sim^1$ & $$ & $\sim^2$ \\
     \midrule
     $C$ & $\{C,\textbf{F}\}$ & $C$ & $\{C,\textbf{F}\}$ \\
     $D$ & $\{D,\textbf{A}\}$ & $F$ & $\{\textbf{F},E\}$ \\
     $$ & $$        & $D$ & $\{D,\textbf{A}\}$ \\
     $$  & $$        & $A$ & $\{\textbf{F},B\}$ \\
     \bottomrule
    \end{tabular}

В получившейся таблице расположение финальных и нефинальных состояний одинаково, следовательно пара $[C;D]$ неразличима цепочками длины $2$.

Поскольку на этом шаге нам не удалось уточнить предыдущие классы неразличимости (множества неразличимых состояний стабилизировались), процесс выделения классов неразличимости можно завершить.

В результате выделены три класса неразличимости: $X = [A;F]$, $Y = [B;E]$ и $Z = [C;D]$. Минимальный автомат $M' = (\{X;Y;Z\},\{a;b\},\delta',X,\{X\})$ задаётся таблицей переходов~\ref{tab8}.
\end{itemize}
\end{myexample}

В примере $4.6.2$ показан принцип разбиения множества состояний конечного автомата на классы неразличимости. Для простых случаев вести <<дневник>> переходов достаточно просто, но если нужно установить неразличимость цепочками длины больше $3$, то запись переходов будет очень громоздкой. В \cite {Hop} изложена методика поиска неразличимых состояний путём заполнения таблицы. В данном случае громоздкий <<дневник>> переходов представляется в компактной форме таблицы неразличимости.
\begin{myexample}
Еще раз рассмотрим автомат из примера $4.6.2$. Пусть $M=(\{A;B;C;D;E;F\},\{a;b\},\delta,A,\{A;F\})$ --- детерминированный полностью определённый конечный автомат, где функция переходов $\delta$ задаётся таблицей~\ref{tab7}. В примере $4.6.2$ установлено, что этот автомат не содержит недостижимых состояний. Подготовим для множества $Q$ таблицу неразличимости следующим образом:
\\
 \begin{center}
\begin{tabular}{llllll}
\cline{2-2}
\multicolumn{1}{l|}{B} & \multicolumn{1}{l|}{} &                       &                       &                       &                       \\ \cline{2-3}
\multicolumn{1}{l|}{C} & \multicolumn{1}{l|}{} & \multicolumn{1}{l|}{} &                       &                       &                       \\ \cline{2-4}
\multicolumn{1}{l|}{D} & \multicolumn{1}{l|}{} & \multicolumn{1}{l|}{} & \multicolumn{1}{l|}{} &                       &                       \\ \cline{2-5}
\multicolumn{1}{l|}{E} & \multicolumn{1}{l|}{} & \multicolumn{1}{l|}{} & \multicolumn{1}{l|}{} & \multicolumn{1}{l|}{} &                       \\ \cline{2-6}
\multicolumn{1}{l|}{\textbf{F}} & \multicolumn{1}{l|}{} & \multicolumn{1}{l|}{} & \multicolumn{1}{l|}{} & \multicolumn{1}{l|}{} & \multicolumn{1}{l|}{} \\ \cline{2-6}
                       & \textbf{A}                     & B                     & C                     & D                     & E
\end{tabular}
 \end{center}
Принцип заполнения таблицы следующий: в ячейку ставим $X$, если пара состояний, соответствующая этой ячейке, различима. На первом шаге расставим $X$ в ячейках, соответсвующих парам финальных и обычных состояний (установим неразличимость состояний цепочками длины $0$):
\\
 \begin{center}
\begin{tabular}{llllll}
\cline{2-2}
\multicolumn{1}{l|}{B}          & \multicolumn{1}{l|}{X} &                        &                        &                        &                        \\ \cline{2-3}
\multicolumn{1}{l|}{C}          & \multicolumn{1}{l|}{X} & \multicolumn{1}{l|}{}  &                        &                        &                        \\ \cline{2-4}
\multicolumn{1}{l|}{D}          & \multicolumn{1}{l|}{X} & \multicolumn{1}{l|}{}  & \multicolumn{1}{l|}{}  &                        &                        \\ \cline{2-5}
\multicolumn{1}{l|}{E}          & \multicolumn{1}{l|}{X} & \multicolumn{1}{l|}{}  & \multicolumn{1}{l|}{}  & \multicolumn{1}{l|}{}  &                        \\ \cline{2-6}
\multicolumn{1}{l|}{\textbf{F}} & \multicolumn{1}{l|}{}  & \multicolumn{1}{l|}{X} & \multicolumn{1}{l|}{X} & \multicolumn{1}{l|}{X} & \multicolumn{1}{l|}{X} \\ \cline{2-6}
                                & \textbf{A}             & B                      & C                      & D                      & E
\end{tabular}
\end{center}
Далее по таблице переходов определяем пары неразличимых состояний цепочками длины $1$. Для этого нам достаточно сравнить строки справа и посмотреть расположение финальных и нефинальных состояний. Получим следующую таблицу неразличимости:
\\
\begin{center}
\begin{tabular}{llllll}
\cline{2-2}
\multicolumn{1}{l|}{B}          & \multicolumn{1}{l|}{X} &                        &                        &                        &                        \\ \cline{2-3}
\multicolumn{1}{l|}{C}          & \multicolumn{1}{l|}{X} & \multicolumn{1}{l|}{X} &                        &                        &                        \\ \cline{2-4}
\multicolumn{1}{l|}{D}          & \multicolumn{1}{l|}{X} & \multicolumn{1}{l|}{X} & \multicolumn{1}{l|}{}  &                        &                        \\ \cline{2-5}
\multicolumn{1}{l|}{E}          & \multicolumn{1}{l|}{X} & \multicolumn{1}{l|}{}  & \multicolumn{1}{l|}{X} & \multicolumn{1}{l|}{X} &                        \\ \cline{2-6}
\multicolumn{1}{l|}{\textbf{F}} & \multicolumn{1}{l|}{}  & \multicolumn{1}{l|}{X} & \multicolumn{1}{l|}{X} & \multicolumn{1}{l|}{X} & \multicolumn{1}{l|}{X} \\ \cline{2-6}
                                & \textbf{A}             & B                      & C                      & D                      & E
\end{tabular}
\end{center}
Незаполненными остались ячейки на пересечении пар состояний $[F;A]$, $[E;B]$, $[D;C]$. Рассмотрим пару $[F;A]$. По таблице переходов из состояния $F$ можно попасть в состояния $F, E$ по одной букве, а из состояния $A$ --- в состояния $F, B$. В результате получается, что буква $a$ не различает состояния $[F;A]$, поскольку переводит автомат в одно и то же состояние $F$. Буква $b$ переводит автомат из состояний $F, A$ в состояния $B, E$ соответственно. В таблице неразличимости на пересечении этих состояний пока нет отметки о неразличимости. Можем условно поставить в ячейку на пересечении состояний $[F;A]$ знак вопроса и перейти к исследованию пары состояний $[E;B]$. Для этой пары состояний буква $a$ переводит автомат в те же состояния (перекрёстно), а буква $b$ переводит автомат в состояния $D, C$. Для этой пары в таблице нет отметки о неразличимости, поэтому ставим знак вопроса и переходим к анализу пары состояний  $[D;C]$. Буква $a$ переводит автомат в ту же пару состояний $C, D$, а $b$ --- в пару состояний $F, A$, для которой мы уже поставили в таблицу знак вопроса. В итоге круг замкунлся, и мы выделили пары неразличимых состояний:
 \\
\begin{center}
\begin{tabular}{llllll}
\cline{2-2}
\multicolumn{1}{l|}{B}          & \multicolumn{1}{l|}{X} &                        &                        &                        &                        \\ \cline{2-3}
\multicolumn{1}{l|}{C}          & \multicolumn{1}{l|}{X} & \multicolumn{1}{l|}{X} &                        &                        &                        \\ \cline{2-4}
\multicolumn{1}{l|}{D}          & \multicolumn{1}{l|}{X} & \multicolumn{1}{l|}{X} & \multicolumn{1}{l|}{O} &                        &                        \\ \cline{2-5}
\multicolumn{1}{l|}{E}          & \multicolumn{1}{l|}{X} & \multicolumn{1}{l|}{O} & \multicolumn{1}{l|}{X} & \multicolumn{1}{l|}{X} &                        \\ \cline{2-6}
\multicolumn{1}{l|}{\textbf{F}} & \multicolumn{1}{l|}{O} & \multicolumn{1}{l|}{X} & \multicolumn{1}{l|}{X} & \multicolumn{1}{l|}{X} & \multicolumn{1}{l|}{X} \\ \cline{2-6}
                                & \textbf{A}             & B                      & C                      & D                      & E
\end{tabular}
\end{center}
Далее можно переобозначить выделенные классы неразличимости: $X = [A;F]$, $Y = [B;E]$ и $Z = [C;D]$. Минимальный автомат $M'$, эквивалентный исходному автомату $M$, уже построен в примере $4.6.2$.
\end{myexample}


\section{Упражнения}
\label{Chapter4Exs}

\subsection*{Построение $\eps$"/НКА по регулярному выражению}

Построить $\eps$"/НКА по следующим регулярным выражениям:
\begin{enumerate}
	\item $R = (a+b+c)(bab)*(a+b)$;
  \item $R = 1(10+10)^*1(0+1)^*$;
  \item $R = (a^*b^*)^*+(ab+b)^*$.
\end{enumerate}
Для каждого полученного $\eps$"/НКА построить соответствующий ему ДКА. 
\subsection*{Построение $\eps$"/НКА по ПЛ"/грамматике}
Построить $\eps$"/НКА по грамматикам со стартовым символом $S$ и продукциями:
\begin{align*}
    \text{(1) }&
        \begin{aligned}%{l}
            S &\to 10A \mid 101A,\\
            A &\to 0A \mid 1B,\\
            B &\to 1 \mid 0;
        \end{aligned}
        \qquad\qquad
    &
    \text{(2) }&
        \begin{aligned}%{l}
            S &\to abaA \mid abB,\\
            A &\to a \mid aaB \mid baC,\\
            B &\to b \mid abaC,\\
            C &\to aaB \mid a;
		        \end{aligned}
        \qquad\qquad
    &
    \text{(3) }&
        \begin{aligned}%{l}
            S &\to xA \mid B,\\
            A &\to yxyA \mid \eps \mid xB,\\
            B &\to b \mid \eps.
        \end{aligned}
\end{align*}
Методом исключения состояний определить язык для каждого полученного автомата.
\subsection*{Минимизация конечного автомата}
Минимизировать конечные автоматы, заданные таблицами переходов:
\begin{multicols}{2}
\begin{enumerate}
  \item
     \begin{tabular}{rlll}
     \toprule
     \multirow{2}{*}{\Large $\delta$}
      & \multicolumn{3}{c}{\text{Вход}} \\
     \cmidrule(rl){2-4}
        & \multicolumn{1}{c}{a}
        & \multicolumn{1}{c}{b}
        &\multicolumn{1}{c}{$\eps$}\\
     \midrule
     ${}\to A$ & $\{B; C\}$ & $\{C\}$  & $\{D\}$\\
     $B$ & $\{C\}$ & $\{D\}$ &  \\
     $C$ & $\{B; C\}$ & $\{D\}$ &  \\
     $\boxed{D}$ & $\emptyset$ & $\emptyset$ &  \\
     \bottomrule
    \end{tabular}
		\qquad\qquad
  \item
     \begin{tabular}{rll}
     \toprule
     \multirow{2}{*}{\Large $\delta$}
      & \multicolumn{2}{c}{\text{Вход}} \\
     \cmidrule(rl){2-3}
        & \multicolumn{1}{c}{a}
        &\multicolumn{1}{c}{b}\\
     \midrule
     ${}\to q_1$ & $q_2$ & $q_4$\\
     $q_2$ & $q_2$ & $q_3$\\
		 \boxed{q_3} & $q_4$ & $q_5$\\
     $q_4$ & $q_4$ & $q_5$\\
     \boxed{q_5} & $q_2$ & $q_3$\\
     \bottomrule
    \end{tabular}
\end{enumerate}
\end{multicols}
Методом исключения состояний определить язык полученных автоматов.
%\subsection*{Удаление бесполезных символов}
%
%Удалить бесполезные символы в грамматиках с продукциями:
%\begin{align*}
    %\text{(1) }&
        %\begin{aligned}%{l}
            %S &\to 0 \mid A,\\
            %A &\to AB,\\
            %B &\to 1;
        %\end{aligned}
        %\qquad\qquad
    %&
    %\text{(2) }&
        %\begin{aligned}%{l}
            %S &\to AB \mid CA,\\
            %A &\to a,\\
            %B &\to BC \mid AB,\\
            %C &\to aB \mid \varepsilon.
        %\end{aligned}
%\end{align*}

\chapter{Булева алгебра регулярных языков}
\label{Chapter5}
\section{Свойства регулярных языков}
\label{Chapter5PropsReg}
Регулярные (а значит, и автоматные, и ПЛ-) языки обладают большим количеством интересных свойств, многие из которых легко доказываются на основе соответствия между упомянутыми классами языков.

Второе свойство регулярных языков, которое мы рассмотрим, назовем свойством замкнутости относительно теоретико"/множественных операций. Иначе, можно сказать, что регулярные множества представляют собой подалгебру алгебры всех подмножеств $\Sigma^*$.

\begin{mytheorem}Класс регулярных языков замкнут относительно операций: объединения, конкатенации, итерации, дополнения, пересечения, разности.
\end{mytheorem}

\begin{myproof}Для регулярных языков $L_1$, $L_2$ рассмотрим описывающие
их регулярные выражения $\RE(L_1)$, $\RE(L_2)$. Язык объединения
(соответственно, конкатенации, итерации) описывается выражением $\RE(L_1) +
\RE(L_2)$ (соответственно, $\RE(L_1)\RE(L_2)$, $\RE(L_1)^\ast$), а значит,
регулярен.

Для регулярного языка $L$ построим распознающий его
детерминированный конечный автомат $\mathcal A(L)$, задаваемый пятеркой $(Q, \Sigma, \delta, q_0, F)$. Легко
проверить, что автомат $\widetilde{\mathcal A}(L) = (Q, \Sigma, \delta, q_0, Q
\setminus F)$ допускает дополнение $\overline L = \Sigma^\ast \setminus L$ языка
$L$, которое является, таким образом, регулярным языком.

Поскольку справедливо $L_1 \cap L_2 = \overline{\overline L_1 \cup \overline
L_2}$, то язык $L_1 \cap L_2$ регулярен, исходя из доказанного выше.
Аналогично регулярность операции взятия разности языков выводится из
равенства $L_1 \setminus L_2 = L_1 \cap \overline{L_2}$.
\end{myproof}

\begin{myproblem}
Задача построения конечного автомата, распознающего дополнение
языка, была сведена к взятию дополнения множества заключительных
состояний детерминированного автомата исходного языка. На самом деле, этот факт
требует отдельного доказательства, которое читателю предлагается
построить самостоятельно. Неожиданным является то, что для
недетерминированного автомата исходного языка
такая конструкция в общем случае не
работает. Приведите контрпример, иллюстрирующий эту проблему.
\end{myproblem}

\begin{myproblem}
Доказательство регулярности $L_1 \cap L_2$ можно провести
конструктивно, а не формально, как это было сделано с помощью
определения операции пересечения через дополнение и разность.
А именно, имеется явная схема для ДКА, допускающего $L_1 \cap L_2$,
внутри которой «параллельно» работают автоматы $\mathcal A(L_1)$
и $\mathcal A(L_2)$ для языков $L_1$ и $L_2$. Попытайтесь придумать эту
схему или разберите теорему~4.8 раздела~4.2.1 учебника~\cite{Hop}.
\end{myproblem}

\begin{myproblem}[контрпример к достаточности леммы о накачке]
Необходимо понимать, что условие леммы о накачке не является
достаточным для регулярности языка. То есть, если
некоторый язык ему удовлетворяет, из этого не следует
регулярность языка. Рассмотрим язык
\[L = \{a^i b^j c^k \mid i, j, k \in \N_0 \wedge (i=1 \Rightarrow j = k)\}.\]
Покажите, что
\begin{enumerate}
    \item $L$ нерегулярен (с помощью свойства замкнутости),
    \item $L$ удовлетворяет условию леммы о накачке.
\end{enumerate}
\end{myproblem}

Пусть $\Sigma$ --- конечный алфавит, $\Sigma^*$ --- полный язык над алфавитом $\Sigma$. Выделим регулярные подмножества $L_1, L_2 \subseteq \Sigma^*$. Если $L_1, L_2$ --- регулярные языки над алфавитом $\Sigma$, то регулярными являются следующие языки:
\begin{enumerate}
\item $L_1 \cup L_2$
\item $L_1 L_2$
\item $L_1^*$
\item $L_1^R = \{ \omega^R \mid \omega \in L_1 \}$
\item $\bar L_1 = \Sigma^* \setminus L_1$
\item $L_1 \cap L_2$
\item $L_1 \setminus L_2$
\item $H(L_1)$
\item $H^{-1}(L_1)$
\end{enumerate}

Язык регулярен, если можно построить конечный автомат, распознающий этот язык. Пункты $1, 2, 3$ подробно рассматривались и доказывались в разделе $3.4$. Построим конечные автоматы для языков из пунктов $4, 5, 6, 7$.

Рассмотрим язык $L_1^R = \{ \omega^R \mid \omega \in L_1 \}$ (пункт $4$). Поскольку язык $L_1$ --- регулярный, для него существует конечный автомат. Пусть $M_1 = (Q,\Sigma, \delta, q_0, F)$ --- конечный автомат, у которого $L(M_1) = L_1$. Построим конечный автомат $M^R = (Q_R, \Sigma, \delta^R, F^R)$ следующим образом:
\begin{enumerate}
\item Если $\mid F \mid > 1$, то ввести новое финальное состояние $q_f$. Из каждого старого финального состояния $p \in F$ ввести $\eps$-переход в новое финальное состояние. Положить $F = \{ q_f \}$.
\item Обратить все стрелки диаграммы переходов автомата $M$.
\item Положить $q_R = q_f$, $F^R = \{ q_0 \}$.
\end{enumerate}
Построенный таким образом автомат $M^R$ начинает распознавать цепочку с последней буквы и входит в допускающее состояние после прочтения первой буквы.

Проверим, что результат обращения регулярного языка в алфавите $\Sigma$ также является регулярным языком:
\begin{enumerate}
\item $\{ \es \}^R = \{ \es \} $, $\{ \eps \}^R = \{ \eps \} $, $\{ a \}^R = \{ a \} \forall a \in \Sigma $.
\item $(L_1 \cup L_2)^R = (L_{1}^R \cup L_{2}^R)$.
\item $(L_1 L_2)^R = (L_{1}^R L_{2}^R)$.
\item $(L_{1}^*)^R = \{ \eps \}^R \cup (L_1)^R \cup (L_2)^R \ldots (L_{1}^R)^*$.
\end{enumerate}

\begin{figure}
\centering
\begin{tikzpicture}[node distance=3cm,>=stealth',auto,every state/.style={thick}]
	\node (init) {};
	\node[state] (q_0) [right=.7cm of init] {$q_0$};
    \node[state, accepting] (q_1) [right of=q_0] {$q_1$};
	\node[state, accepting] (q_2) [right of=q_1] {$q_2$};
    
	\path[->]
    	(init) edge (q_0)
        (q_0) edge node {$0$} (q_1)
        (q_1) edge [loop above] node {$1$} (q_1)
        (q_2) edge [loop above] node {$0$} (q_2)
		(q_1) edge node {$0$} (q_2);
\end{tikzpicture}
\caption{ДКА из примера~\ref{ex-511}, распознающий язык $L = 01^*0^*$}
\label{aut-graph-12}
\end{figure}

\begin{figure}
\centering
\begin{tikzpicture}[node distance=3cm,>=stealth',auto,every state/.style={thick}]
	\node (init) {};
	\node[state] (q_0) [right=.7cm of init] {$q_0$};
    \node[state, accepting] (q_1) [right of=q_0] {$q_1$};
	\node[state, accepting] (q_2) [right of=q_1] {$q_2$};
    
	\path[->]
    	(init) edge (q_0)
        (q_0) edge node {$0$} (q_1)
        (q_1) edge [loop above] node {$1$} (q_1)
        (q_2) edge [loop above] node {$0$} (q_2)
		(q_1) edge node {$0$} (q_2);
\end{tikzpicture}
\caption{ДКА из примера~\ref{ex-511}, распознающий язык $L = 01^*0^*$}
\label{aut-graph-12}
\end{figure}

\begin{myexample}
Пусть $L = 01^*0^*$. Диаграмма автомата, распознающего язык $L$, приведена на рисунке~\ref{aut-graph-12}. В автомате два допускающих состояния $q_1, q_2$. Сведём эти состояния в одно новое допускающее состояние $q_f$, добавив $\eps$-переходы. Диаграмма конечного автомата, допускающего язык $L^R = 0^*1^*0$ представлена на рисунке~\ref{aut-graph-13}.
\end{myexample}

Рассмотрим язык $\bar L_1$ (пункт $5$). Поскольку язык $L_1$ --- регулярный, для него существует конечный автомат. Пусть $M_1 = (Q,\Sigma, \delta, q_0, F)$ --- полностью определённый ДКА, у которого $L(M_1) = L_1$. Рассмотрим конечный автомат $\bar M_1 = (Q,\Sigma, \delta, q_0, Q \setminus F)$. Автомат $\bar M_1$ определяется такой же $\delta$-функцией, что и $M_1$, т. е. их поведение одинаково. Возьмём произвольную цепочку $\omega \in L_1$. Для неё справедливо следующее:
\[ (q_0, \omega) \vdash_{M_1}^* (q_f, \eps), q_f \in F \]
В то же время в автомате $\bar M_1$ для этой цепочки выполняется следующее:
\[ (q_0, \omega) \vdash_{\bar M_1}^* (q_f, \eps), q_f \in (Q \setminus F) \]
Отсюда следует, что если произвольная цепочка допускается в $M_1$, то в автомате $\bar M_1$ эта цепочка приводит к недопускающему состоянию. Таким образом $L(\bar M_1) = \bar L_1$ --- регулярный язык.
\begin{figure}
\centering
\begin{tikzpicture}[node distance=3cm,>=stealth',auto,every state/.style={thick}]
	\node (init) {};
	\node[state] (q_0) [right=.7cm of init] {$q_0$};
    \node[state] (q_1) [right of=q_0] {$q_1$};
	\node[state, accepting] (q_f) [right of=q_1] {$q_f$};
    
	\path[->]
    	(init) edge (q_0)
        (q_0) edge node {$0$} (q_1)
        (q_1) edge [loop above] node {$1$} (q_1)
		(q_1) edge node {$0$} (q_f);
\end{tikzpicture}
\caption{ДКА из примера 5.1.2, распознающий язык $L = 01^*0$}
\label{aut-graph-14}
\end{figure}
\begin{myexample}
Пусть $L = 01^*0$. Диаграмма автомата, распознающего язык $L$, приведена на рисунке~\ref{aut-graph-14}. Представленный ДКА не полностью определён. Идея инвертирования конечного автомата заключается в том, чтобы все недопускающие состояния стали допускающими и наоборот, а в неполностью определенных ДКА дан не весь набор состояний, в который автомат может перейти. Иными словами, в полностью определённых конечных автоматах заложено описание как распознаваемого языка, так и дополнения к нему, а в неполностью определённых --- только описание распознаваемого языка.

Доопределим ДКА на рисунке~\ref{aut-graph-14} следующим образом: введём новое недопускающее состояние $q_E$ и добавим недостающие переходы в это состояние. Получившийся полностью определённый ДКА представлен на рисунке~\ref{aut-graph-15}. Теперь инвертируем состояния и получим ДКА (рисунок~\ref{aut-graph-14}), распознающий язык $\bar L = \eps + 1(0 + 1)^* + 01^*(\eps + 0(0+1)(0+1)^*)$
\end{myexample}
\begin{figure}
\centering
\begin{tikzpicture}[node distance=3cm,>=stealth',auto,every state/.style={thick}]
	\node (init) {};
	\node[state] (q_0) [right=.7cm of init] {$q_0$};
    \node[state] (q_1) [right of=q_0] {$q_1$};
	\node[state, accepting] (q_f) [right of=q_1] {$q_f$};
    \node[state] (q_E) [below of=q_1] {$q_E$};
    
	\path[->]
    	(init) edge (q_0)
        (q_0) edge node {$0$} (q_1)
        (q_0) edge node {$1$} (q_E)
        (q_1) edge [loop above] node {$1$} (q_1)
        (q_1) edge node {$0$} (q_f)
		(q_f) edge node {$0, 1$} (q_E)
		(q_E) edge [loop below] node {$0, 1$} (q_E);
\end{tikzpicture}
\caption{Полностью определённый ДКА из примера 5.1.2, распознающий язык $L = 01^*0$}
\label{aut-graph-15}
\end{figure}
\begin{figure}
\centering
\begin{tikzpicture}[node distance=3cm,>=stealth',auto,every state/.style={thick}]
	\node (init) {};
	\node[state, accepting] (q_0) [right=.7cm of init] {$q_0$};
    \node[state, accepting] (q_1) [right of=q_0] {$q_1$};
	\node[state] (q_f) [right of=q_1] {$q_f$};
    \node[state, accepting] (q_E) [below of=q_1] {$q_E$};
    
	\path[->]
    	(init) edge (q_0)
        (q_0) edge node {$0$} (q_1)
        (q_0) edge node {$1$} (q_E)
        (q_1) edge [loop above] node {$1$} (q_1)
        (q_1) edge node {$0$} (q_f)
		(q_f) edge node {$0, 1$} (q_E)
		(q_E) edge [loop below] node {$0, 1$} (q_E);
\end{tikzpicture}
\caption{ДКА из примера 5.1.2, распознающий язык  $\bar L = \eps + 1(0 + 1)^* + 01^*(\eps + 0(0+1)(0+1)^*)$}
\label{aut-graph-16}
\end{figure}

Рассмотрим язык $L = L_1 \cap L_2$ (пункт $6$). Поскольку языки $L_1, L_2$ --- регулярны, для них существуют конечные автоматы. Пусть $M_1 = (Q_1,\Sigma, \delta_1, q_1, F_1)$ --- ДКА, у которого $L(M_1) = L_1$. $M_2 = (Q_2,\Sigma, \delta_2, q_2, F_2)$ --- ДКА, у которого $L(M_2) = L_2$.

Построим конечный автомат $M = (Q_1 \times Q_2,\Sigma, \delta, (q_1, q_2), F_1 \times F_2)$, у которого
\[ \delta: \forall a \in \Sigma, \forall p \in Q_1, \forall q \in Q_2 \delta((p, q), a) = (\delta_1(p,a), \delta_2(q,a)). \]

Покажем, что построенный автомат распознаёт $L_1 \cap L_2$.

Произвольная цепочка $\omega \in L(M) \Leftrightarrow ((q_1, q_2), \omega) \vdash_{M}^* ((q_{f_1}, q_{f_2}), \eps), q_{f_1} \in F_1, q_{f_2} \in F_2$. Вместе с этим \[(q_1, \omega) \vdash_{M_1}^* (q_{f_1}, \eps),  q_{f_1} \in F_1  \] \[ (q_2, \omega) \vdash_{M_2}^* (q_{f_2}, \eps), q_{f_2} \in F_2\]
Таким образом, цепочка допускается автоматом $M$, если она одновременно допускается и автоматом $M_1$, и автоматом $M_2$.

\begin{myexample}
Пусть $L_1 = a^*bc$ и $L_2 = b^*ca^*$. $M_1 = (\{q_0; q_1; q_f\}, \{ a; b; c \}, \delta_1, q_0, \{ q_f \})$ и $M_2 = (\{p_0; p_f\}, \{ a; b; c \}, \delta_2, p_0, \{ p_f \})$.
\begin{center}
%TODO: поднять таблицы на одну строку
     \begin{tabular}{rlll}
     \toprule
     \multirow{2}{*}{\Large $\delta_1$}
      & \multicolumn{3}{c}{\text{Вход}} \\
    \cmidrule(rll){2-4}
        & \multicolumn{1}{c}{a}
				& \multicolumn{1}{c}{b}
        &\multicolumn{1}{c}{c}\\
     \midrule
     ${}\to q_0$ & $\{q_0\}$ & $\{ q_1 \}$ & $\emptyset$\\
     $q_1$ & $ \emptyset $ & $\emptyset$ & $\{ q_f \}$\\
     $\boxed{q_f}$ & $\emptyset$ & $\emptyset$ & $\emptyset$\\
     \bottomrule
    \end{tabular}
%\vskip
\hspace{4 em}
%\bigskip
     \begin{tabular}{rlll}
     \toprule
     \multirow{2}{*}{\Large $\delta_2$}
      & \multicolumn{3}{c}{\text{Вход}} \\
    \cmidrule(rll){2-4}
        & \multicolumn{1}{c}{a}
				& \multicolumn{1}{c}{b}
        &\multicolumn{1}{c}{c}\\
     \midrule
     ${}\to p_0$ & $\emptyset$ & $\{ p_0 \}$ & $\{ p_f \}$\\
     $\boxed{p_f}$ & $\{ p_f \}$ & $\emptyset$ & $\emptyset$\\
     \bottomrule
    \end{tabular}

\end{center}
Построим новую таблицу функции переходов для автомата, распознающего пересечение исходных языков:
\begin{center}
	     \begin{tabular}{rlll}
     \toprule
     \multirow{2}{*}{\Large $\delta$}
      & \multicolumn{3}{c}{\text{Вход}} \\
    \cmidrule(rll){2-4}
        & \multicolumn{1}{c}{a}
				& \multicolumn{1}{c}{b}
        &\multicolumn{1}{c}{c}\\
     \midrule
     ${}\to (q_0, p_0)$ & $ \{ (q_0, \es) \} $ & $ \{ (q_1, p_0) \} $ & $ \{ (\es, p_1) \} $\\
		 $(q_1, p_0)$ & $ \{ (\es, \es) \} $ & $ \{ (\es, p_0) \} $ & $ \{ (q_f, p_f) \} $\\
     $\boxed{(q_f, p_f)}$ & $\{ (\es, p_1) \}$ & $(\es, \es)$ & $(\es, \es)$\\
     \bottomrule
    \end{tabular}

\end{center}
Таким образом \[ M = (\{ (q_0, p_0); (q_1, p_0); (q_f, p_f)  \}, \Sigma, \delta, (q_0, p_0), \{ (q_f, p_f)  \}. \] Язык $L(M) = \{bc \}.$
\end{myexample}

Рассмотрим язык $L = L_1 \setminus L_2$ (пункт $7$). Поскольку $L_1$ и $L_2$ --- множества (регулярные), то теоретико-множественная операция разности может быть выражена через другие операции следующим образом: $L = L_1 \setminus L_2 = L_1 \cap \bar L_2$. В пунктах $5, 6$ показано, как строить автоматы для операций дополнения и пересечения, так что построение автомата для разности двух регулярных выражений заключается в построении автомата, распознающего дополнение к языку $L_2$, и построении автомата, распознающего пересечение между $L_1$ и дополнением к $L_2$.

\begin{figure}
\centering
\begin{tikzpicture}[node distance=3cm,>=stealth',auto,every state/.style={thick}]
	\node (init) {};
	\node[state, accepting] (q_0) [right=.7cm of init] {$q_0$};
    \node[state] (q_1) [right of=q_0] {$q_1$};
	\node[state] (q_2) [right of=q_1] {$q_2$};
    \node[state, accepting] (q_E) [below of=q_1] {$q_E$};
    
	\path[->]
    	(init) edge (q_0)
        (q_0) edge node {$a$} (q_1)
        (q_0) edge node {$b, c$} (q_E)
        (q_1) edge [loop above] node {$b$} (q_1)
        (q_1) edge node {$c$} (q_2)
        (q_1) edge node {$a$} (q_E)
		(q_2) edge [loop above] node {$c$} (q_2)
		(q_2) edge node {$a, b$} (q_E)
		(q_E) edge [loop below] node {$a, b, c$} (q_E);
\end{tikzpicture}
\caption{ДКА из примера 5.1.4, распознающий язык  $\bar L_2 $}
\label{aut-graph-17}
\end{figure}
\begin{figure}
\centering
\begin{tikzpicture}[node distance=3cm,>=stealth',auto,every state/.style={thick}]
	\node (init) {};
	\node[state] (p_0) [right=.7cm of init] {$p_0$};
    \node[state] (p_1) [right of=p_0] {$p_1$};
	\node[state, accepting] (p_2) [below of=p_1] {$p_2$};
    
	\path[->]
    	(init) edge (p_0)
        (p_0) edge [loop above] node {$a$} (p_0)
        (p_0) edge node {$b$} (p_1)
        (p_0) edge node {$c$} (p_2)
        (p_1) edge [loop above] node {$b$} (p_1)
        (p_1) edge node {$c$} (p_2);
\end{tikzpicture}
\caption{ДКА из примера 5.1.4, распознающий язык  $ L_1 $}
\label{aut-graph-18}
\end{figure}
\begin{myexample}
Пусть $L_1 = a^*b^*c$, $L_2 = ab^*c^*$. Построим конечный автомат $M_{\bar L_2}$, такой что $L(M_{\bar L_2}) = \bar L_2$. Для этого необходимо построить конечный автомат, распознающий язык $L_2$, доопределить его, если это необходимо, и инвертировать состояния. Диаграмма автомата, распознающего язык $\bar L_2$, приведена на рисунке~\ref{aut-graph-17}.

Построим автомат, распознающий язык $L_1$. Поскольку он будет использоваться для построения автомата, распознающего пересечение языков, этот автомат доопределять не будем. Диаграмма автомата, распознающего язык $L_2$, приведена на рисунке~\ref{aut-graph-18}.

Для построения автомата, распознающего $L_1 \cap \bar L_2$, воспользуемся таблицами $\delta$-функций переходов автоматов для языка $L_1$ и $\bar L_2$:

\begin{center}
%TODO: поднять таблицы на одну строку
     \begin{tabular}{rlll}
     \toprule
     \multirow{2}{*}{\Large $\delta_{L_1}$}
      & \multicolumn{3}{c}{\text{Вход}} \\
    \cmidrule(rll){2-4}
        & \multicolumn{1}{c}{a}
				& \multicolumn{1}{c}{b}
        &\multicolumn{1}{c}{c}\\
     \midrule
     ${}\to p_0$ & $\{p_0\}$ & $\{ p_1 \}$ & $\{ p_2 \}$\\
     $p_1$ & $ \emptyset $ & $\{ p_1 \}$ & $\{ p_2 \}$\\
     $\boxed{p_2}$ & $\emptyset$ & $\emptyset$ & $\emptyset$\\
     \bottomrule
    \end{tabular}	
\hspace{4 em}
%\bigskip
     \begin{tabular}{rlll}
     \toprule
     \multirow{2}{*}{\Large $\delta_{\bar L_2}$}
      & \multicolumn{3}{c}{\text{Вход}} \\
    \cmidrule(rll){2-4}
        & \multicolumn{1}{c}{a}
				& \multicolumn{1}{c}{b}
        &\multicolumn{1}{c}{c}\\
     \midrule
     ${}\to q_0$ & ${ q_1 }$ & $\{ q_E \}$ & $\{ q_E \}$\\
		 $q_1$ & ${ q_E }$ & $\{ q_1 \}$ & $\{ q_2 \}$\\
		 $q_2$ & ${ q_E }$ & $\{ q_E \}$ & $\{ q_2 \}$\\
     $\boxed{p_E}$ & $\{ p_E \}$ & $\{ p_E \}$ & $\{ p_E \}$\\
     \bottomrule
    \end{tabular}

\end{center}

На основе построенных таблиц сформируем новую таблицу функции переходов для автомата, распознающего язык $L_1 \cap \bar L_2$:
\begin{center}
	     \begin{tabular}{rlll}
     \toprule
     \multirow{2}{*}{\Large $\delta_{L_1 \cap \bar L_2}$}
      & \multicolumn{3}{c}{\text{Вход}} \\
    \cmidrule(rll){2-4}
        & \multicolumn{1}{c}{a}
				& \multicolumn{1}{c}{b}
        &\multicolumn{1}{c}{c}\\
     \midrule
     ${}\to A = (q_0, p_0)$ & $ \{ (q_1, p_1) \} $ & $ \{ (q_E, p_2) \} $ & $ \{ (q_E, p_3) \} $\\
		 $B = (q_1, p_1)$ & $ \{ (q_E, p_1) \} $ & $ \{ (q_1, p_2) \} $ & $ \{ (q_2, p_3) \} $\\
		 $C = (q_E, p_2)$ & $ \{ (q_E, \es) \} $ & $ \{ (q_E, p_2) \} $ & $ \{ (q_E, p_3) \} $\\
		 $\boxed{D =(q_E, p_3)}$ & $ \{ (q_E, \es) \} $ & $ \{ (q_E, \es) \} $ & $ \{ (q_E, \es) \} $\\
		 $E = (q_E, p_1)$ & $ \{ (q_E, p_1) \} $ & $ \{ (q_E, p_2) \} $ & $ \{ (q_E, p_3) \} $\\
		 $F = (q_1, p_2)$ & $ \{ (q_E, \es) \} $ & $ \{ (q_1, p_2) \} $ & $ \{ (q_2, p_3) \} $\\
		 $G = (q_2, p_3)$ & $ \{ (q_E, \es) \} $ & $ \{ (q_E, \es) \} $ & $ \{ (q_E, \es) \} $\\
     \bottomrule
    \end{tabular}
\end{center}
\begin{figure}
\centering
\begin{tikzpicture}[node distance=3cm,>=stealth',auto,every state/.style={thick}]
	\node (init)  at (0,0) {};
	\node[state] (A) [right=.7cm of init] {$A$};
    \node[state] (B) at (0,-7) {$B$};
    \node[state] (G) at (3,-3) {$G$};
	\node[state, accepting] (D) at (9,0) {$D$};
	\node[state] (F) at (5,-4) {$F$};
	\node[state] (C) at (7,-3) {$C$};
	\node[state] (E) at (9,-7) {$E$};
    
	\path[->]
    	(init) edge (A)
        (A) edge node {$a$} (B)
        (A) edge node {$c$} (D)
        (A) edge node {$b$} (C)
        (B) edge node {$b$} (F)
        (B) edge node {$c$} (G)
        (B) edge node {$a$} (E)
				(C) edge node {$b$} (F)
				(C) edge node {$c$} (D)
				(E) edge [loop below] node {$a$} (E)
				(E) edge node {$c$} (D)
				(E) edge node {$b$} (C)
				(F) edge node {$c$} (G)
        (F) edge [bend left] node {$b$} (C);
\end{tikzpicture}
\caption{ДКА из примера 5.1.4, распознающий язык  $L_1 \cap \bar L_2$}
\label{aut-graph-19}
\end{figure}
Диаграмма автомата $M$, распознающего $L_1 \cap \bar L_2$, представлена на рисунке~\ref{aut-graph-19}.

Вычислив язык автомата $M$ методом исключения состояний, можно убедиться, что $L(M) = c+bc+aaa^*(c+bc)+(ab+bb+aaa^*bb)(bb)^*bc$.
\end{myexample}

\begin{myproblem}
Пусть $L_1, L_2$ --- регулярные языки над алфавитом $\Sigma$. Покажите, что языки $H(L_1)$ и $H^{-1}(L_1)$ являются регулярными.
\end{myproblem}

\section{Замкнутость регулярных множеств относительно базисных операций}
\label{Chapter5Closure}

Рассматривая свойства регулярных языков, мы получили инструкции построения конечных автоматов, распознающих языки --- результаты операций над регулярными множествами. Регулярные множества обладают еще одним важным свойством, которое позволяет определять произвольные операции над языками и строить конечные автоматы для их распознавания.

Напомним, что множество $A$ называется замкнутым относительно $n$-местной операции $\Psi$, если $\Psi(a_1, a_2, \ldots a_n) \in A \forall a_i \in A, 1 \leq i \leq n $. Так, например, множество натуральных чисел замкнуто относительно операции сложения. 

\textbf{\textit{Определение:}} Класс множеств называется булевой алгеброй множеств, если он замкнут относительно дополнения, объединения и пересечения.

\begin{mytheorem}
Пусть $\Sigma$ --- произвольный (необязательно конечный) алфавит. Класс регулярных множеств, содержащихся в $\Sigma^*$, является булевой алгеброй множеств.
\end{mytheorem}
\begin{myproof}
Замкнутость относительно объединения уже доказывалась ранее. Замкнутость относительно пересечения следует из теоретико-множественного закона $\overline{\overline{ A \cap B }} = \overline {\bar A \cup \bar B}$.

Докажем замкнутость относительно дополнения. Пусть $M = (Q, \Delta, \delta, q_0, F)$ --- конечный автомат, у которого $\Delta \subseteq \Sigma$, $\Delta$ --- конечный алфавит. Легко показать, что каждое регулярное множество $L \subseteq \Sigma^* $ допускается некоторым таким автоматом. Тогда конечный автомат $M' = (Q, \Delta, \delta, q_0, Q \setminus F)$ допускает $\Delta^* \setminus L(M)$. При этом автомат $M$ должен быть полностью определён. 

Дополнение $\overline {L(M)}$ относительно $\Sigma^*$ можно представить в виде 
\[
\overline {L(M)} = L(M') \cup \Sigma^*(\Sigma \setminus \Delta)\Sigma^*
\]
Так как множество $\Sigma^*(\Sigma \setminus \Delta)\Sigma^*$ регулярно, то регулярность множества $L(M)$ следует из замкнутости регулярных множеств относительно объединения.
\end{myproof}

Из свойства замкнутости регулярных множеств относительно базисных операций дополнения, объединения и пересечения следует, в частности, что произвольные конечные комбинации этих операций также являются регулярными множествами, и через базисные операции можно вывести любые другие операции над регулярными множествами.

\section{Алгоритмические проблемы регулярных языков}
\label{Chapter5AlgProblems}
Регулярный язык может быть задан конечным образом с помощью трех типов описаний: праволинейной грамматики, регулярного выражения или конечного автомата. Имея одно описание, можно перейти к любому другому с помощью известных алгоритмов.

Для конечных описаний регулярных множеств естественным образом возникают алгоритмические проблемы. Рассмотрим три из них.
\begin{enumerate}
\item \textit{Проблема принадлежности:} Дано конечное описание языка и цепочка $w$. Принадлежит ли цепочка $w$ этому языку?
\item \textit{Проблема пустоты:} Дано конечное описание языка. Пуст ли этот язык?
\item \textit{Проблема эквивалентности:} Даны два конечных описания языка одного типа. Задают ли эти описания один и тот же язык?
\end{enumerate}
Поскольку для каждого типа описания можно построить эквивалентное ему описание другого типа, то достаточно рассмотреть решение этих проблем только для одного типа описания. Будем рассматривать решения обозначенных алгоритмических проблем для конечных автоматов.

\textit{Проблема принадлежности для конечных автоматов} решается с помощью следующего алгоритма. На вход алгоритм получает конечный автомат $M = (Q,\Sigma, \delta, q_0, F)$ и цепочку $w \in \Sigma^*$. На выходе алгоритм возвращает ответ <<ДА>>, если $w \in L(M)$, в противном случае возвращается ответ <<НЕТ>>. Принцип работы алгоритма следующий:\\
Пусть $w = a_1a_2 \ldots a_n$. \\
Найти последовательно состояния $q_1 = \delta(q_0, a_1), q_2 = \delta(q_1, a_2), \ldots , q_n = \delta(q_{n-1}, a_n)$. Если $q_n \in F$, вернуть <<ДА>>. В противном случае вернуть <<НЕТ>>. Иными словами, задача алгоритма --- определить достижимость любого финального состояния из начального по заданной цепочке.

Шаги алгоритма:
\begin{enumerate}
\item  Представить $w = a_1a_2 \ldots a_n$. Если $n = 0$, перейти к Шагу 5.
\item Положить $i = 0$.
\item $q_{i+1} = \delta(q_i, a_{i+1})$.
\item Если $ i \neq n$, $ i = i + 1$ и перейти к Шагу 2.
\item Если $q_i \in F$, вернуть <<ДА>>, иначе вернуть <<НЕТ>>.
\end{enumerate}

\textit{Проблема пустоты языка} конечного автомата $M = (Q,\Sigma, \delta, q_0, F)$ решается путём определения множества состояний, достижимых из $q_0$. Если хотя бы одно финальное состояние принадлежит этому множеству, то алгоритм возвращает ответ <<ДА>>, иначе --- <<НЕТ>>.

Шаги алгоритма:
\begin{enumerate}
\item Положить $Q_0 = { q_0 }, i = 1$.
\item $Q_i = $ {$p \in Q \vdash \forall a \in \Sigma \delta(q, a) = $ {$ p $} $, q \in Q_{i-1},$}.
\item Если $Q_i = Q_{i-1}$, $i = i + 1$ и перейти на Шаг 2.
\item Если $Q_i \cap F \neq \es$, вернуть <<ДА>>, иначе вернуть <<НЕТ>>.
\end{enumerate}

\textit{Проблема эквивалентности для конечных автоматов} может быть решена двумя способами. Первый способ --- это проверка симметрической разности двух исходных языков на непустоту.

Пусть $M_1 = (Q_1,\Sigma_1, \delta_1, q_1, F_1)$ и $M_2 = (Q_2,\Sigma_2, \delta_2, q_2, F_2)$ конечные автоматы, такие что $L_1 = L(M_1)$ и $L_2 = L(M_2)$. Симметрическая разность двух языков может быть построена через теоретико-логические операции:
\[ L_1 \triangle L_2 = (L_1 \setminus L_2) \cup(L_2 \setminus L_1) = (L_1 \cap \bar L_2) \cup (L_2 \cap \bar L_1). \]
В разделе $4.7$ показано, как построить конечные автоматы для результатов операций над регулярными языками. Далее с помощью алгоритма, решающего проблему пустоты языка, можно проверить, пуста ли полученная симметрическая разность. Если множество симметрической разности двух языков пусто, значит языки совпадают между собой. Следовательно, можно сделать вывод, что конечные автоматы, распознающие эти языки, эквивалентны.

Второй способ установления эквивалентности двух конечных автоматов основан на поиске неразличимых состояний в объединённом множестве состояний исходных автоматов. Изложим схему алгоритма установления экваиалентности двух автоматов.

Пусть $M_1 = (Q_1,\Sigma_1, \delta_1, q_1, F_1)$ и $M_2 = (Q_2,\Sigma_2, \delta_2, q_2, F_2)$ конечные автоматы, такие что $L_1 = L(M_1)$ и $L_2 = L(M_2)$. Построим конечный автомат $M_U = (Q_1 \cup Q_2, \Sigma_1 \cup \Sigma_2, q_1, F_1 \cup F_2)$. Поскольку автомат $M_U$ конструируется только для анализа, выбор начального состояния не имеет значения. Далее на множестве $Q_1 \cup Q_2$ выделяются классы неразличимости состояний. Если начальные состояния исходных автоматов находятся в оном классе эквивалентности, то алгоритм возвращает ответ <<ДА>>, иначе --- <<НЕТ>>.

\section{Упражнения}
\label{Chapter5Exs}
\subsection*{Свойства регулярных языков}
Построить регулярные выражения для множеств $L_1 \setminus L_2$, $L_1^R$, $L_1 \cup L_2$, $L_1 \triangle L_2$, если языки  $L_1$ и $L_2$ заданы следующими регулярными выражениями:
\begin{enumerate}
\item $L_1 = a^*b(a+b)^*$ и $L_2 = a^*b(a+b)^*$
\item $L_1 = (00 + 11)^*10(1 + 0)^*$ и $L_2=(00)^*(11)^*0^*1^*$
\end{enumerate}

\subsection*{Доказательство нерегулярности языков}
Пользуясь свойством замкнутости
класса регулярных языков, выяснить, какие из следующих языков регулярны:
%\begin{multicols}{2}
\begin{enumerate}
  \item $\{0^n 1^m \mid n \neq m\}$ (подсказка: воспользуйтесь
  нерегулярностью $\{0^n1^n\}$);
  \item $\{ w \in \{0, 1\}^* \mid
  |w|_0 = |w|_1\}$;
  \item язык из слов $w \in \{0,\ldots, 9\}^*$, которые
  являются десятичной записью чисел, делящихся на 2 или на 3, без
  лишних лидирующих нулей;
  \item $\{ 0^n 1^m 2^{n-m} \mid n \geqslant m \}$;
  \item $\{ a^n b a^m b a^{n+m} \mid n, m \in \N \}$.
\end{enumerate}
%\end{multicols}
%\subsection*{Удаление бесполезных символов}
%
%Удалить бесполезные символы в грамматиках с продукциями:
%\begin{align*}
    %\text{(1) }&
        %\begin{aligned}%{l}
            %S &\to 0 \mid A,\\
            %A &\to AB,\\
            %B &\to 1;
        %\end{aligned}
        %\qquad\qquad
    %&
    %\text{(2) }&
        %\begin{aligned}%{l}
            %S &\to AB \mid CA,\\
            %A &\to a,\\
            %B &\to BC \mid AB,\\
            %C &\to aB \mid \varepsilon.
        %\end{aligned}
%\end{align*}

\chapter{Контекстно"/свободные языки}
\label{cfg-intro}

\section{Деревья выводов в КС"/грамматиках}
\label{Chapter6-trees}

В грамматике может быть несколько выводов, эквивалентных в том смысле, что во всех них применяются одни и те же правила в одних и тех же местах, но в различном порядке, в случае КС-грамматик можно ввести удобное графическое представление класса эквивалентных выводов, называемое деревом вывода.

Далее мы будем использовать некоторые стандартные понятия теории графов: (ориентированное!) дерево, поддерево, корень, доминирование, помеченное дерево, упорядоченное дерево и~т.~п. Предполагается, что читателю эти понятия известны.

Сечением дерева $D$ назовем такое множество $C$ вершин дерева $D$. что выполняются следующие свойства: 
\begin{enumerate}
\item никакие две вершины из $C$ не лежат на одном пути из корня в дереве $D$; 
\item ни одну вершину дерева $D$ нельзя добавить к $C$, не нарушив первого свойства. 
\end{enumerate}

Множество вершин дерева $D$, состоящее из одного корня, является сечением; листья тоже образуют сечение; остальные сечения можно расположить как бы между этими крайними сечениями.

Помеченное упорядоченное дерево $D$ называется
\mydef{деревом вывода} в КС"/грамматике $G(A)=(N,\Sigma,P,A)$,
если выполнены следующие условия.
\begin{enumerate}

\item Корень дерева $D$ помечен $A$.

\item Если корень дерева имеет единственного потомка, помеченного $\eps$, то этот потомок образует дерево, состоящее из единственной вершины, и $A\to\eps$ --- продукция из множества $P$.

\item Если $D_1, \ldots ,D_k$ --- поддеревья, над которыми доминируют прямые потомки корня дерева, и корень дерева $D_i$ помечен $X_i$, то $A\to X_1 \ldots X_k$ --- продукция из множества $P$. Если при этом $X_i$ --- нетерминал, то дерево $D$ должно быть деревом вывода в грамматике $G(X_i)=(N,\Sigma,P,X_i)$, а если $X_i$ --- терминал, то дерево $D_i$ состоит из единственной вершины, помеченной $X_i$.
\end{enumerate}

Далее будем рассматривать естественное упорядочение листьев упорядоченного дерева --- <<слева направо>>. Кроной дерева вывода назовем слово, которое получится, если выписать слева направо метки листьев. Определим крону $\omega$ сечения $C$ дерева $D$ как слово, которое получается конкатенацией (в порядке слева направо) меток вершин, образующих сечение $C$.

\begin{mylemma}
\label{lemma-oSech}
Пусть $S=\alpha_0\To\alpha_1\To \ldots \To \alpha_n$ --- вывод слова $\alpha_n$ в КС"/грамматике $G=(N,\Sigma,P,S)$. Тогда в $G$ можно построить дерево вывода $D$, для которого $\alpha_n$ --- крона, а $\alpha_0,\alpha_1,\ldots ,\alpha_{n-1}$ --- набор крон некоторых сечений.
\end{mylemma}

\begin{myproof}
Построим такую последовательность деревьев выводов $D_i$, где $0\le i\le n$, что --- крона дерева $D_i$.

Пусть $D_0$ --- дерево, состоящее из единственной вершины, помеченной начальным нетерминалом $S$.

Предположим теперь, что дерево $D_i$ с нужными свойствами уже построено, и по нему построим дерево $D_{i+1}$. Допустим, что $\alpha_i=\beta_iA\gamma_i$ и после применения продукции $A\to X_1X_2 \ldots X_k$ получается слово $\alpha_{i+1}=\beta_iX_1X_2 \ldots X_k\gamma_i$. Дерево $D_{i+1}$ построим при помощи $D_i$ добавлением к листу, помеченному выделенным вхождением $A$ (он является ($|\beta_i|+1$)-м символом кроны дерева $D_i$), $k$ прямых потомков, которые помечаются $X_1, X_2, \ldots , X_k$. Ясно, что $\alpha_{i+1}$ --- крона дерева $D_{i+1}$.

Итак, $D_n=D$ --- искомое дерево вывода, a $\alpha_0,\alpha_1, \ldots , \alpha_{n-1}$ --- набор крон сечений этого дерева.
\end{myproof}

\begin{mylemma}
\label{lemma-oKrone}
Пусть $G=(N,\Sigma,P,S)$ --- КС-грамматика, a $D$ --- дерево вывода в $G$ с кроной $\alpha$. Тогда $\alpha\in L(G)$.
\end{mylemma}

\begin{myproof}
Пусть $C_0,C_1,C_2, \ldots , C_n$ --- такая последовательность сечений дерева $D$, что выполняются следующие условия:

\begin{enumerate}
\item $C$ содержит только корень дерева $D$;

\item $C_{i+1}$ для $0\le i<n$ получается из $C_i$ заменой одной нетерминальной вершины ее прямыми потомками;

\item $C_n$ --- множество листьев дерева $D$.
\end{enumerate}

Ясно, что хотя бы одна такая последовательность существует. Если $\alpha_i$ --- крона сечения $C_i$, то $S=\alpha_0\To \alpha_1\To \ldots \To \alpha_n=a$ --- вывод слова $\alpha$ из $S$ в $G$. Поэтому $\alpha\in L(G)$.
\end{myproof}

Непосредственным следствием лемм~\ref{lemma-oSech} и~\ref{lemma-oKrone} является

\begin{mytheorem}
\label{theorem-SechKrona}
Пусть $G=(N,\Sigma,P,S)$ --- КС-грамматика. $S\To^*\alpha$ тогда и только тогда, когда в $G$ существует дерево вывода с кроной $\alpha$.
\end{mytheorem}

По одному дереву вывода с кроной $\alpha$ можно построить разные выводы в грамматике, для которых $S\To^*\alpha$; среди всех таких выводов два вызывают особый интерес. Именно, если в доказательстве леммы~\ref{lemma-oKrone} сечение $C_{i+1}$ получается из $C_i$ заменой самой левой нетерминальной вершины в $C_i$ ее прямыми потомками, то соответствующий вывод $S=\alpha_0$, $\alpha_1, \ldots , \alpha_n$ называется левым выводом слова $\alpha_n$ из $\alpha_0$ в грамматике $G$. Правый вывод определяется аналогично. Заметим, что и левый и правый выводы определяются по дереву однозначно.

Если $S=\alpha_0,\alpha_1, \ldots , \alpha_n=\omega$ --- левый вывод терминального слова $\omega$ и $\alpha_i$ $(0\le i<n)$ имеет вид $x_iA_i\beta_i$, где $x_i\in\Sigma^*$, $A_i\in N$, $\beta_i\in(N\cup\Sigma)^*$, то каждое следующее слово $\alpha_{i+1}$ левого вывода получается из предыдущего слова $\alpha_i$ заменой самого левого нетерминала $A_i$ правой частью некоторой продукции. В правом выводе заменяется самый правый нетерминал.

\begin{myproblem}
Рассмотрим грамматику $G$ из примера~\ref{exampleArithmGrammar}: $G_0=(\{E;T;F\},\\\{a;+;*;(;)\},P,E)$, где $P$ состоит из продукций
\begin{align*}
    E &\to E+T \mid T, \\
    T &\to T*F \mid F, \\
    F &\to (E) \mid a.
\end{align*}
Постройте такое дерево вывода в этой грамматике, кроной которого является слово $a+a$. Укажите левый и правый выводы.
\end{myproblem}

Слово $\alpha$ будем называть левовыводимым в грамматике $G$ и писать $S\To_l^*\alpha$, если существует левый вывод $S=\alpha_0,\alpha_1, \ldots , \alpha_n=\alpha$. Аналогично, слово $\alpha$ будем называть правовыводимым и писать $S\To_r^*\alpha$, если существует правый вывод $S=\alpha_0,\alpha_1, \ldots , \alpha_n=\alpha$. Один шаг левого вывода обозначается через $\To_l$ , а шаг правого вывода --- через $\To_r$.

Если дан вывод $S\To^*\alpha$ в КС"/грамматике $G$, то не всегда можно найти единственное дерево вывода с кроной $\alpha$. Причина этого заключается в том, что есть КС"/грамматики, у которых может быть несколько различных деревьев выводов с одной и той же кроной.

\begin{myproblem}
Пусть $G=(\{S\},\{a,b\},\{S\to aSbS\mid bSaS\mid\eps\},S)$. Постройте разные деревья выводов в этой грамматике, у которых кроной является слово $abab$.
\end{myproblem}

КС-грамматику $G$ называют \mydef{неоднозначной}, если существует хотя бы одно слово $\omega\in L(G)$, которое является кроной двух или более различных деревьев выводов в $G$. В противном случае КС"/грамматика $G$ называется однозначной. Неоднозначность КС"/грамматики можно связать с существованием различных левых и (или) правых выводов.

\begin{myproblem}
Рассмотрим КС-грамматику $G$. Пусть $\omega\in L(G)$, докажите, что следующие утверждения эквивалентны:
\begin{enumerate}
\item $\omega$ --- крона двух различных деревьев выводов в $G$;
\item $\omega$ имеет два различных левых вывода в $G$;
\item $\omega$ имеет два различных правых вывода в $G$.
\end{enumerate}
\end{myproblem}

\section{Проблема непустоты и устранение бесполезных символов}
\label{Chapter6-problemEmptyLang}

Грамматика может, вообще говоря, содержать бесполезные символы и продукции. Например, в грамматике $G=(\{S,A\},\{a,b\},P,S)$, где $P=\{S\to a;A\to b\}$, нетерминал $A$ и терминал $b$ не могут появиться ни в каком выводимом слове. Таким образом, и эти символы, и продукцию $(A\to b)$ можно устранить из грамматики $G$, не изменив языка $L(G)$.

Дадим точное определение. Назовем символ $X\in N\cup\Sigma$ бесполезным в КС"/грамматике $G=(N,\Sigma,P,S)$, если в ней невозможен вывод вида $S\To^*\omega Xy\To^*\omega xy$, где $\omega,x,y\in\Sigma^*$.

Ясно, что если язык $L(G)$ пуст, то все символы бесполезны. Построим алгоритм (см. алгоритм~\ref{algo-KS-NonEmptyLang} ниже), выясняющий, может ли нетерминал порождать какие-нибудь терминальные слова. Далее будет доказано, что этот алгоритм годится для проверки на непустоту произвольных КС-языков. Таким образом, проблема непустоты для КС"/языков разрешима.

\Algo[H]{Проверка на непустоту}
{\label{algo-KS-NonEmptyLang} КС"/грамматика $G=(N,\Sigma,P,S)$.}
{<<ДА>>, если $L(G)\neq\es$, <<НЕТ>>, если $L(G)=\es$.}
{Рекурсивное построение расширяющейся последовательности специального вида множества $N$.}
{
\item Положить $N_0=\es$, $i=1$.

\item Положить $N_i=\{A\mid A\to\alpha\in P, \alpha\in(N_{i-1}\cup\Sigma)^*)\cup N_{i-1}$.

\item Если $N_i\neq N_{i-1}$, то положить $i=i+1$ и перейти к шагу 2, в противном случае положить $N_\Sigma = N_l$.

\item Если $S\in N_\Sigma$, выдать на печать <<ДА>>, в противном случае --- <<НЕТ>>.
}

Так как $N_\Sigma\subseteq N$, то алгоритм~\ref{algo-KS-NonEmptyLang} должен остановиться самое большее после $|N|+1$ повторений шага~2.

Алгоритм~\ref{algo-KS-NonEmptyLang} строит множество <<стабилизации>> $N_\Sigma$, соответствующее алфавиту $\Sigma$. Аналог этого множества можно определить и для произвольного подмножества $\Omega$ множества $\Sigma\cup\{\eps\}$:
\[
    N_\Omega=\{A\mid A\in N, A\To_G^*\alpha, \alpha\in\Omega^*\}.
\]
Множества такого типа встречаются в различных алгоритмах, поэтому выделим ту часть алгоритма~\ref{algo-KS-NonEmptyLang}, которая позволяет построить $N_\Omega$ (см. алгоритм~\ref{algo-KS-NonEmptyLang-Stab}).

\Algo[H]{Построение множества $N_\Omega$}
{\label{algo-KS-NonEmptyLang-Stab}КС-грамматика $G=(M,\Sigma,P,S)$, $\Omega$ --- подмножество множества $\Sigma\cup\{\eps\}$.}
{Множество $N_\Omega$.}
{Рекурсивное построение расширяющейся последовательности подмножеств специального вида множества $N$.}
{
\item Положить $N_0=\es$, $i=1$.

\item Положить $N_i=\{A\mid A\to\alpha\in P, \alpha\in(N_{i-1}\cup \Omega)^*\}\cup N_{i-1}$.

\item Если $N_i\neq N_{i-1}$, то положить $i=i+1$ и перейти к
шагу~2, в противном случае положить $N_\Omega=N_i.$.
}

Рассмотрим КС-грамматику $G=(N,\Sigma,P,S)$ и займемся теперь обоснованием алгоритма~\ref{algo-KS-NonEmptyLang}.

\begin{mylemma}
\label{lemma-NonEmptyAlgoCorr-1}
Пусть $i\in\{0;1;2;\ldots \}$. Если $A\in N_i$, то $A\To_G^*\omega$ для некоторого слова $\omega=\Sigma^*$.
\end{mylemma}

\begin{myproof}
Применим метод математической индукции по $i$.

Случай $i=0$ не нуждается в доказательстве, так как $N_0=\es$.

Предположим, что утверждение верно для $i=k$, и докажем его для $i=k+1$ Рассмотрим $A\in N_{k+1}$. Если $A$ принадлежит также и $N_k$,то $A\To_G^*\omega$ для некоторого слова $\omega\in\Sigma^*$ в силу индуктивного предположения. Если же $A\in N_{k+1}-N_k$, то существует такая продукция $A\to X_1\ldots X_m$, в которой $X\in\Sigma\cup N_k$. Тогда для каждого $X_j$, можно найти такое слово $\omega_j$, что $X_j\To_G^*\omega_j$: если $X_j\in\Sigma$, то $\omega_j=X_j$, в противном случае существование $\omega_j$ следует из индуктивного предположения. Итак, $A \To X_1\ldots X_m \To_G^*\omega_1X_2\ldots X_m \To_G^* \ldots \To_G^* \omega_1\ldots \omega_m$. (Подчеркнем, что случай $m=0$ на первом шаге, т. е. продукция $A\to \eps$, не составляет исключения.)

Таким образом, лемма верна для произвольного $i$.
\end{myproof}

\begin{mylemma}
\label{lemma-NonEmptyAlgoCorr-2}
Пусть $n\in\{1;2;\ldots\}$. Если $A\To_G^n\omega$ для некоторого слова $\omega\in\Sigma^*$, то существует такое $i\in\{0;1;2;\ldots\}$, что $A\in N_i$.
\end{mylemma}

\begin{myproof}
Применим метод математической индукции по $n$.

В случае $n=1$, очевидно, $i=1$.

Допустим, что утверждение верно для $n=k$, и докажем его для $n=k+1$. Пусть $A\To_g^{k+1}\omega$. Тогда $A\To X_1\ldots X_m\To_G^k\omega$, где слово $\omega=\omega_1\ldots \omega_m$ таково, что $X_j\To_G^{n_j}\omega$ для каждого $j$ и $n_j\le k$ (в дереве вывода $A\To_G^{k+1}\omega$ слово $\omega_j$ является кроной поддерева с корнем $X_j$). В силу индуктивного предположения, если $X_j\in N$, то $X_j\in N_{i_j}$ для некоторого $i_j$, а если $X_j\in\Sigma$, то определим $i_j=0$. Пусть $i=1+max(i_1, \ldots , i_k)$. Тогда $A\in N_i$.

Итак, лемма верна для произвольного $n$.
\end{myproof}

\begin{mytheorem}
Алгоритм~\ref{algo-KS-NonEmptyLang} говорит <<Да>> тогда и только тогда, когда $S\To_G^*\omega$ для некоторого слова $\omega$ из $\Sigma^*$.
\end{mytheorem}

\begin{myproof}
Согласно алгоритму~\ref{algo-KS-NonEmptyLang} <<Да>> выводится тогда и только тогда, когда $S\in N_\Sigma$. Для завершения доказательства теоремы теперь достаточно воспользоваться леммами~\ref{lemma-NonEmptyAlgoCorr-1} и~\ref{lemma-NonEmptyAlgoCorr-2} при $A=S$, так как $N_\Sigma=\bigcup_iN_i$.
\end{myproof}

Символ $X\in N\cup\Sigma$ назовем \mydef{недостижимым} в КС"/грамматике $G=(N,\Sigma,P,S)$, если $X$ не появляется ни в одной выводимой цепочке. Иначе говоря, символ $X\in N\cup\Sigma$ достижим, если для него существуют такие слова $\alpha$, $\beta$ из $(N\cup\Sigma)^*$, что $S\To_G^*\alpha X\beta$. Недостижимые символы являются примерами бесполезных символов; их можно устранить из КС"/грамматики с помощью алгоритма~\ref{algo-DelNonAvailSybmbols}.

\Algo[H]{Устранение недостижимых символов}
{\label{algo-DelNonAvailSybmbols}КС"/грамматика $G=(N,\Sigma,P,S)$.}
{КС"/грамматика $G'=(N,\Sigma,P,S)$, у которой $L(G')=L(G)$ и нет недостижимых символов.}
{Рекурсивное построение расширяющейся последовательности подмножеств специального вида множества $N\cup\Sigma$.}
{
\item Положить $V_0=\{S\}$, $i=1$.

\item  Положить $V_i=\{X\mid (A\to\alpha X\beta)\in P$, $A\in V_{i-1} \}\cup V_{i-1}$.

\item  Если $V_i\neq V_{i-1}$, то положить $i=i+1$ перейти к шагу 2, в противном случае положить $V^S=V_i$.

\item  Построить грамматику $G'=(N,\Sigma,P,S)$, где $N'=V^S\cap N$, $\Sigma '=V^S\cap\Sigma$, а в $P'$ включены те и только те продукции из $P$, которые содержат только символы из $V^S$.
}

Так как $V^S\subseteq N\cup\Sigma$, то алгоритм~\ref{algo-DelNonAvailSybmbols} должен остановиться самое большее после $|N| + |\Sigma|+1$ повторений шага 2. Алгоритмы~\ref{algo-KS-NonEmptyLang} и~\ref{algo-DelNonAvailSybmbols} очень похожи; более того, обосновываются они тоже сходным образом.

\begin{myproblem}
Используя метод математической индукции по $i$, докажите, что существует вывод $S\To_{G'}^i\alpha X\beta$ тогда и только тогда, когда $X\in V_i$
\end{myproblem}

\begin{myproblem}
\label{problem-eqOfLangsWithoutUselessSymbols}
Докажите, что алгоритм~\ref{algo-DelNonAvailSybmbols} по КС-грамматике $G=(N,\Sigma,P,S)$ строит такую КС-грамматику $G'=(N' ,\Sigma',P',S)$, у которой $L(G')=L(G)$ и для всех $X\in N'\cup\Sigma'$ существуют такие слова $\alpha$, $\beta$ из $(N'\cup\Sigma')^*$, что $S\To_{G'}^*\alpha X\beta$. (Другими словами, алгоритм~\ref{algo-DelNonAvailSybmbols} строит новую КС-грамматику $G'=(N',\Sigma',P',S)$ без недостижимых символов, для которой $L(G')=L(G)$.)
\end{myproblem}

\Algo[H]{Устранение бесполезных символов}
{\label{algo-DelUselessSybmbols}КС"/грамматика $G=(N,\Sigma,P,S)$, у которой $L(G)\neq\es$.}
{КС"/грамматика $G'=(N',\Sigma',P',S)$. У которой $L(G')=L(G)$ и в $N'\cup\Sigma'$ нет бесполезных символов.}
{Последовательное применение алгоритма~\ref{algo-KS-NonEmptyLang-Stab} для $\Omega=\Sigma$ и алгоритма~\ref{algo-DelNonAvailSybmbols}.}
{
\item К грамматике $G=(N,\Sigma,P,S)$ применить алгоритм~\ref{algo-KS-NonEmptyLang-Stab} и найти множество $N_\Sigma$; построить грамматику $G_1=(N\cap N_\Sigma,\Sigma,P_1,S)$, где в $P_1$ включены те и только те продукции из $P$, которые содержат только символы из $N\cap N_\Sigma$.

\item К грамматике $G_1=(N\cap N_\Sigma,\Sigma,P_1,S)$ применить алгоритм~\ref{algo-DelNonAvailSybmbols} и построить грамматику $G'=(N',\Sigma',P',S)$.
}

Дадим пояснения к алгоритму~~\ref{algo-DelUselessSybmbols} устранения бесполезных символов. На шаге 1 из $G$ устраняются все нетерминалы, которые не могут порождать терминальных слов. Затем на шаге 2 устраняются все недостижимые символы. Каждый символ $X$ результирующей грамматики должен появиться хотя бы в одном выводе вида $S\To^*\omega Xy\To^*\omega xy$. В примере~\ref{example-algosteps} показано, что если сначала применить алгоритм~\ref{algo-DelNonAvailSybmbols}, а потом алгоритм~\ref{algo-KS-NonEmptyLang-Stab}, то в результате может получиться грамматика, содержащая бесполезные символы.

\begin{mytheorem}
\label{theorem-AlgoDelUselessSymbolsCorrectness}
Грамматика $G'$, которую строит алгоритм~\ref{algo-DelUselessSybmbols}, является не содержащей бесполезных символов КС"/грамматикой, и $L(G)=L(G')$.
\end{mytheorem}

\begin{myproof}
При переходе от грамматики $G$ к грамматике $G_1$, исключаются только те нетерминалы и продукции, которые не участвуют в выводах терминальных слов, поэтому $L(G)=L(G_1)$. Равенство $L(G_1)=L(G')$ обсуждалось выше (см. упражнение~\ref{problem-eqOfLangsWithoutUselessSymbols}). Следовательно, $L(G)=L(G')$.

Покажем теперь, что в $G'$ нет бесполезных символов. Предположим, что $A\in N'$ --- бесполезный символ. Тогда по определению бесполезности символа могут представиться два случая:

\begin{enumerate}
\item Вывод $S\To_{G'}^*\alpha A\beta$ ни для каких $\alpha$ и $\beta$ невозможен. Но в этом случае символ $A$ должен был быть устраненным на шаге 2 алгоритма~\ref{algo-DelUselessSybmbols}, что приводит к противоречию.

\item Вывод $S\To_{G'}^*\alpha A\beta$ для некоторых $\alpha$ и $\beta$ возможен, но вывода $A\To_{G'}^*\omega$ для $\omega\in\Sigma'^*$ не существует. Ясно, что если символ $A$ <<проскочил>> шаг 1, то на шаге 2 в рассматриваемой ситуации его уже не устранить. Кроме того, если в этом случае $A\To_G^*\gamma B\delta$, то и символ $B$ не устраним на шаге 2. Итак, предположим, что символ $A$ не устранен на первом шаге, т. е. $A\To_G^*\omega$ для $\omega\in\Sigma^*$. Если возможен вывод $A\To_G^*\gamma B\delta\To_G^*\omega$ то в силу сказанного выше символ $B$ не устраним на шаге 2, это означает, что $A\To_{G'}^*\omega$. Полученное противоречие показывает, что в действительности $A$ устраняется на шаге~1.
\end{enumerate}

Доказательство того, что ни один терминал в $G'$ не может быть бесполезным, проводится аналогично.
\end{myproof}

\begin{myexample}
\label{example-algosteps}
Рассмотрим грамматику $G=(\{S;A;B\},\{a:b\},P,S)$, где $P$ состоит из продукций
\[
S\to a \mid A; \quad A\to AB; \quad B\to b.
\]
Применим к $G$ алгоритм~\ref{algo-DelUselessSybmbols}. На шаге 1 этого алгоритма получим: $N_\eps\{S;B\}$ и $G=(\{S;B\},\{a;b\},\{S\to a;B\to b\},S)$. На втором шаге, применив алгоритм~\ref{algo-DelNonAvailSybmbols}, получим: $V_2=V_1=\{S,a\}$. Итак, $G'=(\{S\},\{a\},\{S\to a\},S)$.

Теперь в алгоритме~\ref{algo-DelUselessSybmbols} поменяем местами шаги 1 и 2. После применения к $G$ алгоритма~\ref{algo-DelNonAvailSybmbols} грамматика не изменится в силу того, что все символы достижимы. Последующее применение алгоритма~\ref{algo-KS-NonEmptyLang-Stab} дает $N_\eps=\{S;B\}$. Следовательно, результирующей будет грамматика $G$ , отличная от $G'$ .
\end{myexample}

\section{Построение приведенной КС"/грамматики}
\label{Chapter6-normalizeGrammar}

В пункте~\ref{Chapter6-problemEmptyLang} было показано, как, не меняя языка, устранить из КС"/грамматики все бесполезные символы. Целью этого пункта является устранение из КС-грамматики <<нехороших>> продукций.

Назовем КС-грамматику $G=(N,\Sigma,P,S)$ \mydef{неукорачивающейся грамматикой}, если либо $P$ вовсе не содержит $\eps$-продукции типа $A\to\eps$, либо в $P$ есть точно одна $\eps$-продукция $S\to\eps$ и $S$, при этом не встречается в правых частях остальных продукций.

\Algo[H]{Преобразование КС"/грамматики в неукорачивающуюся грамматику}
{\label{algo-GrammarToEpsFreeGrammar}КС-грамматика $G=(N,\Sigma,P,S)$.}
{Неукорачивающаяся КС"/грамматика $G'=(N',\Sigma,P',S')$, у которой $L(G')=L(G)$.}
{«Устранение перегородок».}
{
\item Применить алгоритм~\ref{algo-KS-NonEmptyLang-Stab} для $\Omega=\{\eps\}$ и построить множество
\[
    N_\eps=\{A\mid A\in N, A\To_G^*\eps\}.
\]

\item Если
\[
    (A\to\alpha_0B_1\alpha_1B_2\alpha_2 \ldots B_k\alpha_k)\in P,
\]
где $k\ge 0$, $B_i\in N_\eps$ и ни один символ в словах $\alpha_j$ не принадлежит $N_\eps$, то включить в $P'$ все продукции вида $A\to\alpha_0X_1\alpha_1X2 \ldots \alpha_{k-1}X_k\alpha_k$, где $X_i$ --- либо $B_i$, либо $\eps$, но не включать продукцию $A\to\eps$ (это могло бы произойти в случаe, если все $\alpha_i$, равны $\eps$).

\item Если $S\in N_\eps$ , то ввести новый нетерминал $S'$ и дополнительно включить в $P'$ продукции $(S'\to S|\eps)$, в противном случае положить $N'=N, S'=S$.

\item Положить $G'=(N',\Sigma,P',S')$.
}

\begin{mytheorem}
\label{theorem-AlgoDelEpsProductionsCorrectness}
Грамматика $G'$, которую строит алгоритм~\ref{algo-GrammarToEpsFreeGrammar} по КС"/грамматике $G$, является неукорачивающейся КС"/грамматикой и $L(G)=L(G')$.
\end{mytheorem}

\begin{myproof}
Тот факт, что грамматика $G'$ является неукорачивающейся, вытекает из простого анализа алгоритма.

Применяя метод математической индукции по длине слова $\omega$, можно доказать следующее вспомогательное утверждение: для произвольного слова $\omega$ из $\Sigma^*\backslash\{\eps\}$ и буквы $A$ из $N$ вывод $A\To_{G'}^*$ возможен тогда и только тогда, когда $A\To_G^*\omega$. Применим это утверждение для $S=A$ и $\omega\in\Sigma^*\backslash\{\eps\}$ получим: $\omega\in L(G)$ тогда и только тогда, когда $\omega\in L(G')$. Заметим теперь, что $\eps\in L(G)$ тогда и только тогда, когда $\eps\in L(G')$. Таким образом, $L(G)=L(G')$.
\end{myproof}

\begin{myproblem}
\label{problem-GrammarToEpsFreeGrammarWOUselessSymbols}
Докажите, что если на вход алгоритма~\ref{algo-GrammarToEpsFreeGrammar} подается КС"/грамматика без бесполезных символов, то и на выходе алгоритма получается такая же грамматика.
\end{myproblem}

\begin{myexample}
Рассмотрим контекстно"/свободную грамматику
\[
    G=(\{S;A;B\},\{0;1\},P,S),
\]
где $P$ состоит из продукций
\begin{align*}
	S &\to 0A \mid 1B \mid \eps, \\
    A &\to AB \mid 0 \mid \eps, \\
    B &\to 0 \mid A.
\end{align*}
Применим к грамматике $G$ алгоритм~\ref{algo-GrammarToEpsFreeGrammar} и получим неукорачивающуюся КС-грамматику $G'=(\{S';S;A;B\},\{0;1\},P,S)$, где $P'$ состоит из продукций
\begin{align*}
    S' &\to S \mid \eps, \\
    S  &\to 0A \mid 0 \mid 1B \mid 1, \\
    A  &\to AB \mid A \mid B \mid 0, \\
    B  &\to 0 \mid A.
\end{align*}
\end{myexample}

Другое полезное преобразование грамматик --- устранение продукций вида $A\to B$, где $A$ и $B$ --- нетерминалы: такие продукции далее будем называть \mydef{цепными}.

\Algo[H]{Устранение цепных продукций}
{\label{algo-DelCyclicProductions}Неукорачивающаяся КС"/грамматика $G=(N,\Sigma,P,S)$.}
{Неукорачивающаяся КС"/грамматика $G'=(N',\Sigma',P',S')$ без цепных продукций, у которой $L(G')=L(G)$.}
{Для каждого $A$ из $N$ строится подмножество $N^A=\{B\mid A\To^*B)\}$ множества $N$, и на основе этого конструируется новая грамматика.}
{
\item
Для каждого $A$ из $N$ построить $N^A=\{B\mid A\To^*B\}$ следующим образом.
\begin{enumerate}[leftmargin=1cm]
\item Положить $N_0=\{A\}$ и $i=1$.

\item Положить $N_i|=\{C\mid (B\to C)\in P$ и $B\in N_{i-1}\}\cup N_{i-1}$.

\item Если $N_i\neq N_{i-1}$, то положить $i=i+1$ и повторить шаг 1.2, в противном случае положить $N^A=N_i$.
\end{enumerate}

\item
Построить $P'$ так: если продукция $B\to\alpha$ принадлежит $P$ и не является цепной, то включить в $P'$ продукцию $(A\to\alpha)$ для всех таких $A$, что $B\in N^A$.

\item
Положить $G'=(N,\Sigma,P',S)$.
}
\begin{mytheorem}
\label{theorem-AlgoDelEpsProductionsCorrectnessWOEpsProducts}
Грамматика $G'$, которую строит алгоритм~\ref{algo-DelCyclicProductions} по неукорачивающейся КС"/грамматике $G$, является неукорачивающейся КС"/грамматикой без цепных продукций и $L(G)=L(G')$.
\end{mytheorem}

\begin{myproof}
Тот факт, что грамматика $G'$ является неукорачивающейся и не имеет цепных продукций, вытекает из простого анализа алгоритма.

Покажем, что $L(G')\subseteq L(G)$. Пусть $\omega\in L(G')$. Тогда в грамматике $G'$ существует вывод $S\To^1\alpha_0 \To^1 \alpha_1 \To^1 \ldots \To^1 \alpha_n = \omega$. Если при переходе от некоторого $\alpha_i$ к $\alpha_{i+1}$ применяется продукция $A\to\beta$ из $P'$, то тогда существует такой символ $B\in N$ (не исключено, что $B=A$), что $A\To_G^*B\To_G\beta$. Таким образом, $A\To_G^*\beta$ и, следовательно, $\alpha_i\To_G^*\alpha_{i+1}$. Отсюда следует, что $S\To_G^*\omega\in L(G)$, так что $L(G')\subseteq L(G)$.

Теперь покажем, что $L(G)\subseteq L(G')$; причем далее мы будем пользоваться понятиями, которые обсуждались в конце раздела~\ref{Chapter6-trees}. Пусть $\omega\in L(G)$ и $S=\alpha_0\To_l\alpha_1\To_l\ldots\To_l\alpha_n=\omega$ --- левый вывод слова $\omega$ в грамматике $G$. Рассмотрим подпоследовательность $i_1, i_2, \ldots , i_k$ последовательности $1, 2, \ldots , n$, состоящую в точности из тех номеров $j$, для которых на шаге $\alpha j_{j-1}\To_l\alpha_j$ вывода $S=\alpha_0\To_l\alpha_1\To_l\ldots\To_l\alpha_n=\omega$ применяется нецепная продукция. В частности, $i_k=n$, так как вывод терминального слова не может оканчиваться цепной продукцией. Так как мы рассматриваем левый вывод, то последовательные применения нескольких исключительно цепных продукций заменяют нетерминальный символ, занимающий одну и ту же позицию в левовыводимых словах. Используя конструкцию $P$, отсюда получаем:
\[
S\To_{G'}\alpha_{i_1}\To_{G'}\alpha_{i_2}\To_{G'}\ldots \To_{G'} \alpha_{i_k}=\omega.
\]
Таким образом, $\omega\in L(G')$ и, следовательно, $L(G)\subseteq L(G')$.

В итоге получаем: $L(G')=L(G)$.
\end{myproof}

С точки зрения анализа бесполезности символов алгоритм~\ref{algo-DelCyclicProductions}
является <<плохим>>.

\begin{myexample}
Рассмотрим неукорачивающуюся КС"/грамматику $G=(\{S;A\},\{a\},\\\{S\to A;A\to a\},S)$, у которой нет бесполезных символов. Применяя к ней алгоритм~\ref{algo-DelCyclicProductions}, получаем грамматику $G=(\{S;A\},\{a\},\{S\to a;A\to a\},S)$ с бесполезным символом $A$.
\end{myexample}

\begin{myexample}
Рассмотрим неукорачивающуюся КС"/грамматику
\[
    G'=(\{W;S;A;B\},\{0;1\},P',W).
\]
где $P'$ состоит из продукций
\begin{align*}
	W & \to S \mid \eps, &
	S &\to 0A \mid 0 \mid 1B \mid 1, \\
    A &\to AB \mid A \mid B \mid 0, &
	B &\to 0.
\end{align*}
Применим к грамматике $G$ алгоритм~\ref{algo-DelCyclicProductions}  и получим неукорачивающуюся КС"/грамматику $G'=(\{W;S;A;B\},\{0;1\},P',W)$, где $P'$ состоит из продукций
\begin{align*}
	W &\to \eps \mid 0A \mid 0 \mid 1B \mid 1, &
    S &\to 0A \mid 0 \mid 1B \mid 1, \\
    A &\to AB \mid 0, &
    B &\to 0,
\end{align*}
среди которых нет цепных.

Легко обнаружить, что построенная грамматика $G'$ имеет бесполезные символы. Применим к грамматике $G'$ алгоритм~\ref{algo-DelUselessSybmbols} и получим неукорачивающуюся КС"/грамматику $G=(\{W;S;A;B\},\{0;1\},P',W)$, где $P'$ состоит из продукций
\begin{align*}
	W &\to \eps \mid 0A \mid 0 \mid 1B \mid 1, &
    A &\to 0, &
    B &\to 0.
\end{align*}
У нее нет ни цепных продукций, ни бесполезных символов.
\end{myexample}

\begin{myproblem}
\label{problem-GrammarToEpsFreeGrammarWOCyclicSymbols}
Докажите, что если на вход алгоритма~\ref{algo-DelUselessSybmbols} подается неукорачивающаяся КС"/грамматика без цепных продукций, то на выходе алгоритма получится неукорачивающаяся КС"/грамматика без бесполезных символов и цепных продукций.

КС"/грамматика $G=(H,\Sigma,P,S)$ называется грамматикой без циклов, если в ней нет выводов $A\To^*A$ для $A\in N$.
\end{myproblem}

\begin{myproblem}
Докажите, что если неукорачивающаяся КС"/грамматика $G=(N,\Sigma,P,S)$ не имеет цепных продукций, то в ней нет циклов. Существуют ли КС\-грамматики с циклами, но без цепных продукций?
\end{myproblem}

КС"/грамматика $G$ называется \mydef{приведенной}, если она не имеет бесполезных символов, циклов и является неукорачивающейся.

\Algo[H]{Преобразование произвольной КС"/грамматики в приведенную}
{\label{algo-NormalGrammar}КС"/грамматика $G$.}
{Приведенная КС"/грамматика $G'$, для которой $L(G')=L(G)$.}
{Применение алгоритмов~\ref{algo-DelUselessSybmbols},~\ref{algo-GrammarToEpsFreeGrammar}, ~\ref{algo-DelCyclicProductions}.}
{
\item Применить алгоритм~\ref{algo-DelUselessSybmbols} и по КС"/грамматике $G$ построить КС"/грамматику без бесполезных символов $G_1$, для которой $L(G_1)=L(G)$.

\item Применить алгоритм ~\ref{algo-GrammarToEpsFreeGrammar} и по КС"/грамматике $G_1$ построить неукорачивающуюся КС"/грамматику без бесполезных символов $G_2$, для которой $L(G_2)=L(G_1)$.

\item Применить алгоритм~\ref{algo-DelCyclicProductions} и по КС"/грамматике $G_2$ построить неукорачивающуюся КС"/грамматику без цепных продукций $G_3$, для которой $L(G_3)=L(G_2)$.

\item Применить алгоритм~\ref{algo-DelUselessSybmbols} и по КС"/грамматике $G_3$ построить искомую КС"/грамматику $G$, для которой $L(G_3)=L(G')$.
}

\begin{mytheorem}
\label{theorem-NormalGrammarAlgoCorrectness}
Грамматика $G'$, которую строит алгоритм~\ref{algo-NormalGrammar} по произвольной КС"/грамматике $G$, является приведенной КС"/грамматикой и $L(G)=L(G')$.
\end{mytheorem}

\begin{myproof}
Для доказательства этой теоремы достаточно последовательно воспользоваться теоремой~\ref{theorem-AlgoDelUselessSymbolsCorrectness}, теоремой~\ref{theorem-AlgoDelEpsProductionsCorrectness}, упражнением~\ref{problem-GrammarToEpsFreeGrammarWOUselessSymbols}, теоремой~\ref{theorem-AlgoDelEpsProductionsCorrectnessWOEpsProducts}, теоремой~\ref{theorem-AlgoDelUselessSymbolsCorrectness} и упражнением~\ref{problem-GrammarToEpsFreeGrammarWOCyclicSymbols}.
\end{myproof}

\section{Упражнения}
\label{Chapter6Exs}
\subsection*{Построение КС"/грамматик}

Построить КС"/грамматики для следующих языков:
\begin{itemize}
    \item $\{ a^n b^n c^m d^m \mid n, m \in \N \}$;
    \item $\{a^i b^j c^j d^i \mid i, j \in \N\}$;
    \item $\{ a^n b^n c^m d^m \mid n, m \in \N \} \cup \{a^i b^j c^j d^i \mid i, j \in \N\}$.
\end{itemize}


\subsection*{Неоднозначность в КС"/грамматиках}

\begin{enumerate}
    \item Найдите, если это возможно, грамматику без
    неоднозначности для каждого языка из предыдущего упражнения.

    \item Докажите или опровергните контрпримером, что если
     $L_1$ и $L_2$ это КС"/языки,
    не обладающие неоднозначностью, то язык $L_1 \cup L_2$
    тоже не обладает этим свойством.
\end{enumerate}


\subsection*{Алгоритмы для КС-грамматик}

Удалить бесполезные символы в грамматиках с продукциями:
\begin{align*}
    \text{(1) }&
        \begin{aligned}%{l}
            S &\to 0 \mid A,\\
            A &\to AB,\\
            B &\to 1;
        \end{aligned}
        \qquad\qquad
    &
    \text{(2) }&
        \begin{aligned}%{l}
            S &\to AB \mid CA,\\
            A &\to a,\\
            B &\to BC \mid AB,\\
            C &\to aB \mid \varepsilon.
        \end{aligned}
\end{align*}

Преобразовать следующую грамматику к неукорачивающейся:
\begin{align*}
            S &\to  AB,\\
            A &\to aAA \mid \eps,\\
            B &\to bBB \mid \eps;
\end{align*}

Удалить цепные продукции из грамматики с продукциями:
\begin{align*}
            E &\to T \mid E+T,\\
            T &\to F \mid T*F,\\
            F &\to I \mid (E),\\
            I &\to a \mid b \mid Ia \mid Ib \mid I0 \mid I1;\\
\end{align*}

\chapter{Нормальные формы КС"/грамматик}
\label{normal-cfg}

В предыдущем параграфе было показано, как, не меняя языка, устранить из
КС"/грамматики все бесполезные символы, циклы и сделать ее неукорачивающейся.
В этом параграфе процесс упрощения формы записи грамматик для КС"/языков
будет продолжен. Мы рассмотрим две нормальные формы КС"/грамматик,
каждая из которых используется для решения определенной
практической задачи. Эти задачи будут рассмотрены, соответственно,
в конце этой и следующей главы.


\section{Нормальная форма Хомского}
\label{Chapter7NFH}

Говорят, что КС"/грамматика $G=(N,\Sigma,P,S)$ представлена в
\mydef{нормальной форме Хомского}, если продукции из $P$
имеют один из следующих видов:
\begin{enumerate}
    \item $A\to BC$, где $A,B,C\in N$,

    \item $A\to a$, где $A\in N$, $a\in\Sigma$,

    \item $S\to\eps$, если $\eps\in L(G)$,
\end{enumerate}
причем $S$ не встречается в правых частях продукций.

\Algo[H]{Преобразование КС"/грамматики к нормальной форме Хомского}
{\label{algo-Homsky}КС-грамматика $G=(N,\Sigma,P,S)$.}
{КС-грамматика $G'$ в нормальной форме Хомского, у которой $L(G')=L(G)$.}
{Предварительное приведение исходной грамматики и последовательное конструирование продукций новой грамматики.}
{
\item
Ввести новый стартовый нетерминал $S_1$ и добавить к $P$ одну продукцию $S_1\to S$. Применить к модифицированной грамматике алгоритм~\ref{algo-NormalGrammar} и построить тем самым приведенную КС"/грамматику $G_1=(N_1,\Sigma,P_1,S_1)$, в продукциях которой $S_1$ не встречается в правых частях. Перейти на следующий шаг, где начинается построение искомой грамматики $G'=(N',\Sigma,P',S_1)$.

\item Включить в $P'$ все продукции из $P_1$, вида $A\to a$, где $A\in N_1$, $a\in\Sigma$.

\item Включить в $P'$ все продукции из $P_1$ вида $A\to BC$, где $A,B,C\in N_1$.

\item Включить в $P'$ продукцию $S_1\to\eps$, если она была в $P_1$.

\item Для каждой продукции из $P$ вида $A\to X_1 \ldots X_k$ включить в $P'$ продукции
\begin{align*}
    A  &\to X'_1 \la X_2\ldots X_k \ra, \\
    \la X_2 \ldots X_k> &\to X'_2 \la X_3\ldots X_k \ra, \\
     & \ldots\\
    \la X_{k-2} X_{k-1} X_k> &\to X'_{k-2} \la X_{k-1} X_k \ra, \\
    \la X_{k-1} X_k \ra &\to X'_{k-1} X'_k,
\end{align*}
где $X'_i=X_i$, если $X_i\in N$, и $X'_i$~--- новый нетерминал,
если $X_i\in\Sigma$, а $\la X_i\ldots X_k \ra$~--- новый нетерминал в
любом случае.

\item Для каждой продукции из $P$ вида $A\to X_1X_2$, где хотя бы один из символов $X_1$ и $X_2$ принадлежит $\Sigma$, включить в $P'$ продукцию $A\to X'_1X'_2$, где $X'_i$ определяется так же, как на шаге 5.

\item Для каждого нетерминала вида $a'$, введенного на шагах 5 и 6, включить в $P'$ продукцию $a'\to a$.

\item Положить $G'=(N',\Sigma,P',S_1)$.
}

Для обоснования алгоритма преобразования КС"/грамматики к нормальной форме Хом\-ско\-го (см.~алгоритм~\ref{algo-Homsky}) нам понадобится вспомогательное утверждение, которое позволит без изменения языка удалять из грамматики продукции вида $A\to\alpha$. Продукции вида $A\to\alpha$ будем называть \mydef{нетерминальными $A$"/продукциями}.

\begin{mylemma}
\label{lemma-proofOfHomskyMod}
Пусть $G=(N,\Sigma,P,S)$ --- КС"/грамматика и $P$ содержит продукцию $A\to\alpha B\beta$, где $B\in N$, $\alpha,\beta\in(N\cup\Sigma)^*$. Рассмотрим все нетерминальные $B$"/про\-дук\-ции этой грамматики
\[
    B \to \gamma_1 \mid \gamma_2 \mid \ldots \mid \gamma_k |.
\]
Пусть $G'=(N,\Sigma,P',S)$, где
\[
    P' = (P - \{A\to\alpha B\beta\}) \cup
        \{A \to \alpha\gamma_1\beta
            \mid \alpha\gamma_2\beta
            \mid \ldots
            \mid \alpha\gamma_k\beta\}.
\]
Тогда $L(G)=L(G')$.
\end{mylemma}

\begin{myproblem}
Доказать лемму~\ref{lemma-proofOfHomskyMod}.
\end{myproblem}

\begin{mytheorem}
\label{theorem-AlgoHomskyProof}
Пусть $G$ --- произвольная КС"/грамматика. Алгоритм~\ref{algo-Homsky} строит по $G$ такую КС"/грамматику $G'$ в нормальной форме Хомского, что $L(G)=L(G')$.
\end{mytheorem}

\begin{myproof}
Применим к КС"/грамматике $G$ алгоритм~\ref{algo-Homsky}. В силу теоремы~\ref{theorem-NormalGrammarAlgoCorrectness} на шаге 1 этого алгоритма строится приведенная КС"/грамматика $G_1=(N_1,\Sigma, P_1, S_1)$, в продукциях которой $S_1$ не встречается в правых частях и для которой $L(G)=L(G_1)$. Шаги 2"/8 алгоритма~\ref{algo-Homsky} позволяют построить по $G_1$ грамматику $G'$, которая, очевидно, имеет нормальную форму Хомского. Чтобы показать, что $L(G_1)=L(G')$, достаточно применить лемму~\ref{lemma-proofOfHomskyMod} к каждой продукции грамматики $G'$, в правую часть которой входит $a'$, а затем применить эту лемму к продукциям с нетерминалами вида $\la X_i\ldots X_j \ra$.
\end{myproof}

Иногда вместо нормальной формы Хомского рассматривает другую, слегка
измененную, каноническую форму. Будем говорить, что КС"/грамматика
$G=(N,\Sigma,P,S)$ представлена в \mydef{модифицированной нормальной
форме Хомского}, если продукции из $P$ имеют один из следующих видов:
\begin{enumerate}
    \item $A\to BC$, где $A,B,C\in N$;
    \item $A\to a$, где $A\in N$, $a\in\Sigma$;
    \item $S\to\eps$, если $\eps\in L(G)$, причем в этом случае
    $S$ не встречается в правых частях продукций.
\end{enumerate}

Чтобы преобразовать произвольную КС"/грамматику к модифицированной нормальной форме Хомского достаточно слегка подправить алгоритм~\ref{algo-Homsky}.

\begin{AlgoEnv}[H]
\label{algo-Homsky-Mod}
\caption{Преобразование КС-грамматики к модифицированной нормальной форме Хомского}

\AlgoPre%
    {КС"/грамматика $G=(N,\Sigma,P,S)$.}
    {КС"/грамматика $G'$ в модифицированной нормальной форме Хомского, у которой $L(G')=L(G)$.}
    {Модификация одного шага алгоритма~\ref{algo-Homsky}.}

\vspace{-5mm}
\varhrulefill[.2pt]

\begin{algoenum}[leftmargin=2cm]
        \item
 Применить к грамматике $G=(N,\Sigma,P,S)$ алгоритм~\ref{algo-NormalGrammar} и построить тем самым приведенную КС"/грамматику $G_1=(N_1,\Sigma,P_1,S_1)$. Перейти на следующий шаг, где начинается построение искомой грамматики $G'=(N',\Sigma,P',S_1)$.

        \item[\textit{Шаги 2--8}]
 повторяют соответствующие шаги алгоритма~\ref{algo-Homsky}.
\end{algoenum}
\end{AlgoEnv}

\begin{myproblem}
Сформулировать и доказать аналог теоремы~\ref{theorem-AlgoHomskyProof} для обоснования алгоритма~\ref{algo-HomskyMod}.
\end{myproblem}

\begin{myexample}
\label{example-GramToHomsky}
Преобразуем КС"/грамматику
\[
G = (\{A;B;S\},\{a;b;c;d\},P,S)
\]
с продукциями
\begin{align*}
	S &\to \alpha AB \mid BA, \\
	A &\to BBB \mid a, \\
	B &\to AS \mid b,
\end{align*}
к нормальной форме Хомского. Для этого применим к грамматике $G$
алгоритм~\ref{algo-Homsky}. В соответствии с первым шагом алгоритма введем новый
стартовый нетерминал $S_1$ и добавим к $P$ одну продукцию $S_1\to S$.
Применим теперь к полученной грамматике алгоритм~\ref{algo-NormalGrammar} и построим
приведенную КС"/грамматику
\[
    G_1 = (\{A;B;S;S_1\},\{a;b;c;d\},P_1,S_1)
\]
с продукциями
\begin{align*}
	S_1 &\to aAB \mid BA, \\
	S &\to aAB \mid BA, \\
    A &\to BBB \mid a, \\
    B &\to AS \mid b
\end{align*}
(в продукциях стартовый нетерминал $S_1$ в правых частях не
встречается). Начнём преобразовывать грамматику $G_1$ к грамматике
$G'$
в нормальной форме Хомского. Из $P_1$ в $P'$ продукции $S_1\to BA$,
$S\to BA$, $A\to a$, $B\to AS$ и $B\to b$ переносим без изменения.
Заменяем продукцию $S_1\to aAB$ продукциями $S\to a' \la AB \ra$  и $
\la AB \ra \to AB$, $S\to aAB$~--- продукциями $S\to a' \la AB \ra$ и $
\la AB \ra \to AB$, а $A\to BBB$~--- продукциями $A\to B \la BB \ra$ и
$ \la BB \ra \to BB$. Наконец, добавляем к $P'$ продукцию $a'\to a$. В
результате получаем грамматику
\[
    G' = (\{A;B;S;S_1;\la AB \ra; \la BB \ra; a'\},\{a,b\},P',S_1),
\]
с продукциями
\begin{align*}
	S_1 &\to a' \la AB\ra \mid BA, &
    S   &\to a'\la AB\ra  \mid BA, \\
    A   &\to B \la BB\ra \mid a, &
    B   &\to AS \mid b, \\
    \la AB \ra &\to AB, &
    \la BB\ra  &\to BB, \\
    a' &\to a.
\end{align*}
Эта грамматика имеет нормальную форму Хомского. По теореме~\ref{theorem-AlgoHomskyProof}
\[
    L(G)=L(G'),
\]
и, следовательно, $G'$"--- искомая грамматика.
\end{myexample}

\begin{myproblem}
Преобразовать КС-грамматику из примера~\ref{example-GramToHomsky} к модифицированной нормальной форме Хомского.
\end{myproblem}

\section{Проблема принадлежности для КС"/языков}
\label{Chapter7ProblemB}
Нормальные формы, как правило, вводятся для удобства формулировки
алгоритмов, решающих конкретные задачи. Нормальная форма Хомского
позволяет, в частности, сформулировать алгоритм решения проблемы
принадлежности для КС"/языков, а именно, по данному слову и данному
КС"/языку, представленному в виде КС"/грамматики его порождающей,
определить, принадлежит ли это слово языку.
\Algo[H]{Алгоритм Кока—Янгера—Касами («CYK-алгоритм»)}
{\label{algo-CYK}КС-грамматика в нормальной форме Хомского $G=(N,\Sigma,P,S)$,
слово $w\in \Sigma^*$ длины $n$.}
{Истина, если $w \in L(G)$, и ложь в противном случае.}
{Последовательное определение нетерминалов, выводящих
всевозможные подстроки $w$ всё большей длины.}
{
\item Для всех $k\in [1,n]_\N$ положить $N_{kk} = \{ A \in N \mid
    A \to w_i \in P\}$. Таким образом учтены все подстроки $w$ длины
    единица.

\item Если $n=1$, завершить алгоритм и вернуть результат
    проверки $S \in N_{11}$. Иначе положить длину
    рассматриваемых подстрок $s = 2$.

\item Положить $i = 1$.

\item
    Положить $j = i + s - 1$. Положить
    \[
        N_{ij} = \{ A \in N \mid A \to BC \in P; \;
                \exists k \in [i, j-1]_\N \colon B \in N_{ik}, \;
                C \in N_{k+1 j} \}.
    \]

\item Увеличить $i$ на единицу. Если $ i < n - s$, перейти
    к шагу 4.

\item Если $s = n$, завершить алгоритм и вернуть результат
    проверки $S \in N_{1n}$. Иначе увеличить $s$ на единицу и
    перейти к шагу 3.
}
Для описания алгоритма~\ref{algo-CYK} используется
одно важное обозначение. Пусть задана
КС"/грамматика $G=(N,\Sigma,P,S)$ и слово $w\in \Sigma^*$.
Для всех $1 \leqslant i \leqslant j \leqslant n$
определим множество
\[
    N_{ij}^w = \{ A \in N \mid A \Rightarrow^*_G w_i \ldots w_j \}.
\]
То есть $N_{ij}^w$ это множество нетерминалов, которые порождают
подстроку $w_i \ldots w_j$ строки $w$ в грамматике $G$. Когда слово
$w$ фиксировано и ясно из контекста, будем опускать верхний индекс
в обозначении $N_{ij}^w$.


\begin{myproblem}[применение CYK-алгоритма к синтаксическому
анализу] Постройте модификацию CYK"/алгоритма, которая позволяет в случае $w \in
L(G)$ давать на выходе вывод слова $w$ в грамматике $G$.
\end{myproblem}

\begin{myremark}[о структуре CYK-алгоритма]
Заметим, что вычисление множеств $N_{ij}$ в алгоритме~\ref{algo-CYK} происходит
таким образом, что на более поздних итерациях (для больших значений $j-i$)
используются результаты более ранних. Такой подход к проектированию алгоритмов,
когда решение задачи определяется через решение нескольких таких же задач,
но меньшего размера,
называется \emph{динамическим программированием}. Отличие от более простого
подхода \emph{разделяй и властвуй}, обычно основанного на прямой рекурсии,
состоит в том, что подзадачи могут перекрываться. Более точно, результат
решения одной подзадачи может использоваться многократно. В такой ситуации
простая рекурсивная реализация будет заведомо неэффективной. Вместо этого
предлагается решать все подзадачи подряд, начиная с самых маленьких, а их
результаты записывать для дальнейшего использования при решении более
крупных подзадач.
\end{myremark}

Как это часто бывает в динамическом программировании, CYK"/алгоритм
удобно проводить с помощью заполнения некоторой таблицы
(см. таблицу~\ref{tab-cyk}).

\begin{table}[H]
\begin{center}
\begin{tabular}{|ccccc}
$N_{15}$ & & & &\\
$N_{14}$ & $N_{25}$ & & &\\
$N_{13}$ & $N_{24}$ & $N_{35}$ & &\\
$N_{12}$ & $N_{23}$ & $N_{34}$ & $N_{45}$ &\\
$N_{11}$ & $N_{22}$ & $N_{33}$ & $N_{44}$ & $N_{55}$\\
\hline
\end{tabular}
\end{center}
\caption{Пример таблицы CYK-алгоритма для $n=5$}
\label{tab-cyk}
\end{table}

\noindent Таблица заполняется снизу вверх и
слева направо. При этом для подсчёта очередного множества будут
использоваться лишь клетки, расположенные ниже текущей.

Рассмотрим в качестве примера, какие пары множеств нетерминалов $N_{ij}$ потребуется
просматривать, чтобы построить $N_{25}$. В соответствии с
описанием алгоритма (шаг~4, формула для $N_{ij}$) это пары $N_{24}$ и $N_{55}$,
$N_{23}$ и $N_{45}$,  $N_{22}$ и $N_{35}$. Они отмечены в
таблице~\ref{cyk-computeN25} разными типами скобок (фигурными,
квадратными и круглыми соответственно), чтобы подчеркнуть
геометрическую последовательность работы алгоритма. Аналогичная
треугольная форма будет появляться при расчете и всех других множеств,
кроме $N_{ii}$, которые строятся непосредственно по грамматике.

\begin{table}[H]
\begin{center}
\begin{tabular}{|ccccc}
$N_{15}$ & & & &\\
$N_{14}$ & \fbox{$N_{25}$} & & &\\
$N_{13}$ & $\{N_{24}\}$ & $(N_{35})$ & &\\
$N_{12}$ & $[N_{23}]$ & $N_{34}$ & $[N_{45}]$ &\\
$N_{11}$ & $(N_{22})$ & $N_{33}$ & $N_{44}$ & $\{N_{55}\}$\\
\hline
\end{tabular}
\end{center}
\caption{Используемые для расчета $N_{25}$ множества}
\label{cyk-computeN25}
\end{table}


\begin{myremark}[о сложности CYK-алгоритма]
Отметим, что с точки зрения теории син\-так\-сического анализа
CYK"/алгоритм проводит \emph{восходящий (bottom—up) анализ}, то есть
восстанавливает дерево вывода, начиная с кроны, а не с корня.
Нетрудно видеть, что сложность CYK"/алгоритма может быть оценена как
$O(n^3 \cdot |P|)$, это ограничивает применение алгоритма на практике.
В теории компиляторов и приложениях чаще рассматривается подкласс КС"/грамматик,
\emph{детерминированные КС"/грамматики} (по-другому, $LL(k)$- и
$LR(k)$"/грамматики), для которых существуют линейные алгоритмы разбора
(сложность $O(n)$).
\end{myremark}


\section{Матричный метод перехода к нормальной форме Грейбах}
\label{Chapter7NFG-MT}

КС"/грамматика $G=(N,\Sigma,P,S)$ называется грамматикой в нормальной
форме Грейбах, если она является неукорачивающеися и каждая продукция
из $P$, отличная от $S\to\eps$, имеет вид $A\to a\alpha$, где
$a\in\Sigma$ и $\alpha\in N^*$. В $[1]$ описан алгоритм пребразования
произвольной КС"/грамматики к нормальной форме Грейбах, основанный на
предварительном устранении левой рекурсии.

Рассмотрим КС"/грамматику $G=(N,\Sigma,P,S)$, в которой нет цепных
правил и $\eps$"/продукций (даже вида $S\to\eps$), и схематично опишем
принадлежащий Розенкранцу матричный метод преобразования такой
грамматики к нормальной форме Грейбах. Этот метод использует технику,
напоминающую технику работы с регулярными выражениями из главы~\ref{Chapter2}.

Дадим необходимые определения. Пусть $\Delta$ и $\Sigma$ --- два
непересекающихся алфавита; алфавит $\Delta$ будем называть
нетерминальным, а алфавит $\Sigma$ --- терминальным. Системой
определяющих уравнений над конечными алфавитами $\Sigma$ и $\Delta$
назовем систему уравнений вида
\begin{equation}
\label{eq621}
A = \alpha_1 + \ldots + \alpha_k, %(6.2.1)
\end{equation}
где $A\in\Delta$ и $\alpha_i\in(\Delta\cup\Sigma)^*$; если $k=0$, то
уравнение имеет вид $A=\es$. Предполагается, что для каждого
$A\in\Delta$ в системе есть только одно уравнение с левой частью $A$.
Решением системы определяющих уравнений назовем такое отображение $f$
множества $\Delta$ в $P(\Sigma^*)$, что если вместо каждого
$A\in\Delta$ подставить $f(A)$ в уравнения системы, то эти уравнения
превратятся в равенства. Решение $f$ назовем наименьшей неподвижной
точкой, если $f(A)\subseteq g(A)$ для любого решения $g$ и любого
$A\in\Delta$. Стандартные системы линейных уравнений с регулярными
коэффициентами из главы~\ref{Chapter2} входят в класс систем определяющих уравнений.

Выделим в $\Delta$ символ $S$. Системе определяющих уравнений над
конечными алфавитами $\Delta$ и $\Sigma$ с выделенным символом
$S\in\Delta$ можно естественным образом сопоставить КС"/грамматику, в
которой $\Delta$ --- нетерминальный алфавит, $\Sigma$ --- терминальный
алфавит, $S$ --- начальные символ, а продукции определяются по каждому
уравнению~\eqref{eq621} исходной системы следующим образом:
\begin{equation}
\label{eq622}
    A \to  \alpha_1 \mid \alpha_2 \mid \ldots \mid \alpha_k.
\end{equation}
Ясно, что разным системам сопоставляются разные грамматики и
приведенная конструкция осуществляет взаимно однозначное соответствие
между множеством всех систем определяющих уравнений над конечными
алфавитами и множеством КС"/грамматик.

Приведем без доказательства несколько результатов о системах
определяющих уравнений, обобщающих результаты о стандартных системах
линейных уравнений с регулярными коэффициентами. Отметим прежде
всего, что при сделанных предположениях наименьшая неподвижная
точка системы определяющих уравнений над алфавитами $\Delta$
и $\Sigma$ существует, единственна и имеет вид
\[
    f(A) = \{\omega\mid A \To_G^* \omega, \; \omega\in\Sigma^*\},
\]
где $G$ --- соответствующая системе КС"/грамматика.

Мы будем пользоваться далее матричным представлением систем
определяющих уравнений. Именно пусть алфавит $\Delta$ состоит из
нетерминалов
\[
A_1, A_2, \ldots , A_n
\]
и
\begin{enumerate}
    \item $\Delta$ --- вектор"/строка $[A_1;A_2;\ldots ;A_n]$;

    \item $R$ --- квадратная матрица порядка $n$, элементами которой
    служат регулярные выражения над $N\cup\Sigma$: стоящий в $i$"/й
    строке и $j$"/м столбце элемент матрицы $R$ определяется равенством
    \[
        R_{ij} = \alpha_1 + \alpha_2 + \ldots + \alpha_k,
    \]
    где $A_i\alpha_1, A_i\alpha_2, \ldots , A_i\alpha_k$ --- все члены
    уравнения для $A_j$, первой буквой которых является $A_i$;

    \item $B$ --- вектор"/строка, состоящая из $n$ регулярных
    выражений над $N\cup\Sigma$: стоящий на $j$"/м месте элемент $B_j$
    определяется как сумма членов уравнения для $A_j$ которые
    начинаются буквой из $\Sigma$.
\end{enumerate}

Таким образом, $B_j$ и $R_{ij}$ --- такие выражения, что уравнение для $A_j$ можно записать в виде
\[
A_j = A_1R_{1j} + A_2R_{2j} + \ldots + A_iR_{ij} + \ldots + A_nR_{nj} + B_j.
\]

Сложение и умножение векторов и матриц определим как обычно, при этом в качестве <<умножения>> элементов матриц будем рассматривать конкатенацию, а в качестве <<сложения>> операцию объединения. Систему определяющих уравнений теперь будем представлять при помощи матриц:
\begin{equation}
\label{eqGeneralSOS}
\Delta=\Delta R+B.
\end{equation}

\begin{myexample}
\label{example-MatrView}
Рассмотрим грамматику $G=(\{A;B\},\{b;a;c;d\},P,A)$ с продукциями
\begin{align*}
    A &\to AaB \mid BB \mid b, \\
    B &\to aA \mid BAa \mid Bd \mid c.
\end{align*}
Построим систему определяющих уравнений для этой грамматики:
\begin{equation*}
\begin{cases}
	A = AaB + BB + b \\
	B = aA + BAa + Bd + c.
\end{cases}
\end{equation*}

При помощи матриц эту систему можно переписать так:
\begin{equation}
[A;B] = [A;B]
\begin{bmatrix}
aB & \es \\
B & Aa+d
\end{bmatrix} + [b;aA+C].
\end{equation}
\end{myexample}

Для системы~\eqref{eqGeneralSOS} можно найти такую равносильную ей систему определяющих уравнений, что все правые части продукций из КС"/грамматики, которая соответствует новой системе, начинаются терминальными буквами.

\begin{mytheorem}
Пусть~\eqref{eqGeneralSOS} --- записанная в матричном виде система определяющих уравнений над алфавитами $\Delta$ и $\Sigma$, $Q$ --- квадратная матрица порядка $b=|\Delta |$ с $n^2$ различными новыми элементами $q_{i,j}$. $\widetilde Q$ --- множество всех элементов $q_{i,j}$, $\widetilde\Delta = \Delta \cup \widetilde Q$. Тогда система определяющих уравнений

\begin{equation}
\label{eqGeneralWithAddQ}
\begin{cases}
	\Delta = BQ + B, \\
    Q = RQ + R
\end{cases}
\end{equation}
над алфавитами $\widetilde\Delta$ и $\Sigma$ имеет наименьшую неподвижную точку, которая совпадает на $\Delta$ с наименьшей неподвижной точкой системы~\eqref{eqGeneralSOS}.
\end{mytheorem}

Приведем алгоритм преобразования КС"/грамматики к нормальной форме Грейбах, основанный на конструкции из этой теоремы.

\AlgoNoMeth[H]{Преобразование к нормальной форме Грейбах}
{\label{algo-Greybah}Приведенная КС"/грамматика $G=(N,\Sigma,P,S)$, не содержащая продукции $S\to\eps$.}
{КС"/грамматика $G'=(N',\Sigma,P',S)$ в нормальной форме Грейбах.}
{
\item По грамматике $G$ построить систему определяющих уравнений $\Delta=\Delta R+B$ над алфавитами $N$ и $\Sigma$.

\item Ввести $Q$ --- квадратную матрицу порядка $n=|N|$, состоящую из новых нетерминальных букв $q_{ij}$. Построить по $Q$ и исходной системе~\eqref{eqGeneralSOS} новую систему~\eqref{eqGeneralWithAddQ} и соответствующую новой системе КС"/грамматику $G_1$. (Так как в $B$ каждая компонента, отличная от $\es$, начинается терминалом, то для $A\in N$ все нетерминальные $A$"/продукции грамматики $G_1$ будут начинаться терминалами; наличие в $B$ ненулевых компонент вытекает из приведенности грамматики $G$.)

\item (Так как $G$ --- приведенная грамматика, то $\eps$ не встречается среди элементов матрицы $R$; поэтому для каждого элемента $q\in Q$ все нетерминальные $q$"/продукции грамматики $G_1$ начинаются символами из $N\cup\Sigma$.) Для каждого нетерминала $A$ из $N$ в тех правых частях нетерминальных $q$"/продукций грамматики $G_1$, которые начинаются буквой $A$, заменить этот нетерминал правыми частями всех $A$"/продукций. В результате получится грамматика, у которой правые части всех продукций начинаются терминалом.

\item Если в правой части продукции терминал $a$ встречается не на первом месте, заменить его новым нетерминалом $a'$ и добавить продукцию $a'\to a$. В итоге получим искомую КС"/грамматику $G'$.$\blacksquare$
}

\begin{mytheorem}
\label{theorem-AlgoGreybahCorrectness}
Алгоритм~\ref{algo-Greybah} строит грамматику $G'$ в нормальной форме Грейбах и $L(G)=L(G')$.
\end{mytheorem}

\begin{myexample}
Рассмотрим КС"/грамматику $G=(\{A;B\},\{b;a;c\},P,A)$ из примера~\ref{example-MatrView}, где реализован первый шаг алгоритма. На втором шаге введем нетерминальную матрицу
\[
    \begin{bmatrix}
        W & X \\
        Y & Z
    \end{bmatrix}
\]
и построим по этой матрице и системе~\ref{eqGeneralWithAddQ} новую систему
\begin{equation}
\label{eq625}
    \begin{cases}
        [A,B]
            = [b,aA+c]
            \begin{bmatrix}
                W & X \\
                Y & Z
            \end{bmatrix}
            + [b,aA+c], \\

        \begin{bmatrix}
            W & X \\
            Y & Z
        \end{bmatrix}
            =
            \begin{bmatrix}
                ab & \es \\
                B & Aa+d
            \end{bmatrix}
            \begin{bmatrix}
                W & X \\
                Y & Z
            \end{bmatrix}
            +
            \begin{bmatrix}
                aB & \es \\
                B & Aa+d
            \end{bmatrix}
            .
    \end{cases}
\end{equation}
Системе~\ref{eq625} соответствует КС"/грамматика с продукциями
\begin{align*}
	A &\to bW \mid aAY \mid cY \mid b, &
    B &\to bX \mid aAZ \mid cZ \mid aA \mid c, \\
    W &\to aBW \mid aB, &
    X &\to aBX, \\
    Y &\to BW \mid AaY \mid dY \mid B, &
    Z &\to BX \mid AaZ \mid dZ \mid Aa \mid d.
\end{align*}
Заметим, что $X$ --- бесполезный символ. Следовательно, полученные продукции можно упростить:
\begin{align*}
    A &\to bW \mid aAY \mid cY \mid b, &
    B &\to aAZ \mid cZ \mid aA \mid c, \\
    W &\to aBW \mid aB, &
    Y &\to BW~|AaY \mid dY \mid B, \\
    Z &\to AaZ \mid dZ \mid Aa \mid d.
\end{align*}
Итак, в результате второго шага алгоритма построена грамматика
\[
    G_1 = (\{A;B;W;Y;Z\},\{a;b;c;d\},P_1,A),
\]
где $P_1$ --- описанное выше множество.

На третьем шаге алгоритма в пяти <<нехороших>> продукциях
\begin{align*}
    Y &\to BW \mid AaY \mid B, \\
    Z &\to AaZ \mid Aa
\end{align*}
нетерминалы $A$ и $B$ надо заменить правыми частями
соответствующих нетерминальных $A$- и $B$"/продукций.
Например, из продукции $Y \to BW$ получаем:
\[
    Y \to aAZW \mid cZW \mid aAW \mid cW,
\]
а из продукции $Z \to AaZ$:
\[
    Z \to bWaZ \mid aAYaZ \mid cYaZ \mid baZ.
\]
В результате третьего шага алгоритма вместо пяти <<нехороших>> продукций мы построили двадцать новых:
\begin{align*}
    Y &\to aAZW \mid cZW \mid aAW \mid cW \mid bWaY \mid aAYaY
        \mid cYaY \mid baY \mid cAZ \mid cZ \mid aA \mid c, \\
    Z &\to bWaZ \mid aAYaZ \mid cYaZ \mid baZ \mid bWa \mid aAYa
        \mid cYa \mid ba.
\end{align*}
Теперь правые части всех продукций начинаются терминалами.

На четвертом шаге алгоритма в тех продукциях, у которых в правой части терминал $x$ встречается не на первом месте, надо заменить его новым нетерминалом $x'$: после этого надо добавить новую продукцию $x'\to x$. В итоге получаем КС"/грамматику $G' = (\{A;B;W;Y;Z;a'\},\{a;b;c;d\},P',A)$ с множеством продукций $P'$:
\begin{align*}
    A &\to bW \mid aAY \mid cY \mid b, \\
    B &\to aAZ \mid cZ \mid aA \mid c, \\
    W &\to aBW \mid aB, \\
    Y &\to aAZW \mid cZW \mid aAW \mid cW \mid bWa'Y \mid cYa'Y
        \mid ba'Y \mid aAZ \mid cZ \mid aA \mid c \mid dY, \\
    Z &\to bWa'Z \mid aAYa'Z \mid cYa'Z \mid ba'Z \mid bWa'
        \mid aAYa' \mid cYa' \mid ba' \mid dZ \mid d, \\
    a' &\to a.
\end{align*}
Эта грамматика имеет нормальную форму Грейбах. С другой стороны, по теореме~\ref{theorem-AlgoGreybahCorrectness} $L(G)=L(G')$.
\end{myexample}

\section{Упражнения}
\label{Chapter7Exs}

\subsection*{Получение нормальных форм}

Привести к нормальной форме Хомского, а также, к нормальной форме Грей\-бах, грамматики с продукциями:
\begin{align*}
    \text{(1) }&
        \begin{aligned}%{l}
            S &\to ASB \mid \varepsilon,\\
            A &\to aAS \mid a,\\
            B &\to SbS \mid A \mid bb;
        \end{aligned}
        \qquad\qquad{}
    &
    \text{(2) }&
        \begin{aligned}%{l}
            S &\to 0A0 \mid 1B1 \mid BB,\\
            A &\to C,\\
            B &\to S \mid A,\\
            C &\to S \mid \varepsilon;
        \end{aligned}
    \\[.3cm]
    \text{(3) }&
        \begin{aligned}%{l}
            S &\to AAA \mid B,\\
            A &\to aA \mid B,\\
            B &\to \varepsilon;
        \end{aligned}
        \qquad\qquad{}
    &
    \text{(4) }&
        \begin{aligned}%{l}
            S &\to aAa \mid bBb \mid \varepsilon,\\
            A &\to C \mid a,\\
            B &\to C \mid b,\\
            C &\to CDE \mid \varepsilon,\\
            D &\to A \mid B \mid ab.
        \end{aligned}
\end{align*}

\subsection*{CYK"/алгоритм}

Используя CYK"/алгоритм,
\begin{enumerate}
    \item
    для грамматики $G$ с продукциями:
    \[
        S \to AB, \qquad
        A \to BB \mid a, \qquad
        B \to AB \mid b
    \]
    определить, принадлежат ли $L(G)$ строки: (а) $aabbb$, (б) $babab$,
    (в) $b^7$;

    \item
    для грамматики $G$ с продукциями:
    \[
        S \to AB \mid BC,\qquad
        A \to BA \mid a,\qquad
        B \to CC \mid b,\qquad
        C \to AB \mid a
    \]
    определить, принадлежат ли $L(G)$ строки: (а) $ababa$, (б) $baaab$, (в) $aabab$.
\end{enumerate}

\chapter{Автоматы с магазинной памятью}
\label{Chapter8FSMSM}
\section{Определения и примеры}
\label{Chapter8Defines}
Перед тем, как изучать этот параграф, было бы хорошо еще раз просмотреть содержание пункта 3.1.

Автомат с магазинной памятью --- это односторонний недетерминированный распознаватель (см. 1.5), в потенциально бесконечной памяти которого элементы информации хранятся и используются так же, как патроны в магазине автоматического оружия, т. е. в каждый момент доступен только верхний элемент магазина. Можно представлять себе магазин в виде слова, причем верхним символом магазина будем считать самую левую букву.

Автомат с магазинной памятью
(сокращенно МП"/автомат)~--- это семерка
\[P=
    (Q,\Sigma,\Gamma,\delta,q_0,Z_0,F),\]
где
\begin{enumerate}
\item $Q$~--- конечное множество состояний;

\item $\Sigma$~--- конечный входной алфавит;

\item $\Gamma$~--- конечный алфавит магазинных символов;

\item функция переходов $\delta$ ---
отображение множества $Q\times(\Sigma\cup\{\eps\}) \times \Gamma$ во множество $P \left (Q \times \Gamma^* \right)$
конечных подмножеств множества $Q \times\Gamma^*$;

\item
$q_0$ $(\in Q)$ --- начальное состояние;

\item
$Z_0$ $(\in\Gamma)$ --- начальный символ
магазина;

\item
$F$ $(\subseteq Q)$ --- множество
заключительных (финальных) состояний.
\end{enumerate}


Конфигурацией МП"/автомата $P$ называется тройка $(q,\omega,\alpha)$ из $Q\times\Sigma^*\times\Gamma^*$, где $q$ --- текущее состояние управляющего устройства; $\omega$ --- непрочтенная часть входного слова (первая буква слова $\omega$ находится под входной головкой; при этом, если $\omega=\eps$, то считается, что все входное слово прочитано); $\alpha$ --- содержимое магазина (самый левый символ слова $\alpha$ отождествляется с верхним символом магазина; при этом, если $a=\eps$, то магазин считается пустым).

Такт работы МП"/автомата $P$ будем представлять бинарным отношением $\vdash_p$ (или, короче, $\vdash$), определенным на конфигурациях. Именно, если $(q,\gamma)\in\delta(q,a,\Sigma)$, то будем писать $(q,a\omega,Z\alpha)\}\vdash(q',\omega,\gamma\alpha)$, где $q,q'\in Q$, $a\in\Sigma\cup\{\eps\}$, $\omega\in\Sigma^*$, $Z\in\Gamma$ и $\alpha,\gamma\in\Gamma^*$. Если $a\neq\eps$, то формула ($q,a\omega,Z\alpha)\vdash(q',\omega,\gamma\alpha)$ говорит о том, что МП"/автомат $P$ сначала находился в состоянии $q$, имел $a$ в качестве текущей входной буквы и $Z$ в качестве верхнего символа магазина; после чего он перешел в новое состояние $q$, сдвинул входную головку на одну ячейку вправо и заменил верхний символ магазина словом $\gamma$, составленным из магазинных букв. Если же $a=\eps$ (такой такт называется $\eps$"/тактом), то текущая входная буква не принимается во внимание и входная головка не двигается, однако состояние управляющего устройства и содержимое памяти могут измениться.

Подчеркнем, что $\eps$"/такт может происходить и тогда, когда все входное слово прочитано; при пустом же магазине следующий такт невозможен.

Обычным образом вводятся обозначения $\vdash_P^i$ для $i\ge 0$, $\vdash_P^*$ и $\vdash_P^+$ (далее значок $P$ мы часто будем в этих обозначениях пропускать).

Начальной конфигурацией МП"/автомата $P$ называется конфигурация вида $\{q_0,\omega,\Sigma_0\}$; в этом случае управляющее устройство находится в начальном состоянии, входная лента содержит некоторое слово из $\Sigma^*$, а в магазине есть только начальный символ $Z_0$.

Заключительная конфигурация --- это конфигурация вида $(q,\eps,\alpha)$, где
\[
    q\in F \quad \text{и} \quad \alpha\in\Gamma^*.
\]
Говорят, что слово $\omega$ допускается
МП"/автоматом $P$, если $(q_0,\omega,Z_0)\vdash^*
(q,\eps,\alpha)$ для некоторых $q\in F$ и
$\alpha\in\Gamma^*$. Языком, определяемым (или
допускаемым) автоматом $P$, называют множество всех
слов, допускаемых автоматом $P$; этот язык
обозначается $L(P)$.

Как и в случае обычных автоматов, недетерминированность удобно интепретировать как наличие нескольких параллельно работающих экземпляров исходного МП"/автомата.

\begin{myexample}
Для того, чтобы задать язык
\[L=\{0^n1^n\mid n\ge 0\},\]
рассмотрим МП"/автомат
\[P=(\{q_0;q_1;q_2\},\{0;1\},\{Z;0\},\delta,q_0,Z,\{q_0\}),\]
где
\[
\delta(q_0,0,Z)=\{(q_1,0Z)\}, \quad
    \delta(q_1,0,0)=\{(q_1,00)\}, \quad
    \delta(q_1,1,0)=\{(q_2,\eps)\},
\]
а для остальных элементов из $Q\times(\Sigma\cup\{\eps\})\times\Gamma$ функция переходов не определена.

Работа автомата $P$ состоит в том, что он копирует в магазин состоящий из нулей префикс входного слова, а затем устраняет из магазина по одному нулю на каждую единицу, которую он обнаруживает на входе.

Например, для входного слова $0011$ автомат $P$ проделает такую последовательность тактов:
\[
    (q_0,0011,Z) \vdash (g_1,011,0Z) \vdash
        (q_1,11,00Z) \vdash (q_2,1,0Z) \vdash
        (q_2,\eps,Z)\vdash(q_0,\eps,\eps).
\]

Докажем, что $L\subseteq L(P)$. Прежде всего заметим, что осуществимы следующие пять последовательностей тактов:
%\begin{equation*}
%\begin{array}{l}
\begin{align*}
(q_0,0,Z)           &\vdash      (q_1,\eps,0Z), \\
(q_1,0^i,0Z)        &\vdash^i    (q_1,\eps,0^{i+1}Z), \\
(q_1,1,0^{i+1}Z)    &\vdash      (q_2,\eps,0^iZ), \\
(q_2,1^i,0^iZ)      &\vdash^i    (q_2,\eps,Z), \\
(q_2,\eps,Z)       &\vdash      (q_0,\eps,\eps).
\end{align*}
%\end{array}
%\end{equation*}
Отсюда вытекает, что
\[
(q_0,\eps,Z) \vdash^0 (q_0,\eps,Z)
\]
и для $n \ge 1$ возможна последовательность
\[
(q_0,0^n1^n,Z) \vdash^{2n+1} (q_0, \eps, \eps).
\]
Таким образом, $L\subseteq L(P)$.

Докажем, что $L\supseteq L(P)$. Анализ функции переходов показывает, что если $P$ допускает непустое слово, то он обязан последовательно пройти через состояния $q_0, q_1, q_2, q_0$. Если для $i\ge 1$
\[
(q_0,\omega,Z)\vdash^i(q_1,\eps,\alpha),
\]
то $\omega=0^i$, $\alpha=0^iZ$. Аналогично, если для $i\ge 1$
\[
(q_2,\omega,\alpha)\vdash^i(q_2,\eps,\beta),
\]
то $\omega=1^i$, $\alpha=0^i\beta$. Такт
\[
(q_1,\omega,\alpha)\vdash(q_2,\eps,\beta)
\]
возможен тогда и только тогда, когда $\omega=1$, $\alpha=0\beta$, а последовательность тактов
\[
(q_2,\omega,Z)\vdash^*(q_0,\eps,\eps)
\]
возможна только и только тогда, когда $\omega=\eps$. Таким образом, если
\[
(q_0,\omega,Z) \vdash^i (q_0,\eps,\alpha)
\]
для некоторого $i\ge 0$, то либо $\omega=\omega$ и $i=0$, либо $\omega=0^n1^n$, $i=2n+1$ и $\alpha=\eps$. Следовательно, $L\supseteq L(P)$.
\end{myexample}

\begin{myexample}
Построим МП"/автомат, допускающий язык
\[
L = \{\omega\omega^R\mid\omega\in\{a,b\}^+\}.
\]
Пусть $P=(\{q_0;q_1;q_2\},\{a;b\},\{Z;a;b\},\delta,q_0,Z,\{q_2\}$, где
\begin{enumerate}
\item $\delta(q_0,a,Z) = \{(q_0,aZ)\}$,
\item $\delta(q_0,b,Z) = \{(q_0,bZ)\}$,
\item $\delta(g_0,a,a) = \{(g_,aa),(q_1,\eps)\}$,
\item $\delta(q_0,a,b) = \{(q_0,a,b)\}$,
\item $\delta(q_0,b,a) = \{(q_0,ba)\}$,
\item $\delta(q_0,b,b) = \{(g_0,bb),(q_1,\eps)\}$,
\item $\delta(q_1,a,a) = \{(q_1^\eps)\}$,
\item $\delta(q_1,b,b) = \{(q_1,\eps)\}$,
\item $\delta(q_1,\eps,Z) = \{(q_2,\eps)\}$,
\end{enumerate}
а для остальных элементов из $Q\times(\Sigma\cup\{\eps\})\times\Gamma$ функция пареходов не определена.

Опишем работу МП"/автомата $P$. Сначала $P$ копирует в магазине какую"/то часть входного слова по правилам (1), (2), (4), (5) и первым альтернативам правил (3) и (6). Однако в том случае, когда текущая входная буква совпадет с верхним символом магазина, автомат может перейти (если пожелает!) в состояние $q_1$ и начать сравнивать слово в магазине с оставшейся частью входного слова. Эта возможность гарантируется вторыми альтернативами правил (3) и (6), а по правилам (7) и (8) происходит сравнение символов. Если в хода сравнения обнаруживается несовпадение очередных букв, то соответствующий экземпляр МП"/автомата <<умирает>>. Однако в силу недетерминированности разные экземпляры $P$ могут проделывать все возможные для него такты, и если реализация какой"/нибудь последовательности тактов приводит к тому, что $Z$ снова оказывается верхним (и единственным!) символом магазина, то по правилу (8) $P$ стирает $Z$ и попадает в состояние $q_2$. МП"/автомат $P$ должен допустить слово тогда и только тогда, когда все сравнения обнаружили совпадение букв.

Рассмотрим работу МП"/автомата $P$ в случав, когда $aabbaa$ --- входное слово. Ясно, что среда прочих возможны <<смертельные>> последовательности тактов, например:
\[
(q_0,aabbaa,Z) \vdash (q_0,abbaa,aZ) \vdash (q_1,bbaa,Z)
\]
или
\begin{multline*}
(q_0,aabbaa,Z) \vdash (q_0,abbaa,aZ) \vdash (q_0,bbaa,aaZ) \vdash (q_0,baa,baaZ) \vdash \\ (q_0,aa,bbaaZ) \vdash (q_0,a,abbaaZ) \vdash (q_0,\eps,aabbaaZ).
\end{multline*}
С другой стороны, возможна последовательность, оканчивающаяся заключительной конфигурацией:
\[
(q_0,abba,Z) \vdash (q_0,bba,aZ) \vdash (q_0,ba,baZ) \vdash (q_1,a,aZ) \vdash (q_1,\eps,Z) \vdash (q_2,\eps,\eps).
\]
Это означает, что МП"/автомат $P$ допускает входное слово $aabbaa$.

Докажем, что $L\subseteq L(P)$. Пусть $\omega=c_1c_2\ldots c_{n-1}c_nc_{n-1}\ldots c_1$, где $C_i\in\{a,b\}$ для $1\le i\le n$. Возможна следующая последовательность тактов:
\begin{multline*}
(q_0,\omega,Z) \vdash^n (q_0,c_n,c_{n-1}\ldots c_1,c_nc_{n-1}\ldots c_1Z) \vdash \\ (q_1,c_{n-1}\ldots c_1,c_{n-1}\ldots c_1,c_{n-1}\ldots c_1Z) \vdash^{n-1} (q_1,\eps,Z) \vdash {q_2,\eps,\eps}.
\end{multline*}
Таким образом, $L\subseteq L(P)$.

Доказательство вложения $L\supseteq L(P)$ не приводим.
\end{myexample}

\begin{myproblem}
Завершить доказательство равенства $L=L(P)$ в примере 7.1.2, то есть доказать, что если $(q_0,\omega,Z)\vdash^*(q_2,\eps,\alpha)$, где $\alpha\in\Gamma^*$, то $\omega=xx^R$ для некоторого $x\in(a+b)^+$ и $\alpha=\eps$.
\end{myproblem}

\begin{myproblem}
Для произвольного МП"/автомата
\[
P=(Q,\Sigma,\Gamma,\delta,q_0,Z_0,F)
\]
доказать, что если $(q,\omega,A)\vdash^n(q',\eps,\eps)$, то $(q,\omega,A\alpha)\vdash^n(q',\eps,\alpha)$ для всех $A\in\Gamma$ и $\alpha\in\Gamma^*$. Этот факт можно было бы сформулировать так: <<То, что происходит с верхним символом магазина, не зависит от того, что находится в магазине под ним>>.
\end{myproblem}

\section{Расширенный МП"/автомат}
\label{Chapter8FSMSMVariants}

Напомним, что МП"/автомат мог на каждом такте заменять лишь один верхний символ магазина. Теперь определение МП"/автомата будет слегка изменено, именно, автомату будет позволено заменять за один такт какой-нибудь магазинный префикс.

\mydef{Расширенным МП"/автоматом} (РМП"/автоматом) назовем семерку
\[
P=(Q,\Sigma,\Gamma,\delta,q_0,Z_0,F),
\]
где $\delta$ --- отображение конечного подмножества множества
$Q\times(\Sigma\cup\{\eps\})\times\Gamma^*$ во множество конечных подмножеств множества $Q\times\Gamma^*$, а все другие символы имеют такой же смысл, как и в~\ref{Chapter8Defines}.

Ясно, что каждый обычный МП"/автомат является РМП"/автоматом. Конфигурация определяется как и прежде. Мы будем писать
\[
(q,a\omega,\alpha\gamma)\vdash(q',\omega,\beta\gamma),
\]
если $(q',\beta)\in\delta(q,a,\alpha)$, где $q\in Q$, $a\in\Sigma\cup\{\eps\}$, $\alpha,  \beta, \gamma\in\Gamma^*$. Языком $L(P)$, определяемым РМП"/автоматом $P$, называется множество всех таких слов $\omega$, что $(q_0,\omega,Z_0)\vdash^*(q,\eps,\alpha)$ для некоторых $q\in Q$ и $\alpha \in\Gamma^*$.

Отметим, что в отличие от МП"/автомата РМП"/автомат обладает способностью продолжать работу и тогда, когда магазин пуст.

\begin{myexample}
\label{example-langwwr-rmp}
Построим РМП"/автомат $P$, распознающий язык $L=\{\omega\omega^R\mid \omega\in\{a,b\}^*\}$. Пусть $P=(\{q;p\},\{a;b\},\{a;b;S;Z\},\delta,q,Z,\{p\})$, где

\begin{enumerate}
\item $\delta(q,a,\eps) = \{(q,a)\}$,
\item $\delta(q,b,\eps) = \{(q,b)\}$,
\item $\delta(q,\eps,\eps) = \{(q,S)\}$,
\item $(q,\eps,aSa) = \{(q,S)\}$,
\item $(q,\eps,bSb) = \{(q,S)\}$,
\item $(q,\eps,SZ) = \{(p,\eps)\}$,
\end{enumerate}
а для остальных элементов из $Q\times(\Sigma\cup\{\eps\})*\Gamma$ функция переходов не определена.

Сначала автомат $P$ записывает в магазине некоторый префикс входного слова (правила (1),(2)). Далее автомат может предположить, что середина слова достигнута, и записать верхним символом магазина маркер $S$ (правило (3)). Можно проверить, что если автомат неправильно угадает середину слова, то рано или поздно он обязательно <<умрет>>, но если же автомат в нужном слове угадает середину правильно, то он имеет возможность выжить и достигнуть заключительного состояния. После угадывания середины слова автомат $P$ помещает в магазин очередную входную букву и заменяет в магазине $aSa$ или $bSb$ на $S$ (правила (1),(4) или (2),(5)). Автомат $P$ работает до тех пор, пока не исчерпается все входное слово. Если после этого в магазине останется слово $SZ$, то $P$ сотрет его и тем самым будет получена заключительная конфигурация.

Рассмотрим работу РМП"/автомата $P$ в случае, когда $aabbaa$ --- входное слово. Разумеется, как это обычно бывает, среди прочих возможны <<смертельные>> последовательности тактов, например:
\begin{multline*}
(q,aabbaa,Z) \vdash (q,abbaa,aZ) \vdash (q,abbaa,SaZ) \vdash (q,bbaa,aSaZ) \vdash \\ (q,bbaa,SZ) \vdash (q,baa,bSZ) \vdash (q,baa,SbSZ) \vdash (q,aa,bSbSZ) \vdash \\ (q,aa,SSZ) \vdash (q,a,aSSZ) \vdash (q,a,SaSSZ) \vdash (q,a,SSZ) \vdash (q,\eps,aSSZ)
\end{multline*}
или
\begin{multline*}
(q,aabbaa,Z) \vdash (q,abbaa,aZ) \vdash (q,bbaa,aaZ) \vdash (q,bbaa,SaaZ) \vdash \\ (q,aa,bbSaaZ) \vdash (q,a,abbSaaZ) \vdash (q,\eps,aabbSaaZ).
\end{multline*}
С другой стороны, возможна последовательность, оканчивающаяся заключительной конфигурацией:
\begin{multline*}
    (q,aabbaa,Z) \vdash
    (q,abbaa,aZ) \vdash
    (q,bbaa,aaZ) \vdash
    (q,baa,baaZ) \\
    %
    \vdash
    (q,baa,SbaaZ) \vdash
    (q,aa,bSbaaZ) \vdash
    (q,aa,SaaZ) \vdash
    (q,a,aSaaZ) \\
    %
    \vdash
    (q,a,SaZ) \vdash
    (q,\eps,aSaZ) \vdash
    (q,\eps,SZ) \vdash
    (q,\eps,\eps).
\end{multline*}
Это означает, что РМП"/автомат $P$ допускает входное слово $aabbaa$.
\end{myexample}

\begin{myproblem}
Доказать, что построенный в примере~\ref{example-langwwr-rmp} РМП"/автомат $P$, действительно распознает язык $L=\{\omega\omega^R\mid\omega\in\{a,b\}^*\}$.
\end{myproblem}

\begin{mytheorem}
\label{theorem-eqMPandRMP}
Класс языков, определяемых МП"/автоматами, совпадает с классом языков, определяемых РМП"/автоматами.
\end{mytheorem}

\begin{myproof}
Пусть $P=(Q,\Sigma,\Gamma,\delta,q_0,Z_0,F)$ --- произвольный РМП"/автомат. Для доказательства теоремы достаточно построить такой МП"/автомат $P_1$, что $L(P_1)=L(P)$. Ниже будет предъявлена конструкция автомата и приведена схема доказательства равенства.

Введем обозначение:
\[
    m = \max\left\{|\alpha|: \delta(q,a,\alpha)\neq\es, \alpha\in\Gamma^*,
                q\in Q, a\in\Sigma\cup\{\eps\}\right\}.
\]
Построим МП"/автомат $P_1$, который будет моделировать автомат $P$, храня верхние $m$ символов его магазина в <<буфере>> длины $m$, занимающим часть памяти управляющего устройства автомата $P_1$. При этом автомат $P_1$ сможет сообщить в начале каждого такта, каковы $m$ верхних символов магазина автомата $P$. Если в некотором такте $P$ заменяет слово из $k$ верхних символов магазина словом из $l$ символов, то $P_1$ заменит $k$ первых символов в буфере этим словом длины $l$. Если $l<k$, то $P_1$ сделает $k-i$ вспомогательных $\eps$"/тактов, в течение которых $k-i$ символов перейдут из верхней части магазина в буфер управляющего устройства; после этого буфер окажется заполненным, и $P_1$ будет готов моделировать очередной такт автомата $P$. Если $i>k$, то символы передаются из буфера в магазин. В качестве состояний МП"/автомата $P_1$ будут рассматриваться упорядоченные пары $[q,\alpha]$, где $q$ --- состояние из множества $Q, \alpha (\in\Gamma_1^*)$ --- буфер, $0\le|\alpha|\le m$.

Итак, рассмотрим МП"/автомат $P_1=(Q_1,\Sigma,\Gamma_1,\delta_1,q_1,Z_1,F_1)$, где
\begin{itemize}
\item[] $\Gamma_1 = \Gamma\cup\{Z_1\}$,
\item[] $Q_1 = \{[q,\alpha]\mid q\in Q, \alpha\in\Gamma^*_1, 0\le\alpha\le m\}$,
\item[] $q_1 = [q_0,Z_0Z_1^{m-1}]$,
\item[] $F_1 = \{[q,\alpha]\mid q\in F, \alpha\in\Gamma_1^*\}$,
\end{itemize}
а функция переходов $\delta_1$ определяется по правилам:
\begin{enumerate}
    \item если $(r,Y_1 \ldots Y_l)\in\delta(q,a,X_1\ldots X_k)$, то
    для всех $Z\in\Gamma_1$ и всех таких $\alpha\in\Gamma_1^*$,
    у которых $|\alpha|=m-k$ положим

    \begin{enumerate}
        \renewcommand{\labelenumii}{\arabic{enumi}.\arabic{enumii})}
        \item при $l\ge k$ для всех $\beta\in \Gamma_{1}^*$, у которых
        $\beta\gamma=Y_1\ldots Y_l\alpha$ и $|\beta|=m$:
        %
        \[
        ([r,\beta],\gamma Z)\in\delta_1([q,X_1\ldots X_k\alpha],a,Z),
        \]

        \item при $l<k$:
        %
        \[
        ([r,Y_1\ldots Y_l\alpha Z],\eps) \in
            \delta_1([q,X_1\ldots X_k\alpha],a,Z);
        \]
    \end{enumerate}

    \item $\delta_1([q,\alpha],\eps,Z)=\{([q,\alpha Z],\eps)\}$ для
    всех $q\in Q$, $\Sigma\in\Gamma_1$ и всех таких
    $\alpha\in\Gamma_1^*$, у которых $|\alpha|<m$.

\end{enumerate}
По этим правилам осуществляется заполнение буфера управляющего устройства, который содержит $m$ символов. Отметим, что в начальный момент буфер содержит символ $Z_0$ наверху и $(m-1)$ символов $Z_1$ ниже. Символ $Z_1$ используется как специальный маркер, отмечающий <<дно>> магазина.

Анализ конструкции автомата $P_1$ позволяет показать, что такт
\[
    (q,a\omega,X_1\ldots X_kX_{k+1}\ldots X_n) \vdash_P
        (r,\omega,Y_1 \ldots Y_lX_{k+1} \ldots X_n)
\]
происходит тогда и только тогда, когда
\[
    ( [q,\alpha],a\omega,\beta ) \vdash_{P_1}^+
        ([r,\alpha'],\omega,\beta' ),
\]
где
\[
    \alpha\beta = X_1\ldots X_nZ_1^m, \alpha'\beta' =
        Y_1 \ldots Y_lX_{k+1} \ldots X_nZ_1^m, |\alpha| = |\alpha'| = m,
\]
и между двумя
конфигурациями $([q,\alpha],a\omega,\beta)$ и
$([r,\alpha'],\omega,\beta')$ МП"/автомата $P_1$ нет ни
одной, в которой состояние имело бы вторую компоненту (буфер) длины
$m$.

Таким образом, для некоторых $q\in F$ и $\alpha\in\Gamma^*$ такт
\[
(q_0,\omega,\Sigma_0) \vdash_P^* (q,\eps,\alpha)
\]
происходит тогда и только тогда, когда
\[
([q_0,Z_0Z_1^{m-1}],\omega,Z_1) \vdash_{P_1}^*([q,\beta],\eps,\gamma),
\]
где $|\beta|=m$ и $\beta\gamma=\alpha\Sigma_1^m$. Отсюда вытекает, что $L(P_1)=L(P)$.
\end{myproof}

Теорема~\ref{theorem-eqMPandRMP} предоставляет и обосновывает все необходимые конструкции для перехода от РМП к МП"/автомату. Сформулируем на основе этой теоремы следующий алгоритм.

\Algo[H]{Построение МП"/автомата по РМП"/автомату}
{\label{algo-RMPtoMP} РМП-автомат $P = (Q, \Sigma, \Gamma, q_0, Z_0, F)$. }
{МП"/автомат $P' = (Q_1, \Sigma, \Gamma_1, q_1, Z_1, F_1)$, такой что $L(P') = L(P).$}
{ конструирование элементов автомата $P'$ по правилам теоремы~\ref{theorem-eqMPandRMP}.}
{
\item Положить $m = max \{|\alpha| \mid \alpha \in \Gamma^*, \delta(q, a, \alpha) \neq \es, q \in Q, a \in \Sigma \cup \{\eps\}\}$.

\item Положить $\Gamma_1 = \Gamma \cup \{Z_1\}$.

\item Положить $Q_1 = \{ [q, \alpha] \mid q \in Q \cup \{ q_{new} \}, \alpha \in \Gamma_1^*, 0 \leq|\alpha| \leq m \}$.

\item Определить $\delta_1$ следующим образом:
	\begin{enumerate}
		\item если $\delta(q, a, X_1 \ldots X_k) \ni (r, Y_1 \ldots Y_l)$ и $l \geq k$, то для всех $Z \in \Gamma_1$ и $\alpha \in \Gamma_1^*, |\alpha| = m - k$,
		\[
			\delta_1([q, X_1 \ldots X_k\alpha], a, Z) \ni ([r, \beta], \gamma Z),
		\]
		где $\beta\gamma = Y_1\ldots Y_l\alpha$ и $|\beta| = m $;
		\item если $\delta(q, a, X_1 \ldots X_k) \ni (r, Y_1 \ldots Y_l)$ и $l \le k$, то для всех $Z \in \Gamma_1$ и $\alpha \in \Gamma_1^*, |\alpha| = m - k$,
		\[
			\delta_1([q, X_1 \ldots X_k\alpha], a, Z) \ni ([r, Y_1 \ldots \alpha Z], \eps);
		\]
		\item для всех $q \in Q, Z \in \Gamma_1$ и $\alpha \in \Gamma_1^*, |\alpha| \le m$,
		\[
			\delta_1(q_1, \eps, Z_1) = \{ ([q, \alpha Z], \eps) \};
		\]
		\item $\delta_1(q_1, \eps, Z_1) = \{ ([q_0, Z_0Z_1^{m-1}], Z_1Z_1) \}$.
  \end{enumerate}  
\item Положить $q_1 = [q_{new}, Z_1^m]$.

\item Положить $F_1 = \{ [q, \alpha] \mid q \in F, \alpha \in \Gamma_1^* \}$.

\item Вернуть МП"/автомат $P' = (Q_1, \Sigma, \Gamma_1, \delta_1, q_1, Z_1, F_1)$ в качестве результата.
} 

\section{Автомат, допускающий слово опустошением магазина }
\label{MPeps-fsm}
Напомним, что согласно данным ранее определениям, слово $\omega$ из
$\Sigma^*$ допускается РМП"/автоматом $P=
(G,\Sigma,\Gamma,\delta,q_0,Z_0,F)$ тогда и только тогда, когда
\[
(q_0,\omega,Z_0)\vdash^*(q,\eps,\alpha)
\]
для некоторых $q\in F$ и
$\alpha\in\Gamma^*$, а языком $L(P)$, определяемым автоматом $P$,
называют множество всех слов, допускаемых автоматом $P$. Теперь изменим условие допускаемости слова.

Пусть $P=(Q,\Sigma,\Gamma,\delta,q_0,Z_0,P)$ --- РМП"/автомат. Будем говорить, что автомат $P$ допускает слово $\omega\in\Sigma^*$ \mydef{опустошением магазина}, если $(q_0,\omega,Z_0)\vdash^+(q,\eps,\eps)$ для некоторого $q\in Q$. Пусть $L_\eps(P)$ --- множество всех слов, допускаемых автоматом $P$ опустошением магазина. Далее, если нас в автомате (МП или РМП) будет интересовать только язык $L_\eps(P)$, то такие автоматы будем называть МП$\eps$"/автомат и РМП$\eps$"/автомат соответственно.

Если два РМП"/автомата. $P$ и $R$, отличаются друг от друга только множествами заключительных состояний, то $L_\eps(P)=L'_\eps(R)$, хотя, разумеется, языки $L(P)$ и $L(R)$ не обязаны совпадать. Поэтому при изучении языков, допускаемых РМП-автоматами опустошением магазина, обычно рассматривают только такие автоматы, у которых $P=\es$.

Выясним, как связано новое условие допускаемости слов с прежним.

\begin{mytheorem}
\label{theorem-eqMPandMPeps}
Пусть $P=(Q,\Sigma,\Gamma,\delta,q_0,Z_0,F)$ --- РМП"/автомат. Тогда можно построить такой МП$\eps$"/автомат $P'$, что $L_\eps(P')=L(P)$.
\end{mytheorem}

\begin{myproof}
Ввиду теоремы~\ref{theorem-eqMPandRMP} будем, не теряя общности, полагать, что $P$ --- МП"/автомат.

Предположим, что автомат $P'$ моделирует действия автомата $P$, и прикинем, каким требованиям он обязан в этом случае удовлетворять. Введем специальное состояние $q_\eps$, которое позволяет опустошать магазин; всякий раз, когда $P$ переходит в заключительное состояние, $P'$ должен решать, продолжать ли моделирование $P$ или перейти в состояние $q_\eps$. Второй важный момент, который надо учесть, состоит в том, что для некоторого входного слова $\omega$ автомат $P$ может сделать последовательность тактов, приводящую к опустошению магазина, но управляющее устройство окажется при этом не в заключительном состоянии; тогда для того, чтобы помешать $P'$ допустить в этом случае слово $\omega$, надо добавить к $P'$ специальный маркер $Z'$, отмечающий <<дно>> магазина, который автомат $P'$ может устранить только в состоянии $q_\eps$. Этот же символ, $Z'$, будет начальным символом магазина.

Итак, пусть
\[
P' = (Q\cup\{q_\eps;q'\},\Sigma,\Gamma\cup\{Z'\},\delta',q',Z',\es),
\]
где $\delta'$ определяется так:
\begin{enumerate}
    \item $\delta'(q',\eps,Z')=\{(q_0,Z_0,Z')\}$;

    \item $\forall q\in Q$, $\forall a\in\Sigma\cup\{\eps\}$,
    $\forall Z\in\Gamma$:
        $(r,\gamma)\in\delta(q,a,Z) \To (r,\gamma)\in\delta'(q,a,Z)$;

    \item $\forall q\in F$, $\forall Z\in\Gamma\cup\{Z'\}$:
        $(q_\eps,\eps)\in\delta'(q,\eps,Z)$;

    \item $\forall Z\in\Gamma\cup\{Z'\}$:
        $\delta'(q_\eps,\eps,Z)=\{(q_\eps,\eps)\}$.
\end{enumerate}

Отметим, что на первом такте автомат $P'$ записывает в магазин $Z_0Z'$ и переходит в начальное состояние $q_0$ автомата $P$, a $Z'$ начинает играть роль маркера, отмечающего <<дно>> магазина.

Анализ функций переходов $\delta$ и $\delta'$ показывает, что для некоторых натуральных $r$ и $n$, произвольного $q$ из $F$ и произвольных слов $Y_1, \ldots , Y_{r-1}, Y_r=Z'$ из $\Gamma^*$ последовательность тактов автомата $P'$
\[
    (q',\omega,Z')
        \vdash_{P'} (q_0,\omega,Z_0,Z')
        \vdash_{P'}^n (q,\eps,Y_1 \ldots Y_r)
        \vdash_{P'} (q_\eps,\eps,Y_2 \ldots Y_r)
        \vdash_{P'}^{r-1} (q_\eps,\eps,\eps)
\]
осуществима тогда и только тогда, когда
\[
    (q_0,\omega,Z_0) \vdash_P^n (q, \eps, Y_1 \ldots Y_{r-1}).
\]
Следовательно, $L_\eps(P')=L(P)$.
\end{myproof}

Сформулируем основной результат теоремы~\ref{theorem-eqMPandMPeps} в виде следующего алгоритма.

\Algo[H]{Построение МП$\eps$"/автомата по МП"/автомату}
{\label{algo-MPtoMPeps} МП-автомат $P = (Q, \Sigma, \Gamma, q_0, Z_0, F)$. }
{МП"/автомат $P' = (Q_1, \Sigma, \Gamma_1, q_1, Z_1, \es)$, такой что $L_\eps(P') = L(P).$}
{ конструирование элементов автомата $P'$ по правилам теоремы~\ref{theorem-eqMPandMPeps}.}
{
\item Положить $Q_1 = Q \cup \{ q_\eps, q_1 \}$.

\item Положить $\Gamma_1 = \Gamma \cup \{Z_1\}$.

\item Определить $\delta_1$ следующим образом:
	\begin{enumerate}
		\item $\delta_1(q_1, \eps, Z_1) = \{ (q_0, Z_0Z_1) \}$;
		\item если $\delta(q, a, Z) \ni (r, \gamma)$, то $\delta_1(q, a, Z) \ni (r, \gamma)$ для всех $q \in Q, a \in \Sigma \cup \{ \eps \}$ и $Z \in \Gamma$;
		\item $\delta_1(q, \eps, Z) \ni (q_\eps, \eps)$ для всех $a \in F$ и $Z \in \Gamma_1$;
		\item $\delta_1(q_\eps, \eps, Z) = \{ (q_\eps, \eps) \}$ для всех $Z \in \Gamma_1$.
  \end{enumerate}  

\item Вернуть МП"/автомат $P' = (Q_1, \Sigma, \Gamma_1, \delta_1, q_1, Z_1, \es)$ в качестве результата.
} 

Заметим, что алгоритм~\ref{algo-MPtoMPeps} <<заворачивает>> исходный МП"/автомат в автоматную обёртку, цель которой --- очистить магазин после перехода исходного автомата в завершающую конфигурацию. 

Справедливо обращение теоремы~\ref{theorem-eqMPandMPeps}.

\begin{mytheorem}
\label{theorem-eqRMPandMPeps}
Пусть $P=(Q,\Sigma,\Gamma,\delta,q_0,\Sigma_0,\es)$ --- РМП$\eps$"/автомат. Тогда можно построить такой МП"/автомат $P'$, что $L(P')=L_\eps(P)$.
\end{mytheorem}

\begin{myproof}
Ввиду теоремы~\ref{theorem-eqMPandRMP} будем, не теряя общности, полагать, что
$P$~--- МП"/автомат.

Как и при доказательстве теоремы~\ref{theorem-eqMPandMPeps} предположим, что автомат $P'$ моделирует действия автомата $P$, и выясним, каким требованиям он обязан в этом случае удовлетворять. Введем заключительное состояние $g_f$ и маркер $Z'$ , отмечающий <<дно>> магазина нового автомата ($Z'$ будет, кроме того, начальным магазинным символом). В тот момент, когда автомат $P'$ может прочесть $Z'$ , он будет переходить в новое заключительное состояние $q_f$.

Итак, пусть
\[
P' = (Q\cup\{q_f;q'\},\Sigma,\Gamma\cup\{Z'\},\delta' ,q' ,Z' ,\{q_f\}),
\]
где $\delta'$ определяется так:
\begin{enumerate}
    \item $\delta' (q',\eps,Z')=\{(q_0,Z_0,Z')\}$;

    \item $\forall q\in Q$, $\forall a\in\Sigma\cup\{\eps\}$,
    $\forall\ Z\in\Gamma$:
        $(r,\gamma)\in\delta(q,a,Z)\To(r,\gamma)\in\delta'(q,a,Z)$;

    \item $\forall q\in Q$: $\delta'(q,\eps,Z')=\{(q_f,\eps)\}$.
\end{enumerate}

Доказательство равенства $L(P')=L_\eps(P)$ не приводим.
\end{myproof}

\begin{myproblem}
Доказать равенство $L(P')=L_\eps(P)$ из теоремы~\ref{theorem-eqMPandMPeps}.
\end{myproblem}

Сформулируем основной результат теоремы~\ref{theorem-eqRMPandMPeps} в виде следующего алгоритма.

\Algo[H]{Построение МП"/автомата по МП$\eps$"/автомату}
{\label{algo-RMPtoMPeps} МП$\eps$-автомат $P = (Q, \Sigma, \Gamma, q_0, Z_0, \es)$. }
{МП"/автомат $P' = (Q_1, \Sigma, \Gamma_1, q_1, Z_1, F_1)$, такой что $L_(P') = L\eps(P).$}
{ конструирование элементов автомата $P'$ по правилам теоремы~\ref{theorem-eqRMPandMPeps}.}
{
\item Положить $Q_1 = Q \cup \{ q_f, q_1 \}$.

\item Положить $\Gamma_1 = \Gamma \cup \{Z_1\}$.

\item Положить $F_1 = \{q_f\}$.

\item Определить $\delta_1$ следующим образом:
	\begin{enumerate}
		\item $\delta_1(q_1, \eps, Z_1) = \{ (q_0, Z_0Z_1) \}$;
		\item если $\delta(q, a, Z) \ni (r, \gamma)$, то $\delta_1(q, a, Z) \ni (r, \gamma)$ для всех $q \in Q, a \in \Sigma \cup \{ \eps \}$ и $Z \in \Gamma$;
		\item $\delta_1(q, \eps, Z_1) = \{ (q_f, \eps) \}$ для всех $q \in Q$.
  \end{enumerate}  

\item Вернуть МП"/автомат $P' = (Q_1, \Sigma, \Gamma_1, \delta_1, q_1, Z_1, F_1)$ в качестве результата.
} 

Как и в случае алгоритма~\ref{algo-MPtoMPeps}, алгоритм~\ref{algo-RMPtoMPeps}  создаёт над исходным МП"/автоматом обёртку, цель которой --- перевести результирующий автомат в финальное состояние, когда исходный автомат допустил цепочку опустошением магазина. 

\section {Эквивалентность МП"/автоматов и КС"/грамматик}
\label{Chapter8GrammarEqFSM}

Сформулируем теперь один из фундаментальных результатов теории КС"/языков, показывающий, что языки, определяемые МП"/автоматами, --- это в точности КС"/языки.

\begin{mytheorem}
\label{theorem-eqKSandMP}
Пусть $\Sigma$ --- конечный алфавит, $L$ --- язык над этим алфавитом. Тогда следующие условия эквивалентны:
\begin{enumerate}
\item $L=L(G)$ для некоторой КС"/грамматики $G$;
\item $L=L(P_1)$ для некоторого МП"/автомата $P_1$;
\item $L=L(P_2)$ для некоторого РМП"/автомата $P_2$;
\item $L=L_\eps(P_3)$ для некоторого МП"/автомата $P_3$.
\end{enumerate}
\end{mytheorem}

\begin{myproof}
Эквивалентность утверждений 2) и 3) доказана в теореме~\ref{theorem-eqMPandRMP}. Эквивалентность утверждении 3) и 4) вытекает из теорем~\ref{theorem-eqMPandMPeps} и~\ref{theorem-eqRMPandMPeps}. Ниже будут доказаны еще две теоремы --- ~\ref{cfg2pda} и~\ref{pda2cfg}. В силу теоремы~\ref{cfg2pda} утверждение 4) --- следствие утверждения 1), а в силу теоремы~\ref{pda2cfg} утверждение 1) --- следствие утверждения 4). Это завершает доказательство теоремы~\ref{theorem-eqKSandMP}.
\end{myproof}

\begin{mytheorem}\label{cfg2pda}
Пусть $G=(N,\Sigma,P,S)$ --- КС"/грамматика. Тогда можно построить такой МП"/автомат $R$, что $L_\eps(R)=L(G)$.
\end{mytheorem}

\begin{myproof}
Построим МП"/автомат $R$ так, чтобы он моделировал все левые выводы в $G$ (см. пункт~\ref{Chapter6-trees}). Именно, пусть $R=(\{q\},\Sigma,N\cup\Sigma,\delta,q,S,\es)$, и функция переходов $\delta$ полностью определяется правилами:
\begin{enumerate}[label=(\emph{\roman*})]
    \item если $A\to\alpha\in P$, то $(q,\alpha)\in\delta(q,\eps,A)$, где, напомним, $\alpha\in(N\cup\Sigma)^*$;
    \item $\delta(q,a,a)=\{(q,\eps)\}$ для всех $a\in\Sigma$.
\end{enumerate}
Первое правило моделирует продукции грамматики $G$, а второе --- позволяет эти продукции применять.

Перед тем, как проверить равенство $L_\eps(R)=L(G)$, докажем методом математической индукции два вспомогательных утверждения.

$A)$ Для произвольного натурального числа $m$ из возможности вывода $A\To^m\omega(\in\Sigma^*)$ вытекает, что $(q,\omega,A)\vdash^+(q,\eps,\eps)$.

Пусть $A\To^*\omega$. Если $m=1$, $\omega=a_1 \ldots a_k$, где $k\ge 1$, то из $i)$ и $ii)$ получаем:
\[
(q,a_1 \ldots a_k,A) \vdash (q,a_1 \ldots a_k,a_1 \ldots a_k)\vdash^k (q,\eps,\eps).
\]
Если же $m=1$ и $\omega=\eps$, то, как легко видеть, $(q,\eps,A)\vdash (q,\eps,\eps)$.

Теперь предположим, что утверждение верно для $m\le j$ и докажем его для $m=j+1$. Итак, пусть $A\To^{j+1}\omega$. Первый шаг этого вывода должен иметь вид $A\To X_1X_2 \ldots X_k$, где $X_i\To^*x_i$, $x_i\in\Sigma$ и $x_1x_2\ldots x_k=\omega_i$. Тогда в силу правила $i)$ $(q,\omega,A) \vdash (q,\omega,X_1X_2\ldots X_k)$. Если $X_i\in N$, то по предположению индукции $(q,x_i,X_i)\vdash^*(q,\eps,\eps)$. Если же $X_i=x_i\in\Sigma$, то $(q,x_i,X_i) \vdash  (q,\eps,\eps)$. Объединяя эти последовательности тактов, получаем: $(q,\omega,A) \vdash^+ (q,\eps,\eps)$.

Таким образом, утверждение $A)$ верно.

$B)$ Для произвольного натурального числа $n$ из существования последовательности тактов $(q,\omega,A)\vdash^n(q,\eps,\eps)$ вытекает, что $A\To^+\omega$.

Пусть $n=1$. Тогда, как нетрудно убедиться, $\omega=\eps$ и $(A\to \eps)\in P$. Таким образом, $A\To^+\omega$.

Теперь предположим, что утверждение верно для $n\le j$ и докажем его для $n=j+1$. Первый такт, сделанный МП"/автоматом $R$, должен иметь вид $(q,\omega,A)\vdash(q,\omega,X_1,X_2\ldots X_k)$, где $(q,x_i,X_i)\vdash^+(q,\eps,\eps)$, $x_i\in\Sigma$ и $x_1x_2\ldots x_k=\omega_i$ (см. упражнение~\ref{myproblem-612}). Тогда $(A\to X_1\ldots X_k)\in P$, а по предположению индукции $X_i\To^+x_i$ для $X_i\in N$ и $X_i\To^0x_1$ при $X_i\in\Sigma$. Таким образом, искомый вывод $A\To^+\omega$ строится так:
\[
    A   \To X_i \ldots X_k \To^* x_1X_2\ldots X_k
        \To^* x_1x_2\ldots x_{k-1}X_k
        \To^* x_1x_2\ldots x_{k-1}x_k=\omega.
\]

Итак, утверждение $B)$ верно.

Из утверждений $A)$ и $B)$ вытекает, что $S\To^+\omega$ тогда и только тогда, когда $(q,\omega,S)\vdash^+(q,\eps,\eps)$. Следовательно, $L_\eps(R)=L(G)$.
\end{myproof}

Сформулируем результаты теоремы~\ref{cfg2pda} в виде следующего алгоритма.

\Algo[H]{Построение МП$\eps$"/автомата по КС"/грамматике.}
{\label{algo-KStoMPeps} КС"/грамматика $G = (N, \Sigma, P, S)$. }
{МП$\eps$"/автомат $P = (Q, \Sigma, \Gamma, q, Z, \es)$, такой что $L\eps(P) = L(G).$}
{ конструирование элементов автомата $P$ по правилам теоремы~\ref{cfg2pda}.}
{
\item Положить $Q = \{ q \}$.

\item Положить $\Gamma = N \cup \Sigma$.

\item Определить $\delta$ следующим образом:
	\begin{enumerate}
		\item если $A \to \alpha \in P$, то $\delta(q, \eps, A) \ni \{ (q, \alpha) \}$, для всех $A \in N$ и $\alpha \in (N + \Sigma)^*$;
		\item $\delta(q, a, a) = \{ (q, \eps) \}$ для всех $a \in \Sigma$.
  \end{enumerate}  

\item Вернуть МП"/автомат $P = (Q, \Sigma, \Gamma, \delta, q, S, \es)$ в качестве результата.
}

\begin{myexample}
Рассмотрим КС"/грамматику $G = (\{S\}, \{a, b\}, P, S)$, заданную множеством $P$ = \{
$S \to aSbb \mid \eps \}$.
После применения алгоритма~\ref{algo-KStoMPeps} получим автомат $P = (\{q\},  \{a, b\}, \{a, b, S\}, \delta, q, S, \es)$, где функция переходов $\delta$ определяется следующим образом:
\begin{align*}
    \delta(q, \eps, S) 	&= \{ (q, aSbb), (q, \eps) \}; \\
    \delta(q, a, a) 		&= \{ (q, \eps) \}; \\
    \delta(q, b, b) 		&= \{ (q, \eps) \}. \\
\end{align*}
\end{myexample}

Из теоремы~\ref{theorem-eqRMPandMPeps} следует, что если у нас есть МП$\eps$"/автомат, то по нему всегда можно получить МП"/автомат, распознающий тот же язык. Без обосновывающей теоремы приведём алгоритм, который строит РМП"/автомат, моделирующий все правые выводы в заданной КС"/грамматике.

\Algo[H]{Построение РМП"/автомата по КС"/грамматике.}
{\label{algo-KStoRMP} КС"/грамматика $G = (N, \Sigma, P, S)$. }
{РМП"/автомат $P = (Q, \Sigma, \Gamma, q, \mathdollar, F)$, такой что $L(P) = L(G).$}
{преобразование правил грамматики в такты автомата.}
{
\item Положить $Q = \{ q, q_f \}$.

\item Положить $\Gamma = N \cup \Sigma \cup \{ \mathdollar \}$.
\item Положить $F = \{ q_f \}$. 

\item Определить $\delta$ следующим образом:
	\begin{enumerate}
		\item $\delta(q, a, \eps) = \{ (q, a) \}$ для всех $a \in \Sigma$;
		\item если $A \to \alpha \in P$, то $\delta(q, \eps, \alpha^R) \ni \{ (q, A) \}$, для всех $A \in N$ и $\alpha \in (N + \Sigma)^*$;
		\item $\delta(q, \eps, S\mathdollar) = \{ (q_f, \eps) \}$.
  \end{enumerate}  

\item Вернуть МП"/автомат $P = (Q, \Sigma, \Gamma, \delta, q, \mathdollar, F)$ в качестве результата.
}

\begin{myexample}
Рассмотрим КС"/грамматику $G = (\{S\}, \{a, b\}, P, S)$, заданную множеством $P$ = \{ $S \to aSbb \mid \eps \}$.
После применения алгоритма~\ref{algo-KStoRMP} получим автомат $P = (\{q, q_f\},  \{a, b\}, \{a, b, S, \mathdollar\}, \delta, q, \mathdollar, \{q_f\})$, где функция переходов $\delta$ определяется следующим образом:
\begin{align*}
    \delta(q, a, \eps) 	&= \{ (q, a) \}; \\
    \delta(q, b, \eps) 	&= \{ (q, b) \}; \\
    \delta(q, \eps, \eps) 		&= \{ (q, S) \}; \\
    \delta(q, \eps, aSbb) 		&= \{ (q, S) \}; \\
    \delta(q, \eps, S\mathdollar) 		&= \{ (q_f, \eps) \}. \\
\end{align*}
\end{myexample}

Покажем теперь, что язык, определяемый МП"/автоматом, контекстно"/свободен.

\begin{mytheorem}
\label{pda2cfg}
Пусть $R=(Q,\Sigma,\Gamma,\delta,q_0,Z_0,F)$ --- МП"/автомат. Можно построить такую КС"/грамматику $G$, что $L(G)=L_\eps(R)$.
\end{mytheorem}

\begin{myproof}
Начнем строить грамматику $G=(N,\Sigma,P,S)$. Нетерминальные символы будем записывать в виде $[qZr]$, где $q,r\in Q$ и $Z\in\Gamma$, т. е. определим множество нетерминалов равенством
\[
N =\{[qZr]\mid q,r\in Q, Z\in\Gamma\} \cup \{S\}.
\]
Множество продукций зададим следующими условиями:
\begin{enumerate}
\item если $(r,X_1\ldots X_k)\in\delta(q,a,Z)$, где $k\ge 1$, то для каждой последовательности $s_1, s_2, \ldots , s_k$ состояний из $Q$ отнесем к $P$ все продукции вида
\[
[qZs_k] \to a[rX_1s_1][s_1X_2s_2]\ldots [s_{k-1}X_ks_k];
\]
\item если $(r,\eps)\in\delta(q,a,Z)$, то отнесем к $P$ продукцию $[qZr]\to a$; \\
\item для каждого $q\in Q$ отнесем к $P$ продукцию $S\to[q_0\Sigma_0q]$.
\end{enumerate}

Индукцией по числу продукций и числу тактов доказывается следующее вспомогательное утверждению: для любых $q,r\in Q$ и $Z\in\Gamma$ $[qZr]\To^+\omega$ тогда и только тогда, когда $(q,\omega,Z)\vdash^+(r,\eps,\eps)$. Из этого утверждения следует, что $S\To[q_0Z_0q]\To^+\omega$ тогда и только тогда, когда $(q_0,\omega,Z_0)\vdash^+(q,\eps,\eps)$ для $q\in Q$. Таким образом, $L_\eps(R)=L(G)$.
\end{myproof}

На базе теоремы~\ref{pda2cfg} сформулируем следующий алгоритм.

\Algo[H]{Построение КС"/грамматики по МП"/автомату.}
{\label{algo-MPetoKS} МП"/автомат $R = (Q, \Sigma, \Gamma, q, S, F)$. }
{КС"/грамматика $G = (N, \Sigma, P, S)$, такая что $L(G) = L_\eps(R)$.}
{ конструирование элементов грамматики $G$ по правилам теоремы~\ref{cfg2pda}. }
{
\item Положить $N = \{ [qZr] \mid q, r \in Q, Z \in \Gamma \} \cup \{ S \}$.

\item Положить $P = \es$.

\item Если $\delta(q, a, Z) \ni (r, X_1 \ldots X_k)$, где $k \geq 1, \quad q, r \in Q, \quad a \in \Sigma, \quad Z, X_1 \ldots X_k \in \Gamma$, то включить в $P$ все правила вида 
\[
	[qZs_k] \to a[rX_1s_1][s_1X_2s_2] \ldots [s_{k-1}X_ks_k]
\]  
для каждой последовательности $s_1, s_2, \ldots , s_k$ состояний из $Q$.

\item Если $\delta(q, a, Z) \ni (r, \eps)$, где $q, r \in Q, \quad a \in \Sigma, \quad Z \in \Gamma$, то включить в $P$ правило $[qZr] \to a$.

\item Включить в $P$ правила $S \to [q_0Z_0q]$ для каждого $q \in Q$.

\item Вернуть КС"/грамматику $G = (N, \Sigma, P, S)$ в качестве результата.
}

\begin{myexample}
Рассмотрим МП"/автомат $R = (\{ q_0, q_1, q_2 \}, \{a, b\}, \{a, b, Z_0\}, q_0, Z_0, \{ q_0 \})$, функция переходов $\delta$ которого определяется следующим образом:
\begin{align*}
    &\delta(q_0, a, Z_0) 	= \{ (q_1, aaZ_0) \}; \\
    &\delta(q_1, a, a) 	  = \{ (q_1, aaa) \}; \\
    &\delta(q_1, b, a) 		= \{ (q_2, \eps) \}; \\
    &\delta(q_2, b, a) 		= \{ (q_2, \eps) \}; \\
    &\delta(q_2, \eps, Z_0) 		= \{ (q_0, \eps) \}. \\
\end{align*}

После применения алгоритма~\ref{algo-MPetoKS} получим следующие элементы результирующей грамматики $G = (N, \{a, b\}, P, S)$.

Множество нетерминалов $N = \{ [qZr] \mid q, r \in \{ q_0, q_1, q_2 \}, Z \in \{a, Z_0\} \cup \{S\} \}$.
Множество продукций $P$ без бесполезных символов:
\begin{align*}
    S	&\to [q_0Z_0q_0]; \\
    [q_0Z_0q_0] 	  &\to a[q_1aq_2][q_2aq_2][q_2Z_0q_0]; \\
    [q_1aq_2] 		&\to a[q_1aq_2][q_2aq_2][q_2aq_2]; \\
    [q_1aq_2] 		&\to b; \\
    [q_2aq_2] 		&\to b; \\
    [q_2Z_0q_0] 		&\to  \eps. \\
\end{align*}

После переименования множество продукций $P$ имеет следующий вид:
\begin{align*}
    S	&\to aABC; \\
    A 	  &\to aABB \mid b; \\
    B 		 &\to b; \\
    C 		&\to  \eps. \\
\end{align*}

Упростим правила грамматики и получим $P = \{ S \to aAb; A \to aAbb \mid b \}$.


В результате искомая грамматика имеет вид $G = (\{S, A\}, \{a, b\}, P, S)$, у которой $L(G) = \{ a^nb^{2n} \mid n \ge 1 \}$.

\end{myexample}

\section{Детерминированный МП"/автомат}
Выше отмечалось, что для каждой КС"/грамматики $G$ можно построить МП"/автомат, распознающий $L(G)$ (теорема~\ref{cfg2pda}). Однако построенный автомат был недетерминированный, а в приложениях более удобны детерминированные МП"/автоматы, которые в каждой конфигурации могут сделать не более одного очередного такта.

Дадим точное определение: МП"/автомат $P=(Q,\Sigma,\Gamma,\delta,q_0,Z_0,F)$ называется \mydef{детерминированным} (сокращенно ДМП"/автоматом), если для каждых $q\in Q$ и $\Sigma\in\Gamma$ либо $\delta(q,a,Z)$ содержит не более одного элемента для каждого $a\in\Sigma$ и $(q,\eps,Z)=\es$, либо $\delta(q,a,Z)=\es$ для всех $a\in\Sigma$ и $\delta(q,\eps,Z)$ содержит не более одного элемента. В силу этих двух ограничений ДМП"/автомат в любой конфигурации может выбрать не более одного такта.

В теореме~\ref{theorem-reduction-NKAtoDKA} было показано, что класс языков, определяемых недетерминированными конечными автоматами, совпадает с классом языков, определяемых полностью определенными детерминированными конечными автоматами. Но, к сожалению, ДМП"/автоматы не так мощны по своей распознавательной способности, как недетерминированные МП"/автоматы, и существуют КС"/языки, которые нельзя определить детерминированными МП"/автоматами. При этом для более сильного
вычислительного формализма (машин Тьюринга) снова справедлива эквивалентность
детерминированной и недетерминированной версии.

ДМП-автоматные языки интересно соотносятся с уже известными классами языков.
Укажем ещё два факта об этих отношениях, кроме упомянутого выше строгого включения
в класс МП-автоматных языков.
\begin{enumerate}
    \item Хотя исключить недетерминизм без уменьшения вычислительной мощности в случае МП"/автоматов не удаётся, можно избавиться от $\eps$"/переходов,
    не меняя класса распознаваемых языков. Этот факт получается,
   если адаптировать доказательство теоремы~\ref{cfg2pda} к КС"/грамматике,
   которая находится в нормальной форме Грейбах. Подумайте, как
   реализовать эту идею.

    \item ДМП-автоматные языки содержат в себе класс регулярных языков как собственное подмножество.

    \item Все ДМП-автоматные языки не являются существенно неоднозначными, то есть для
    них существуют КС"/грамматики без неоднозначности. В то же время,
    существуют языки без неоднозначности, которые не являются ДМП"/автоматными.
     К последним относится, например, язык палиндромов четной длины
     $L_{ww^r}$.
\end{enumerate}

\section{Упражнения}
\label{Chapter8Exs}

Для каждого из следующих языков
\begin{enumerate}
    \item $\{ a^n b^n c^m d^m \mid n,m \in \N\}$,
    \item $\{ a^i b^j c^j d^i \mid i,j \in \N \}$,
    \item $\{ a^i b^j c^k \mid i,j,k \in \N \text{ и } i+j=k \}$,
    \item $\left\{  x \in \{ a,b,c \}^* \mid |x|_a + |x|_b = |x|_c \right\}$
\end{enumerate}
построить МП-автомат:
\begin{enumerate}[label=\asbuk*)]
   \item распознающий $L$,
   \item распознающий $L$ опустошением магазина.
\end{enumerate}
Напомним, что запись вида $|w|_z$  (см.~язык 4) означает количество вхождений символа $z$ в строку $w$.

% \addchap{Литература}

\renewcaptionname{russian}{\bibname}{Список литературы}
\begin{thebibliography}{9}
\addcontentsline{toc}{chapter}{Список литературы}


\bibitem{ASU} \emph{Ахо~А., Лам~М., Сети~Р., Ульман~Дж.} Компиляторы: принципы, технологии и инструментарий, 2-е изд. М.: Вильямс, 2008.

\bibitem{AU} \emph{Ахо~А., Ульман~Дж.} Теория синтаксического анализа, перевода и компиляции. Т.~1. Синтаксический анализ. М.: Мир, 1978. 

\bibitem{Bau} \emph{Бауэр~Ф.\,Л., Гооз~Г.} Информатика. М.: Мир, 1990. 

\bibitem{BelTka} \emph{Белоусов~А.\,И., Ткачёв~С.\,Б.} Дискретная математика. М.: МГТУ, 2010.

\bibitem{Hro} \emph{Громкович~Ю.} Теоретическая информатика. Введение в теорию автоматов, теорию вычислимости, теорию сложности, теорию алгоритмов, рандомизацию, теорию связи и криптографию. СПб: БХВ-Петербург, 2010.

\bibitem{Sal} \emph{Саломаа~А.} Жемчужины теории формальных языков. М.: Мир, 1986. 

%\bibitem{Kot} \emph{В.\,Е.~Котов.} Сети Петри. М.: Наука, 1984. 

\bibitem{Hoa} \emph{Хоар~Ч.} Взаимодействующие последовательные процессы. М.: Мир, 1889.

\bibitem{Hop} \emph{Хопкрофт~Дж., Мотвани~Р., Ульман~Дж.}  Введение в теорию автоматов, языков и
вычислений. — 2-е изд. М.: Вильямс, 2008.

%\bibitem{Glu} \emph{В.\,М.~Глушков, Г.\,Е.~Цейтлин, Е.\,Л.~Ющенко.} Алгебра, языки, программирование. Киев: Наукова думка, 1989. 

\end{thebibliography}


\appendix
\addtocontents{toc}{\def\protect\cftchappresnum{\normalfont{}Приложение }%
\addtolength{\cftchapnumwidth}{\widthof{\normalfont{}Приложение } - \widthof{\normalfont{}Глава }} }

\renewcommand{\theAlgoEnv}{\Alph{chapter}.\arabic{AlgoEnv}}

\chapter{Алгоритмы для контекстно-свободных грамматик}

В этом приложении алгоритмы, рассмотренные в
главах~\ref{cfg-intro} и~\ref{normal-cfg}, приведены в виде
псевдокода, что может облегчить программирование, а иногда и
понимание этих алгоритмов.

\textbf{Некоторые обозначения, используемые при описании алгоритмов.}
%\paragraph{Некоторые обозначения, используемые при описании алгоритмов}
\begin{itemize}
    \item
${A \gets B}$~— операция присваивания («$A$ присвоить $B$»);

    \item
$A \hookleftarrow a$~— операция добавления элемента в множество («добавить в
множество $A$ элемент $a$»);

    \item
$A \hookleftarrow B$~— операция добавления
множества элементов во множество («добавить в множество $A$ все элементы из
$B$»);

    \item
${\rhd}$~— начало комментария, продолжающегося до конца строки.
\end{itemize}

\AlgoPseudoCode[H]{Нахождение порождающих символов}
{\label{algo-gen}КС-грамматика $G=(\Sigma, N, \mathcal P, S \in N)$.}
{$\Gen_G(\Sigma)$  — множество порождающих в $G$ символов.}
{%
\li $\Gen_G(\Sigma) \gets \Sigma$ \Comment Терминалы являются порождающими
символами
\zi\Comment Ищем продукции, в правых частях которых только порождающие символы\ldots
\li \While $\exists A \to X_1 \ldots X_n \in \mathcal P$, $n \geqslant 0$,
такое что $\forall i \; X_i \in \Gen_G(\Sigma)$  \label{gen-induction}
\zi     \Do
        $\Gen_G(\Sigma) \hookleftarrow A$ \Comment левые части таких
        продукций добавляются в $\Gen_G(\Sigma)$
        \End
}

\begin{myremark}[о многократном просмотре продукций]
В цикле на шаге~\ref{gen-induction} алгоритма~\ref{algo-gen}
одна и та же продукция $A \to X_1 \ldots X_n$
может быть просмотрена несколько раз, причём если на ранних итерациях она не
удовлетворяла условию $\forall i \; X_i \in \Gen_G(\Sigma)$, то на поздних,
когда множество $\Gen_G(\Sigma)$ станет достаточно большим, положение дел может
измениться (условие $\forall i \; X_i \in \Gen_G(\Sigma)$ выполнится и $A$ надо
будет добавить в $\Gen_G(\Sigma)$).

Аналогичное справедливо для просмотра продукций в алгоритме~\ref{algo-reach}.
\end{myremark}

\newcommand{\GenGS}{\ensuremath{\Gen_G(\Sigma)}}

\AlgoPseudoCode[h]{Нахождение достижимых символов}{\label{algo-reach}%
КС-грамматика $G=(\Sigma, N, \mathcal P, S \in N)$.}
{$\Reach_G$  — множество достижимых в $G$ символов.}
{%
\li $\Reach_G \gets \{ S \}$ \Comment $S$ достижим в $G$
\zi\Comment Ищем продукции, в левых частях которых стоит достижимый символ\ldots
\li \While $\exists A \to X_1 \ldots X_n \in \mathcal P$, такое что
    $A \in \Reach_G$
    % $\forall i \; X_i \in \Gen_G(\Sigma^*)$  \label{gen-induction}
\zi \Comment \ldots символы из правых частей таких продукций добавляются
в $\Reach_G$:
\zi     \Do
        $\Reach_G \hookleftarrow \{ X_i \}_{i=1}^n$
        \End
    \End
}

\AlgoNoMeth[h]{Удаление бесполезных символов}{\label{algo-del-useless}%
КС-грамматика $G=(\Sigma, N, \mathcal P, S \in N)$.}
{КС-грамматика $G'=(\Sigma', N', \mathcal P', S \in N')$, не
содержащая бесполезных символов, такая что $L(G') = L(G)$, либо сигнал о том,
что язык исходной грамматики пуст и не существует эквивалентной $G$
грамматики без бесполезных символов.}
{
  \item Построить множество \GenGS, используя алгоритм~\ref{algo-gen}. Если
  $S \not \in \GenGS$ — завершение алгоритма, сообщение о пустоте языка исходной
  грамматики. Иначе удалить из $G$ все символы, не вошедшие в \GenGS.
  \item Построить множество $\Reach_G$, используя алгоритм~\ref{algo-reach}.
  Удалить из $G$ все символы, не вошедшие в $\Reach_G$. Получившуюся грамматику
  обозначить $G'$ и подать её на выход алгоритма.
}

\begin{myremark}[об операции удаления символа из грамматики]
Когда в алгоритме требуется удалить символ $X$ из грамматики $G$, необходимо
не только исключить его из множества $N$ или $\Sigma$, но и \emph{удалить из
$\mathcal P$ все продукции, в которых он участвует}.
\end{myremark}

\Algo[h]{Удаление $\varepsilon$-правил}{\label{algo-del-eps}%
КС-грамматика $G=(\Sigma, N, \mathcal P, S \in N)$.}
{КС-грамматика $G'=(\Sigma, N, \mathcal P', S \in N)$, без
$\varepsilon$-правил (продукций вида $A \to \varepsilon$), такая что
$L(G')=L(G) \setminus \{ \varepsilon \}$.}
{«Устранение перегородок».}
{
  \item Построить множество $\Gen_G(\varepsilon) \subset N$ всех порождающих
  $\varepsilon$ нетерминалов, используя следующую процедуру:
  \begin{codebox}
  \li   \For $A \to \varepsilon \in \mathcal P$
  \zi   \Do
            $\Gen_G(\varepsilon) \hookleftarrow A$
        \End
  \li   \While $\exists A \to X_1 \ldots X_n \in \mathcal P$,
        где $\{ X_i \}_{i=0}^n \subset \Gen_G(\varepsilon)$
  \zi       \Do
            $\Gen_G(\varepsilon) \hookleftarrow A$
            \End
        \End
  \end{codebox}
  \item\label{remove-barriers} Выполнить следующие действия:
  \begin{codebox}
  \zi \For $A \to \alpha_0 B_1 \alpha_1 \ldots B_n \alpha_n \in \mathcal P$, где
  $\forall i \; B_i \in \Gen_G(\varepsilon) \wedge \;
  \alpha_i \in ((N \cup \Sigma) \setminus \Gen_G(\varepsilon))^*$
  \zi   \Do
        $\mathcal P \hookleftarrow
            \{ \alpha_0 X_1 \alpha_1 \ldots X_n \alpha_n \mid
            \forall i \; X_i = \varepsilon \vee X_i = B_i \}$
        \End
  \end{codebox}
  \item Удалить из $\mathcal P$ все $\varepsilon$-правила, обозначить
  получившееся множество правил $\mathcal P'$ и подать на выход алгоритма
  грамматику $G' = (N, \Sigma, \mathcal P', S)$.
}

\begin{myremark}[о методе «устранения перегородок»]
На шаге 2 алгоритма~\ref{algo-del-eps} (с.~\pageref{algo-del-eps}) каждая 
продукция $\mathcal P$ просматривается ровно
один раз. Понятно, что для подходящего $n$ любая продукция может быть
представлена в виде $A \to \alpha_0 B_1 \alpha_1 \ldots B_n \alpha_n$ с
указанными условиями для $B_i$ и $\alpha_i$. Например, если в продукции нет
$\varepsilon$"/порождающих символов, то это продукция вида $A \to
\alpha_0$ и для неё нет необходимости добавлять новые продукции.

\emph{Пример}. Продукция $C \to aDbD$, где $D \in \Gen_G(\varepsilon)$, $a, b \in
\Sigma$, это продукция вида $A \to \alpha_0 B_1 \alpha_1 B_2 \alpha_2$, где $A =
C$, $\alpha_0 = a$, $B_1 = B_2 = D$, $\alpha_2 = \varepsilon$, и для неё нужно
добавить три новых продукции. Укажем их, пояснив название метода «удаления
перегородок». Можно считать, что $\varepsilon$-порождающие символы $B_i$ являются
«перегородками» в продукции $A \to \alpha_0 B_1 \alpha_1 \ldots B_n \alpha_n$ и
новые продукции получаются из данной при помощи устранения этих перегородок \emph{всеми
возможными способами}. Для продукции $C \to aDbD$ это: $C \to abD$, $A \to aDb$ и
$A \to ab$. Исходная продукция остаётся в множестве $\mathcal P$.
\end{myremark}

\AlgoNoMeth[h]{Удаление цепных продукций}{\label{algo-del-chains}%
КС-грамматика $G=(\Sigma, N, \mathcal P, S \in N)$, не
содержащая $\varepsilon$"/правил.}
{КС-грамматика
$G'=(\Sigma, N, \mathcal P', S \in N)$, без цепных правил (продукций вида $A \to B$), такая что $L(G')=L(G)$.}
{
  \item Для каждого нетерминала $A \in N$ построить множество циклически
  достижимых из $A$ нетерминалов $C(A)$, используя процедуру:
  \begin{codebox}
  \li   $C(A) \gets \{ A \}$
  \li   \While $\exists D \to E \in \mathcal P$, такая что $D \in C(A)$
  \zi   \Do
        $C(A) \hookleftarrow E$
        \End
  \end{codebox}
  \item Выполнить следующую процедуру:
  \begin{codebox}
  \zi   \For $A \in N$
  \zi   \Do \For $B \to \alpha \in \mathcal P$
  \zi       \Do \If $B \in C(A)$
  \zi           \Then $\mathcal P \hookleftarrow A \to \alpha$
                \End
            \End
        \End
  \end{codebox}
  \item Удалить из $\mathcal P$ все цепные правила, обозначить
  получившееся множество правил $\mathcal P'$ и подать на выход алгоритма
  грамматику $G' = (N, \Sigma, \mathcal P', S)$.
}

\begin{myremark}[о введении новых нетерминалов в алгоритме~\ref{to-cnf}]
Отметим, что на каждой итерации цикла шага~4 в алгоритме~\ref{to-cnf} 
(с.~\pageref{to-cnf}), если
есть необходимость ввести новые не\-тер\-ми\-на\-лы, то они должны отличаться не только
от тех, которые присутствовали в грамматике до начала этого цикла, но и от тех,
которые были введены на предыдущих итерациях этого цикла. Таким образом, одного
комплекта букв $\{C_i\}$ для выполнения цикла может не хватить. Для борьбы с
нехваткой букв можно вводить нетерминалы, помеченные частями исходного слова
$B_1\ldots B_n$, которое «разбивается на слоги». Например, вместо набора
$\{C_i\}_{i=1}^{n-2}$ можно использовать набор $\{ \langle B_{i+1}\ldots B_n
\rangle \}_{i=1}^{n-2}$, где для каждого $i$ выражение $\langle B_{i+1}\ldots B_n \rangle$
понимается как \emph{один новый} нетерминал. Аналогично можно поступать на
шаге~3, добавляя для рассматриваемого терминала $a$ новый
нетерминал $\langle a \rangle$, а не $A'$.
\end{myremark}

\Algo[h]{Приведение к нормальной форме Хомского}{\label{to-cnf}%
КС-грамматика $G=(\Sigma, N, \mathcal P, S \in N)$, не
содержащая $\varepsilon$"/правил.}
{КС-грамматика
$G'=(\Sigma', N', \mathcal P', S \in N)$ в НФХ, такая что $L(G')=L(G) \setminus
\{ \varepsilon \}$ или сообщение о пустоте языка грамматики.}
{«Разбиение слов на слоги».}
{
    \item Последовательно воспользоваться
    алгоритмами~\ref{algo-del-eps} и \ref{algo-del-chains}, полученную
    грамматику обозначить $G'' = (N, \Sigma, \mathcal P'', S)$.
    \item Применить к $G''$ алгоритм~\ref{algo-del-useless}. Если получен ответ
    $L(G'')=\es$, остановить алгоритм и подать на выход сигнал о пустоте языка
    исходной грамматики. Иначе получена грамматика $G' = (N', \Sigma',
    \mathcal P', S)$.
    \item\label{process-terminals}К $G'$ применить следующую процедуру:
    \begin{codebox}
    \zi \For $a \in \Sigma$
    \zi \Do
            \If $\exists A \to X_1 \ldots X_n \in \mathcal P''$, такая что
            $n > 1 \wedge \exists i\: X_i = a$
    \zi     \Then
    \li     добавить в $N$ \emph{новый} нетерминал $A'$
    \li     заменить $a$ на $A'$ в продукциях с правой частью длиннее 1
    \li     $\mathcal P'' \hookleftarrow A' \to a$
            \End
        \End
    \end{codebox}
    \item\label{breaking-words}К $G'$ применить следующую процедуру:
    \begin{codebox}
    \zi \For $A \to B_1 B_2 \ldots B_n \in \mathcal P'$,
        где $n > 2$, $B_i \in N$
    \zi \Do
    \li     добавить в $N$ \emph{новые} нетерминалы $C_1, \ldots C_{n-2}$
    \li     $\mathcal P' \hookleftarrow
            \{ A \to B_1C_1\} \cup \{ C_i \to B_{i+1} C_{i+1} \}_{i=1}^{n-3}
            \cup \{ C_{n-2} \to B_{n-1}B_n \}$
    \li     удалить $A \to B_1 B_2 \ldots B_n$ из $\mathcal P'$
        \End
    \end{codebox}
    Подать $G'$ на выход алгоритма.
}

\Algo[h]{Решение проблемы принадлежности для КС"/языков, CYK-алгоритм}
{\label{cyk}Грамматика $G=(\Sigma, N, \mathcal P, S \in N)$ в НФХ,
слово $w = w_1 \ldots w_n \in \Sigma^*$.}
{Истина, если $w \in L(G)$, ложь  в противном случае.}
{Последовательное определение нетерминалов, выводящих
всевозможные подстроки $w$ всё большей длины.}
{
\item Построить множества $N_{ij}$, используя процедуру:
\begin{codebox}
\zi\For $i \gets 1$ \kw{to} $n$
\zi     \Do
        $N_{ii} \gets \{ A \in N \mid A \to w_i \in \mathcal P  \}$
        \Comment Подстроки $w$ длины $1$
        \End
\zi\For $s \gets 2$ \kw{to} $n$ \Comment Цикл по длине подстроки
\zi     \Do
        \For $i \gets 1$ \kw{to} $n - s + 1$ \Comment Цикл по месту начала подстроки
\zi         $j \gets i + s - 1$
            \Comment Позиция конца подстроки с началом в $w_i$ длины $s$
\zi         $N_{ij} \gets \{ A \in N \mid A \to BC \in \mathcal P; \;
                \exists k \in [i, j-1]_{\mathbb Z} \colon B \in N_{ik}, \;
                C \in N_{k+1 j} \} $
        \End
\end{codebox}

\item
Если $S \in N_{1n}$, то подать на выход алгоритма «да», иначе — «нет».
}

\begin{myremark}[о табличной форме алгоритма~\ref{cyk}, с.~\pageref{cyk}]
Алгоритм удобно выполнять, заполняя таблицу с $N_{ij}$ в ячейках. Ячейки таблицы
расположены в системе координат $(i,s)$, в позиции $(i,s)$ находится множество
$N_{i, i+s-1}$. Подробно этот процесс описан в разделе~\ref{Chapter7ProblemB}.
\end{myremark}

\renewcommand{\theAlgoEnv}{\Alph{chapter}.\arabic{AlgoEnv}}

\chapter{Задание к курсовой работе}
\begin{itemize}
\item[1.] Для языка $L$, выбранного в соответствии с вариантом, выполнить:
\begin{itemize}
\item[(i)] Построить ПЛ-грамматику $G$, порождающую $L$;
\item[(ii)] Доказать вложения $L\subseteq L(G)$, $L(G)\subseteq L$;
\item[(iii)] Путем решения системы линейных уравнений с регулярными коэффициентами построить регулярное выражение, описывающее $L$;
\item[(iv)] Построить НКА или $e$-НКА $M^{ND}$, распознающий язык $L$, предъявить его граф;
\item[(v)] Построить ДКА $M^{D}$ путем детерминизации $M^{ND}$, предъявить его граф;
\item[(vi)] Доказать, что $L(M^{ND})=L(M^{D})=L$;
\item[(vii)] Минимизировать полученный конечный автомат, распознающий язык $L$, или доказать его минимальность;
\item[(viii)] Методом последовательного исключения состояний выписать регулярные выражения для $L(M^{ND})$, $L(M^{D})$;
\item[(ix)] Постоить ДКА $\overline{M^D}$, распознающий дополнение $\overline{L}$ к языку $L$, записать $\overline{L}$ в виде регулярного выражения;
\item[(x)] На произвольно выбранном языке программирования написать программу, позволяющую распознать принадлежность вводимой строки языку $L$.
\end{itemize}

\item[2.] Для множества слов $A$ над алфавитом $\Sigma$, заданных в соответствии с вариантом, выполнить:
\begin{itemize}
\item[(i)] Для каждого слова $w_i\in A$ построить НКА $M^{ND}_i$, распознающий наличие в произвольной строке $s\in\Sigma^\ast$ подстроки $w_i$;
\item[(ii)] Для каждого НКА $M^{ND}_i$ построить соответствующий ДКА $M^D_i$;
\item[(iii)] На произвольно выбранном языке программирования написать программу, позволяющую подсчитать количество вхождений каждого слова из множества $A$ во входную строку $s\in\Sigma^\ast$ (при этом программно реализовать хотя бы один автомат $M^{ND}_i$ и хотя бы один автомат $M^D_j$).
\end{itemize}

\pagebreak

\item[3.] Для языков $L_1$, $L_2$, выбранных в соответствии с вариантом, выполнить:
\begin{itemize}
\item[(i)] Вычислить регулярное выражение, определяющее язык $L_1\cap L_2$;
\item[(ii)] Вычислить регулярное выражение, определяющее язык $L_1\Delta L_2$;
\item[(iii)] Определить: совпадают ли языки $L_1$ и $L_2$, является ли $L_1$ дополнением $L_2$;
\item[(iv)] Построить $e$-НКА, распознающий один из языков $L_1^R$ или $L_2^R$;
\item[(v)] Вычислить регулярное выражение по построенному $e$-НКА;
\item[(vi)] Построить $e$-НКА, распознающий один из языков $L_1L_2$, $L_1\cup L_2$ или $L_1^\ast$;
\item[(vii)] Детерминизировать один из построенных $e$-НКА;
\item[(viii)] На произвольно выбранном языке программирования написать программу, реализующую какой-либо $e$-НКА и какой-либо ДКА из этого пункта.
\end{itemize}
\renewcommand{\theAlgoEnv}{\Alph{chapter}.\arabic{AlgoEnv}}

\chapter{Варианты заданий}
Задание 1
\begin{enumerate}
\item Язык над алфавитом $\Sigma=\{0,1\}$, состоящий из всех слов, в которых единицы встречаются только четное число раз, а первая сдвоенная пара нулей может появиться только после появления двух единиц.
\item Язык над алфавитом $\Sigma=\{0,1\}$, состоящий из всех слов, в которых единицы встречаются только нечетное число раз, а после последней сдвоенной пары нулей может появиться не более трех единиц.
\item Язык над алфавитом $\Sigma=\{0,1\}$, состоящий из всех слов, в которых после второй сдвоенной пары нулей количество единиц четно.
\item Язык над алфавитом $\Sigma=\{0,1\}$, состоящий из всех слов, в которых до второй сдвоенной пары нулей количество единиц четно.
\item Язык над алфавитом $\Sigma=\{0,1\}$, состоящий из всех слов, в которых, если в слове содержится три последовательные буквы, то оно содержит хотя бы два нуля.
\item Язык над алфавитом $\Sigma=\{0,1\}$, состоящий из всех слов, в которых число единиц делится на 2, а число нулей -- на 3.
\item Язык над алфавитом $\Sigma=\{0,1\}$, состоящий из всех слов, которые начинаются или оканчиваются (или и то и другое) последовательностью 01.
\item Язык над алфавитом $\Sigma=\{0,1\}$, состоящий из всех слов, в которых хотя бы на одной из последних трёх позиций стоит 1, и не встречается более двух нулей подряд.
\item Язык над алфавитом $\Sigma=\{0,1\}$, состоящий либо из повторяющихся один или несколько раз фрагментов 01, либо повторяющихся один или несколько раз фрагментов 010, либо всех слов четной длины и содержащих хотя бы одно подслово 00.
\item Язык над алфавитом $\Sigma=\{0,1\}$, состоящий из всех слов, в которых если встречается три подряд идущих нуля, то слово имеет чётную длину, а если встречается две подряд идущие единицы, то длина слова нечётная.
\item Язык над алфавитом $\Sigma=\{0,1\}$, состоящий из всех слов, в которых если встречается хотя бы одна пара 01, то пара 10 должна встретиться не менее двух раз.
\item Язык над алфавитом $\Sigma=\{0,1\}$, состоящий из всех слов, в которых если слово содержит пару 10, то оно имеет четную длину и должно обязательно заканчиваться двумя подряд идущими единицами.
\item Язык над алфавитом $\Sigma=\{0,1\}$, состоящий из всех слов, в которых не встречаются подслова 101 и 110 одновременно.
\item Язык над алфавитом $\Sigma=\{0,1\}$, состоящий из всех слов, в которых четное число единиц может быть тогда и только тогда, когда слово содержит послово 101.
\item Язык над алфавитом $\Sigma=\{0,1\}$, состоящий из всех слов, которые имеют четную длину тогда и только тогда, когда они содержат подслово 011.
    
\item Язык над алфавитом $\Sigma=\{0,1\}$, состоящий из всех слов, в которых количество единиц четно, а пара сдвоенных нулей встречается не более трех раз.
\item Язык над алфавитом $\Sigma=\{0,1\}$, состоящий из всех слов, содержащих не более трёх подряд идущих нулей, и количество единиц в которых четно.
\item Язык над алфавитом $\Sigma=\{0,1\}$, состоящий из всех слов, в которых после третьей слева единицы стоит четное число групп вида $01$.
\item Язык над алфавитом $\Sigma=\{0,1\}$, состоящий из всех слов, в которых количество нулей четно, а первая сдвоенная пара единиц появляется не раньше группы строенных нулей.
\item Язык над алфавитом $\Sigma=\{0,1\}$, состоящий из всех слов, содержащих нечетное количество нулей и единиц, начинаться и заканчиваться слова могут не более чем тремя подряд идущими нулями.
\item Язык над алфавитом $\Sigma=\{0,1\}$, состоящий из всех слов, не содержащих двух смежных нулей, и в которых группа цифр $010$ повторяется не более одного раза.
\item Язык над алфавитом $\Sigma=\{0,1\}$, состоящий из всех слов, содержащих три смежные единицы, и в которых группа вида $101$ содержится не более одного раза.
\item Язык над алфавитом $\Sigma=\{0,1\}$, состоящий из всех слов, в которых между двумя парами единиц может находиться не более чем три подряд идущих нуля.
\item Язык над алфавитом $\Sigma=\{0,1\}$, состоящий из всех слов, в которых после группы $010$ следует чётное количество подряд идущих единиц. 
\item Язык над алфавитом $\Sigma=\{0,1\}$, состоящий из всех слов, в которых перед тремя подряд идущими нулями находится четное количество единиц.
\item Язык над алфавитом $\Sigma=\{0,1\}$, состоящий из всех слов, таких что если слово начинается с нуля, то количество единиц в слове четно, а если с единицы, то нечетно.
\item Язык над алфавитом $\Sigma=\{0,1\}$, состоящий из всех слов, в которых если встречается хотя бы одна группа из трех подряд идущих нулей, то пара сдвоенных единиц должна появиться не менее двух раз.
\item Язык над алфавитом $\Sigma=\{0,1\}$, состоящий из всех слов, в которых если слово содержит группу цифр 010, то оно имеет нечетную длину.
\item Язык над алфавитом $\Sigma=\{0,1\}$, состоящий из всех слов, в которых группа строенных нулей и строенных единиц не встречаются одновременно.
\item Язык над алфавитом $\Sigma=\{0,1\}$, состоящий из всех слов, имеющих четную длину тогда и только тогда, когда они начинаются тремя нулями, а заканчиваются тремя единицами.

\end{enumerate}

\bigskip

Задание 2
\begin{enumerate}
\item $\Sigma = \{0,1,2\}$, $A = \{01110, 1011, 11100, 010101\}$.
\item $\Sigma = \{0,1,2\}$, $A = \{110, 1110, 0000, 0111\}$.
\item $\Sigma = \{0,1,2\}$, $A = \{1011, 111, 0110, 10011\}$.
\item $\Sigma = \{a,b,c\}$, $A = \{bbab, ccac, bacb, baabc\}$.
\item $\Sigma = \{a,b,f\}$, $A = \{faafb, ffba, bfab, affa\}$.
\item $\Sigma = \{1,2,3\}$, $A = \{2112, 1233, 1322, 131313\}$.
\item $\Sigma = \{4,5,6\}$, $A = \{5555, 4665, 666, 4455, 665\}$.
\item $\Sigma = \{9,10,11,12\}$, $A = \{9101112, 101111, 121110, 121211\}$.
\item $\Sigma = \{0,1,2,3,4,5\}$, $A = \{22231, 1213, 0001, 11101\}$.
\item $\Sigma = \{0,1,2,3,4,5\}$, $A = \{000, 010, 221, 3334, 0110\}$.
\item $\Sigma = \{a,b,c,d,e\}$, $A = \{aaac, bbd, bbe, abbc\}$.
\item $\Sigma = \{a,b,c,d,e\}$, $A = \{aaaa, bbbba, bbaa, cccac\}$.
\item $\Sigma = \{x,y,v,w,z\}$, $A = \{xyz, xyzz, zzzw, xyyw\}$.
\item $\Sigma = \{0,1,2,3,4,5\}$, $A = \{01112, 32544, 12124, 5551\}$.
\item $\Sigma = \{a,b,w,z\}$, $A = \{wwwz, bbbww,zzbz,wzwzw\}$.

\item $\Sigma = \{0, 1, 2\}$, $A = \{010100, 111001, 110, 1110\}$.
\item $\Sigma = \{0, 1, 2\}$, $A = \{001100, 111, 0101, 1000\}$.
\item $\Sigma = \{0, 1, 2\}$, $A = \{1011, 0000, 1111, 1011101\}$.
\item $\Sigma = \{0, 1, 2\}$, $A = \{011100, 11101, 1010, 111\}$.
\item $\Sigma = \{x, y, z\}$, $A = \{xxyy, xxxy, xyxyyy, xyxyyx\}$.
\item $\Sigma = \{x, y, z\}$, $A = \{xy, yxxy, yxyxxy, xyxxx\}$.
\item $\Sigma = \{x, y, z\}$, $A = \{xyx, xxy, yyyxxy, xyxyx\}$.
\item $\Sigma = \{3, 5, 0\}$, $A = \{03530, 0033, 30335, 0350\}$.
\item $\Sigma = \{9, 8, 7\}$, $A = \{987, 99787, 87799, 9777\}$.
\item $\Sigma = \{3, 5, 0\}$, $A = \{3330, 05033, 30330, 535\}$.
\item $\Sigma = \{x, a, 35\}$, $A = \{35aaxx, xx35ax, 35xa, 3535ax\}$.
\item $\Sigma = \{10, 23, 45, 96\}$, $A = \{101045, 45109623, 23102345, 1010\}$.
\item $\Sigma = \{0, 1, 2, 3, 4, 5\}$, $A = \{0112, 0033, 30121, 331\}$.
\item $\Sigma = \{0, 1, 2, 3, 4, 5\}$, $A = \{53150, 53555, 5510, 0001\}$.
\item $\Sigma = \{a, b, c, d, e\}$, $A = \{abaac, ccbac, ccc, abac\}$.
\end{enumerate}

\bigskip

Задание 3
\begin{enumerate}
\item $L_1 = (10+1)^\ast11(0+1)^\ast$, $L_2 = (101)^\ast111(0+1)^\ast$.
\item $L_1 = (0+1)^\ast1^\ast01(11+0)^\ast$, $L_2 = 1^\ast(0+1+11)^\ast0$.
\item $L_1 = 11(01+10)^\ast001^\ast$, $L_2 = 11((10)^\ast(01)^\ast)^\ast0011^\ast$.
\item $L_1 = (00+11)^\ast10(10+01)^\ast$, $L_2 = (0+1)^\ast1$.
\item $L_1 = (1+01)^\ast(00+1)^\ast$, $L_2 = (0+10)^\ast(11+0)^\ast$.
\item $L_1 = (111+01)^\ast1(00+11)^\ast$, $L_2 = 1^\ast0^\ast$.
\item $L_1 = (01+11)^\ast(10+11)^\ast$, $L_2 = (10+11)^\ast(01+11)^\ast$.
\item $L_1 = (a+bb)^\ast bba^\ast$, $L_2 = a^\ast (bb)^\ast a^\ast$.
\item $L_1 = b^\ast ab(ab+a)^\ast$, $L_2 = b^\ast a^\ast ab$.
\item $L_1 = a^\ast b^\ast(a+b)(bb)^\ast$, $L_2 = a^\ast(aa+b)^\ast$.
\item $L_1 = (b+ab)^\ast(a+ab)b^\ast$, $L_2 = (ab)^\ast b^\ast$.
\item $L_1 = (aba+bab)^\ast$, $L_2 = (ab + ba)^\ast$.
\item $L_1 = b(a+ba)^\ast (ab)^\ast$, $L_2 = b(a+ba+ab)^\ast$.
\item $L_1 = (aa+bb)^\ast ba(a+b)^\ast$, $L_2 = (aa)^\ast(bb)^\ast a^\ast b^\ast$.
\item $L_1 = ((a+b)^\ast + (ab+ba)^\ast)^\ast$, $L_2 = (a+b+ab+ba)^\ast$.

\item $L_1 = (1+01)^\ast010(10+01)^\ast$, $L_2 = (10)^\ast(0+10)10(1+0)^\ast$.
\item $L_1 = (0+1)^\ast010^\ast1(111+01)^\ast$, $L_2 = 0^\ast(10+11)^\ast0^\ast(1+0)^\ast$.
\item $L_1 = 01(0+1+10)^\ast01^\ast0$, $L_2 = 011((11)^\ast(10)^\ast)^\ast1^\ast001^\ast$.
\item $L_1 = (00+11)^\ast01(101+01)^\ast$, $L_2 = (0+1)^\ast1^\ast(0+1)^\ast0^\ast10^\ast$.
\item $L_1 = (11+00)^\ast(101+0)^\ast$, $L_2 = (11+000)^\ast(1+01)^\ast$.
\item $L_1 = (101+111)^\ast0(01+10)^\ast1$, $L_2 = 1^\ast0^\ast$.
\item $L_1 = (1+011)^\ast(01+11)^\ast$, $L_2 = (11+00)^\ast(01+10)^\ast$.
\item $L_1 = (aa+bbb)^\ast b^\ast ab^\ast$, $L_2 = a^\ast (bab)^\ast b^\ast$.
\item $L_1 = b^\ast bba^\ast (ab+b)^\ast$, $L_2 = a^\ast ba^\ast ab$.
\item $L_1 = a^\ast b^\ast (a+b+ab+ba)ab^\ast$, $L_2 = a^\ast (bb+aa)^\ast$.
\item $L_1 = (a+ab)^\ast b^\ast (b+ba)a^\ast$, $L_2 = (a+b)b^\ast (ab+bb)^\ast$.
\item $L_1 = (aba+b)^\ast (ab+a)^\ast$, $L_2 = (abb+baa)^\ast$.
\item $L_1 = b(a+b)^\ast (bb+aab)^\ast$, $L_2 = a(a+ab)(b+ab)^\ast$.
\item $L_1 = (aa+bb)^\ast a^\ast ab (a+ba)^\ast$, $L_2 = (aa)^\ast a(a+b)^\ast (bb)^\ast$.
\item $L_1 = (ab)^\ast (a+b)(b+a)(a+bab)^\ast$, $L_2 = ((a+b)^\ast + (aa+bb)^\ast)^\ast$.
\end{enumerate}
\end{itemize}
\renewcommand{\theAlgoEnv}{\Alph{chapter}.\arabic{AlgoEnv}}

\chapter{Пример выполнения заданий курсовой работы}
\section*{Задание 1}
Язык над алфавитом $\Sigma = \{0,1\}$, состоящий из всех слов, в которых хотя бы на одной из последних трех позиций стоит $1$
\begin{enumerate}[label=(\roman{*})]
	\item Построим ПЛ-грамматику $G$, порождающую $L$. Вначале будем выводить (возможно пустой) префикс слова, а затем последние три символа, контролируя момент первого написания единицы.\\
		$G = (\{S, P, T_{0F}, T_{1F}, T_{1T}, T_{2F}, T_{2T}\}, \{0, 1\}, P, S)$
		\begin{align*}
			S &\longrightarrow P\\
			P &\longrightarrow 0P | 1P | T_{0F}\\ 
			T_{0F} &\longrightarrow 0T_{1F} | 1T_{1T}\\
			T_{1F} &\longrightarrow 0T_{2F} | 1T_{2T}\\
			T_{1T} &\longrightarrow 0T_{2T} | 1T_{2T}\\
			T_{2F} &\longrightarrow 1\\
			T_{2T} &\longrightarrow 0 | 1
		\end{align*}
	\item
		\begin{description}
			\item Покажем $L \subset L(G)$:\\
				Предоставим алгоритм вывода произвольного слова $w \in L$ в нашей грамматике. Разобьем слово на префикс и последние три символа $w = pt, |t| = 3$. Префикс выведем продукциями $P \longrightarrow 0P | 1P$. Пусть $k$ - номер первой единицы в подслове $t$. Тогда первые k-1 символов слова $t$ выводятся продукциями вида $T_{i-1F} \longrightarrow 0T_{iF} (i \in \{1, .., k-1\})$, единицу выведем $T_{k-1F} \longrightarrow 1T_{kT}$, конец слова допишем продукциями $T_{i-1T} \longrightarrow 0T_{iT} | 1T_{iT} (i \in \{k+1, .., 3\})$. Таким образом мы можем вывести любое слово из $L$ в грамматике $G$ $\Rightarrow L \subset L(G)$. 
			\item Покажем $L(G) \subset L$:\\
				Все слова, которые выводятся в грамматике $G$, можно разбить на префикс и 3х-символьный суффикс.\\
				$\forall w \in L(G), w = pt, |t| = 3, p \in \{0,1\}^*$ Суффикс же содержит по крайней мере одну единицу, так как вывод завершается, либо продукцией $T_{9F} \longrightarrow 1$ и тогда единственная единица в слове стоит в самом конце слова, либо $T_{9T} \longrightarrow 0 | 1$, а нетерминалы вида $T_{iT}$ выводятся только после написания первой единицы суффикса. Следовательно, все выводимые слова это слова написанные буквами из $\{0,1\}$ такие, что в последних трех символах стоит по меньшей мере одна единица.
		\end{description}

	\item Решим систему линейных уравнений с регулярными коэффициентами
		\begin{align}
			S &= P\\
			P &= 0P + 1P + T_{0F}\\ 
			T_{0F} &= 0T_{1F} + 1T_{1T}\\
			T_{1F} &= 0T_{2F} + 1T_{2T}\\
			T_{1T} &= 0T_{2T} + 1T_{2T}\\
			T_{2F} &= 1\\
			T_{2T} &= 0 + 1
		\end{align}
		\\\\Для уравнений с $(2)$ по $(7)$ совершим последовательную подстановку снизу вверх и упрощая. Получим:
		\begin{align}
			S &= P\\
			P &= (0+1)P + 1(0+1)^2 + (0+1)1(0+1) + (0+1)^21\nonumber
		\end{align}
		Затем для $(9)$ по формуле $A = \alpha A + \beta \Rightarrow A = \alpha^*\beta$ получим подставив сразу в $(8)$:
		\begin{align*}
			S &= (0+1)^*(1(0+1)^2 + (0+1)1(0+1) + (0+1)^21)
		\end{align*}

	\item Построим $M^{ND}$:\\
		\begin{tikzpicture}[initial text={},->,>=stealth',shorten >=1pt,auto,node distance=2.5cm, semithick]
		  \tikzstyle{every state}=[fill=none,draw=black,text=black]

		  \node[initial,state]    (S)              {$q_0$};
		  \node[state]            (1F) [right of=S] {$q_{1F}$};
		  \node[state]            (2F) [right of=1F] {$q_{2F}$};

		  \node[state]            (1T) [below of=1F] {$q_{1T}$};
		  \node[state]            (2T) [right of=1T] {$q_{2T}$};
		  \node[state,accepting]  (3T) [right of=2T] {$q_{3T}$};

		  \path (S) edge [loop below]  node {$0+1$}  (S)
		            edge              node {0} 		(1F)
		            edge              node {1} 		(1T)
		        (1F) edge              node {0} 		(2F)
		        	 edge              node {1} 		(2T)
		        (2F) edge              node {1} 		(3T)

		        (1T) edge              node {$1+0$} 		(2T)
				(2T) edge              node {$1+0$} 		(3T);
		\end{tikzpicture}\\
	\item Детерминируем $M^{ND}$ воспользовавшись алгоритмом детерминизации конечных автоматов.\\
		$\begin{array}{|c||c|c|c|}
			\hline
			q' & $q$ & $0$ & $1$\\
			\hline
			q_0' & \rightarrow\{q_0\} & \{q_0, q_{1F}\} & \{q_0, q_{1T}\}\\
			\hline
			q_1' & \{q_0, q_{1F}\} & \{q_0, q_{1F}, q_{2F}\} & \{q_0, q_{1T}, q_{2T}\}\\
			\hline
			q_2' & \{q_0, q_{1T}\} & \{q_0, q_{1F}, q_{2T}\} & \{q_0, q_{1T}, q_{2T}\}\\
			
			\hline
			q_3' & \{q_0, q_{1F}, q_{2F}\} & \{q_0, q_{1F}, q_{2F}, q_{3F}\} & \{q_0, q_{1T}, q_{2T}, q_{3T}\}\\
			\hline
			q_4' & \{q_0, q_{1F}, q_{2T}\} & \{q_0, q_{1F}, q_{2F}, q_{3T}\} & \{q_0, q_{1T}, q_{2T}, q_{3T}\}\\
			\hline
			q_5' & \{q_0, q_{1T}, q_{2T}\} & \{q_0, q_{1F}, q_{2T}, q_{3T}\} & \{q_0, q_{1T}, q_{2T}, q_{3T}\}\\
			
			\hline
			q_6' & \{q_0, q_{1F}, q_{2F}, q_{3F}\} & \{q_0, q_{1F}, q_{2F}, q_{3F}\} & \{q_0, q_{1T} q_{2T}, q_{3T}\}\\
			\hline
			q_7' & \tikz \node[draw,shape=rounded rectangle, inner sep=1pt]{$\{q_0, q_{1F}, q_{2F}, q_{3T}\}$}; & \{q_0, q_{1F}, q_{2F}, q_{3F}\} & \{q_0, q_{1T}, q_{2T}, q_{3T}\}\\
			\hline
			q_8' & \tikz \node[draw,shape=rounded rectangle, inner sep=1pt]{$\{q_0, q_{1F}, q_{2T}, q_{3T}\}$}; & \{q_0, q_{1F}, q_{2F}, q_{3T}\} & \{q_0, q_{1T}, q_{2T}, q_{3T}\}\\
			\hline
			q_9' & \tikz \node[draw,shape=rounded rectangle, inner sep=1pt]{$\{q_0, q_{1T}, q_{2T}, q_{3T}\}$}; & \{q_0, q_{1F}, q_{2T}, q_{3T}\} & \{q_0, q_{1T}, q_{2T}, q_{3T}\}\\
			\hline
		\end{array}$\\

		$M^{D}$:\\
		\begin{tikzpicture}[initial text={}, ->,>=stealth',shorten >=1pt,auto,node distance=4.0cm, semithick]
		  \tikzstyle{every state}=[fill=none,draw=black,text=black]

		  \node[initial,state]    (0)              {$q_0'$};
		  \node[state]            (1) [right of=0] {$q_1'$};
		  \node[state]            (3) [right of=1] {$q_3'$};
		  \node[state]            (6) [right of=3] {$q_6'$};

		  \node[state]            (2) [below of=0] {$q_2'$};
		  \node[state]            (5) [below right of=0] {$q_5'$};
		  
		  \node[state]            (4) [below of=2] {$q_4'$};

		  \node[state,accepting]  (7) [below right of=2] {$q_7'$};

		  \node[state,accepting]  (8) [below left of=3] {$q_8'$};
		  \node[state,accepting]  (9) [below right of=7] {$q_9'$};

		  \path (0) edge              node {0} 		(1)
		  		(0) edge              node {1} 		(2)

		  		(1) edge              node {0} 		(3)
		  		(1) edge              node {1} 		(5)

		  		(2) edge              node {0} 		(4)
		  		(2) edge              node {1} 		(5)

		  		(3) edge              node {0} 		(6)
		  		(3) edge              node {1} 		(9)

		  		(4) edge              node {0} 		(7)
		  		(4) edge              node {1} 		(9)

		  		(5) edge              node {0} 		(8)
		  		(5) edge              node {1} 		(9)

		  		(6) edge [loop below] node {0} 		(6)
		  		(6) edge              node {1} 		(9)

		  		(7) edge              node {0} 		(6)
		  		(7) edge              node {1} 		(9)

		  		(8) edge              node {0} 		(7)
		  		(8) edge [bend left=10] node {1} 		(9)

		  		(9) edge [bend left=10] node {0} 		(8)
				(9) edge [loop below]  node {1} 		(9);
		\end{tikzpicture}
	\item
		Покажем, что $L(M^{ND}) \subset L$: Рассмотрим произвольное слово $w \in L, |w| = k$. Рассмотрим работу автомата с этим словом на входе. Первые $k-3$ символа прочтутся переходами по петле $0+1$ (конечно, будут клоны, которые будут переходить в состояния $q_{1F}$, $q_{1T}$ и дальше, но они будут безусловно умирать, так как им на прочтение будет оставаться больше трех символов), затем, оставшиеся три символа будут прочтены автоматом, а так как это слово из $L$, т.е. содержит по меньшей мере одну единицу в суффиксе длины $3$, то мы непременно прочтем в этих трех символах единицу и остановимся в $q_{3T}$.\\
		Покажем, что $L \subset L(M^{ND})$: Рассмотрим пути на графе автомата, которые заканчиваются в финальном состоянии $q_{3T}$. Они все состоят из некоторого числа проходов по петле $0+1$, после чего следует последовательность из трех переходов, безусловно содержащих по меньшей мере одну единицу.\\
		Таким образом доказано, что $L(M^{ND}) = L$.\\
		Из того, что мы строили $M^D$ по $M^{ND}$ с помощью алгоритма детерминизации конечных автоматов для которого доказана эквивалентность в смысле равенства распознаваемых языков следует, что $L(M^{ND}) = L(M^D)$.
	\item Для минимизации $M^{D}$ вначале построим множества неразличимых состояний. Для этого на первой итерации образуем два множества неразличимых пустым словом состояний: финальные и не финальные. После чего на каждой итерации будем проверять, что два состояния из одного и того же множества неразличимы по каждой букве, т.е. переходят в одно и тоже множество из прошлой итерации. Если на некоторой итерации $i$ они становятся различимы, то разделяем множество на подмножества неразличимых состояний словами длины $i$. Повторяем этот процесс пока множества не перестанут разделятся. Для минимизации автомата отождествим все неразличимые вершины графа:
		\begin{enumerate}[label=Итерация №\arabic*:]
			\item $\{q_0', q_1', q_2', q_3', q_4', q_5', q_6'\}; \{q_7', q_8', q_9'\};$
			\item $\{q_0', q_1', q_2'\}; \{q_3', q_6'\}; \{q_4', q_5'\}; \{q_7'\}; \{q_8', q_9'\};$
			\item $\{q_0'\}; \{q_1'\}; \{q_2'\}; \{q_3', q_6'\}; \{q_4'\}; \{q_5'\}; \{q_7'\}; \{q_8'\}; \{q_9'\};$
			\item $\{q_0'\}; \{q_1'\}; \{q_2'\}; \{q_3', q_6'\}; \{q_4'\}; \{q_5'\}; \{q_7'\}; \{q_8'\}; \{q_9'\};$
		\end{enumerate}
		
		$M^{D}_{min}$:\\
		\begin{tikzpicture}[initial text={}, ->,>=stealth',shorten >=1pt,auto,node distance=4.0cm, semithick]
		  \tikzstyle{every state}=[fill=none,draw=black,text=black]

		  \node[initial,state]    (0)              {$q_0'$};
		  \node[state]            (1) [right of=0] {$q_1'$};
		  \node[state]            (3) [right of=1] {$q_3'$};

		  \node[state]            (2) [below of=0] {$q_2'$};
		  \node[state]            (5) [below right of=0] {$q_5'$};
		  
		  \node[state]            (4) [below of=2] {$q_4'$};

		  \node[state,accepting]  (7) [below right of=2] {$q_7'$};

		  \node[state,accepting]  (8) [below left of=3] {$q_8'$};
		  \node[state,accepting]  (9) [below right of=7] {$q_9'$};

		  \path (0) edge              node {0} 		(1)
		  		(0) edge              node {1} 		(2)

		  		(1) edge              node {0} 		(3)
		  		(1) edge              node {1} 		(5)

		  		(2) edge              node {0} 		(4)
		  		(2) edge              node {1} 		(5)

		  		(3) edge              node {1} 		(9)
		  		(3) edge [loop right] node {0} 		(3)

		  		(4) edge              node {0} 		(7)
		  		(4) edge              node {1} 		(9)

		  		(5) edge              node {0} 		(8)
		  		(5) edge              node {1} 		(9)

		  		(7) edge [bend right=20] node {0} 		(3)
		  		(7) edge              node {1} 		(9)

		  		(8) edge              node {0} 		(7)
		  		(8) edge [bend left=10] node {1} 		(9)

		  		(9) edge [bend left=10] node {0} 		(8)
				(9) edge [loop below]  node {1} 		(9);
		\end{tikzpicture}\\
		Для доказательства минимальности $M^{ND}$ воспользуемся следующими рассуждениями: для распознавания слов из $L$ и не распознавания слов из $\overline{L}$ нам необходимо читать префикс слова, что выполняет петля из $q_0$ по $0+1$ и читать суффикс убеждаясь, что он имеет длину равную $3$ и содержит как минимум одну единицу. Для подсчета числа прочитанных символов из суффикса достаточно трех вершин, но, так как необходимо контролировать еще и факт прочтения единицы в суффиксе, число необходимых состояний удваивается за счет того, что на каждом из трех этапов мы можем как уже прочесть единицу, так и нет. Но так как состояние обозначающее, что мы прочли три символа суффикса и не встретили единицу - тупиковая ветвь нашего автомата, то её можно выбросить допустив смерть клона читающего ноль в состоянии "прочтено 2 символа и не встречена единица". Таким образом у нас остается 6 состояний все из которых необходимы для успешного функционирования автомата и мы убедились, что $M^{ND}$ минимален.

	\item Воспользуемся методом последовательного исключения состояний для того, чтобы выписать регулярное выражение:\\
	\begin{description}
		\item[$L(M^{ND})$:]\hfill \\
			\begin{enumerate}[label=\bfseries Шаг \arabic*:]
				\item \hfill \\
				    \begin{tikzpicture}[initial text={},->,>=stealth',shorten >=1pt,auto,node distance=2.5cm, semithick]
					  \tikzstyle{every state}=[fill=none,draw=black,text=black]

					  \node[initial,state]    (S)              {$q_0$};
					  \node[state]            (1F) [right of=S] {$q_{1F}$};
					  \node[state]            (2F) [right of=1F] {$q_{2F}$};

					  \node[state]            (1T) [below of=1F] {$q_{1T}$};
					  \node[state]            (2T) [right of=1T] {$q_{2T}$};
					  \node[state,accepting]  (3T) [right of=2T] {$q_{3T}$};

					  \path (S) edge [loop below]  node {$0+1$}  (S)
					            edge              node {0} 		(1F)
					            edge              node {1} 		(1T)
					        (1F) edge              node {0} 		(2F)
					        	 edge              node {1} 		(2T)
					        (2F) edge              node {1} 		(3T)

					        (1T) edge              node {$1+0$} 		(2T)
							(2T) edge              node {$1+0$} 		(3T);
					\end{tikzpicture}\\

				\item \hfill \\
				    \begin{tikzpicture}[initial text={},->,>=stealth',shorten >=1pt,auto,node distance=2.5cm, semithick]
					  \tikzstyle{every state}=[fill=none,draw=black,text=black]

					  \node[initial,state]    (S)              {$q_0$};
					  \node[state]            (1F) [right of=S] {$q_{1F}$};
					  %\node[state]            (2F) [right of=1F] {$q_{2F}$};

					  %\node[state]            (1T) [below of=1F] {$q_{1T}$};
					  \node[state]            (2T) [below of=1F] {$q_{2T}$};
					  \node[state,accepting]  (3T) [right of=2T] {$q_{3T}$};

					  \path (S) edge [loop below]  node {$0+1$}  (S)
					            edge              node {$0$} 		(1F)
					            edge              node [sloped, anchor=center, above]{$1(1+0)$} 		(2T)
					        (1F) edge              node [sloped, anchor=center, above]{$01$} 		(3T)
					        	 edge              node {$1$} 		(2T)
					   %     (2F) edge              node {1} 		(3T)

					    %    (1T) edge              node {1} 		(2T)
							(2T) edge              node {$1+0$} 		(3T);
					\end{tikzpicture}\\

				\item \hfill \\
				    \begin{tikzpicture}[initial text={},->,>=stealth',shorten >=1pt,auto,node distance=3.5cm, semithick]
					  \tikzstyle{every state}=[fill=none,draw=black,text=black]

					  \node[initial,state]    (S)              {$q_0$};
					  %\node[state]            (1F) [right of=S] {$q_{1F}$};
					  %\node[state]            (2F) [right of=1F] {$q_{2F}$};

					  %\node[state]            (1T) [below of=1F] {$q_{1T}$};
					  \node[state]            (2T) [below right of=S] {$q_{2T}$};
					  \node[state,accepting]  (3T) [right of=2T] {$q_{3T}$};

					  \path (S) edge [loop above]  node {$0+1$}  (S)
					            edge              node [sloped, anchor=center, above]{$001$} 		(3T)
					            edge node [sloped, anchor=center, below]{$1(1+0)+01$} 		(2T)
					    %    (1F) edge              node {01} 		(3T)
					    %    	 edge              node {1} 		(2T)
					   %     (2F) edge              node {1} 		(3T)

					    %    (1T) edge              node {1} 		(2T)
							(2T) edge              node {$1+0$} 		(3T);
					\end{tikzpicture}\\

				\item \hfill \\
				    \begin{tikzpicture}[initial text={},->,>=stealth',shorten >=1pt,auto,node distance=6.0cm, semithick]
					  \tikzstyle{every state}=[fill=none,draw=black,text=black]

					  \node[initial,state]    (S)              {$q_0$};
					  %\node[state]            (1F) [right of=S] {$q_{1F}$};
					  %\node[state]            (2F) [right of=1F] {$q_{2F}$};

					  %\node[state]            (1T) [below of=1F] {$q_{1T}$};
					  %\node[state]            (2T) [below of=S] {$q_{2T}$};
					  \node[state,accepting]  (3T) [right of=S] {$q_{3T}$};

					  \path (S) edge [loop above]  node {$0+1$}  (S)
					            edge              node [sloped, anchor=center, above]{$(1(1+0)+01)(1+0) + 001$} 		(3T)
					     		%edge [bend right=50]  node {11+01} 		(2T)
					    %    (1F) edge              node {01} 		(3T)
					    %    	 edge              node {1} 		(2T)
					   %     (2F) edge              node {1} 		(3T)

					    %    (1T) edge              node {1} 		(2T)
						%	(2T) edge              node {1} 		(3T)
							;
					\end{tikzpicture}\\
			\end{enumerate}
			Таким образом мы получаем регулярное выражение $L(M^{ND}) = \\
            (0+1)^*((1(1+0)+01)(1+0) + 001) = \\
            (0+1)^*((11 + 10 + 01)(1+0) + 001) = \\
            (0+1)^*(001 + 010 + 011 + 100 + 101 + 110 + 111)$\\

		\item[$L(M^D)$:]\hfill \\
			\begin{enumerate}[label=\bfseries Шаг \arabic*:]
				\item \hfill \\
				    \begin{tikzpicture}[initial text={}, ->,>=stealth',shorten >=1pt,auto,node distance=4.0cm, semithick]
					  \tikzstyle{every state}=[fill=none,draw=black,text=black]

					  \node[initial,state]    (0)              {$q_0'$};
					  \node[state]            (1) [right of=0] {$q_1'$};
					  \node[state]            (3) [right of=1] {$q_3'$};
					  \node[state]            (6) [right of=3] {$q_6'$};

					  \node[state]            (2) [below of=0] {$q_2'$};
					  \node[state]            (5) [below right of=0] {$q_5'$};
					  
					  \node[state]            (4) [below of=2] {$q_4'$};

					  \node[state,accepting]  (7) [below right of=2] {$q_7'$};

					  \node[state,accepting]  (8) [below left of=3] {$q_8'$};
					  \node[state,accepting]  (9) [below right of=7] {$q_9'$};

					  \path (0) edge              node[sloped, anchor=center, above] {0} 		(1)
					  		(0) edge              node[sloped, anchor=center, above] {1} 		(2)

					  		(1) edge              node[sloped, anchor=center, above] {0} 		(3)
					  		(1) edge              node[sloped, anchor=center, above] {1} 		(5)

					  		(2) edge              node[sloped, anchor=center, above] {0} 		(4)
					  		(2) edge              node[sloped, anchor=center, above] {1} 		(5)

					  		(3) edge              node[sloped, anchor=center, above] {0} 		(6)
					  		(3) edge              node[sloped, anchor=center, above] {1} 		(9)

					  		(4) edge              node[sloped, anchor=center, above] {0} 		(7)
					  		(4) edge              node[sloped, anchor=center, above] {1} 		(9)

					  		(5) edge              node[sloped, anchor=center, above] {0} 		(8)
					  		(5) edge              node[sloped, anchor=center, above] {1} 		(9)

					  		(6) edge [loop below] node[sloped, anchor=center, above] {0} 		(6)
					  		(6) edge              node[sloped, anchor=center, above] {1} 		(9)

					  		(7) edge              node[sloped, anchor=center, above] {0} 		(6)
					  		(7) edge              node[sloped, anchor=center, above] {1} 		(9)

					  		(8) edge              node[sloped, anchor=center, above] {0} 		(7)
					  		(8) edge [bend left=10] node[sloped, anchor=center, above] {1} 		(9)

					  		(9) edge [bend left=10] node[sloped, anchor=center, above] {0} 		(8)
							(9) edge [loop below]  node[sloped, anchor=center, above] {1} 		(9);
					\end{tikzpicture}
				\item \hfill \\
				    \begin{tikzpicture}[initial text={}, ->,>=stealth',shorten >=1pt,auto,node distance=4.0cm, semithick]
					  \tikzstyle{every state}=[fill=none,draw=black,text=black]

					  \node[initial,state]    (0)              {$q_0'$};
					  %\node[state]            (1) [right of=0] {$q_1'$};
					  \node[state]            (3) [right of=1] {$q_3'$};
					  \node[state]            (6) [right of=3] {$q_6'$};

					  %\node[state]            (2) [below of=0] {$q_2'$};
					  \node[state]            (5) [below right of=0] {$q_5'$};
					  
					  \node[state]            (4) [below of=2] {$q_4'$};

					  \node[state,accepting]  (7) [below right of=2] {$q_7'$};

					  \node[state,accepting]  (8) [below left of=3] {$q_8'$};
					  \node[state,accepting]  (9) [below right of=7] {$q_9'$};

					  \path (0) edge              node[sloped, anchor=center, above] {$00$} 		(3)
					  		(0) edge              node[sloped, anchor=center, above] {$01+11$} 		(5)
					  		(0) edge              node[sloped, anchor=center, above] {$10$} 		(4)

					  		%(1) edge              node[sloped, anchor=center, above] {0} 		(3)
					  		%(1) edge              node[sloped, anchor=center, above] {1} 		(5)

					  		%(2) edge              node[sloped, anchor=center, above] {0} 		(4)
					  		%(2) edge              node[sloped, anchor=center, above] {1} 		(5)

					  		(3) edge              node[sloped, anchor=center, above] {0} 		(6)
					  		(3) edge              node[sloped, anchor=center, above] {1} 		(9)

					  		(4) edge              node[sloped, anchor=center, above] {0} 		(7)
					  		(4) edge              node[sloped, anchor=center, above] {1} 		(9)

					  		(5) edge              node[sloped, anchor=center, above] {0} 		(8)
					  		(5) edge              node[sloped, anchor=center, above] {1} 		(9)

					  		(6) edge [loop below] node[sloped, anchor=center, above] {0} 		(6)
					  		(6) edge              node[sloped, anchor=center, above] {1} 		(9)

					  		(7) edge              node[sloped, anchor=center, above] {0} 		(6)
					  		(7) edge              node[sloped, anchor=center, above] {1} 		(9)

					  		(8) edge              node[sloped, anchor=center, above] {0} 		(7)
					  		(8) edge [bend left=10] node[sloped, anchor=center, above] {1} 		(9)

					  		(9) edge [bend left=10] node[sloped, anchor=center, above] {0} 		(8)
							(9) edge [loop below]  node[sloped, anchor=center, above] {1} 		(9);
					\end{tikzpicture} 
				\item \hfill \\
				    \begin{tikzpicture}[initial text={}, ->,>=stealth',shorten >=1pt,auto,node distance=4.0cm, semithick]
					  \tikzstyle{every state}=[fill=none,draw=black,text=black]

					  \node[initial,state]    (0)              {$q_0'$};
					  %\node[state]            (1) [right of=0] {$q_1'$};
					  %\node[state]            (3) [right of=1] {$q_3'$};
					  \node[state]            (6) [right of=3] {$q_6'$};

					  %\node[state]            (2) [below of=0] {$q_2'$};
					  %\node[state]            (5) [below right of=0] {$q_5'$};
					  
					  \node[state]            (4) [below of=2] {$q_4'$};

					  \node[state,accepting]  (7) [below right of=2] {$q_7'$};

					  \node[state,accepting]  (8) [below left of=3] {$q_8'$};
					  \node[state,accepting]  (9) [below right of=7] {$q_9'$};

					  \path (0) edge              node[sloped, anchor=center, above] {$000$} 		(6)
					  		(0) edge              node[sloped, anchor=center, above] {$(01+11)0$} 		(8)
					  		(0) edge              node[sloped, anchor=center, above] {$001 + (01+11)1$} 		(9)
					  		(0) edge              node[sloped, anchor=center, above] {$10$} 		(4)

					  		%(1) edge              node[sloped, anchor=center, above] {0} 		(3)
					  		%(1) edge              node[sloped, anchor=center, above] {1} 		(5)

					  		%(2) edge              node[sloped, anchor=center, above] {0} 		(4)
					  		%(2) edge              node[sloped, anchor=center, above] {1} 		(5)

					  		%(3) edge              node[sloped, anchor=center, above] {0} 		(6)
					  		%(3) edge              node[sloped, anchor=center, above] {1} 		(9)

					  		(4) edge              node[sloped, anchor=center, above] {0} 		(7)
					  		(4) edge              node[sloped, anchor=center, above] {1} 		(9)

					  		%(5) edge              node[sloped, anchor=center, above] {0} 		(8)
					  		%(5) edge              node[sloped, anchor=center, above] {1} 		(9)

					  		(6) edge [loop below] node[sloped, anchor=center, above] {0} 		(6)
					  		(6) edge              node[sloped, anchor=center, above] {1} 		(9)

					  		(7) edge              node[sloped, anchor=center, above] {0} 		(6)
					  		(7) edge              node[sloped, anchor=center, above] {1} 		(9)

					  		(8) edge              node[sloped, anchor=center, above] {0} 		(7)
					  		(8) edge [bend left=10] node[sloped, anchor=center, above] {1} 		(9)

					  		(9) edge [bend left=10] node[sloped, anchor=center, above] {0} 		(8)
							(9) edge [loop below]  node[sloped, anchor=center, above] {1} 		(9);
					\end{tikzpicture} 
				\item \hfill \\
				    \begin{tikzpicture}[initial text={}, ->,>=stealth',shorten >=1pt,auto,node distance=6.0cm, semithick]
					  \tikzstyle{every state}=[fill=none,draw=black,text=black]

					  \node[initial,state]    (0)              {$q_0'$};
					  %\node[state]            (1) [right of=0] {$q_1'$};
					  %\node[state]            (3) [right of=1] {$q_3'$};
					  %\node[state]            (6) [right of=3] {$q_6'$};

					  %\node[state]            (2) [below of=0] {$q_2'$};
					  %\node[state]            (5) [below right of=0] {$q_5'$};
					  
					  %\node[state]            (4) [below of=2] {$q_4'$};

					  \node[state,accepting]  (7) [below of=0] {$q_7'$};

					  \node[state,accepting]  (8) [right of=0] {$q_8'$};
					  \node[state,accepting]  (9) [right of=7] {$q_9'$};

					  \path %(0) edge              node {$000$} 		(6)
					  		(0) edge              node[sloped, anchor=center, above] {$(01+11)0$} 		(8)
					  		(0) edge              node [sloped, xshift=-0.2em,anchor=center, above]{$001 + (01+11)1 + 101 + 0000^*1$} 		(9)
					  		(0) edge              node[sloped, anchor=center, above] {$100$} 		(7)

					  		%(1) edge              node[sloped, anchor=center, above] {0} 		(3)
					  		%(1) edge              node[sloped, anchor=center, above] {1} 		(5)

					  		%(2) edge              node[sloped, anchor=center, above] {0} 		(4)
					  		%(2) edge              node[sloped, anchor=center, above] {1} 		(5)

					  		%(3) edge              node[sloped, anchor=center, above] {0} 		(6)
					  		%(3) edge              node[sloped, anchor=center, above] {1} 		(9)

					  		%(4) edge              node[sloped, anchor=center, above] {0} 		(7)
					  		%(4) edge              node[sloped, anchor=center, above] {1} 		(9)

					  		%(5) edge              node[sloped, anchor=center, above] {0} 		(8)
					  		%(5) edge              node[sloped, anchor=center, above] {1} 		(9)

					  		%(6) edge [loop below] node[sloped, anchor=center, above] {0} 		(6)
					  		%(6) edge              node[sloped, anchor=center, above] {1} 		(9)

					  		%(7) edge              node[sloped, anchor=center, above] {0} 		(6)
					  		(7) edge              node[sloped, anchor=center, above] {$1 + 00^*1$} 		(9)

					  		(8) edge              node  [sloped, xshift=-5.0em, anchor=west, above]{$0$} 		(7)
					  		(8) edge [bend left=10] node[sloped, anchor=center, above] {1} 		(9)

					  		(9) edge [bend left=10] node[sloped, anchor=center, above] {0} 		(8)
							(9) edge [loop below]  node[sloped, anchor=center, above] {1} 		(9);
					\end{tikzpicture} 
				\item Добавим новое состояние, которое сделаем финальным и в которое пустим спонтанные переходы из текущих финальных. Также текущие финальные сделаем нефинальными. \\
				    \begin{tikzpicture}[initial text={}, ->,>=stealth',shorten >=1pt,auto,node distance=6.0cm, semithick]
					  \tikzstyle{every state}=[fill=none,draw=black,text=black]

					  \node[initial,state]    (0)              {$q_0'$};
					  %\node[state]            (1) [right of=0] {$q_1'$};
					  %\node[state]            (3) [right of=1] {$q_3'$};
					  %\node[state]            (6) [right of=3] {$q_6'$};

					  %\node[state]            (2) [below of=0] {$q_2'$};
					  %\node[state]            (5) [below right of=0] {$q_5'$};
					  
					  %\node[state]            (4) [below of=2] {$q_4'$};

					  \node[state]  (7) [below of=0] {$q_7'$};

					  \node[state]  (8) [right of=0] {$q_8'$};
					  \node[state]  (9) [right of=7] {$q_9'$};
					  \node[state,accepting]  (F) [below right of=9 ] {$q_f'$};

					  \path %(0) edge              node {$000$} 		(6)
					  		(0) edge              node[sloped, anchor=center, above] {$(01+11)0$} 		(8)
					  		(0) edge              node [sloped, xshift=-0.2em,anchor=center, above]{$001 + (01+11)1 + 101 + 0000^*1$} 		(9)
					  		(0) edge              node[sloped, anchor=center, above] {$100$} 		(7)

					  		%(1) edge              node {0} 		(3)
					  		%(1) edge              node {1} 		(5)

					  		%(2) edge              node {0} 		(4)
					  		%(2) edge              node {1} 		(5)

					  		%(3) edge              node {0} 		(6)
					  		%(3) edge              node {1} 		(9)

					  		%(4) edge              node {0} 		(7)
					  		%(4) edge              node {1} 		(9)

					  		%(5) edge              node {0} 		(8)
					  		%(5) edge              node {1} 		(9)

					  		%(6) edge [loop below] node {0} 		(6)
					  		%(6) edge              node {1} 		(9)

					  		%(7) edge              node {0} 		(6)
					  		(7) edge              node[sloped, anchor=center, above] {$1 + 00^*1$} 		(9)

					  		(8) edge              node  [sloped, xshift=-5.0em, anchor=west, above]{$0$} 		(7)
					  		(8) edge [bend left=10] node[sloped, anchor=center, above] {1} 		(9)

					  		(9) edge [bend left=10] node[sloped, anchor=center, above] {0} 		(8)
							(9) edge [loop below]  node[sloped, anchor=center, above] {1} 		(9)

							(7) edge node [sloped, anchor=center, above]{$e$} 		(F)
							(8) edge node [sloped, anchor=center, above]{$e$} 		(F)
							(9) edge node [sloped, anchor=center, above]{$e$} 		(F);
					\end{tikzpicture} 
				\item \hfill \\
				    \begin{tikzpicture}[initial text={}, ->,>=stealth',shorten >=1pt,auto,node distance=6.0cm, semithick]
					  \tikzstyle{every state}=[fill=none,draw=black,text=black]

					  \node[initial,state]    (0)              {$q_0'$};
					  %\node[state]            (1) [right of=0] {$q_1'$};
					  %\node[state]            (3) [right of=1] {$q_3'$};
					  %\node[state]            (6) [right of=3] {$q_6'$};

					  %\node[state]            (2) [below of=0] {$q_2'$};
					  %\node[state]            (5) [below right of=0] {$q_5'$};
					  
					  %\node[state]            (4) [below of=2] {$q_4'$};

					  \node[state]  (7) [below of=0] {$q_7'$};

					  \node[state]  (8) [right of=0] {$q_8'$};
					  %\node[state]  (9) [right of=7] {$q_9'$};
					  \node[state,accepting]  (F) [below right of=9 ] {$q_f'$};

					  \path %(0) edge              node {$000$} 		(6)
					  		(0) edge              node [sloped, anchor=center, above, text width=5.0cm]{$(01+11)0 + (001 + (01+11)1 + 101 + 0000^*1)1$} 		(8)
					  		(0) edge              node [sloped, xshift=0.4em,anchor=center, above]{$001 + (01+11)1 + 101 + 0000^*1$} 		(F)
					  		(0) edge              node [sloped, anchor=center, above]{$100$} 		(7)

					  		%(1) edge              node {0} 		(3)
					  		%(1) edge              node {1} 		(5)

					  		%(2) edge              node {0} 		(4)
					  		%(2) edge              node {1} 		(5)

					  		%(3) edge              node {0} 		(6)
					  		%(3) edge              node {1} 		(9)

					  		%(4) edge              node {0} 		(7)
					  		%(4) edge              node {1} 		(9)

					  		%(5) edge              node {0} 		(8)
					  		%(5) edge              node {1} 		(9)

					  		%(6) edge [loop below] node {0} 		(6)
					  		%(6) edge              node {1} 		(9)

					  		%(7) edge              node {0} 		(6)
					  		%(7) edge              node [sloped, anchor=center, above]{$1 + 00^*1$} 		(9)
					  		(7) edge [bend left]             node [sloped, anchor=center, above]{$(1 + 00^*1)1^*0$} 		(8)

					  		(8) edge [bend left]             node  [sloped, xshift=-5.0em, anchor=west, above]{$0$} 		(7)
					  		(8) edge [loop above] node[sloped, anchor=center, above] {$10$} 		(8)

					  		%(9) edge [bend left=10] node {0} 		(8)
							%(9) edge [loop below]  node {1} 		(9)

							(7) edge node [sloped, anchor=center, above]{$e + (1+00^*1)1^*$} 		(F)
							(8) edge node [sloped, anchor=center, above]{$e + 11^*$} 		(F)
							%(9) edge node [sloped, anchor=center, above]{$e$} 		(F)
							;
					\end{tikzpicture} 

				\item \hfill \\
				    \begin{tikzpicture}[initial text={}, ->,>=stealth',shorten >=1pt,auto,node distance=8.3cm, semithick]
					  \tikzstyle{every state}=[fill=none,draw=black,text=black]

					  \node[initial,state]    (0)              {$q_0'$};
					  %\node[state]  (7) [below of=0] {$q_7'$};

					  \node[state]  (8) [right of=0] {$q_8'$};
					  \node[state,accepting]  (F) [below of=8 ] {$q_f'$};

					  \path 
					  		(0) edge              node [sloped, anchor=center, above, text width=5.0cm]{$(01+11)0 + (001 + (01+11)1 + 101 + 0000^*1)1 + 100(1+00^*1)1^*0$} 		(8)
					  		(0) edge              node [sloped, xshift=0.4em,anchor=center, above]{$001 + (01+11)1 + 101 + 0000^*1 + 100(e + (1+00^*1)1^*)$} 		(F)
					  		(8) edge [loop right] node {$10 + 0(1+00^*1)1^*0$} 		(8)
							(8) edge node [sloped, anchor=center, above]{$e + 11^* + 0(e + (1+00^*1)1^*)$} 		(F)
							;
					\end{tikzpicture} 
				\item \hfill \\
				    \begin{tikzpicture}[initial text={}, ->,>=stealth',shorten >=1pt,auto,node distance=9.0cm, semithick]
					  \tikzstyle{every state}=[fill=none,draw=black,text=black]

					  \node[initial,state]    (0)              {$q_0'$};
					  %\node[state]  (8) [right of=0] {$q_8'$};
					  \node[state,accepting]  (F) [right of=0 ] {$q_f'$};

					  \path 
					  		(0) edge              node [sloped, xshift=0.4em,anchor=center, above, text width=8.0cm]{$001 + (01+11)1 + 101 + 0000^*1 + 100(e + (1+00^*1)1^*) + ((01+11)0 + (001 + (01+11)1 + 101 + 0000^*1)1 + 100(1+00^*1)1^*0)(10 + 0(1+00^*1)1^*0)^*(e + 11^* + 0(e + (1+00^*1)1^*))$} 		(F);
					\end{tikzpicture}
			\end{enumerate}
			Таким образом мы получаем регулярное выражение $L(M^{D}) = 001 + (01+11)1 + 101 + 0000^*1 + 100(e + (1+00^*1)1^*) + ((01+11)0 + (001 + (01+11)1 + 101 + 0000^*1)1 + 100(1+00^*1)1^*0)(10 + 0(1+00^*1)1^*0)^*(e + 11^* + 0(e + (1+00^*1)1^*))$\\
	\end{description}

	\item $\overline{M^D}$: \\
		\begin{tikzpicture}[initial text={}, ->,>=stealth',shorten >=1pt,auto,node distance=4.0cm, semithick]
		  \tikzstyle{every state}=[fill=none,draw=black,text=black]

		  \node[initial,state,accepting]    (0)              {$q_0'$};
		  \node[state,accepting]            (1) [right of=0] {$q_1'$};
		  \node[state,accepting]            (3) [right of=1] {$q_3'$};

		  \node[state,accepting]            (2) [below of=0] {$q_2'$};
		  \node[state,accepting]            (5) [below right of=0] {$q_5'$};
		  
		  \node[state,accepting]            (4) [below of=2] {$q_4'$};

		  \node[state]  (7) [below right of=2] {$q_7'$};

		  \node[state]  (8) [below left of=3] {$q_8'$};
		  \node[state]  (9) [below right of=7] {$q_9'$};

		  \path (0) edge              node {0} 		(1)
		  		(0) edge              node {1} 		(2)

		  		(1) edge              node {0} 		(3)
		  		(1) edge              node {1} 		(5)

		  		(2) edge              node {0} 		(4)
		  		(2) edge              node {1} 		(5)

		  		(3) edge              node {1} 		(9)
		  		(3) edge [loop right] node {0} 		(3)

		  		(4) edge              node {0} 		(7)
		  		(4) edge              node {1} 		(9)

		  		(5) edge              node {0} 		(8)
		  		(5) edge              node {1} 		(9)

		  		(7) edge [bend right=20] node {0} 		(3)
		  		(7) edge              node {1} 		(9)

		  		(8) edge              node {0} 		(7)
		  		(8) edge [bend left=10] node {1} 		(9)

		  		(9) edge [bend left=10] node {0} 		(8)
				(9) edge [loop below]  node {1} 		(9);
		\end{tikzpicture}\\
		$\overline{L} = e + 0 + 1 + 00 + 01 + 10 + 11 + (0+1)^*000$
\end{enumerate}
\newpage
\section*{Задание 2}
$\Sigma = \{a, b, c\}, A = \{bbab, ccac, bacb, baabc\}.$
\begin{enumerate}[label=(\roman{*})]
	\item Для каждого слова $w_i \in A$ построить НКА $M^{ND}_i$, распознающий наличие в произвольной строке $s \in \Sigma^*$ подстроки $w_i$.
	
	
	\begin{enumerate}
		\item НКА $M^{ND}_1$, распознающий наличие в произвольной строке $s \in \Sigma^*$ подстроки $w_1$, где $w_1$ = bbab :
		
		$M^{ND}_1 = (\{q_0, q_1, q_2, q_3, q_4, q_f\}, \{a, b, c\}, \delta, q_0, \{q_f\})$  
		
		
			\begin{center}
			\begin{tabular}{llll}
				\toprule
				\multicolumn{1}{c}{\multirow{2}{*}{\Large $\delta$}}
				& \multicolumn{3}{c}{Вход} \\
				\cmidrule(rl){2-4}
				& \multicolumn{1}{c}{$a$}
				& \multicolumn{1}{c}{$b$} 
				& \multicolumn{1}{c}{$c$} \\
				\midrule
				$\{q_0\}$       & $\{q_0\}$      		 & $\{q_1\}$     &$\{q_0\}$  \\
				$\{q_1\}$       & $\{q_0\}$    			 & $\{q_2\}$     &$\{q_0\}$ \\
				$\{q_2\}$       & $\{q_4\}$    			 & $\{q_3\}$     &$\{q_0\}$  \\
				$\{q_3\}$       & $\{q_4\}$    			 & $\{q_2, q_3\}$     &$\{q_0\}$  \\
				$\{q_4\}$       & $\{q_0\}$    			 & $\{q_f\}$     &$\{q_0\}$  \\
				$\{q_f\}$       & $\{\varnothing\}$    	 & $\{\varnothing\}$     &$\{\varnothing\}$  \\
				\bottomrule
			\end{tabular}
		\end{center}
		
		
		\item НКА $M^{ND}_2$, распознающий наличие в произвольной строке $s \in \Sigma^*$ подстроки $w_2$, где $w_2$ = ссaс :
		
		$M^{ND}_2 = (\{q_0, q_1, q_2, q_3, q_4, q_f\}, \{a, b, c\}, \delta, q_0, \{q_f\})$
		
		\begin{center}
			\begin{tabular}{llll}
				\toprule
				\multicolumn{1}{c}{\multirow{2}{*}{\Large $\delta$}}
				& \multicolumn{3}{c}{Вход} \\
				\cmidrule(rl){2-4}
				& \multicolumn{1}{c}{$a$}
				& \multicolumn{1}{c}{$b$} 
				& \multicolumn{1}{c}{$c$} \\
				\midrule
				$\{q_0\}$       & $\{q_0\}$      		 & $\{q_0\}$     &$\{q_1\}$  \\
				$\{q_1\}$       & $\{q_0\}$    			 & $\{q_0\}$     &$\{q_2\}$ \\
				$\{q_2\}$       & $\{q_4\}$    			 & $\{q_0\}$     &$\{q_3\}$  \\
				$\{q_3\}$       & $\{q_4\}$    			 & $\{q_0\}$     &$\{q_2, q_3\}$  \\
				$\{q_4\}$       & $\{q_0\}$    			 & $\{q_0\}$     &$\{q_f\}$  \\
				$\{q_f\}$       & $\{\varnothing\}$    	 & $\{\varnothing\}$     &$\{\varnothing\}$  \\
				\bottomrule
			\end{tabular}
		\end{center}
		
		
		\item НКА $M^{ND}_3$, распознающий наличие в произвольной строке $s \in \Sigma^*$ подстроки $w_3$, где $w_3$ = bacb :
		
		$M^{ND}_3 = (\{q_0, q_1, q_2, q_3, q_4, q_f\}, \{a, b, c\}, \delta, q_0, \{q_f\})$
		
		\begin{center}
			\begin{tabular}{llll}
				\toprule
				\multicolumn{1}{c}{\multirow{2}{*}{\Large $\delta$}}
				& \multicolumn{3}{c}{Вход} \\
				\cmidrule(rl){2-4}
				& \multicolumn{1}{c}{$a$}
				& \multicolumn{1}{c}{$b$} 
				& \multicolumn{1}{c}{$c$} \\
				\midrule
				$\{q_0\}$       & $\{q_0\}$      		 & $\{q_1\}$     &$\{q_0\}$  \\
				$\{q_1\}$       & $\{q_3\}$    			 & $\{q_2\}$     &$\{q_0\}$ \\
				$\{q_2\}$       & $\{q_3\}$    			 & $\{q_1, q_2\}$     &$\{q_0\}$  \\
				$\{q_3\}$       & $\{q_0\}$    			 & $\{q_1, q_2\}$     &$\{q_4\}$  \\
				$\{q_4\}$       & $\{q_0\}$    			 & $\{q_f\}$     &$\{q_0\}$  \\
				$\{q_f\}$       & $\{\varnothing\}$    	 & $\{\varnothing\}$     &$\{\varnothing\}$  \\
				\bottomrule
			\end{tabular}
		\end{center}
		
		\item НКА $M^{ND}_4$, распознающий наличие в произвольной строке $s \in \Sigma^*$ подстроки $w_4$, где $w_4$ = baabc :
		
		$M^{ND}_4 = (\{q_0, q_1, q_2, q_3, q_4, q_5, q_f\}, \{a, b, c\}, \delta, q_0, \{q_f\})$
		
		\begin{center}
			\begin{tabular}{llll}
				\toprule
				\multicolumn{1}{c}{\multirow{2}{*}{\Large $\delta$}}
				& \multicolumn{3}{c}{Вход} \\
				\cmidrule(rl){2-4}
				& \multicolumn{1}{c}{$a$}
				& \multicolumn{1}{c}{$b$} 
				& \multicolumn{1}{c}{$c$} \\
				\midrule
				$\{q_0\}$       & $\{q_0\}$      		 & $\{q_1\}$     &$\{q_0\}$  \\
				$\{q_1\}$       & $\{q_3\}$    			 & $\{q_2\}$     &$\{q_0\}$ \\
				$\{q_2\}$       & $\{q_3\}$    			 & $\{q_2\}$     &$\{q_0\}$  \\
				$\{q_3\}$       & $\{q_4\}$    			 & $\{q_1, q_2\}$     &$\{q_0\}$  \\
				$\{q_4\}$       & $\{q_0\}$    			 & $\{q_5\}$     &$\{q_0\}$  \\
				$\{q_5\}$       & $\{q_3\}$    			 & $\{q_1, q_2\}$     &$\{q_f\}$  \\
				$\{q_f\}$       & $\{\varnothing\}$    	 & $\{\varnothing\}$     &$\{\varnothing\}$  \\
				\bottomrule
			\end{tabular}
		\end{center}
		
	\end{enumerate}

	
	\item Для каждого НКА $M^{ND}_i$ построить соответствующий ДКА $M^D_i$.
		
	\begin{enumerate}
		
		\item ДКА $M^{D}_1$, соответствующий НКА  $M^{ND}_1$ :
		\newline
		
		Переобозначим : $\{q_2, q_3\} = \{q_5\}$ 
		\newline 
		
		$M^{D}_1 = (\{q_0, q_1, q_2, q_3, q_4, q_5, q_f\}, \{a, b, c\}, \delta, q_0, \{q_f\})$
		
		
		\begin{center}
			\begin{tabular}{llll}
				\toprule
				\multicolumn{1}{c}{\multirow{2}{*}{\Large $\delta$}}
				& \multicolumn{3}{c}{Вход} \\
				\cmidrule(rl){2-4}
				& \multicolumn{1}{c}{$a$}
				& \multicolumn{1}{c}{$b$} 
				& \multicolumn{1}{c}{$c$} \\
				\midrule
				$\{q_0\}$       & $\{q_0\}$      		 & $\{q_1\}$     &$\{q_0\}$  \\
				$\{q_1\}$       & $\{q_0\}$    			 & $\{q_2\}$     &$\{q_0\}$ \\
				$\{q_2\}$       & $\{q_4\}$    			 & $\{q_3\}$     &$\{q_0\}$  \\
				$\{q_3\}$       & $\{q_4\}$    			 & $\{q_5\}$     &$\{q_0\}$  \\
				$\{q_4\}$       & $\{q_0\}$    			 & $\{q_f\}$     &$\{q_0\}$  \\
				$\{q_5\}$ 		& $\{q_4\}$    			 & $\{q_5\}$     &$\{q_0\}$  \\
				$\{q_f\}$       & $\{\varnothing\}$    	 & $\{\varnothing\}$     &$\{\varnothing\}$  \\
				\bottomrule
			\end{tabular}
		\end{center}
		
		
		\item ДКА $M^{D}_2$, соответствующий НКА  $M^{ND}_2$ :
		\newline
		
		Переобозначим : $\{q_2, q_3\} = \{q_5\}$ 
		\newline 
		
		$M^{D}_2 = (\{q_0, q_1, q_2, q_3, q_4, q_5, q_f\}, \{a, b, c\}, \delta, q_0, \{q_f\})$
		
			\begin{center}
			\begin{tabular}{llll}
				\toprule
				\multicolumn{1}{c}{\multirow{2}{*}{\Large $\delta$}}
				& \multicolumn{3}{c}{Вход} \\
				\cmidrule(rl){2-4}
				& \multicolumn{1}{c}{$a$}
				& \multicolumn{1}{c}{$b$} 
				& \multicolumn{1}{c}{$c$} \\
				\midrule
				$\{q_0\}$       & $\{q_0\}$      		 & $\{q_0\}$     &$\{q_1\}$  \\
				$\{q_1\}$       & $\{q_0\}$    			 & $\{q_0\}$     &$\{q_2\}$ \\
				$\{q_2\}$       & $\{q_4\}$    			 & $\{q_0\}$     &$\{q_3\}$  \\
				$\{q_3\}$       & $\{q_4\}$    			 & $\{q_0\}$     &$\{q_5\}$  \\
				$\{q_4\}$       & $\{q_0\}$    			 & $\{q_0\}$     &$\{q_f\}$  \\
				$\{q_5\}$       & $\{q_4\}$    			 & $\{q_0\}$     &$\{q_5\}$  \\
				$\{q_f\}$       & $\{\varnothing\}$    	 & $\{\varnothing\}$     &$\{\varnothing\}$  \\
				\bottomrule
			\end{tabular}
		\end{center}
		
		\item ДКА $M^{D}_3$, соответствующий НКА  $M^{ND}_3$ :
		\newline
		
		Переобозначим : $\{q_1, q_2\} = \{q_5\}$ 
		\newline 
		
		$M^{D}_3 = (\{q_0, q_1, q_2, q_3, q_4, q_5, q_f\}, \{a, b, c\}, \delta, q_0, \{q_f\})$
		
			\begin{center}
			\begin{tabular}{llll}
				\toprule
				\multicolumn{1}{c}{\multirow{2}{*}{\Large $\delta$}}
				& \multicolumn{3}{c}{Вход} \\
				\cmidrule(rl){2-4}
				& \multicolumn{1}{c}{$a$}
				& \multicolumn{1}{c}{$b$} 
				& \multicolumn{1}{c}{$c$} \\
				\midrule
				$\{q_0\}$       & $\{q_0\}$      		 & $\{q_1\}$     &$\{q_0\}$  \\
				$\{q_1\}$       & $\{q_3\}$    			 & $\{q_2\}$     &$\{q_0\}$ \\
				$\{q_2\}$       & $\{q_3\}$    			 & $\{q_5\}$     &$\{q_0\}$  \\
				$\{q_3\}$       & $\{q_0\}$    			 & $\{q_5\}$     &$\{q_4\}$  \\
				$\{q_4\}$       & $\{q_0\}$    			 & $\{q_f\}$     &$\{q_0\}$  \\
				$\{q_5\}$       & $\{q_3\}$    			 & $\{q_5\}$     &$\{q_0\}$ \\
				$\{q_f\}$       & $\{\varnothing\}$    	 & $\{\varnothing\}$     &$\{\varnothing\}$  \\
				\bottomrule
			\end{tabular}
		\end{center}
		
		\item ДКА $M^{D}_4$, соответствующий НКА  $M^{ND}_4$ :
		\newline
		
		Переобозначим : $\{q_1, q_2\} = \{q_6\}$ 
		\newline 
		
		$M^{D}_4 = (\{q_0, q_1, q_2, q_3, q_4, q_5, q_6, q_f\}, \{a, b, c\}, \delta, q_0, \{q_f\})$
		\newline
			
		\begin{center}
			\begin{tabular}{llll}
				\toprule
				\multicolumn{1}{c}{\multirow{2}{*}{\Large $\delta$}}
				& \multicolumn{3}{c}{Вход} \\
				\cmidrule(rl){2-4}
				& \multicolumn{1}{c}{$a$}
				& \multicolumn{1}{c}{$b$} 
				& \multicolumn{1}{c}{$c$} \\
				\midrule
				$\{q_0\}$       & $\{q_0\}$      		 & $\{q_1\}$     &$\{q_0\}$  \\
				$\{q_1\}$       & $\{q_3\}$    			 & $\{q_2\}$     &$\{q_0\}$ \\
				$\{q_2\}$       & $\{q_3\}$    			 & $\{q_2\}$     &$\{q_0\}$  \\
				$\{q_3\}$       & $\{q_4\}$    			 & $\{q_6\}$     &$\{q_0\}$  \\
				$\{q_4\}$       & $\{q_0\}$    			 & $\{q_5\}$     &$\{q_0\}$  \\
				$\{q_5\}$       & $\{q_3\}$    			 & $\{q_6\}$     &$\{q_f\}$  \\
				$\{q_6\}$       & $\{q_3\}$    			 & $\{q_2\}$     &$\{q_0\}$ \\
				$\{q_f\}$       & $\{\varnothing\}$    	 & $\{\varnothing\}$     &$\{\varnothing\}$  \\
				\bottomrule
			\end{tabular}
		\end{center}
		
	\end{enumerate}
\end{enumerate}

\newpage
\section*{Задание 3}
$L_1 = (aa + bb)^*a^*ab(a + ba)^*$, $L_2 = (aa)^*a(a + b)^*(bb)^*$
\newline
	
	Вспомогательные построения автоматов для решения задания 3 :
	\newline
	
	НКА для $L_1$:
	
	\begin{center}	
		\begin{tikzpicture}[auto,>=stealth', node distance=3cm,auto,every state/.style={thick}]
		\node (init) {};
		\node[state] (1) [right=.7cm of init] {$1$};
		\node[state] (2) [above of=1] {$2$};
		\node[state] (3) [below of=1] {$3$};
		\node[state] (4) [right of=1] {$4$};	
		\node[state, accepting] (5) [right of=4] {$5$};	
		\node[state] (7) [right of=5, below of=5] {$7$};	
		\node[state, accepting] (6) [right of=7, above of=7] {$6$};	
		\node[state] (8) [right of=6] {$8$};
			
		\path[->]
		(init) edge (1)
		(1) edge[bend right] node[right] {$a$} (2)
		(2) edge[bend right] node[left] {$a$} (1)
		(1) edge[bend left] node[right] {$b$} (3)
		(3) edge[bend left] node[left] {$b$} (1)
		(1) edge node {$a$} (4)
		(4) edge [loop above] node {$a$} (4)
		(4) edge node {$b$} (5)
		(5) edge node {$a$} (6)
		(5) edge node {$b$} (7)
		(7) edge node {$a$} (6)
		(6) edge [loop above] node {$a$} (6)
		(6) edge[bend left] node {$b$} (8)
		(8) edge[bend left] node {$a$} (6)
		;
		\end{tikzpicture}
	\end{center}

	\begin{center}
		\begin{tabular}{lll}
			\toprule
			\multicolumn{1}{c}{\multirow{2}{*}{\Large $\delta$}}
			& \multicolumn{2}{c}{Вход} \\
			\cmidrule(rl){2-3}
			& \multicolumn{1}{c}{$a$}
			& \multicolumn{1}{c}{$b$}  \\
			\midrule
			$\to \{1\}$       & $\{2\}, \{4\}$      		 & $\{3\}$      \\
			$\{2\}$       & $\{1\}$    			 & $\{\varnothing\}$      \\
			$\{3\}$       & $\{\varnothing\}$    			 & $\{1\}$       \\
			$\{4\}$       & $\{4\}$    			 & $\{5\}$       \\
			$\{\textbf{5}\}$       & $\{6\}$    			 & $\{7\}$       \\
			$\{\textbf{6}\}$       & $\{6\}$    			 & $\{8\}$     \\
			$\{7\}$       & $\{6\}$    			 & $\{\varnothing\}$     \\
			$\{8\}$       & $\{6\}$    	 & $\{\varnothing\}$  \\
			\bottomrule
		\end{tabular}
	\end{center}
	
	НКА для $L_2$:
	
		\begin{center}	
		\begin{tikzpicture}[auto,>=stealth', node distance=3cm,auto,every state/.style={thick}]
		\node (init) {};
		\node[state] (1) [right=.7cm of init] {$1$};
		\node[state] (2) [above of=1] {$2$};
		\node[state, accepting] (3) [right of=1] {$3$};	
		\node[state] (4) [right of=3] {$4$};	
		\node[state, accepting] (5) [right of=4] {$5$};	
		\node[state] (6) [right of=5] {$6$};	
		
		\path[->]
		(init) edge (1)
		(1) edge[bend right] node[right] {$a$} (2)
		(2) edge[bend right] node[left] {$a$} (1)
		(1) edge node {$a$} (3)
		(3) edge[loop above] node {$a, b$} (3)
		(3) edge node {$b$} (4)
		(4) edge node {$b$} (5)
		(5) edge[bend left] node {$b$} (6)
		(6) edge[bend left] node {$b$} (5)
		;
		\end{tikzpicture}
	\end{center}

	\begin{center}
		\begin{tabular}{lll}
			\toprule
			\multicolumn{1}{c}{\multirow{2}{*}{\Large $\delta$}}
			& \multicolumn{2}{c}{Вход} \\
			\cmidrule(rl){2-3}
			& \multicolumn{1}{c}{$a$}
			& \multicolumn{1}{c}{$b$}  \\
			\midrule
			$\to \{1\}$       & $\{2\}, \{3\}$      		 & $\{\varnothing\}$      \\
			$\{2\}$       & $\{1\}$    			 & $\{\varnothing\}$      \\
			$\{\textbf{3}\}$       & $\{3\}$    			 & $\{3\}, \{4\}$       \\
			$\{4\}$       & $\{\varnothing\}$    			 & $\{5\}$       \\
			$\{\textbf{5}\}$       & $\{\varnothing\}$    			 & $\{6\}$       \\
			$\{6\}$       & $\{\varnothing\}$    			 & $\{5\}$     \\
			\bottomrule
		\end{tabular}
	\end{center}
	
		
	\begin{enumerate}[label=(\roman{*})]
	\item Вычислить регулярное выражение, определяющее язык $L_1 \cap  L_2$.
	
	\begin{center}
		\begin{tabular}{lll}
			\toprule
			\multicolumn{1}{c}{\multirow{2}{*}{\Large $\delta$}}
			& \multicolumn{2}{c}{Вход} \\
			\cmidrule(rl){2-3}
			& \multicolumn{1}{c}{$a$}
			& \multicolumn{1}{c}{$b$}  \\
			\midrule
			$\to \{1, 1\}$       & $\{2, 2\}, \{2, 3\}, \{4, 2\}, \{4, 3\}$     		 & $\{\varnothing\}$      \\
			$\{2, 2\}$       & $\{1, 1\}$    			 & $\{\varnothing\}$      \\
			$\{2, 3\}$       & $\{1, 3\}$    			 & $\{\varnothing\}$      \\
			$\{1, 3\}$       & $\{2, 3\}, \{4, 3\}$     &  $\{3, 3\}, \{3, 4\}$  \\
			$\{3, 3\}$       & $\{\varnothing\}$     &  $\{1, 3\}, \{1, 4\}$  \\
			$\{3, 4\}$       & $\{\varnothing\}$     &  $\{1, 5\}$  \\
			$\{1, 4\}$       & $\{\varnothing\}$     &  $\{3, 5\}$  \\
			$\{1, 5\}$       & $\{\varnothing\}$     &  $\{3, 6\}$  \\
			$\{3, 5\}$       & $\{\varnothing\}$     &  $\{1, 6\}$  \\
			$\{3, 6\}$       & $\{\varnothing\}$     &  $\{1, 5\}$  \\
			$\{1, 6\}$       & $\{\varnothing\}$     &  $\{3, 5\}$  \\
			$\{4, 2\}$       & $\{4, 1\}$    			 & $\{\varnothing\}$      \\
			$\{4, 3\}$       & $\{4, 3\}$    			 & $\{5, 3\}, \{5, 4\}$      \\
			$\{4, 1\}$       & $\{4, 2\}, \{4, 3\}$    			 & $\{\varnothing\}$      \\
			$\textbf{\{5, 3\}}$       & $\{6, 3\}$    			 & $\{7, 3\}, \{7, 4\}$      \\
			$\{5, 4\}$       & $\{\varnothing\}$    			 & $\{7, 5\}$      \\
			$\textbf{\{6, 3\}}$       & $\{6, 3\}$    			 & $\{8, 3\}, \{8, 4\}$      \\
			$\{7, 3\}$       & $\{6, 3\}$    			 & $\{\varnothing\}$      \\
			$\{7, 4\}$       & $\{\varnothing\}$    			 & $\{\varnothing\}$      \\
			$\{7, 5\}$       & $\{\varnothing\}$    			 & $\{\varnothing\}$      \\
			$\{8, 3\}$       & $\{6, 3\}$    			 & $\{\varnothing\}$      \\
			$\{8, 4\}$       & $\{\varnothing\}$    			 & $\{\varnothing\}$      \\
			\bottomrule
		\end{tabular}
	\end{center}
	
	\begin{center}	
		\begin{tikzpicture}[auto,>=stealth', node distance=2.5cm,auto,every state/.style={thick}]
		\node (init) {};
		\node[state] (11) [right=.7cm of init] {$11$};
		\node[state] (22) [above of=11] {$22$};
		\node[state] (42) [below of=11] {$42$};
		\node[state] (23) [right of=11] {$23$};
		\node[state] (43) [below of=23] {$43$};
		\node[state] (13) [right of=23] {$13$};
		\node[state] (33) [right of=13] {$33$};
		\node[state] (34) [below of=33] {$34$};
		\node[state] (14) [right of=33] {$14$};	
		\node[state] (15) [right of=34] {$15$};
		\node[state] (35) [right of=14] {$35$};
		\node[state] (36) [right of=15] {$36$};
		\node[state] (16) [right of=35] {$16$};
		\node[state] (41) [below of=42] {$41$};
		\node[state] (54) [right of=43, below of=43] {$54$};
		\node[state, accepting] (53) [below of=54] {$53$};
		\node[state] (75) [right of=54] {$75$};
		\node[state, accepting] (63) [right of=53] {$63$};
		\node[state] (73) [right of=53, below of=53] {$73$};
		\node[state] (74) [below of=53] {$74$};
		\node[state] (84) [right of=63, above of=63] {$84$};
		\node[state] (83) [right of=84, below of=84] {$83$};
		
		\path[->]
		(init) edge (11)
		(11) edge node[right] {$a$} (22)
		(11) edge node {$a$} (23)
		(11) edge node[left] {$a$} (42)
		(11) edge node {$a$} (43)
		
		(22) edge[bend right] node[left] {$a$} (11)
		
		(23) edge node[below] {$a$} (13)
		
		(13) edge[bend right] node[above] {$a$} (23)
		(13) edge node {$a$} (43)
		(13) edge node[below]  {$b$}(33)
		(13) edge node[near end] {$b$} (34)
		
		(33) edge[bend right] node[above] {$b$} (13)
		(33) edge node {$b$} (14)
		(34) edge node {$b$} (15)
		
		(14) edge node {$b$} (35)
		(15) edge node[below] {$b$} (36)
		
		(35) edge node[below] {$b$} (16)
		(36) edge[bend right] node[above] {$b$} (15)
		(16) edge[bend right] node[above] {$b$} (35)
		
		(42) edge node {$a$} (41)
		(41) edge[bend left] node[left] {$a$} (42)
		(41) edge node[near end] {$a$} (43)
		
		(43) edge[loop below] node {$a$} (43)
		(43) edge[bend left] node[near end] {$b$} (54)
		(43) edge node[near end, left] {$b$} (53)
		
		(54) edge node {$b$} (75)
		(53) edge node {$a$} (63)
		(53) edge node[near end] {$b$} (73)
		(53) edge node[left] {$b$} (74)
		
		(63) edge[loop above] node {$a$} (63)
		(63) edge node {$b$} (83)
		(63) edge node {$b$} (84)
		
		(73) edge node[right] {$a$} (63)
		(83) edge[bend left] node {$a$} (63)
		;
		\end{tikzpicture}
	\end{center}

	Состояния 84, 74, 54, 75, 14, 35, 16, 34, 15, 36 являются непродуктивными, поэтому их можно удалить.
	
	\begin{center}	
		\begin{tikzpicture}[auto,>=stealth', node distance=2.5cm,auto,every state/.style={thick}]
		\node (init) {};
		\node[state] (11) [right=.7cm of init] {$11$};
		\node[state] (22) [above of=11] {$22$};
		\node[state] (42) [below of=11] {$42$};
		\node[state] (23) [right of=11] {$23$};
		\node[state] (43) [below of=23] {$43$};
		\node[state] (13) [right of=23] {$13$};
		\node[state] (33) [right of=13] {$33$};
		\node[state] (41) [below of=42] {$41$};
		\node[state, accepting] (53) [right of=43] {$53$};
		\node[state, accepting] (63) [right of=53] {$63$};
		\node[state] (73) [right of=53, below of=53] {$73$};
		\node[state] (83) [right of=63] {$83$};
		
		\path[->]
		(init) edge (11)
		(11) edge[bend right] node[right] {$a$} (22)
		(11) edge node {$a$} (23)
		(11) edge node[left] {$a$} (42)
		(11) edge node {$a$} (43)
		
		(22) edge[bend right] node[left] {$a$} (11)
		
		(23) edge node[below] {$a$} (13)
		
		(13) edge[bend right] node[above] {$a$} (23)
		(13) edge node {$a$} (43)
		(13) edge node[below]  {$b$}(33)
		
		(33) edge[bend right] node[above] {$b$} (13)
		
		(42) edge node {$a$} (41)
		(41) edge[bend left] node[left] {$a$} (42)
		(41) edge node[near end] {$a$} (43)
		
		(43) edge[loop below] node {$a$} (43)
		(43) edge[bend right] node {$b$} (53)
		
		(53) edge node {$a$} (63)
		(53) edge node[near end] {$b$} (73)
		
		(63) edge[loop above] node {$a$} (63)
		(63) edge node {$b$} (83)
		
		(73) edge node[right] {$a$} (63)
		(83) edge[bend left] node {$a$} (63)
		;
		\end{tikzpicture}
	\end{center}

	Воспользуемся методом последовательного исключения состояний. Пустим $\varepsilon$-переходы из всех допускающих состояний в новое состояние $q_f$. Все допускающие состояния сделаем недопускающими, а новое состояние $q_f$ - допускающим.
	
	
	\begin{center}	
		\begin{tikzpicture}[auto,>=stealth', node distance=2.5cm,auto,every state/.style={thick}]
		\node (init) {};
		\node[state] (11) [right=.7cm of init] {$11$};
		\node[state] (22) [above of=11] {$22$};
		\node[state] (42) [below of=11] {$42$};
		\node[state] (23) [right of=11] {$23$};
		\node[state] (43) [below of=23] {$43$};
		\node[state] (13) [right of=23] {$13$};
		\node[state] (33) [right of=13] {$33$};
		\node[state] (41) [below of=42] {$41$};
		\node[state] (53) [right of=43, below of=43] {$53$};
		\node[state] (63) [right of=53] {$63$};
		\node[state] (73) [right of=53, below of=53] {$73$};
		\node[state] (83) [right of=63] {$83$};
		\node[state, accepting] (qf) [right of=43] {$q_f$};
		
		\path[->]
		(init) edge (11)
		(11) edge[bend right] node[right] {$a$} (22)
		(11) edge node {$a$} (23)
		(11) edge node[left] {$a$} (42)
		(11) edge node {$a$} (43)
		
		(22) edge[bend right] node[left] {$a$} (11)
		
		(23) edge node[below] {$a$} (13)
		
		(13) edge[bend right] node[above] {$a$} (23)
		(13) edge node {$a$} (43)
		(13) edge node[below]  {$b$}(33)
		
		(33) edge[bend right] node[above] {$b$} (13)
		
		(42) edge node {$a$} (41)
		(41) edge[bend left] node[left] {$a$} (42)
		(41) edge node[near end] {$a$} (43)
		
		(43) edge[loop below] node {$a$} (43)
		(43) edge node {$b$} (53)
		
		(53) edge node {$a$} (63)
		(53) edge node[near end] {$b$} (73)
		
		(63) edge[loop above] node {$a$} (63)
		(63) edge node {$b$} (83)
		
		(73) edge node[right] {$a$} (63)
		(83) edge[bend left] node {$a$} (63)
		
		(53) edge node {$\varepsilon$} (qf)
		(63) edge node {$\varepsilon$} (qf)
		
		;
		\end{tikzpicture}
	\end{center}

	1) Исключим состояние $83$:
	
		\begin{center}	
		\begin{tikzpicture}[auto,>=stealth', node distance=2.5cm,auto,every state/.style={thick}]
		\node (init) {};
		\node[state] (11) [right=.7cm of init] {$11$};
		\node[state] (22) [above of=11] {$22$};
		\node[state] (42) [below of=11] {$42$};
		\node[state] (23) [right of=11] {$23$};
		\node[state] (43) [below of=23] {$43$};
		\node[state] (13) [right of=23] {$13$};
		\node[state] (33) [right of=13] {$33$};
		\node[state] (41) [below of=42] {$41$};
		\node[state] (53) [right of=43, below of=43] {$53$};
		\node[state] (63) [right of=53] {$63$};
		\node[state] (73) [right of=53, below of=53] {$73$};
		\node[state, accepting] (qf) [right of=43] {$q_f$};
		
		\path[->]
		(init) edge (11)
		(11) edge[bend right] node[right] {$a$} (22)
		(11) edge node {$a$} (23)
		(11) edge node[left] {$a$} (42)
		(11) edge node {$a$} (43)
		
		(22) edge[bend right] node[left] {$a$} (11)
		
		(23) edge node[below] {$a$} (13)
		
		(13) edge[bend right] node[above] {$a$} (23)
		(13) edge node {$a$} (43)
		(13) edge node[below]  {$b$}(33)
		
		(33) edge[bend right] node[above] {$b$} (13)
		
		(42) edge node {$a$} (41)
		(41) edge[bend left] node[left] {$a$} (42)
		(41) edge node[near end] {$a$} (43)
		
		(43) edge[loop below] node {$a$} (43)
		(43) edge node {$b$} (53)
		
		(53) edge node {$a$} (63)
		(53) edge node[near end] {$b$} (73)
		
		(63) edge[loop above] node {$a + ba$} (63)
		
		(73) edge node[right] {$a$} (63)
	
		(53) edge node {$\varepsilon$} (qf)
		(63) edge node {$\varepsilon$} (qf)
		
		;
		\end{tikzpicture}
	\end{center}

	2) Исключим состояние $73$:
	
\begin{center}	
	\begin{tikzpicture}[auto,>=stealth', node distance=2.5cm,auto,every state/.style={thick}]
	\node (init) {};
	\node[state] (11) [right=.7cm of init] {$11$};
	\node[state] (22) [above of=11] {$22$};
	\node[state] (42) [below of=11] {$42$};
	\node[state] (23) [right of=11] {$23$};
	\node[state] (43) [below of=23] {$43$};
	\node[state] (13) [right of=23] {$13$};
	\node[state] (33) [right of=13] {$33$};
	\node[state] (41) [below of=42] {$41$};
	\node[state] (53) [right of=43, below of=43] {$53$};
	\node[state] (63) [right of=53] {$63$};
	\node[state, accepting] (qf) [right of=43] {$q_f$};
	
	\path[->]
	(init) edge (11)
	(11) edge[bend right] node[right] {$a$} (22)
	(11) edge node {$a$} (23)
	(11) edge node[left] {$a$} (42)
	(11) edge node {$a$} (43)
	
	(22) edge[bend right] node[left] {$a$} (11)
	
	(23) edge node[below] {$a$} (13)
	
	(13) edge[bend right] node[above] {$a$} (23)
	(13) edge node {$a$} (43)
	(13) edge node[below]  {$b$}(33)
	
	(33) edge[bend right] node[above] {$b$} (13)
	
	(42) edge node {$a$} (41)
	(41) edge[bend left] node[left] {$a$} (42)
	(41) edge node[near end] {$a$} (43)
	
	(43) edge[loop below] node {$a$} (43)
	(43) edge node {$b$} (53)
	
	(53) edge node {$a + ba$} (63)
	
	(63) edge[loop above] node {$a + ba$} (63)
	
	(53) edge node {$\varepsilon$} (qf)
	(63) edge node {$\varepsilon$} (qf)
	
	;
	\end{tikzpicture}
\end{center}

	3) Исключим состояние $22$:

\begin{center}	
	\begin{tikzpicture}[auto,>=stealth', node distance=2.5cm,auto,every state/.style={thick}]
	\node (init) {};
	\node[state] (11) [right=.7cm of init] {$11$};
	\node[state] (42) [below of=11] {$42$};
	\node[state] (23) [right of=11] {$23$};
	\node[state] (43) [below of=23] {$43$};
	\node[state] (13) [right of=23] {$13$};
	\node[state] (33) [right of=13] {$33$};
	\node[state] (41) [below of=42] {$41$};
	\node[state] (53) [right of=43, below of=43] {$53$};
	\node[state] (63) [right of=53] {$63$};
	\node[state, accepting] (qf) [right of=43] {$q_f$};
	
	\path[->]
	(init) edge (11)
	(11) edge[loop above] node {$aa$} (11)
	
	(11) edge node {$a$} (23)
	(11) edge node[left] {$a$} (42)
	(11) edge node {$a$} (43)
	
	(23) edge node[below] {$a$} (13)
	
	(13) edge[bend right] node[above] {$a$} (23)
	(13) edge node {$a$} (43)
	(13) edge node[below]  {$b$}(33)
	
	(33) edge[bend right] node[above] {$b$} (13)
	
	(42) edge node {$a$} (41)
	(41) edge[bend left] node[left] {$a$} (42)
	(41) edge node[near end] {$a$} (43)
	
	(43) edge[loop below] node {$a$} (43)
	(43) edge node {$b$} (53)
	
	(53) edge node {$a + ba$} (63)
	
	(63) edge[loop above] node {$a + ba$} (63)
	
	(53) edge node {$\varepsilon$} (qf)
	(63) edge node {$\varepsilon$} (qf)
	
	;
	\end{tikzpicture}
\end{center}

	4) Исключим состояние $33$:
	
	\begin{center}	
		\begin{tikzpicture}[auto,>=stealth', node distance=2.5cm,auto,every state/.style={thick}]
		\node (init) {};
		\node[state] (11) [right=.7cm of init] {$11$};
		\node[state] (42) [below of=11] {$42$};
		\node[state] (23) [right of=11] {$23$};
		\node[state] (43) [below of=23] {$43$};
		\node[state] (13) [right of=23] {$13$};
		\node[state] (41) [below of=42] {$41$};
		\node[state] (53) [right of=43, below of=43] {$53$};
		\node[state] (63) [right of=53] {$63$};
		\node[state, accepting] (qf) [right of=43] {$q_f$};
		
		\path[->]
		(init) edge (11)
		(11) edge[loop above] node {$aa$} (11)
		
		(11) edge node {$a$} (23)
		(11) edge node[left] {$a$} (42)
		(11) edge node {$a$} (43)
		
		(23) edge node[below] {$a$} (13)
		
		(13) edge[bend right] node[above] {$a$} (23)
		(13) edge node {$a$} (43)
		(13) edge[loop right] node {$bb$} (13)
		
		(42) edge node {$a$} (41)
		(41) edge[bend left] node[left] {$a$} (42)
		(41) edge node[near end] {$a$} (43)
		
		(43) edge[loop below] node {$a$} (43)
		(43) edge node {$b$} (53)
		
		(53) edge node {$a + ba$} (63)
		
		(63) edge[loop above] node {$a + ba$} (63)
		
		(53) edge node {$\varepsilon$} (qf)
		(63) edge node {$\varepsilon$} (qf)
		
		;
		\end{tikzpicture}
	\end{center}

	5) Исключим состояние $13$:
	
	\begin{center}	
		\begin{tikzpicture}[auto,>=stealth', node distance=2.5cm,auto,every state/.style={thick}]
		\node (init) {};
		\node[state] (11) [right=.7cm of init] {$11$};
		\node[state] (42) [below of=11] {$42$};
		\node[state] (23) [right of=11] {$23$};
		\node[state] (43) [below of=23] {$43$};
		\node[state] (41) [below of=42] {$41$};
		\node[state] (53) [right of=43, below of=43] {$53$};
		\node[state] (63) [right of=53] {$63$};
		\node[state, accepting] (qf) [right of=43] {$q_f$};
		
		\path[->]
		(init) edge (11)
		(11) edge[loop above] node {$aa$} (11)
		
		(11) edge node {$a$} (23)
		(11) edge node[left] {$a$} (42)
		(11) edge node {$a$} (43)
		
		(23) edge[loop above] node {$a(bb)^*a$} (23)
		
		(23) edge node[right] {$a(bb)^*a$} (43)
		
		(42) edge node {$a$} (41)
		(41) edge[bend left] node[left] {$a$} (42)
		(41) edge node[near end] {$a$} (43)
		
		(43) edge[loop below] node {$a$} (43)
		(43) edge node {$b$} (53)
		
		(53) edge node {$a + ba$} (63)
		
		(63) edge[loop above] node {$a + ba$} (63)
		
		(53) edge node {$\varepsilon$} (qf)
		(63) edge node {$\varepsilon$} (qf)
		
		;
		\end{tikzpicture}
	\end{center}

	6) Исключим состояние $41$:

\begin{center}	
	\begin{tikzpicture}[auto,>=stealth', node distance=2.5cm,auto,every state/.style={thick}]
	\node (init) {};
	\node[state] (11) [right=.7cm of init] {$11$};
	\node[state] (42) [below of=11] {$42$};
	\node[state] (23) [right of=11] {$23$};
	\node[state] (43) [below of=23] {$43$};
	\node[state] (53) [right of=43, below of=43] {$53$};
	\node[state] (63) [right of=53] {$63$};
	\node[state, accepting] (qf) [right of=43] {$q_f$};
	
	\path[->]
	(init) edge (11)
	(11) edge[loop above] node {$aa$} (11)
	
	(11) edge node {$a$} (23)
	(11) edge node[left] {$a$} (42)
	(11) edge node {$a$} (43)
	
	(23) edge[loop above] node {$a(bb)^*a$} (23)
	
	(23) edge node[right] {$a(bb)^*a$} (43)
	
	(42) edge[loop left] node {$aa$} (42)
	(42) edge node {$aa$} (43)
	
	(43) edge[loop below] node {$a$} (43)
	(43) edge node {$b$} (53)
	
	(53) edge node {$a + ba$} (63)
	
	(63) edge[loop above] node {$a + ba$} (63)
	
	(53) edge node {$\varepsilon$} (qf)
	(63) edge node {$\varepsilon$} (qf)
	
	;
	\end{tikzpicture}
\end{center}

	7) Исключим состояние $42$:

\begin{center}	
	\begin{tikzpicture}[auto,>=stealth', node distance=2.5cm,auto,every state/.style={thick}]
	\node (init) {};
	\node[state] (11) [right=.7cm of init] {$11$};
	\node[state] (23) [right of=11] {$23$};
	\node[state] (43) [below of=23] {$43$};
	\node[state] (53) [right of=43, below of=43] {$53$};
	\node[state] (63) [right of=53] {$63$};
	\node[state, accepting] (qf) [right of=43] {$q_f$};
	
	\path[->]
	(init) edge (11)
	(11) edge[loop above] node {$aa$} (11)
	
	(11) edge node {$a$} (23)
	(11) edge node[left] {$a + a(aa)^*aa$} (43)
	
	(23) edge[loop above] node {$a(bb)^*a$} (23)
	
	(23) edge node[right] {$a(bb)^*a$} (43)
	
	
	(43) edge[loop below] node {$a$} (43)
	(43) edge node {$b$} (53)
	
	(53) edge node {$a + ba$} (63)
	
	(63) edge[loop above] node {$a + ba$} (63)
	
	(53) edge node {$\varepsilon$} (qf)
	(63) edge node {$\varepsilon$} (qf)
	
	;
	\end{tikzpicture}
\end{center}

8) Исключим состояние $23$:

\begin{center}	
	\begin{tikzpicture}[auto,>=stealth', node distance=2.5cm,auto,every state/.style={thick}]
	\node (init) {};
	\node[state] (11) [right=.7cm of init] {$11$};
	\node[state] (43) [below of=11] {$43$};
	\node[state] (53) [right of=43, below of=43] {$53$};
	\node[state] (63) [right of=53] {$63$};
	\node[state, accepting] (qf) [right of=43] {$q_f$};
	
	\path[->]
	(init) edge (11)
	(11) edge[loop above] node {$aa$} (11)
	
	(11) edge node[left] {$a + a(aa)^*aa + a(a(bb)^*a)^*a(bb)^*a$} (43)
	
	(43) edge[loop below] node {$a$} (43)
	(43) edge node {$b$} (53)
	
	(53) edge node {$a + ba$} (63)
	
	(63) edge[loop above] node {$a + ba$} (63)
	
	(53) edge node {$\varepsilon$} (qf)
	(63) edge node {$\varepsilon$} (qf)
	
	;
	\end{tikzpicture}
\end{center}

9) Исключим состояние $43$:

\begin{center}	
	\begin{tikzpicture}[auto,>=stealth', node distance=2.5cm,auto,every state/.style={thick}]
	\node (init) {};
	\node[state] (11) [right=.7cm of init] {$11$};
	\node[state] (53) [below of=11] {$53$};
	\node[state] (63) [right of=53] {$63$};
	\node[state, accepting] (qf) [below of=63] {$q_f$};
	
	\path[->]
	(init) edge (11)
	(11) edge[loop above] node {$aa$} (11)
	
	(11) edge node[left] {$(a + a(aa)^*aa + a(a(bb)^*a)^*a(bb)^*a)a^*b$} (53)
	
	
	(53) edge node {$a + ba$} (63)
	
	(63) edge[loop right] node {$a + ba$} (63)
	
	(53) edge node {$\varepsilon$} (qf)
	(63) edge node {$\varepsilon$} (qf)
	
	;
	\end{tikzpicture}
\end{center}
	
	10) Исключим состояние $63$:
	
	\begin{center}	
		\begin{tikzpicture}[auto,>=stealth', node distance=7cm,auto,every state/.style={thick}]
		\node (init) {};
		\node[state] (11) [right=.7cm of init] {$11$};
		\node[state] (53) [right of=11] {$53$};
		\node[state, accepting] (qf) [right of=53] {$q_f$};
		
		\path[->]
		(init) edge (11)
		(11) edge[loop below] node {$aa$} (11)
		
		(11) edge[bend left] node {$(a + a(aa)^*aa + a(a(bb)^*a)^*a(bb)^*a)a^*b$} (53)
		
		(53) edge node {$\varepsilon + (a + ba)(a + ba)^*\varepsilon$} (qf)
		;
		\end{tikzpicture}
	\end{center}
	
		11) Исключим состояние $53$:
	
	\begin{center}	
		\begin{tikzpicture}[auto,>=stealth', node distance=14cm,auto,every state/.style={thick}]
		\node (init) {};
		\node[state] (11) [right=.7cm of init] {$11$};
		\node[state, accepting] (qf) [right of=11] {$q_f$};
		
		\path[->]
		(init) edge (11)
		(11) edge[loop above] node {$aa$} (11)
		(11) edge node {$(a + a(aa)^*aa + a(a(bb)^*a)^*a(bb)^*a)a^*b(\varepsilon + (a + ba)(a + ba)^*)$} (qf)
		;
		\end{tikzpicture}
	\end{center}

Получаем : $Reg = (aa)^*(a + a(aa)^*aa + a(a(bb)^*a)^*a(bb)^*a)a^*b(\varepsilon + (a + ba)(a + ba)^*)$
		
	\item Вычислить регулярное выражение, определяющее язык $L_1 \bigtriangleup L_2$.
	
	$L_1 \bigtriangleup L_2 = (L_1 \backslash L_2)\cup(L_2 \backslash L_1) = (L_1 \cap \overline{L_2}) \cup (L_2 \cap \overline{L_1})$
	
	
	\begin{center}
		\begin{tabular}{lll}
			\toprule
			\multicolumn{1}{c}{\multirow{2}{*}{\Large $\delta$}}
			& \multicolumn{2}{c}{Вход} \\
			\cmidrule(rl){2-3}
			& \multicolumn{1}{c}{$a$}
			& \multicolumn{1}{c}{$b$}  \\
			\midrule
			$\to \{1, 1\}$       & $\{2, 2\}, \{2, 3\}, \{4, 2\}, \{4, 3\}$     		 & $\{\varnothing\}$      \\
			$\{2, 2\}$       & $\{1, 1\}$    			 & $\{\varnothing\}$      \\
			$\textbf{\{2, 3\}}$       & $\{1, 3\}$    			 & $\{\varnothing\}$      \\
			$\textbf{\{1, 3\}}$       & $\{2, 3\}, \{4, 3\}$     &  $\{3, 3\}, \{3, 4\}$  \\
			$\textbf{\{3, 3\}}$       & $\{\varnothing\}$     &  $\{1, 3\}, \{1, 4\}$  \\
			$\{3, 4\}$       & $\{\varnothing\}$     &  $\{1, 5\}$  \\
			$\{1, 4\}$       & $\{\varnothing\}$     &  $\{3, 5\}$  \\
			$\textbf{\{1, 5\}}$       & $\{\varnothing\}$     &  $\{3, 6\}$  \\
			$\textbf{\{3, 5\}}$       & $\{\varnothing\}$     &  $\{1, 6\}$  \\
			$\{3, 6\}$       & $\{\varnothing\}$     &  $\{1, 5\}$  \\
			$\{1, 6\}$       & $\{\varnothing\}$     &  $\{3, 5\}$  \\
			$\{4, 2\}$       & $\{4, 1\}$    			 & $\{\varnothing\}$      \\
			$\textbf{\{4, 3\}}$       & $\{4, 3\}$    			 & $\{5, 3\}, \{5, 4\}$      \\
			$\{4, 1\}$       & $\{4, 2\}, \{4, 3\}$    			 & $\{\varnothing\}$      \\
			$\{5, 3\}$       & $\{6, 3\}$    			 & $\{7, 3\}, \{7, 4\}$      \\
			$\textbf{\{5, 4\}}$       & $\{\varnothing\}$    			 & $\{7, 5\}$      \\
			$\{6, 3\}$       & $\{6, 3\}$    			 & $\{8, 3\}, \{8, 4\}$      \\
			$\textbf{\{7, 3\}}$       & $\{6, 3\}$    			 & $\{\varnothing\}$      \\
			$\{7, 4\}$       & $\{\varnothing\}$    			 & $\{\varnothing\}$      \\
			$\textbf{\{7, 5\}}$       & $\{\varnothing\}$    			 & $\{\varnothing\}$      \\
			$\textbf{\{8, 3\}}$       & $\{6, 3\}$    			 & $\{\varnothing\}$      \\
			$\{8, 4\}$       & $\{\varnothing\}$    			 & $\{\varnothing\}$      \\
			\bottomrule
		\end{tabular}
	\end{center}


		
	\begin{center}	
		\begin{tikzpicture}[auto,>=stealth', node distance=2.3cm,auto,every state/.style={thick}]
		\node (init) {};
		\node[state] (11) [right=.7cm of init] {$11$};
		\node[state] (22) [above of=11] {$22$};
		\node[state] (42) [below of=11] {$42$};
		\node[state, accepting] (23) [right of=11] {$23$};
		\node[state, accepting] (43) [below of=23] {$43$};
		\node[state, accepting] (13) [right of=23] {$13$};
		\node[state, accepting] (33) [right of=13] {$33$};
		\node[state] (34) [below of=33] {$34$};
		\node[state] (14) [right of=33] {$14$};	
		\node[state, accepting] (15) [right of=34] {$15$};
		\node[state, accepting] (35) [right of=14] {$35$};
		\node[state] (36) [right of=15] {$36$};
		\node[state] (16) [right of=35] {$16$};
		\node[state] (41) [below of=42] {$41$};
		\node[state, accepting] (54) [right of=43, below of=43] {$54$};
		\node[state] (53) [below of=54] {$53$};
		\node[state, accepting] (75) [right of=54] {$75$};
		\node[state] (63) [right of=53] {$63$};
		\node[state, accepting] (73) [right of=53, below of=53] {$73$};
		\node[state] (74) [below of=53] {$74$};
		\node[state] (84) [right of=63, above of=63] {$84$};
		\node[state, accepting] (83) [right of=84, below of=84] {$83$};
		
		\path[->]
		(init) edge (11)
		(11) edge node[right] {$a$} (22)
		(11) edge node {$a$} (23)
		(11) edge node[left] {$a$} (42)
		(11) edge node {$a$} (43)
		
		(22) edge[bend right] node[left] {$a$} (11)
		
		(23) edge node[below] {$a$} (13)
		
		(13) edge[bend right] node[above] {$a$} (23)
		(13) edge node {$a$} (43)
		(13) edge node[below]  {$b$}(33)
		(13) edge node[near end] {$b$} (34)
		
		(33) edge[bend right] node[above] {$b$} (13)
		(33) edge node {$b$} (14)
		(34) edge node {$b$} (15)
		
		(14) edge node {$b$} (35)
		(15) edge node[below] {$b$} (36)
		
		(35) edge node[below] {$b$} (16)
		(36) edge[bend right] node[above] {$b$} (15)
		(16) edge[bend right] node[above] {$b$} (35)
		
		(42) edge node {$a$} (41)
		(41) edge[bend left] node[left] {$a$} (42)
		(41) edge node[near end] {$a$} (43)
		
		(43) edge[loop below] node {$a$} (43)
		(43) edge[bend left] node[near end] {$b$} (54)
		(43) edge node[near end, left] {$b$} (53)
		
		(54) edge node {$b$} (75)
		(53) edge node {$a$} (63)
		(53) edge node[near end] {$b$} (73)
		(53) edge node[left] {$b$} (74)
		
		(63) edge[loop above] node {$a$} (63)
		(63) edge node {$b$} (83)
		(63) edge node {$b$} (84)
		
		(73) edge node[right] {$a$} (63)
		(83) edge[bend left] node {$a$} (63)
		;
		\end{tikzpicture}
	\end{center}
	
	Состояния 74, 84 являются непродуктивными, поэтому их можно удалить.
	
			
	\begin{center}	
		\begin{tikzpicture}[auto,>=stealth', node distance=2.5cm,auto,every state/.style={thick}]
		\node (init) {};
		\node[state] (11) [right=.7cm of init] {$11$};
		\node[state] (22) [above of=11] {$22$};
		\node[state] (42) [below of=11] {$42$};
		\node[state, accepting] (23) [right of=11] {$23$};
		\node[state, accepting] (43) [below of=23] {$43$};
		\node[state, accepting] (13) [right of=23] {$13$};
		\node[state, accepting] (33) [right of=13] {$33$};
		\node[state] (34) [below of=33] {$34$};
		\node[state] (14) [right of=33] {$14$};	
		\node[state, accepting] (15) [right of=34] {$15$};
		\node[state, accepting] (35) [right of=14] {$35$};
		\node[state] (36) [right of=15] {$36$};
		\node[state] (16) [right of=35] {$16$};
		\node[state] (41) [below of=42] {$41$};
		\node[state, accepting] (54) [right of=43] {$54$};
		\node[state] (53) [below of=43] {$53$};
		\node[state, accepting] (75) [below of=34] {$75$};
		\node[state] (63) [right of=53] {$63$};
		\node[state, accepting] (73) [below of=53] {$73$};
		\node[state, accepting] (83) [below of=75] {$83$};
		
		\path[->]
		(init) edge (11)
		(11) edge node[right] {$a$} (22)
		(11) edge node {$a$} (23)
		(11) edge node[left] {$a$} (42)
		(11) edge node {$a$} (43)
		
		(22) edge[bend right] node[left] {$a$} (11)
		
		(23) edge node[below] {$a$} (13)
		
		(13) edge[bend right] node[above] {$a$} (23)
		(13) edge node {$a$} (43)
		(13) edge node[below]  {$b$}(33)
		(13) edge node[near end] {$b$} (34)
		
		(33) edge[bend right] node[above] {$b$} (13)
		(33) edge node {$b$} (14)
		(34) edge node {$b$} (15)
		
		(14) edge node {$b$} (35)
		(15) edge node[below] {$b$} (36)
		
		(35) edge node[below] {$b$} (16)
		(36) edge[bend right] node[above] {$b$} (15)
		(16) edge[bend right] node[above] {$b$} (35)
		
		(42) edge node {$a$} (41)
		(41) edge[bend left] node[left] {$a$} (42)
		(41) edge node[near end] {$a$} (43)
		
		(43) edge[loop left] node {$a$} (43)
		(43) edge node[near end] {$b$} (54)
		(43) edge node[near end, left] {$b$} (53)
		
		(54) edge node {$b$} (75)
		(53) edge node {$a$} (63)
		(53) edge node[near end] {$b$} (73)
		
		(63) edge[loop above] node {$a$} (63)
		(63) edge node {$b$} (83)
		
		(73) edge node[right] {$a$} (63)
		(83) edge[bend left] node {$a$} (63)
		;
		\end{tikzpicture}
	\end{center}

	Воспользуемся методом последовательного исключения состояний. Пустим $\varepsilon$-переходы из всех допускающих состояний в новое состояние $q_f$. Все допускающие состояния сделаем недопускающими, а новое состояние $q_f$ - допускающим.
	
				
	\begin{center}	
		\begin{tikzpicture}[auto,>=stealth', node distance=2.5cm,auto,every state/.style={thick}]
		\node (init) {};
		\node[state] (11) [right=.7cm of init] {$11$};
		\node[state] (22) [above of=11] {$22$};
		\node[state] (42) [below of=11] {$42$};
		\node[state] (23) [right of=11] {$23$};
		\node[state] (43) [below of=23] {$43$};
		\node[state] (13) [right of=23] {$13$};
		\node[state] (33) [right of=13] {$33$};
		\node[state] (34) [below of=33] {$34$};
		\node[state] (14) [right of=33, above of=33] {$14$};	
		\node[state] (15) [right of=34] {$15$};
		\node[state] (35) [right of=14, below of=14] {$35$};
		\node[state] (36) [above of=15] {$36$};
		\node[state] (16) [above of=35] {$16$};
		\node[state] (41) [below of=42] {$41$};
		\node[state] (54) [below of=43] {$54$};
		\node[state] (53) [below of=41, left of=41] {$53$};
		\node[state] (75) [right of=54] {$75$};
		\node[state] (63) [right of=53, below of=53] {$63$};
		\node[state] (73) [above of=63, right of=63] {$73$};
		\node[state] (83) [right of=63] {$83$};
		\node[state, accepting] (qf) [right of=75] {$q_f$};
		
		\path[->]
		(init) edge (11)
		(11) edge node[right] {$a$} (22)
		(11) edge node {$a$} (23)
		(11) edge node[left] {$a$} (42)
		(11) edge node {$a$} (43)
		
		(22) edge[bend right] node[left] {$a$} (11)
		
		(23) edge node[below] {$a$} (13)
		
		(13) edge[bend right] node[above] {$a$} (23)
		(13) edge node[near end] {$a$} (43)
		(13) edge node[below]  {$b$}(33)
		(13) edge node[near end] {$b$} (34)
		
		(33) edge[bend right] node[above] {$b$} (13)
		(33) edge node {$b$} (14)
		(34) edge node {$b$} (15)
		
		(14) edge node[near start] {$b$} (35)
		(15) edge node[right] {$b$} (36)
		
		(35) edge node[right] {$b$} (16)
		(36) edge[bend right] node[left] {$b$} (15)
		(16) edge[bend right] node[left] {$b$} (35)
		
		(42) edge node {$a$} (41)
		(41) edge[bend left] node[left] {$a$} (42)
		(41) edge node {$a$} (43)
		
		(43) edge[loop left] node {$a$} (43)
		(43) edge node[near end] {$b$} (54)
		(43) edge[bend left] node[left] {$b$} (53)
		
		(54) edge node {$b$} (75)
		(53) edge node {$a$} (63)
		(53) edge node[near end] {$b$} (73)
		
		(63) edge[loop below] node {$a$} (63)
		(63) edge node {$b$} (83)
		
		(73) edge node[right] {$a$} (63)
		(83) edge[bend left] node {$a$} (63)
		
		(75) edge node {$\varepsilon$} (qf)
		(54) edge[bend right] node[below] {$\varepsilon$} (qf)
		(73) edge[bend right] node[below] {$\varepsilon$} (qf)
		(83) edge[bend right] node[below] {$\varepsilon$} (qf)
		(15) edge node[right] {$\varepsilon$} (qf)
		(43) edge node {$\varepsilon$} (qf)
		(13) edge node {$\varepsilon$} (qf)
		(33) edge[bend left] node[near start, left] {$\varepsilon$} (qf)
		(35) edge[bend left] node {$\varepsilon$} (qf)
		(23) edge node {$\varepsilon$} (qf)
		;
		\end{tikzpicture}
	\end{center}
	
	1) Исключим состояния $22$, $36$ и $16$: 		
	\begin{center}	
		\begin{tikzpicture}[auto,>=stealth', node distance=2.3cm,auto,every state/.style={thick}]
		\node (init) {};
		\node[state] (11) [right=.7cm of init] {$11$};
		\node[state] (42) [below of=11] {$42$};
		\node[state] (23) [right of=11] {$23$};
		\node[state] (43) [below of=23] {$43$};
		\node[state] (13) [right of=23] {$13$};
		\node[state] (33) [right of=13] {$33$};
		\node[state] (34) [below of=33] {$34$};
		\node[state] (14) [right of=33] {$14$};	
		\node[state] (15) [right of=34] {$15$};
		\node[state] (35) [right of=14] {$35$};
		\node[state] (41) [below of=42] {$41$};
		\node[state] (54) [below of=43] {$54$};
		\node[state] (53) [below of=41, left of=41] {$53$};
		\node[state] (75) [right of=54] {$75$};
		\node[state] (63) [right of=53, below of=53] {$63$};
		\node[state] (73) [above of=63, right of=63] {$73$};
		\node[state] (83) [right of=63] {$83$};
		\node[state, accepting] (qf) [right of=75] {$q_f$};
		
		\path[->]
		(init) edge (11)
		(11) edge[loop above] node {$aa$} (11)
		(11) edge node {$a$} (23)
		(11) edge node[left] {$a$} (42)
		(11) edge node {$a$} (43)
		
		(23) edge node[below] {$a$} (13)
		
		(13) edge[bend right] node[above] {$a$} (23)
		(13) edge node[near end] {$a$} (43)
		(13) edge node[below]  {$b$}(33)
		(13) edge node[near end] {$b$} (34)
		
		(33) edge[bend right] node[above] {$b$} (13)
		(33) edge node {$b$} (14)
		(34) edge node {$b$} (15)
		
		(14) edge node[near start] {$b$} (35)
		(15) edge[loop above] node {$bb$} (15)
		
		(35) edge[loop above] node {$bb$} (35)
		
		(42) edge node {$a$} (41)
		(41) edge[bend left] node[left] {$a$} (42)
		(41) edge node {$a$} (43)
		
		(43) edge[loop left] node {$a$} (43)
		(43) edge node[near end] {$b$} (54)
		(43) edge[bend left] node[left] {$b$} (53)
		
		(54) edge node {$b$} (75)
		(53) edge node {$a$} (63)
		(53) edge node[near end] {$b$} (73)
		
		(63) edge[loop below] node {$a$} (63)
		(63) edge node {$b$} (83)
		
		(73) edge node[right] {$a$} (63)
		(83) edge[bend left] node {$a$} (63)
		
		(75) edge node {$\varepsilon$} (qf)
		(54) edge[bend right] node[below] {$\varepsilon$} (qf)
		(73) edge[bend right] node[below] {$\varepsilon$} (qf)
		(83) edge[bend right] node[below] {$\varepsilon$} (qf)
		(15) edge node[right] {$\varepsilon$} (qf)
		(43) edge node {$\varepsilon$} (qf)
		(13) edge node {$\varepsilon$} (qf)
		(33) edge[bend left] node[near start, left] {$\varepsilon$} (qf)
		(35) edge[bend left] node {$\varepsilon$} (qf)
		(23) edge node {$\varepsilon$} (qf)
		;
		\end{tikzpicture}
	\end{center}

	2) Исключим состояние $75$: 		
\begin{center}	
	\begin{tikzpicture}[auto,>=stealth', node distance=2.4cm,auto,every state/.style={thick}]
	\node (init) {};
	\node[state] (11) [right=.7cm of init] {$11$};
	\node[state] (42) [below of=11] {$42$};
	\node[state] (23) [right of=11] {$23$};
	\node[state] (43) [below of=23] {$43$};
	\node[state] (13) [right of=23] {$13$};
	\node[state] (33) [right of=13] {$33$};
	\node[state] (34) [below of=33] {$34$};
	\node[state] (14) [right of=33] {$14$};	
	\node[state] (15) [right of=34] {$15$};
	\node[state] (35) [right of=14] {$35$};
	\node[state] (41) [below of=42] {$41$};
	\node[state] (54) [below of=43] {$54$};
	\node[state] (53) [below of=41, left of=41] {$53$};
	\node[state] (73) [below of=54] {$73$};
	\node[state] (63) [right of=73] {$63$};
	\node[state] (83) [right of=63] {$83$};
	\node[state, accepting] (qf) [below of=34] {$q_f$};
	
	\path[->]
	(init) edge (11)
	(11) edge[loop above] node {$aa$} (11)
	(11) edge node {$a$} (23)
	(11) edge node[left] {$a$} (42)
	(11) edge node {$a$} (43)
	
	(23) edge node[below] {$a$} (13)
	
	(13) edge[bend right] node[above] {$a$} (23)
	(13) edge node[near end] {$a$} (43)
	(13) edge node[below]  {$b$}(33)
	(13) edge node[near end] {$b$} (34)
	
	(33) edge[bend right] node[above] {$b$} (13)
	(33) edge node {$b$} (14)
	(34) edge node {$b$} (15)
	
	(14) edge node[near start] {$b$} (35)
	(15) edge[loop above] node {$bb$} (15)
	
	(35) edge[loop above] node {$bb$} (35)
	
	(42) edge node {$a$} (41)
	(41) edge[bend left] node[left] {$a$} (42)
	(41) edge node {$a$} (43)
	
	(43) edge[loop left] node {$a$} (43)
	(43) edge node[near end] {$b$} (54)
	(43) edge[bend left] node[left] {$b$} (53)
	
	(53) edge[bend right] node {$a$} (63)
	(53) edge node[near end] {$b$} (73)
	
	(63) edge[loop below] node {$a$} (63)
	(63) edge node {$b$} (83)
	
	(73) edge node {$a$} (63)
	(83) edge[bend left] node {$a$} (63)
	
	(54) edge node[above] {$\varepsilon + b\varepsilon$} (qf)
	(73) edge node {$\varepsilon$} (qf)
	(83) edge node[right] {$\varepsilon$} (qf)
	(15) edge node[right] {$\varepsilon$} (qf)
	(43) edge node {$\varepsilon$} (qf)
	(13) edge node {$\varepsilon$} (qf)
	(33) edge[bend left] node[near start, left] {$\varepsilon$} (qf)
	(35) edge[bend left] node {$\varepsilon$} (qf)
	(23) edge node {$\varepsilon$} (qf)
	;
	\end{tikzpicture}
\end{center}

	3) Исключим состояние $83$: 		
\begin{center}	
	\begin{tikzpicture}[auto,>=stealth', node distance=2.4cm,auto,every state/.style={thick}]
	\node (init) {};
	\node[state] (11) [right=.7cm of init] {$11$};
	\node[state] (42) [below of=11] {$42$};
	\node[state] (23) [right of=11] {$23$};
	\node[state] (43) [below of=23] {$43$};
	\node[state] (13) [right of=23] {$13$};
	\node[state] (33) [right of=13] {$33$};
	\node[state] (34) [below of=33] {$34$};
	\node[state] (14) [right of=33] {$14$};	
	\node[state] (15) [right of=34] {$15$};
	\node[state] (35) [right of=14] {$35$};
	\node[state] (41) [below of=42] {$41$};
	\node[state] (54) [below of=43] {$54$};
	\node[state] (53) [below of=41, left of=41] {$53$};
	\node[state] (73) [below of=54] {$73$};
	\node[state] (63) [right of=73] {$63$};
	\node[state, accepting] (qf) [below of=34] {$q_f$};
	
	\path[->]
	(init) edge (11)
	(11) edge[loop above] node {$aa$} (11)
	(11) edge node {$a$} (23)
	(11) edge node[left] {$a$} (42)
	(11) edge node {$a$} (43)
	
	(23) edge node[below] {$a$} (13)
	
	(13) edge[bend right] node[above] {$a$} (23)
	(13) edge node[near end] {$a$} (43)
	(13) edge node[below]  {$b$}(33)
	(13) edge node[near end] {$b$} (34)
	
	(33) edge[bend right] node[above] {$b$} (13)
	(33) edge node {$b$} (14)
	(34) edge node {$b$} (15)
	
	(14) edge node[near start] {$b$} (35)
	(15) edge[loop above] node {$bb$} (15)
	
	(35) edge[loop above] node {$bb$} (35)
	
	(42) edge node {$a$} (41)
	(41) edge[bend left] node[left] {$a$} (42)
	(41) edge node {$a$} (43)
	
	(43) edge[loop left] node {$a$} (43)
	(43) edge node[near end] {$b$} (54)
	(43) edge[bend left] node[left] {$b$} (53)
	
	(53) edge[bend right] node {$a$} (63)
	(53) edge node[near end] {$b$} (73)
	
	(63) edge[loop below] node {$a + ba$} (63)
	
	(73) edge node {$a$} (63)
	
	(54) edge node[above] {$\varepsilon + b\varepsilon$} (qf)
	(73) edge node {$\varepsilon$} (qf)
	(63) edge[bend right] node[right] {$b\varepsilon$} (qf)
	(15) edge node[right] {$\varepsilon$} (qf)
	(43) edge node {$\varepsilon$} (qf)
	(13) edge node {$\varepsilon$} (qf)
	(33) edge[bend left] node[near start, left] {$\varepsilon$} (qf)
	(35) edge[bend left] node {$\varepsilon$} (qf)
	(23) edge node {$\varepsilon$} (qf)
	;
	\end{tikzpicture}
\end{center}
	
		4) Исключим состояние $35$: 		
	\begin{center}	
		\begin{tikzpicture}[auto,>=stealth', node distance=2.4cm,auto,every state/.style={thick}]
		\node (init) {};
		\node[state] (11) [right=.7cm of init] {$11$};
		\node[state] (42) [below of=11] {$42$};
		\node[state] (23) [right of=11] {$23$};
		\node[state] (43) [below of=23] {$43$};
		\node[state] (13) [right of=23] {$13$};
		\node[state] (33) [right of=13] {$33$};
		\node[state] (34) [below of=33] {$34$};
		\node[state] (15) [right of=34] {$15$};
		\node[state] (14) [right of=15, above of=15] {$14$};		
		\node[state] (41) [below of=42] {$41$};
		\node[state] (54) [below of=43] {$54$};
		\node[state] (53) [below of=41, left of=41] {$53$};
		\node[state] (73) [below of=54] {$73$};
		\node[state] (63) [right of=73] {$63$};
		\node[state, accepting] (qf) [below of=34] {$q_f$};
		
		\path[->]
		(init) edge (11)
		(11) edge[loop above] node {$aa$} (11)
		(11) edge node {$a$} (23)
		(11) edge node[left] {$a$} (42)
		(11) edge node {$a$} (43)
		
		(23) edge node[below] {$a$} (13)
		
		(13) edge[bend right] node[above] {$a$} (23)
		(13) edge node[near end] {$a$} (43)
		(13) edge node[below]  {$b$}(33)
		(13) edge node[near end] {$b$} (34)
		
		(33) edge[bend right] node[above] {$b$} (13)
		(33) edge node {$b$} (14)
		(34) edge node {$b$} (15)
		
		(15) edge[loop above] node {$bb$} (15)
		
		(42) edge node {$a$} (41)
		(41) edge[bend left] node[left] {$a$} (42)
		(41) edge node {$a$} (43)
		
		(43) edge[loop left] node {$a$} (43)
		(43) edge node[near end] {$b$} (54)
		(43) edge[bend left] node[left] {$b$} (53)
		
		(53) edge[bend right] node {$a$} (63)
		(53) edge node[near end] {$b$} (73)
		
		(63) edge[loop below] node {$a + ba$} (63)
		
		(73) edge node {$a$} (63)
		
		(54) edge node[above] {$\varepsilon + b$} (qf)
		(73) edge node {$\varepsilon$} (qf)
		(63) edge[bend right] node[right] {$b$} (qf)
		(15) edge node[right] {$\varepsilon$} (qf)
		(43) edge node {$\varepsilon$} (qf)
		(13) edge node {$\varepsilon$} (qf)
		(33) edge[bend left] node[near start, left] {$\varepsilon$} (qf)
		(14) edge[bend left] node {$b(bb)^*\varepsilon$} (qf)
		(23) edge node {$\varepsilon$} (qf)
		;
		\end{tikzpicture}
	\end{center}
	
	5) Исключим состояние $41$: 		
	\begin{center}	
		\begin{tikzpicture}[auto,>=stealth', node distance=2.4cm,auto,every state/.style={thick}]
		\node (init) {};
		\node[state] (11) [right=.7cm of init] {$11$};
		\node[state] (42) [below of=11] {$42$};
		\node[state] (23) [right of=11] {$23$};
		\node[state] (43) [below of=23] {$43$};
		\node[state] (13) [right of=23] {$13$};
		\node[state] (33) [right of=13] {$33$};
		\node[state] (34) [below of=33] {$34$};
		\node[state] (15) [right of=34] {$15$};
		\node[state] (14) [right of=15, above of=15] {$14$};		
		\node[state] (54) [below of=43] {$54$};
		\node[state] (73) [below of=54] {$73$};
		\node[state] (53) [left of=73] {$53$};
		\node[state] (63) [right of=73] {$63$};
		\node[state, accepting] (qf) [below of=34] {$q_f$};
		
		\path[->]
		(init) edge (11)
		(11) edge[loop above] node {$aa$} (11)
		(11) edge node {$a$} (23)
		(11) edge node[left] {$a$} (42)
		(11) edge node {$a$} (43)
		
		(23) edge node[below] {$a$} (13)
		
		(13) edge[bend right] node[above] {$a$} (23)
		(13) edge node[near end] {$a$} (43)
		(13) edge node[below]  {$b$}(33)
		(13) edge node[near end] {$b$} (34)
		
		(33) edge[bend right] node[above] {$b$} (13)
		(33) edge node {$b$} (14)
		(34) edge node {$b$} (15)
		
		(15) edge[loop above] node {$bb$} (15)
		
		(42) edge[loop left] node {$aa$} (42)
		(42) edge node[above] {$aa$} (43)
		
		(43) edge[loop above] node {$a$} (43)
		(43) edge node[near end] {$b$} (54)
		(43) edge[bend right] node[left] {$b$} (53)
		
		(53) edge[bend right] node[below] {$a$} (63)
		(53) edge node[near end] {$b$} (73)
		
		(63) edge[loop below] node {$a + ba$} (63)
		
		(73) edge node {$a$} (63)
		
		(54) edge node[above] {$\varepsilon + b$} (qf)
		(73) edge node {$\varepsilon$} (qf)
		(63) edge[bend right] node[right] {$b$} (qf)
		(15) edge node[right] {$\varepsilon$} (qf)
		(43) edge node {$\varepsilon$} (qf)
		(13) edge node {$\varepsilon$} (qf)
		(33) edge[bend left] node[near start, left] {$\varepsilon$} (qf)
		(14) edge[bend left] node {$b(bb)^*\varepsilon$} (qf)
		(23) edge node {$\varepsilon$} (qf)
		;
		\end{tikzpicture}
	\end{center}
	
		6) Исключим состояние $15$: 		
	\begin{center}	
		\begin{tikzpicture}[auto,>=stealth', node distance=2.4cm,auto,every state/.style={thick}]
		\node (init) {};
		\node[state] (11) [right=.7cm of init] {$11$};
		\node[state] (42) [below of=11] {$42$};
		\node[state] (23) [right of=11] {$23$};
		\node[state] (43) [below of=23] {$43$};
		\node[state] (13) [right of=23] {$13$};
		\node[state] (34) [right of=13, below of=13] {$34$};
		\node[state] (33) [right of=34, above of=34] {$33$};
		\node[state] (14) [right of=33] {$14$};		
		\node[state] (54) [below of=43] {$54$};
		\node[state] (73) [below of=54] {$73$};
		\node[state] (53) [left of=73] {$53$};
		\node[state] (63) [right of=73] {$63$};
		\node[state, accepting] (qf) [below of=34] {$q_f$};
		
		\path[->]
		(init) edge (11)
		(11) edge[loop above] node {$aa$} (11)
		(11) edge node {$a$} (23)
		(11) edge node[left] {$a$} (42)
		(11) edge node {$a$} (43)
		
		(23) edge node[below] {$a$} (13)
		
		(13) edge[bend right] node[above] {$a$} (23)
		(13) edge node[near end] {$a$} (43)
		(13) edge node[below]  {$b$}(33)
		(13) edge node[near end] {$b$} (34)
		
		(33) edge[bend right] node[above] {$b$} (13)
		(33) edge node {$b$} (14)
		
		(42) edge[loop left] node {$aa$} (42)
		(42) edge node[above] {$aa$} (43)
		
		(43) edge[loop above] node {$a$} (43)
		(43) edge node[near end] {$b$} (54)
		(43) edge[bend right] node[left] {$b$} (53)
		
		(53) edge[bend right] node[below] {$a$} (63)
		(53) edge node[near end] {$b$} (73)
		
		(63) edge[loop below] node {$a + ba$} (63)
		
		(73) edge node {$a$} (63)
		
		(54) edge node[above] {$\varepsilon + b$} (qf)
		(73) edge node {$\varepsilon$} (qf)
		(63) edge[bend right] node[right] {$b$} (qf)
		(34) edge node[right] {$b(bb)^*\varepsilon$} (qf)
		(43) edge node {$\varepsilon$} (qf)
		(13) edge node {$\varepsilon$} (qf)
		(33) edge[bend left] node[near start, right] {$\varepsilon$} (qf)
		(14) edge[bend left] node {$b(bb)^*$} (qf)
		(23) edge node {$\varepsilon$} (qf)
		;
		\end{tikzpicture}
	\end{center}

	7) Исключим состояние $34$: 		
\begin{center}	
	\begin{tikzpicture}[auto,>=stealth', node distance=2.4cm,auto,every state/.style={thick}]
	\node (init) {};
	\node[state] (11) [right=.7cm of init] {$11$};
	\node[state] (42) [below of=11] {$42$};
	\node[state] (23) [right of=11] {$23$};
	\node[state] (43) [below of=23] {$43$};
	\node[state] (13) [right of=23] {$13$};
	\node[state] (33) [right of=13] {$33$};
	\node[state] (14) [right of=33] {$14$};		
	\node[state] (54) [below of=43] {$54$};
	\node[state] (73) [below of=54] {$73$};
	\node[state] (53) [left of=73] {$53$};
	\node[state] (63) [right of=73] {$63$};
	\node[state, accepting] (qf) [right of=54] {$q_f$};
	
	\path[->]
	(init) edge (11)
	(11) edge[loop above] node {$aa$} (11)
	(11) edge node {$a$} (23)
	(11) edge node[left] {$a$} (42)
	(11) edge node {$a$} (43)
	
	(23) edge node[below] {$a$} (13)
	
	(13) edge[bend right] node[above] {$a$} (23)
	(13) edge node[near start] {$a$} (43)
	(13) edge node[below]  {$b$}(33)
	
	(33) edge[bend right] node[above] {$b$} (13)
	(33) edge node {$b$} (14)
	
	(42) edge[loop left] node {$aa$} (42)
	(42) edge node[above] {$aa$} (43)
	
	(43) edge[loop above] node {$a$} (43)
	(43) edge node[near end] {$b$} (54)
	(43) edge[bend right] node[left] {$b$} (53)
	
	(53) edge[bend right] node[below] {$a$} (63)
	(53) edge node[near end] {$b$} (73)
	
	(63) edge[loop below] node {$a + ba$} (63)
	
	(73) edge node {$a$} (63)
	
	(54) edge node[above] {$\varepsilon + b$} (qf)
	(73) edge node {$\varepsilon$} (qf)
	(63) edge[bend right] node[right] {$b$} (qf)
	(43) edge node {$\varepsilon$} (qf)
	(13) edge node {$\varepsilon + bb(bb)^*$} (qf)
	(33) edge[bend left] node[near start, right] {$\varepsilon$} (qf)
	(14) edge[bend left] node {$b(bb)^*$} (qf)
	(23) edge node {$\varepsilon$} (qf)
	;
	\end{tikzpicture}
\end{center}

	8) Исключим состояние $14$: 		
\begin{center}	
	\begin{tikzpicture}[auto,>=stealth', node distance=2.4cm,auto,every state/.style={thick}]
	\node (init) {};
	\node[state] (11) [right=.7cm of init] {$11$};
	\node[state] (42) [below of=11] {$42$};
	\node[state] (23) [right of=11] {$23$};
	\node[state] (43) [below of=23] {$43$};
	\node[state] (13) [right of=23] {$13$};
	\node[state] (33) [right of=13] {$33$};	
	\node[state] (54) [below of=43] {$54$};
	\node[state] (73) [below of=54] {$73$};
	\node[state] (53) [left of=73] {$53$};
	\node[state] (63) [right of=73] {$63$};
	\node[state, accepting] (qf) [right of=54] {$q_f$};
	
	\path[->]
	(init) edge (11)
	(11) edge[loop above] node {$aa$} (11)
	(11) edge node {$a$} (23)
	(11) edge node[left] {$a$} (42)
	(11) edge node {$a$} (43)
	
	(23) edge node[below] {$a$} (13)
	
	(13) edge[bend right] node[above] {$a$} (23)
	(13) edge node[near start] {$a$} (43)
	(13) edge node[below]  {$b$}(33)
	
	(33) edge[bend right] node[above] {$b$} (13)
	
	(42) edge[loop left] node {$aa$} (42)
	(42) edge node[above] {$aa$} (43)
	
	(43) edge[loop above] node {$a$} (43)
	(43) edge node[near end] {$b$} (54)
	(43) edge[bend right] node[left] {$b$} (53)
	
	(53) edge[bend right] node[below] {$a$} (63)
	(53) edge node[near end] {$b$} (73)
	
	(63) edge[loop below] node {$a + ba$} (63)
	
	(73) edge node {$a$} (63)
	
	(54) edge node[above] {$\varepsilon + b$} (qf)
	(73) edge node {$\varepsilon$} (qf)
	(63) edge[bend right] node[right] {$b$} (qf)
	(43) edge node {$\varepsilon$} (qf)
	(13) edge node {$\varepsilon + bb(bb)^*$} (qf)
	(33) edge[bend left] node[near start, right] {$\varepsilon + bb(bb)^*$} (qf)
	(23) edge node {$\varepsilon$} (qf)
	;
	\end{tikzpicture}
\end{center}

	9) Исключим состояние $33$: 		
	\begin{center}	
		\begin{tikzpicture}[auto,>=stealth', node distance=2.4cm,auto,every state/.style={thick}]
		\node (init) {};
		\node[state] (11) [right=.7cm of init] {$11$};
		\node[state] (42) [below of=11] {$42$};
		\node[state] (23) [right of=11] {$23$};
		\node[state] (43) [below of=23] {$43$};
		\node[state] (13) [right of=23] {$13$};
		\node[state] (54) [below of=43] {$54$};
		\node[state] (73) [below of=54] {$73$};
		\node[state] (53) [left of=73] {$53$};
		\node[state] (63) [right of=73] {$63$};
		\node[state, accepting] (qf) [above of=63, right of=63] {$q_f$};
		
		\path[->]
		(init) edge (11)
		(11) edge[loop above] node {$aa$} (11)
		(11) edge node {$a$} (23)
		(11) edge node[left] {$a$} (42)
		(11) edge node {$a$} (43)
		
		(23) edge node[below] {$a$} (13)
		
		(13) edge[bend right] node[above] {$a$} (23)
		(13) edge node[near end] {$a$} (43)
		(13) edge[loop above] node  {$bb$}(13)
		
		(42) edge[loop left] node {$aa$} (42)
		(42) edge node[above] {$aa$} (43)
		
		(43) edge[loop above] node {$a$} (43)
		(43) edge node[near end] {$b$} (54)
		(43) edge[bend right] node[left] {$b$} (53)
		
		(53) edge[bend right] node[below] {$a$} (63)
		(53) edge node[near end] {$b$} (73)
		
		(63) edge[loop below] node {$a + ba$} (63)
		
		(73) edge node {$a$} (63)
		
		(54) edge node[above] {$\varepsilon + b$} (qf)
		(73) edge node {$\varepsilon$} (qf)
		(63) edge[bend right] node[right] {$b$} (qf)
		(43) edge node {$\varepsilon$} (qf)
		(13) edge node {$\varepsilon + bb(bb)^* + b(\varepsilon + bb(bb)^*)$} (qf)
		(23) edge node {$\varepsilon$} (qf)
		;
		\end{tikzpicture}
	\end{center}
	
	10) Исключим состояние $54$: 		
	\begin{center}	
		\begin{tikzpicture}[auto,>=stealth', node distance=2.7cm,auto,every state/.style={thick}]
		\node (init) {};
		\node[state] (11) [right=.7cm of init] {$11$};
		\node[state] (42) [below of=11] {$42$};
		\node[state] (23) [right of=11] {$23$};
		\node[state] (43) [below of=23] {$43$};
		\node[state] (13) [right of=23] {$13$};
		\node[state] (73) [below of=43] {$73$};
		\node[state] (53) [left of=73] {$53$};
		\node[state] (63) [right of=73] {$63$};
		\node[state, accepting] (qf) [above of=63, right of=63] {$q_f$};
		
		\path[->]
		(init) edge (11)
		(11) edge[loop above] node {$aa$} (11)
		(11) edge node {$a$} (23)
		(11) edge node[left] {$a$} (42)
		(11) edge node {$a$} (43)
		
		(23) edge node[above] {$a$} (13)
		
		(13) edge[bend right] node[above] {$a$} (23)
		(13) edge node[left] {$a$} (43)
		(13) edge[loop above] node  {$bb$}(13)
		
		(42) edge[loop left] node {$aa$} (42)
		(42) edge node[above] {$aa$} (43)
		
		(43) edge[loop above] node {$a$} (43)
		(43) edge node[left] {$b$} (53)
		
		(53) edge[bend right] node[below] {$a$} (63)
		(53) edge node[near end] {$b$} (73)
		
		(63) edge[loop below] node {$a + ba$} (63)
		
		(73) edge node {$a$} (63)
		
		(73) edge node {$\varepsilon$} (qf)
		(63) edge[bend right] node[right] {$b$} (qf)
		(43) edge node {$\varepsilon + b(\varepsilon + b)$} (qf)
		(13) edge node {$\varepsilon + bb(bb)^* + b(\varepsilon + bb(bb)^*)$} (qf)
		(23) edge node {$\varepsilon$} (qf)
		;
		\end{tikzpicture}
	\end{center}
	
		11) Исключим состояние $73$: 		
	\begin{center}	
		\begin{tikzpicture}[auto,>=stealth', node distance=2.7cm,auto,every state/.style={thick}]
		\node (init) {};
		\node[state] (11) [right=.7cm of init] {$11$};
		\node[state] (42) [below of=11] {$42$};
		\node[state] (23) [right of=11] {$23$};
		\node[state] (43) [below of=23] {$43$};
		\node[state] (13) [right of=23] {$13$};
		\node[state] (53) [below of=42] {$53$};
		\node[state] (63) [right of=53] {$63$};
		\node[state, accepting] (qf) [below of=13, right of=13] {$q_f$};
		
		\path[->]
		(init) edge (11)
		(11) edge[loop above] node {$aa$} (11)
		(11) edge node {$a$} (23)
		(11) edge node[left] {$a$} (42)
		(11) edge node {$a$} (43)
		
		(23) edge node[above] {$a$} (13)
		
		(13) edge[bend right] node[above] {$a$} (23)
		(13) edge node[left] {$a$} (43)
		(13) edge[loop above] node  {$bb$}(13)
		
		(42) edge[loop left] node {$aa$} (42)
		(42) edge node[above] {$aa$} (43)
		
		(43) edge[loop above] node {$a$} (43)
		(43) edge node[left] {$b$} (53)
		
		(53) edge node[below] {$a + ba$} (63)
		(63) edge[loop below] node {$a + ba$} (63)
		
		(53) edge node {$b\varepsilon$} (qf)
		(63) edge[bend right] node {$b$} (qf)
		(43) edge node {$\varepsilon + b(\varepsilon + b)$} (qf)
		(13) edge node {$\varepsilon + bb(bb)^* + b(\varepsilon + bb(bb)^*)$} (qf)
		(23) edge node {$\varepsilon$} (qf)
		;
		\end{tikzpicture}
	\end{center}
	
	
	12) Исключим состояние $63$: 		
	\begin{center}	
		\begin{tikzpicture}[auto,>=stealth', node distance=2.7cm,auto,every state/.style={thick}]
		\node (init) {};
		\node[state] (11) [right=.7cm of init] {$11$};
		\node[state] (42) [below of=11] {$42$};
		\node[state] (23) [right of=11] {$23$};
		\node[state] (43) [below of=23] {$43$};
		\node[state] (13) [right of=23] {$13$};
		\node[state] (53) [below of=42] {$53$};
		\node[state, accepting] (qf) [below of=13, right of=13] {$q_f$};
		
		\path[->]
		(init) edge (11)
		(11) edge[loop above] node {$aa$} (11)
		(11) edge node {$a$} (23)
		(11) edge node[left] {$a$} (42)
		(11) edge node {$a$} (43)
		
		(23) edge node[above] {$a$} (13)
		
		(13) edge[bend right] node[above] {$a$} (23)
		(13) edge node[left] {$a$} (43)
		(13) edge[loop above] node  {$bb$}(13)
		
		(42) edge[loop left] node {$aa$} (42)
		(42) edge node[above] {$aa$} (43)
		
		(43) edge[loop above] node {$a$} (43)
		(43) edge node[left] {$b$} (53)
		
		(53) edge[bend right] node[right] {$b + (a + ba)(a + ba)^*b$} (qf)
		(43) edge node {$\varepsilon + b(\varepsilon + b)$} (qf)
		(13) edge node {$\varepsilon + bb(bb)^* + b(\varepsilon + bb(bb)^*)$} (qf)
		(23) edge node {$\varepsilon$} (qf)
		;
		\end{tikzpicture}
	\end{center}
	
	13) Исключим состояние $42$: 		
	\begin{center}	
		\begin{tikzpicture}[auto,>=stealth', node distance=2.7cm,auto,every state/.style={thick}]
		\node (init) {};
		\node[state] (11) [right=.7cm of init] {$11$};
		\node[state] (23) [right of=11] {$23$};
		\node[state] (43) [below of=23] {$43$};
		\node[state] (13) [right of=23] {$13$};
		\node[state] (53) [below of=43, left of=43] {$53$};
		\node[state, accepting] (qf) [below of=13, right of=13] {$q_f$};
		
		\path[->]
		(init) edge (11)
		(11) edge[loop above] node {$aa$} (11)
		(11) edge node {$a$} (23)
		(11) edge[bend right] node[left] {$a + a(aa)^*aa$} (43)
		
		(23) edge node[above] {$a$} (13)
		
		(13) edge[bend right] node[above] {$a$} (23)
		(13) edge node[left] {$a$} (43)
		(13) edge[loop above] node  {$bb$}(13)
		
		(43) edge[loop above] node {$a$} (43)
		(43) edge node[left] {$b$} (53)
		
		(53) edge node[right] {$b + (a + ba)(a + ba)^*b$} (qf)
		(43) edge node {$\varepsilon + b(\varepsilon + b)$} (qf)
		(13) edge node {$\varepsilon + bb(bb)^* + b(\varepsilon + bb(bb)^*)$} (qf)
		(23) edge node {$\varepsilon$} (qf)
		;
		\end{tikzpicture}
	\end{center}
	
	
	14) Исключим состояние $13$: 		
	\begin{center}	
		\begin{tikzpicture}[auto,>=stealth', node distance=3.5cm,auto,every state/.style={thick}]
		\node (init) {};
		\node[state] (11) [right=.7cm of init] {$11$};
		\node[state] (23) [right of=11] {$23$};
		\node[state] (43) [below of=23] {$43$};
		\node[state] (53) [below of=43, left of=43] {$53$};
		\node[state, accepting] (qf) [right of=43] {$q_f$};
		
		\path[->]
		(init) edge (11)
		(11) edge[loop above] node {$aa$} (11)
		(11) edge node {$a$} (23)
		(11) edge[bend right] node[left] {$a + a(aa)^*aa$} (43)
		
		(23) edge[loop above] node {$aa$} (23)
		(23) edge node {$aa$} (43)
		
		(43) edge[loop below] node {$a$} (43)
		(43) edge node[left] {$b$} (53)
		
		(53) edge[bend right] node[right] {$b + (a + ba)(a + ba)^*b$} (qf)
		(43) edge node {$\varepsilon + b(\varepsilon + b)$} (qf)
		(23) edge node {$\varepsilon + a(bb)^*(\varepsilon + bb(bb)^* + b(\varepsilon + bb(bb)^*))$} (qf)
		;
		\end{tikzpicture}
	\end{center}

15) Исключим состояние $23$: 		
\begin{center}	
	\begin{tikzpicture}[auto,>=stealth', node distance=8cm,auto,every state/.style={thick}]
	\node (init) {};
	\node[state] (11) [right=.7cm of init] {$11$};
	\node[state] (43) [below of=11] {$43$};
	\node[state] (53) [right of=43] {$53$};
	\node[state, accepting] (qf) [right of=11] {$q_f$};
	
	\path[->]
	(init) edge (11)
	(11) edge[loop right] node {$aa$} (11)
	(11) edge node[left] {$a + a(aa)^*aa + a(aa)^*aa$} (43)

	(43) edge[loop below] node {$a$} (43)
	(43) edge node[above] {$b$} (53)
	
	(53) edge node[left] {$b + (a + ba)(a + ba)^*b$} (qf)
	(43) edge[bend left] node {$\varepsilon + b(\varepsilon + b)$} (qf)
	(11) edge[bend left] node {$a(aa)^*(\varepsilon + a(bb)^*(\varepsilon + bb(bb)^* + b(\varepsilon + bb(bb)^*)))$} (qf)
	;
	\end{tikzpicture}
\end{center}
	
	16) Исключим состояние $53$: 		
	\begin{center}	
		\begin{tikzpicture}[auto,>=stealth', node distance=8cm,auto,every state/.style={thick}]
		\node (init) {};
		\node[state] (11) [right=.7cm of init] {$11$};
		\node[state, accepting] (qf) [right of=11] {$q_f$};
		\node[state] (43) [below of=qf] {$43$};
		
		\path[->]
		(init) edge (11)
		(11) edge[loop below] node {$aa$} (11)
		(11) edge node[left] {$a + a(aa)^*aa + a(aa)^*aa$} (43)
		
		(43) edge[loop below] node {$a$} (43)
		
		(43) edge node[right] {$\varepsilon + b(\varepsilon + b) + b(b + (a + ba)(a + ba)^*b)$} (qf)
		(11) edge[bend left] node {$a(aa)^*(\varepsilon + a(bb)^*(\varepsilon + bb(bb)^* + b(\varepsilon + bb(bb)^*)))$} (qf)
		;
		\end{tikzpicture}
	\end{center}
	
	
	17) Исключим состояние $43$: 		

			\begin{center}	
			\begin{tikzpicture}[auto,>=stealth', node distance=8cm,auto,every state/.style={thick}]
			\node (init) {};
			\node[state] (11) [right=.7cm of init] {$11$};
			\node[state, accepting] (qf) [right of=11] {$q_f$};
			
			\path[->]
			(init) edge (11)
			(11) edge[loop right] node {$aa$} (11)
			
			(11) edge[bend left] node {$a(aa)^*(\varepsilon + a(bb)^*(\varepsilon + bb(bb)^* + b(\varepsilon + bb(bb)^*)))$} (qf)
			
			(11) edge[bend right] node[below] {$(a + a(aa)^*aa + a(aa)^*aa)a^*(\varepsilon + b(\varepsilon + b) + b(b + (a + ba)(a + ba)^*b))$} (qf)
			;
			\end{tikzpicture}
		\end{center}
	
	Получаем $Reg = (aa)^*(a(aa)^*(\varepsilon + a(bb)^*(\varepsilon + bb(bb)^* + b(\varepsilon + bb(bb)^*))) + (a + a(aa)^*aa + + a(aa)^*aa)a^*(\varepsilon + b(\varepsilon + b) + b(b + (a + ba)(a + ba)^*b)))$ 
		
		
	\item Определить: совпадают ли языки $L_1$ и $L_2$, является ли $L_1$ дополнением $L_2$.
	
	1) Допустим, $L_1$ совпадает с $L_2$. Приведем контрпример. Слово $ bbaba \in L_1$,   
	
	но $ bbaba \notin L_2$. Делаем вывод, что $L_1 \ne L_2$.
	\newline
	
	2) Допустим, $L_1 = \overline{L_2}$. Приведем контрпример. Слово  $aaaab \in L_1$, 
	
	но при этом  $aaaab \in L_2$.
	 $L_1 \cup L_2 \ne \varnothing$, что противоречит определению дополнения языка.

	
	\item Построить $\varepsilon$-НКА, распознающий один из языков $L^R_1$ или $L^R_2$.
	
	НКА, распознающий язык $L_2$ имеет вид :
	
	\begin{center}	
		\begin{tikzpicture}[auto,>=stealth', node distance=3cm,auto,every state/.style={thick}]
		\node (init) {};
		\node[state] (1) [right=.7cm of init] {$1$};
		\node[state] (2) [above of=1] {$2$};
		\node[state, accepting] (3) [right of=1] {$3$};	
		\node[state] (4) [right of=3] {$4$};	
		\node[state, accepting] (5) [right of=4] {$5$};	
		\node[state] (6) [right of=5] {$6$};	
		
		\path[->]
		(init) edge (1)
		(1) edge[bend right] node[right] {$a$} (2)
		(2) edge[bend right] node[left] {$a$} (1)
		(1) edge node {$a$} (3)
		(3) edge[loop above] node {$a, b$} (3)
		(3) edge node {$b$} (4)
		(4) edge node {$b$} (5)
		(5) edge[bend left] node {$b$} (6)
		(6) edge[bend left] node {$b$} (5)
		;
		\end{tikzpicture}
	\end{center}
	
	Следуя алгоритму, преобразуем его к $\varepsilon$-НКА, распознающему язык $L_2^R$:
	
	1) Меняем направления дуг.
	
	2) Начальное состояние автомата делаем единственным конечным.
	
	3) Создаём новое начальное состояние $q_s$ с $\varepsilon$-переходами во все исходные допускающие состояния.
	
	$M = (\{q_s, 1, 2, 3, 4, 5, 6\}, \{a, b\}, \delta, q_s, \{1\})$

	\begin{center}	
		\begin{tikzpicture}[auto,>=stealth', node distance=3cm,auto,every state/.style={thick}]
		\node (init) {};
		\node[state] (qs) [right=.7cm of init] {$q_s$};
		\node[state] (4) [below of=qs] {$4$};
		\node[state] (3) [left of=4] {$3$};	
		\node[state, accepting] (1) [left of=3] {$1$};
		\node[state] (2) [above of=1] {$2$};
		\node[state] (5) [right of=4] {$5$};	
		\node[state] (6) [right of=5] {$6$};	
		
		\path[->]
		(init) edge (qs)
		(1) edge[bend left] node[left] {$a$} (2)
		(2) edge[bend left] node[right] {$a$} (1)
		(3) edge node {$a$} (1)
		(3) edge[loop below] node {$a, b$} (3)
		(4) edge node {$b$} (3)
		(5) edge node {$b$} (4)
		(5) edge[bend right] node[below] {$b$} (6)
		(6) edge[bend right] node[above] {$b$} (5)
		(qs) edge node[left] {$\varepsilon$} (3)
		(qs) edge node {$\varepsilon$} (5)
		;
		\end{tikzpicture}
	\end{center}
	
	\item Вычислить регулярное выражение по построенному $\varepsilon$-НКА.
	
	Воспользуемся методом исключения состояний:
	
	1) Исключим состояние $6$:
	
	\begin{center}	
		\begin{tikzpicture}[auto,>=stealth', node distance=3cm,auto,every state/.style={thick}]
		\node (init) {};
		\node[state] (qs) [right=.7cm of init] {$q_s$};
		\node[state] (4) [below of=qs] {$4$};
		\node[state] (3) [left of=4] {$3$};	
		\node[state, accepting] (1) [left of=3] {$1$};
		\node[state] (2) [above of=1] {$2$};
		\node[state] (5) [right of=4] {$5$};		
		
		\path[->]
		(init) edge (qs)
		(1) edge[bend left] node[left] {$a$} (2)
		(2) edge[bend left] node[right] {$a$} (1)
		(3) edge node {$a$} (1)
		(3) edge[loop below] node {$a, b$} (3)
		(4) edge node {$b$} (3)
		(5) edge node {$b$} (4)
		(5) edge[loop right] node {$bb$} (5)
		(qs) edge node[left] {$\varepsilon$} (3)
		(qs) edge node {$\varepsilon$} (5)
		;
		\end{tikzpicture}
	\end{center}

2) Исключим состояние $4$:

\begin{center}	
	\begin{tikzpicture}[auto,>=stealth', node distance=3cm,auto,every state/.style={thick}]
	\node (init) {};
	\node[state] (qs) [right=.7cm of init] {$q_s$};
	\node[state] (5) [right of=qs, below of=qs] {$5$};	
	\node[state] (3) [left of=qs, below of=qs] {$3$};	
	\node[state, accepting] (1) [left of=3] {$1$};
	\node[state] (2) [above of=1] {$2$};
		
	
	\path[->]
	(init) edge (qs)
	(1) edge[bend left] node[left] {$a$} (2)
	(2) edge[bend left] node[right] {$a$} (1)
	(3) edge node {$a$} (1)
	(3) edge[loop below] node {$a, b$} (3)
	(5) edge[loop right] node {$bb$} (5)
	(5) edge node {$bb$} (3)
	(qs) edge node[left] {$\varepsilon$} (3)
	(qs) edge node {$\varepsilon$} (5)
	;
	\end{tikzpicture}
\end{center}

3) Исключим состояние $2$:

\begin{center}	
	\begin{tikzpicture}[auto,>=stealth', node distance=3cm,auto,every state/.style={thick}]
	\node (init) {};
	\node[state] (qs) [right=.7cm of init] {$q_s$};
	\node[state] (5) [right of=qs, below of=qs] {$5$};	
	\node[state] (3) [left of=qs, below of=qs] {$3$};	
	\node[state, accepting] (1) [left of=3] {$1$};
		
	\path[->]
	(init) edge (qs)
	(1) edge[loop above] node {$aa$} (1)
	(3) edge node {$a$} (1)
	(3) edge[loop below] node {$a, b$} (3)
	(5) edge[loop right] node {$bb$} (5)
	(5) edge node {$bb$} (3)
	(qs) edge node[left] {$\varepsilon$} (3)
	(qs) edge node {$\varepsilon$} (5)
	;
	\end{tikzpicture}
\end{center}

4) Исключим состояние $5$:

\begin{center}	
	\begin{tikzpicture}[auto,>=stealth', node distance=3cm,auto,every state/.style={thick}]
	\node (init) {};
	\node[state] (qs) [right=.7cm of init] {$q_s$};
	\node[state] (3) [below of=qs] {$3$};	
	\node[state, accepting] (1) [left of=3] {$1$};
	
	\path[->]
	(init) edge (qs)
	(1) edge[loop above] node {$aa$} (1)
	(3) edge node {$a$} (1)
	(3) edge[loop below] node {$a, b$} (3)
	(qs) edge node[right] {$\varepsilon + \varepsilon(bb)^*bb$} (3)
	;
	\end{tikzpicture}
\end{center}

5) Исключим состояние $3$:

\begin{center}	
	\begin{tikzpicture}[auto,>=stealth', node distance=8cm,auto,every state/.style={thick}]
	\node (init) {};
	\node[state] (qs) [right=.7cm of init] {$q_s$};	
	\node[state, accepting] (1) [right of=qs] {$1$};
	
	\path[->]
	(init) edge (qs)
	(1) edge[loop above] node {$aa$} (1)
	(qs) edge node {$(\varepsilon + (bb)^*bb)(a + b)^*a$} (1)
	;
	\end{tikzpicture}
\end{center}

Получаем : $Reg = (\varepsilon + (bb)^*bb)(a + b)^*a(aa)^*$

	\item Построить $\varepsilon$-НКА, распознающий один из языков $L_1L_2$, $L_1 \cup L_2$ или $L^*_1$.
	\newline

	Будем строить $\varepsilon$-НКА, распознающий язык $L_1^*$:
	
	\begin{center}	
		\begin{tikzpicture}[auto,>=stealth', node distance=1.8cm,auto,every state/.style={thick}]
		\node (init) {};
		\node[state] (30) [right=.7cm of init] {$30$};
		\node[state] (0) [right of=30] {$0$};
		\node[state] (1) [right of=0] {$1$};
		\node[state] (2) [right of=1] {$2$};
		\node[state] (3) [right of=1, below of=1] {$3$};
		\node[state] (4) [right of=2] {$4$};
		\node[state] (5) [right of=3] {$5$};
		\node[state] (6) [right of=4] {$6$};
		\node[state] (7) [right of=5] {$7$};
		\node[state] (8) [right of=6] {$8$};
		\node[state] (9) [right of=7] {$9$};
		\node[state] (10) [right of=8] {$10$};
		\node[state] (11) [right of=10] {$11$};
		\node[state] (12) [below of=11] {$12$};
		\node[state] (13) [below of=12] {$13$};
		\node[state] (14) [below of=13] {$14$};
		\node[state] (15) [below of=14] {$15$};
		\node[state] (16) [below of=15] {$16$};
		\node[state] (17) [below of=16] {$17$};
		\node[state] (18) [below of=17] {$18$};
		\node[state] (19) [below of=18] {$19$};
		\node[state] (20) [below of=19] {$20$};
		\node[state] (21) [left of=20] {$21$};
		\node[state] (22) [left of=21, above of=21] {$22$};
		\node[state] (23) [left of=21] {$23$};
		\node[state] (24) [left of=22] {$24$};
		\node[state] (25) [left of=23] {$25$};
		\node[state] (26) [left of=24] {$26$};
		\node[state] (27) [left of=26] {$27$};
		\node[state] (28) [left of=27, below of=27] {$28$};
		\node[state] (29) [left of=28] {$29$};
		\node[state, accepting] (31) [left of=29] {$31$};
		
		\path[->]
		(init) edge (30)
		(30) edge node {$\varepsilon$} (0)
		(0) edge node {$\varepsilon$} (1)
		(1) edge node {$\varepsilon$} (2)
		(1) edge node {$\varepsilon$} (3)
		(2) edge node {$a$} (4)
		(3) edge node {$b$} (5)
		(4) edge node {$\varepsilon$} (6)
		(5) edge node {$\varepsilon$} (7)
		(6) edge node {$a$} (8)
		(7) edge node {$b$} (9)
		(8) edge node {$\varepsilon$} (10)
		(9) edge node {$\varepsilon$} (10)
		(10) edge[bend right] node[above] {$\varepsilon$} (1)
		(10) edge node {$\varepsilon$} (11)
		(0) edge[bend left] node[above] {$\varepsilon$} (11)
		(11) edge node {$\varepsilon$} (12)
		(12) edge node {$\varepsilon$} (13)
		(12) edge[bend left] node {$\varepsilon$} (15)
		(13) edge node {$a$} (14)
		(14) edge[bend left] node {$\varepsilon$} (13)
		(14) edge node {$\varepsilon$} (15)
		(15) edge node {$\varepsilon$} (16)
		(16) edge node {$a$} (17)
		(17) edge node {$\varepsilon$} (18)
		(18) edge node {$b$} (19)
		(19) edge node {$\varepsilon$} (20)
		(20) edge node[above] {$\varepsilon$} (21)
		(21) edge node[right] {$\varepsilon$} (22)
		(21) edge node[above] {$\varepsilon$} (23)
		(22) edge node[above] {$b$} (24)
		(23) edge node[above] {$a$} (25)
		(24) edge node[above] {$\varepsilon$} (26)
		(25) edge node[above] {$\varepsilon$} (28)
		(26) edge node[above] {$a$} (27)
		(27) edge node[left] {$\varepsilon$} (28)
		(28) edge node[above] {$\varepsilon$} (29)
		(28) edge[bend right] node[above] {$\varepsilon$} (21)
		(20) edge[bend left] node {$\varepsilon$} (29)
		(29) edge node[above] {$\varepsilon$} (31)
		(29) edge node[left] {$\varepsilon$} (0)
		(30) edge node[left] {$\varepsilon$} (31)
		;
		\end{tikzpicture}
	\end{center}

	\item Детерминизировать один из построенных $\varepsilon$-НКА.
	
	$\varepsilon$-НКА, соответствующий языку $L_2^R$:
	
	\begin{center}
		\begin{tabular}{llll}
			\toprule
			\multicolumn{1}{c}{\multirow{2}{*}{\Large $\delta$}}
			& \multicolumn{3}{c}{Вход} \\
			\cmidrule(rl){2-4}
			& \multicolumn{1}{c}{$a$}
			& \multicolumn{1}{c}{$b$} 
			& \multicolumn{1}{c}{$\varepsilon$} \\
			\midrule
			$\{q_s\}$       & $\{\varnothing\}$      		 & $\{\varnothing\}$     &$\{3\}, \{5\}$  \\
			$\{1\}$       & $\{2\}$    			 & $\{\varnothing\}$     &$\{\varnothing\}$ \\
			$\{2\}$       & $\{1\}$    			 & $\{\varnothing\}$     &$\{\varnothing\}$  \\
			$\{3\}$       & $\{1\}, \{3\}$    			 & $\{3\}$     &$\{\varnothing\}$  \\
			$\{4\}$       & $\{\varnothing\}$    			 & $\{3\}$     &$\{\varnothing\}$  \\
			$\{5\}$ 		& $\{\varnothing\}$    			 & $\{4\}, \{6\}$     &$\{\varnothing\}$  \\
			$\{6\}$       & $\{\varnothing\}$    	 & $\{5\}$     &$\{\varnothing\}$  \\
			\bottomrule
		\end{tabular}
	\end{center}

	Детерминизируем его :
		
		\begin{center}
		\begin{tabular}{lll}
			\toprule
			\multicolumn{1}{c}{\multirow{2}{*}{\Large $\delta$}}
			& \multicolumn{2}{c}{Вход} \\
			\cmidrule(rl){2-3}
			& \multicolumn{1}{c}{$a$}
			& \multicolumn{1}{c}{$b$}  \\
			\midrule
			$\{q_s, 3, 5\}$       & $\{1, 3\}$      		 & $\{3, 4, 6\}$ \\
			$\{1, 3\}$       & $\{1, 2, 3\}$    			 & $\{3\}$    \\
			$\{3, 4, 6\}$       & $\{1, 3\}$    			 & $\{3, 5\}$     \\
			$\{1, 2, 3\}$       & $\{1, 2, 3\}$    			 & $\{3\}$     \\
			$\{3\}$       & $\{1, 3\}$    			 & $\{3\}$     \\
			$\{3, 5\}$       & $\{1, 3\}$    	 & $\{3, 4, 6\}$   \\
			\bottomrule
		\end{tabular}
	\end{center}

Введем обозначения :
	\begin{flalign*}
	&\{q_s, 3, 5\} = A \\
	&\{1, 3\} = B \\
	&\{3, 4, 6\} = C \\
	&\{1, 2, 3\} = D \\
	&\{3\} = E \\
	&\{3, 5\} = F\\
	\end{flalign*}
Представим его граф:

	
		\begin{center}	
		\begin{tikzpicture}[auto,>=stealth', node distance=2.8cm,auto,every state/.style={thick}]
		\node (init) {};
		\node[state] (A) [right=.7cm of init] {$A$};
		\node[state, accepting] (B) [right of=A] {$B$};
		\node[state] (C) [right of=B] {$C$};
		\node[state, accepting] (D) [below of=A] {$D$};
		\node[state] (E) [below of=B] {$E$};
		\node[state] (F) [below of=C] {$F$};
		
		\path[->]
		(init) edge (A)
		(A) edge node {$a$} (B)
		(A) edge[bend left] node {$b$} (C)
		
		(B) edge[bend right] node[left] {$a$} (D)
		(B) edge node[left] {$b$} (E)
		
		(C) edge node[above] {$a$} (B)
		(C) edge node[left] {$b$} (F)
		
		(D) edge[loop left] node {$a$} (D)
		(D) edge node {$b$} (E)
		
		(E) edge[bend right] node[right] {$a$} (B)
		(E) edge[loop below] node {$b$} (E)
		
		(F) edge node[right] {$a$} (B)
		(F) edge[bend right] node[right] {$b$} (C)	
		;
		\end{tikzpicture}
	\end{center}
	
\end{enumerate}
\end{document}
