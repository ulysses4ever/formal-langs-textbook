
\renewcommand{\theAlgoEnv}{\Alph{chapter}.\arabic{AlgoEnv}}

\chapter{Алгоритмы для контекстно-свободных грамматик}

В этом приложении алгоритмы, рассмотренные в
главах~\ref{cfg-intro} и~\ref{normal-cfg}, приведены в виде
псевдокода, что может облегчить программирование, а иногда и
понимание этих алгоритмов.

\textbf{Некоторые обозначения, используемые при описании алгоритмов.}
%\paragraph{Некоторые обозначения, используемые при описании алгоритмов}
\begin{itemize}
    \item
${A \gets B}$~— операция присваивания («$A$ присвоить $B$»);

    \item
$A \hookleftarrow a$~— операция добавления элемента в множество («добавить в
множество $A$ элемент $a$»);

    \item
$A \hookleftarrow B$~— операция добавления
множества элементов во множество («добавить в множество $A$ все элементы из
$B$»);

    \item
${\rhd}$~— начало комментария, продолжающегося до конца строки.
\end{itemize}

\AlgoPseudoCode[H]{Нахождение порождающих символов}
{\label{algo-gen}КС-грамматика $G=(\Sigma, N, \mathcal P, S \in N)$.}
{$\Gen_G(\Sigma)$  — множество порождающих в $G$ символов.}
{%
\li $\Gen_G(\Sigma) \gets \Sigma$ \Comment Терминалы являются порождающими
символами
\zi\Comment Ищем продукции, в правых частях которых только порождающие символы\ldots
\li \While $\exists A \to X_1 \ldots X_n \in \mathcal P$, $n \geqslant 0$,
такое что $\forall i \; X_i \in \Gen_G(\Sigma)$  \label{gen-induction}
\zi     \Do
        $\Gen_G(\Sigma) \hookleftarrow A$ \Comment левые части таких
        продукций добавляются в $\Gen_G(\Sigma)$
        \End
}

\begin{myremark}[о многократном просмотре продукций]
В цикле на шаге~\ref{gen-induction} алгоритма~\ref{algo-gen}
одна и та же продукция $A \to X_1 \ldots X_n$
может быть просмотрена несколько раз, причём если на ранних итерациях она не
удовлетворяла условию $\forall i \; X_i \in \Gen_G(\Sigma)$, то на поздних,
когда множество $\Gen_G(\Sigma)$ станет достаточно большим, положение дел может
измениться (условие $\forall i \; X_i \in \Gen_G(\Sigma)$ выполнится и $A$ надо
будет добавить в $\Gen_G(\Sigma)$).

Аналогичное справедливо для просмотра продукций в алгоритме~\ref{algo-reach}.
\end{myremark}

\newcommand{\GenGS}{\ensuremath{\Gen_G(\Sigma)}}

\AlgoPseudoCode[h]{Нахождение достижимых символов}{\label{algo-reach}%
КС-грамматика $G=(\Sigma, N, \mathcal P, S \in N)$.}
{$\Reach_G$  — множество достижимых в $G$ символов.}
{%
\li $\Reach_G \gets \{ S \}$ \Comment $S$ достижим в $G$
\zi\Comment Ищем продукции, в левых частях которых стоит достижимый символ\ldots
\li \While $\exists A \to X_1 \ldots X_n \in \mathcal P$, такое что
    $A \in \Reach_G$
    % $\forall i \; X_i \in \Gen_G(\Sigma^*)$  \label{gen-induction}
\zi \Comment \ldots символы из правых частей таких продукций добавляются
в $\Reach_G$:
\zi     \Do
        $\Reach_G \hookleftarrow \{ X_i \}_{i=1}^n$
        \End
    \End
}

\AlgoNoMeth[h]{Удаление бесполезных символов}{\label{algo-del-useless}%
КС-грамматика $G=(\Sigma, N, \mathcal P, S \in N)$.}
{КС-грамматика $G'=(\Sigma', N', \mathcal P', S \in N')$, не
содержащая бесполезных символов, такая что $L(G') = L(G)$, либо сигнал о том,
что язык исходной грамматики пуст и не существует эквивалентной $G$
грамматики без бесполезных символов.}
{
  \item Построить множество \GenGS, используя алгоритм~\ref{algo-gen}. Если
  $S \not \in \GenGS$ — завершение алгоритма, сообщение о пустоте языка исходной
  грамматики. Иначе удалить из $G$ все символы, не вошедшие в \GenGS.
  \item Построить множество $\Reach_G$, используя алгоритм~\ref{algo-reach}.
  Удалить из $G$ все символы, не вошедшие в $\Reach_G$. Получившуюся грамматику
  обозначить $G'$ и подать её на выход алгоритма.
}

\begin{myremark}[об операции удаления символа из грамматики]
Когда в алгоритме требуется удалить символ $X$ из грамматики $G$, необходимо
не только исключить его из множества $N$ или $\Sigma$, но и \emph{удалить из
$\mathcal P$ все продукции, в которых он участвует}.
\end{myremark}

\Algo[h]{Удаление $\varepsilon$-правил}{\label{algo-del-eps}%
КС-грамматика $G=(\Sigma, N, \mathcal P, S \in N)$.}
{КС-грамматика $G'=(\Sigma, N, \mathcal P', S \in N)$, без
$\varepsilon$-правил (продукций вида $A \to \varepsilon$), такая что
$L(G')=L(G) \setminus \{ \varepsilon \}$.}
{«Устранение перегородок».}
{
  \item Построить множество $\Gen_G(\varepsilon) \subset N$ всех порождающих
  $\varepsilon$ нетерминалов, используя следующую процедуру:
  \begin{codebox}
  \li   \For $A \to \varepsilon \in \mathcal P$
  \zi   \Do
            $\Gen_G(\varepsilon) \hookleftarrow A$
        \End
  \li   \While $\exists A \to X_1 \ldots X_n \in \mathcal P$,
        где $\{ X_i \}_{i=0}^n \subset \Gen_G(\varepsilon)$
  \zi       \Do
            $\Gen_G(\varepsilon) \hookleftarrow A$
            \End
        \End
  \end{codebox}
  \item\label{remove-barriers} Выполнить следующие действия:
  \begin{codebox}
  \zi \For $A \to \alpha_0 B_1 \alpha_1 \ldots B_n \alpha_n \in \mathcal P$, где
  $\forall i \; B_i \in \Gen_G(\varepsilon) \wedge \;
  \alpha_i \in ((N \cup \Sigma) \setminus \Gen_G(\varepsilon))^*$
  \zi   \Do
        $\mathcal P \hookleftarrow
            \{ \alpha_0 X_1 \alpha_1 \ldots X_n \alpha_n \mid
            \forall i \; X_i = \varepsilon \vee X_i = B_i \}$
        \End
  \end{codebox}
  \item Удалить из $\mathcal P$ все $\varepsilon$-правила, обозначить
  получившееся множество правил $\mathcal P'$ и подать на выход алгоритма
  грамматику $G' = (N, \Sigma, \mathcal P', S)$.
}

\begin{myremark}[о методе «устранения перегородок»]
На шаге 2 алгоритма~\ref{algo-del-eps} (с.~\pageref{algo-del-eps}) каждая 
продукция $\mathcal P$ просматривается ровно
один раз. Понятно, что для подходящего $n$ любая продукция может быть
представлена в виде $A \to \alpha_0 B_1 \alpha_1 \ldots B_n \alpha_n$ с
указанными условиями для $B_i$ и $\alpha_i$. Например, если в продукции нет
$\varepsilon$"/порождающих символов, то это продукция вида $A \to
\alpha_0$ и для неё нет необходимости добавлять новые продукции.

\emph{Пример}. Продукция $C \to aDbD$, где $D \in \Gen_G(\varepsilon)$, $a, b \in
\Sigma$, это продукция вида $A \to \alpha_0 B_1 \alpha_1 B_2 \alpha_2$, где $A =
C$, $\alpha_0 = a$, $B_1 = B_2 = D$, $\alpha_2 = \varepsilon$, и для неё нужно
добавить три новых продукции. Укажем их, пояснив название метода «удаления
перегородок». Можно считать, что $\varepsilon$-порождающие символы $B_i$ являются
«перегородками» в продукции $A \to \alpha_0 B_1 \alpha_1 \ldots B_n \alpha_n$ и
новые продукции получаются из данной при помощи устранения этих перегородок \emph{всеми
возможными способами}. Для продукции $C \to aDbD$ это: $C \to abD$, $A \to aDb$ и
$A \to ab$. Исходная продукция остаётся в множестве $\mathcal P$.
\end{myremark}

\AlgoNoMeth[h]{Удаление цепных продукций}{\label{algo-del-chains}%
КС-грамматика $G=(\Sigma, N, \mathcal P, S \in N)$, не
содержащая $\varepsilon$"/правил.}
{КС-грамматика
$G'=(\Sigma, N, \mathcal P', S \in N)$, без цепных правил (продукций вида $A \to B$), такая что $L(G')=L(G)$.}
{
  \item Для каждого нетерминала $A \in N$ построить множество циклически
  достижимых из $A$ нетерминалов $C(A)$, используя процедуру:
  \begin{codebox}
  \li   $C(A) \gets \{ A \}$
  \li   \While $\exists D \to E \in \mathcal P$, такая что $D \in C(A)$
  \zi   \Do
        $C(A) \hookleftarrow E$
        \End
  \end{codebox}
  \item Выполнить следующую процедуру:
  \begin{codebox}
  \zi   \For $A \in N$
  \zi   \Do \For $B \to \alpha \in \mathcal P$
  \zi       \Do \If $B \in C(A)$
  \zi           \Then $\mathcal P \hookleftarrow A \to \alpha$
                \End
            \End
        \End
  \end{codebox}
  \item Удалить из $\mathcal P$ все цепные правила, обозначить
  получившееся множество правил $\mathcal P'$ и подать на выход алгоритма
  грамматику $G' = (N, \Sigma, \mathcal P', S)$.
}

\begin{myremark}[о введении новых нетерминалов в алгоритме~\ref{to-cnf}]
Отметим, что на каждой итерации цикла шага~4 в алгоритме~\ref{to-cnf} 
(с.~\pageref{to-cnf}), если
есть необходимость ввести новые не\-тер\-ми\-на\-лы, то они должны отличаться не только
от тех, которые присутствовали в грамматике до начала этого цикла, но и от тех,
которые были введены на предыдущих итерациях этого цикла. Таким образом, одного
комплекта букв $\{C_i\}$ для выполнения цикла может не хватить. Для борьбы с
нехваткой букв можно вводить нетерминалы, помеченные частями исходного слова
$B_1\ldots B_n$, которое «разбивается на слоги». Например, вместо набора
$\{C_i\}_{i=1}^{n-2}$ можно использовать набор $\{ \langle B_{i+1}\ldots B_n
\rangle \}_{i=1}^{n-2}$, где для каждого $i$ выражение $\langle B_{i+1}\ldots B_n \rangle$
понимается как \emph{один новый} нетерминал. Аналогично можно поступать на
шаге~3, добавляя для рассматриваемого терминала $a$ новый
нетерминал $\langle a \rangle$, а не $A'$.
\end{myremark}

\Algo[h]{Приведение к нормальной форме Хомского}{\label{to-cnf}%
КС-грамматика $G=(\Sigma, N, \mathcal P, S \in N)$, не
содержащая $\varepsilon$"/правил.}
{КС-грамматика
$G'=(\Sigma', N', \mathcal P', S \in N)$ в НФХ, такая что $L(G')=L(G) \setminus
\{ \varepsilon \}$ или сообщение о пустоте языка грамматики.}
{«Разбиение слов на слоги».}
{
    \item Последовательно воспользоваться
    алгоритмами~\ref{algo-del-eps} и \ref{algo-del-chains}, полученную
    грамматику обозначить $G'' = (N, \Sigma, \mathcal P'', S)$.
    \item Применить к $G''$ алгоритм~\ref{algo-del-useless}. Если получен ответ
    $L(G'')=\es$, остановить алгоритм и подать на выход сигнал о пустоте языка
    исходной грамматики. Иначе получена грамматика $G' = (N', \Sigma',
    \mathcal P', S)$.
    \item\label{process-terminals}К $G'$ применить следующую процедуру:
    \begin{codebox}
    \zi \For $a \in \Sigma$
    \zi \Do
            \If $\exists A \to X_1 \ldots X_n \in \mathcal P''$, такая что
            $n > 1 \wedge \exists i\: X_i = a$
    \zi     \Then
    \li     добавить в $N$ \emph{новый} нетерминал $A'$
    \li     заменить $a$ на $A'$ в продукциях с правой частью длиннее 1
    \li     $\mathcal P'' \hookleftarrow A' \to a$
            \End
        \End
    \end{codebox}
    \item\label{breaking-words}К $G'$ применить следующую процедуру:
    \begin{codebox}
    \zi \For $A \to B_1 B_2 \ldots B_n \in \mathcal P'$,
        где $n > 2$, $B_i \in N$
    \zi \Do
    \li     добавить в $N$ \emph{новые} нетерминалы $C_1, \ldots C_{n-2}$
    \li     $\mathcal P' \hookleftarrow
            \{ A \to B_1C_1\} \cup \{ C_i \to B_{i+1} C_{i+1} \}_{i=1}^{n-3}
            \cup \{ C_{n-2} \to B_{n-1}B_n \}$
    \li     удалить $A \to B_1 B_2 \ldots B_n$ из $\mathcal P'$
        \End
    \end{codebox}
    Подать $G'$ на выход алгоритма.
}

\Algo[h]{Решение проблемы принадлежности для КС"/языков, CYK-алгоритм}
{\label{cyk}Грамматика $G=(\Sigma, N, \mathcal P, S \in N)$ в НФХ,
слово $w = w_1 \ldots w_n \in \Sigma^*$.}
{Истина, если $w \in L(G)$, ложь  в противном случае.}
{Последовательное определение нетерминалов, выводящих
всевозможные подстроки $w$ всё большей длины.}
{
\item Построить множества $N_{ij}$, используя процедуру:
\begin{codebox}
\zi\For $i \gets 1$ \kw{to} $n$
\zi     \Do
        $N_{ii} \gets \{ A \in N \mid A \to w_i \in \mathcal P  \}$
        \Comment Подстроки $w$ длины $1$
        \End
\zi\For $s \gets 2$ \kw{to} $n$ \Comment Цикл по длине подстроки
\zi     \Do
        \For $i \gets 1$ \kw{to} $n - s + 1$ \Comment Цикл по месту начала подстроки
\zi         $j \gets i + s - 1$
            \Comment Позиция конца подстроки с началом в $w_i$ длины $s$
\zi         $N_{ij} \gets \{ A \in N \mid A \to BC \in \mathcal P; \;
                \exists k \in [i, j-1]_{\mathbb Z} \colon B \in N_{ik}, \;
                C \in N_{k+1 j} \} $
        \End
\end{codebox}

\item
Если $S \in N_{1n}$, то подать на выход алгоритма «да», иначе — «нет».
}

\begin{myremark}[о табличной форме алгоритма~\ref{cyk}, с.~\pageref{cyk}]
Алгоритм удобно выполнять, заполняя таблицу с $N_{ij}$ в ячейках. Ячейки таблицы
расположены в системе координат $(i,s)$, в позиции $(i,s)$ находится множество
$N_{i, i+s-1}$. Подробно этот процесс описан в разделе~\ref{Chapter7ProblemB}.
\end{myremark}
