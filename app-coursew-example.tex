\renewcommand{\theAlgoEnv}{\Alph{chapter}.\arabic{AlgoEnv}}
%\renewcommand{\theequation}{\arabic{chapter}.\arabic{}}
\renewcommand{\thesection}{\arabic{section}}

\chapter{Пример выполнения заданий курсовой работы}

\section*{Задание 1}
\setcounter{section}{1}
\cohead{Задание 1}

Язык над алфавитом $\Sigma = \{0,1\}$, состоящий из всех слов, в которых хотя бы на одной из последних трех позиций стоит $1$.
\begin{enumerate}[label=(\roman{*})]
	\item Построим ПЛ-грамматику $G$, порождающую $L$. Вначале будем выводить (возможно пустой) префикс слова, а затем последние три символа, контролируя момент первого написания единицы.\\
		$G = (\{S, P, T_{0F}, T_{1F}, T_{1T}, T_{2F}, T_{2T}\}, \{0, 1\}, P, S)$
		\begin{align*}
			S &\longrightarrow P &
			P &\longrightarrow 0P \mid 1P \mid T_{0F}&
			T_{2T} &\longrightarrow 0 \mid 1\\
			T_{1F} &\longrightarrow 0T_{2F} \mid 1T_{2T} &
			T_{1T} &\longrightarrow 0T_{2T} \mid 1T_{2T} &
			T_{2F} &\longrightarrow 1 \\
			T_{0F} &\longrightarrow 0T_{1F} \mid 1T_{1T}
		\end{align*}
	\item
		\begin{description}
			\item Покажем $L \subset L(G)$:\\
				Предоставим алгоритм вывода произвольного слова $w \in L$ в нашей грамматике. Разобьем слово на префикс и последние три символа $w = pt, |t| = 3$. Префикс выведем продукциями $P \longrightarrow 0P | 1P$. Пусть $k$ --- номер первой единицы в подслове $t$. Тогда первые k-1 символов слова $t$ выводятся продукциями вида $T_{i-1F} \longrightarrow 0T_{iF} (i \in \{1, .., k-1\})$, единицу выведем $T_{k-1F} \longrightarrow 1T_{kT}$, конец слова допишем продукциями $T_{i-1T} \longrightarrow 0T_{iT} | 1T_{iT} (i \in \{k+1, .., 3\})$. Таким образом мы можем вывести любое слово из $L$ в грамматике $G$ $\Rightarrow L \subset L(G)$. 
			\item Покажем $L(G) \subset L$:\\
				Все слова, которые выводятся в грамматике $G$, можно разбить на префикс и 3х-символьный суффикс.\\
				$\forall w \in L(G), w = pt, |t| = 3, p \in \{0,1\}^*$ Суффикс же содержит по крайней мере одну единицу, так как вывод завершается, либо продукцией $T_{9F} \longrightarrow 1$ и тогда единственная единица в слове стоит в самом конце слова, либо $T_{9T} \longrightarrow 0 | 1$, а нетерминалы вида $T_{iT}$ выводятся только после написания первой единицы суффикса. Следовательно, все выводимые слова это слова написанные буквами из $\{0,1\}$ такие, что в последних трех символах стоит по меньшей мере одна единица.
		\end{description}

	\item Решим систему линейных уравнений с регулярными коэффициентами:
		\begin{align}
			S &= P\\
			P &= 0P + 1P + T_{0F}\label{app-ex-examp-1}\\
			T_{0F} &= 0T_{1F} + 1T_{1T}\\
			T_{1F} &= 0T_{2F} + 1T_{2T}\\
			T_{1T} &= 0T_{2T} + 1T_{2T}\\
			T_{2F} &= 1\\
			T_{2T} &= 0 + 1\label{app-ex-examp-2}
		\end{align}
		Для уравнений с \ref{app-ex-examp-2} по \ref{app-ex-examp-1} совершим последовательную подстановку снизу вверх, проводя упрощения. Получим:
		\begin{align}
			S &= P\label{app-ex-examp-3}\\
			P &= (0+1)P + 1(0+1)^2 + (0+1)1(0+1) + (0+1)^21\label{app-ex-examp-4}
		\end{align}
		Затем для \ref{app-ex-examp-4} применим формулу $A = \alpha A + \beta \Rightarrow A = \alpha^*\beta$ и подставим результат в \ref{app-ex-examp-3}:
		\begin{align*}
			S &= (0+1)^*(1(0+1)^2 + (0+1)1(0+1) + (0+1)^21)
		\end{align*}

	\item Построим $M^{ND}$ (рис.~\ref{app-ex-fig-1}, с.~\pageref{app-ex-fig-1}).
\begin{figure}
\centering
		\begin{tikzpicture}[initial text={},->,>=stealth',shorten >=1pt,auto,node distance=2.5cm, semithick]
		  \tikzstyle{every state}=[fill=none,draw=black,text=black]

		  \node[initial,state]    (S)              {$q_0$};
		  \node[state]            (1F) [right of=S] {$q_{1F}$};
		  \node[state]            (2F) [right of=1F] {$q_{2F}$};

		  \node[state]            (1T) [below of=1F] {$q_{1T}$};
		  \node[state]            (2T) [right of=1T] {$q_{2T}$};
		  \node[state,accepting]  (3T) [right of=2T] {$q_{3T}$};

		  \path (S) edge [loop below]  node {$0+1$}  (S)
		            edge              node {0} 		(1F)
		            edge              node {1} 		(1T)
		        (1F) edge              node {0} 		(2F)
		        	 edge              node {1} 		(2T)
		        (2F) edge              node {1} 		(3T)

		        (1T) edge              node {$1+0$} 		(2T)
				(2T) edge              node {$1+0$} 		(3T);
		\end{tikzpicture}
        \caption{}\label{app-ex-fig-1}
\end{figure}

	\item Детерминируем $M^{ND}$ воспользовавшись алгоритмом детерминизации конечных автоматов: на рис.~\ref{app-ex-fig-2} (с.~\pageref{app-ex-fig-2}) проведена таблица работы алгоритма, на рис.~\ref{app-ex-fig-3} (с.~\pageref{app-ex-fig-3}) преведен граф переходов результирующего детерминированного автомата~$M^{D}$.

\begin{figure}
\centering
		$\begin{array}{|c||c|c|c|}
			\hline
			q' & $q$ & $0$ & $1$\\
			\hline
			q_0' & \rightarrow\{q_0\} & \{q_0, q_{1F}\} & \{q_0, q_{1T}\}\\
			\hline
			q_1' & \{q_0, q_{1F}\} & \{q_0, q_{1F}, q_{2F}\} & \{q_0, q_{1T}, q_{2T}\}\\
			\hline
			q_2' & \{q_0, q_{1T}\} & \{q_0, q_{1F}, q_{2T}\} & \{q_0, q_{1T}, q_{2T}\}\\
			
			\hline
			q_3' & \{q_0, q_{1F}, q_{2F}\} & \{q_0, q_{1F}, q_{2F}, q_{3F}\} & \{q_0, q_{1T}, q_{2T}, q_{3T}\}\\
			\hline
			q_4' & \{q_0, q_{1F}, q_{2T}\} & \{q_0, q_{1F}, q_{2F}, q_{3T}\} & \{q_0, q_{1T}, q_{2T}, q_{3T}\}\\
			\hline
			q_5' & \{q_0, q_{1T}, q_{2T}\} & \{q_0, q_{1F}, q_{2T}, q_{3T}\} & \{q_0, q_{1T}, q_{2T}, q_{3T}\}\\
			
			\hline
			q_6' & \{q_0, q_{1F}, q_{2F}, q_{3F}\} & \{q_0, q_{1F}, q_{2F}, q_{3F}\} & \{q_0, q_{1T} q_{2T}, q_{3T}\}\\
			\hline
			q_7' & \tikz \node[draw,shape=rounded rectangle, inner sep=1pt]{$\{q_0, q_{1F}, q_{2F}, q_{3T}\}$}; & \{q_0, q_{1F}, q_{2F}, q_{3F}\} & \{q_0, q_{1T}, q_{2T}, q_{3T}\}\\
			\hline
			q_8' & \tikz \node[draw,shape=rounded rectangle, inner sep=1pt]{$\{q_0, q_{1F}, q_{2T}, q_{3T}\}$}; & \{q_0, q_{1F}, q_{2F}, q_{3T}\} & \{q_0, q_{1T}, q_{2T}, q_{3T}\}\\
			\hline
			q_9' & \tikz \node[draw,shape=rounded rectangle, inner sep=1pt]{$\{q_0, q_{1T}, q_{2T}, q_{3T}\}$}; & \{q_0, q_{1F}, q_{2T}, q_{3T}\} & \{q_0, q_{1T}, q_{2T}, q_{3T}\}\\
			\hline
		\end{array}$\\
        \caption{}\label{app-ex-fig-2}
\end{figure}

\begin{figure}
\centering
		\begin{tikzpicture}[initial text={}, ->,>=stealth',shorten >=1pt,auto,node distance=4.0cm, semithick]
		  \tikzstyle{every state}=[fill=none,draw=black,text=black]

		  \node[initial,state]    (0)              {$q_0'$};
		  \node[state]            (1) [right of=0] {$q_1'$};
		  \node[state]            (3) [right of=1] {$q_3'$};
		  \node[state]            (6) [right of=3] {$q_6'$};

		  \node[state]            (2) [below of=0] {$q_2'$};
		  \node[state]            (5) [below right of=0] {$q_5'$};
		  
		  \node[state]            (4) [below of=2] {$q_4'$};

		  \node[state,accepting]  (7) [below right of=2] {$q_7'$};

		  \node[state,accepting]  (8) [below left of=3] {$q_8'$};
		  \node[state,accepting]  (9) [below right of=7] {$q_9'$};

		  \path (0) edge              node {0} 		(1)
		  		(0) edge              node {1} 		(2)

		  		(1) edge              node {0} 		(3)
		  		(1) edge              node {1} 		(5)

		  		(2) edge              node {0} 		(4)
		  		(2) edge              node {1} 		(5)

		  		(3) edge              node {0} 		(6)
		  		(3) edge              node {1} 		(9)

		  		(4) edge              node {0} 		(7)
		  		(4) edge              node {1} 		(9)

		  		(5) edge              node {0} 		(8)
		  		(5) edge              node {1} 		(9)

		  		(6) edge [loop below] node {0} 		(6)
		  		(6) edge              node {1} 		(9)

		  		(7) edge              node {0} 		(6)
		  		(7) edge              node {1} 		(9)

		  		(8) edge              node {0} 		(7)
		  		(8) edge [bend left=10] node {1} 		(9)

		  		(9) edge [bend left=10] node {0} 		(8)
				(9) edge [loop below]  node {1} 		(9);
		\end{tikzpicture}
        \caption{}\label{app-ex-fig-3}
\end{figure}

	\item
		Покажем, что $L(M^{ND}) \subset L$: Рассмотрим произвольное слово $w \in L, |w| = k$. Рассмотрим работу автомата с этим словом на входе. Первые $k-3$ символа прочтутся переходами по петле $0+1$ (конечно, будут клоны, которые будут переходить в состояния $q_{1F}$, $q_{1T}$ и дальше, но они будут безусловно умирать, так как им на прочтение будет оставаться больше трех символов), затем, оставшиеся три символа будут прочтены автоматом, а так как это слово из $L$, т. е. содержит по меньшей мере одну единицу в суффиксе длины $3$, то мы непременно прочтем в этих трех символах единицу и остановимся в $q_{3T}$.\\
		Покажем, что $L \subset L(M^{ND})$: Рассмотрим пути на графе автомата, которые заканчиваются в финальном состоянии $q_{3T}$. Они все состоят из некоторого числа проходов по петле $0+1$, после чего следует последовательность из трех переходов, безусловно содержащих по меньшей мере одну единицу.\\
		Таким образом доказано, что $L(M^{ND}) = L$.\\
		Из того, что мы строили $M^D$ по $M^{ND}$ с помощью алгоритма детерминизации конечных автоматов для которого доказана эквивалентность в смысле равенства распознаваемых языков следует, что $L(M^{ND}) = L(M^D)$.
        
	\item Для минимизации $M^{D}$ вначале построим множества неразличимых состояний. Для этого на первой итерации образуем два множества неразличимых пустым словом состояний: финальные и не финальные. После чего на каждой итерации будем проверять, что два состояния из одного и того же множества неразличимы по каждой букве, т. е. переходят в одно и тоже множество из прошлой итерации. Если на некоторой итерации $i$ они становятся различимы, то разделяем множество на подмножества неразличимых состояний словами длины $i$. Повторяем этот процесс пока множества не перестанут разделятся. Для минимизации автомата отождествим все неразличимые вершины графа:

		\begin{enumerate}[label={Итерация №\arabic*:},leftmargin=\widthof{Итерация №9:}+\labelsep]
			\item $\{q_0', q_1', q_2', q_3', q_4', q_5', q_6'\}; \{q_7', q_8', q_9'\};$
			\item $\{q_0', q_1', q_2'\}; \{q_3', q_6'\}; \{q_4', q_5'\}; \{q_7'\}; \{q_8', q_9'\};$
			\item $\{q_0'\}; \{q_1'\}; \{q_2'\}; \{q_3', q_6'\}; \{q_4'\}; \{q_5'\}; \{q_7'\}; \{q_8'\}; \{q_9'\};$
			\item $\{q_0'\}; \{q_1'\}; \{q_2'\}; \{q_3', q_6'\}; \{q_4'\}; \{q_5'\}; \{q_7'\}; \{q_8'\}; \{q_9'\};$
		\end{enumerate}
		
		Результат минимизации, $M^{D}_{min}$, показан на рис.~\ref{app-ex-fig-4} (с.~\pageref{app-ex-fig-4}).
        
\begin{figure}
\centering
		\begin{tikzpicture}[initial text={}, ->,>=stealth',shorten >=1pt,auto,node distance=4.0cm, semithick]
		  \tikzstyle{every state}=[fill=none,draw=black,text=black]

		  \node[initial,state]    (0)              {$q_0'$};
		  \node[state]            (1) [right of=0] {$q_1'$};
		  \node[state]            (3) [right of=1] {$q_3'$};

		  \node[state]            (2) [below of=0] {$q_2'$};
		  \node[state]            (5) [below right of=0] {$q_5'$};
		  
		  \node[state]            (4) [below of=2] {$q_4'$};

		  \node[state,accepting]  (7) [below right of=2] {$q_7'$};

		  \node[state,accepting]  (8) [below left of=3] {$q_8'$};
		  \node[state,accepting]  (9) [below right of=7] {$q_9'$};

		  \path (0) edge              node {0} 		(1)
		  		(0) edge              node {1} 		(2)

		  		(1) edge              node {0} 		(3)
		  		(1) edge              node {1} 		(5)

		  		(2) edge              node {0} 		(4)
		  		(2) edge              node {1} 		(5)

		  		(3) edge              node {1} 		(9)
		  		(3) edge [loop right] node {0} 		(3)

		  		(4) edge              node {0} 		(7)
		  		(4) edge              node {1} 		(9)

		  		(5) edge              node {0} 		(8)
		  		(5) edge              node {1} 		(9)

		  		(7) edge [bend right=20] node {0} 		(3)
		  		(7) edge              node {1} 		(9)

		  		(8) edge              node {0} 		(7)
		  		(8) edge [bend left=10] node {1} 		(9)

		  		(9) edge [bend left=10] node {0} 		(8)
				(9) edge [loop below]  node {1} 		(9);
		\end{tikzpicture}
\caption{}\label{app-ex-fig-4}
\end{figure}


		Для доказательства минимальности $M^{ND}$ воспользуемся следующими рассуждениями: для распознавания слов из $L$ и нераспознавания слов из $\overline{L}$ нам необходимо читать префикс слова, что выполняет петля из $q_0$ по $0+1$ и читать суффикс убеждаясь, что он имеет длину, равную $3$, и содержит как минимум одну единицу. Для подсчета числа прочитанных символов из суффикса достаточно трех вершин, но, так как необходимо контролировать еще и факт прочтения единицы в суффиксе, число необходимых состояний удваивается за счет того, что на каждом из трех этапов мы можем как уже прочесть единицу, так и нет. Но так как состояние обозначающее, что мы прочли три символа суффикса и не встретили единицу --- тупиковая ветвь нашего автомата, то её можно выбросить допустив смерть клона читающего ноль в состоянии <<прочтено два символа и не встречена единица>>. Таким образом у нас остается шесть состояний все из которых необходимы для успешного функционирования автомата и мы убедились, что $M^{ND}$ минимален.

	\item Воспользуемся методом последовательного исключения состояний для того, чтобы выписать регулярное выражения для автоматов $L(M^{ND})$ и $L(M^{D})$.
    
    Обработка автомата $L(M^{ND})$ максимально проста и последовательно показана на рисунках \ref{app-ex-fig-5} (a)--(d) (с. \pageref{app-ex-fig-5}).

\begin{figure}
\begin{subfigure}[b]{\linewidth}
\centering
    \begin{tikzpicture}[initial text={},->,>=stealth',shorten >=1pt,auto,node distance=3cm, semithick]
      \tikzstyle{every state}=[fill=none,draw=black,text=black]

      \node[initial,state]    (S)              {$q_0$};
      \node[state]            (1F) [right of=S] {$q_{1F}$};
      \node[state]            (2F) [right of=1F] {$q_{2F}$};

      \node[state]            (1T) [below of=1F] {$q_{1T}$};
      \node[state]            (2T) [right of=1T] {$q_{2T}$};
      \node[state,accepting]  (3T) [right of=2T] {$q_{3T}$};

      \path (S) edge [loop below]  node {$0+1$}  (S)
                edge              node {0} 		(1F)
                edge              node {1} 		(1T)
            (1F) edge              node {0} 		(2F)
                 edge              node {1} 		(2T)
            (2F) edge              node {1} 		(3T)

            (1T) edge              node {$1+0$} 		(2T)
            (2T) edge              node {$1+0$} 		(3T);
    \end{tikzpicture}
    \caption{}\label{}
\end{subfigure}%
\\
\begin{subfigure}[b]{\linewidth}
\centering
    \begin{tikzpicture}[initial text={},->,>=stealth',shorten >=1pt,auto,node distance=3cm, semithick]
      \tikzstyle{every state}=[fill=none,draw=black,text=black]

      \node[initial,state]    (S)              {$q_0$};
      \node[state]            (1F) [right of=S] {$q_{1F}$};
      %\node[state]            (2F) [right of=1F] {$q_{2F}$};

      %\node[state]            (1T) [below of=1F] {$q_{1T}$};
      \node[state]            (2T) [below of=1F] {$q_{2T}$};
      \node[state,accepting]  (3T) [right of=2T] {$q_{3T}$};

      \path (S) edge [loop below]  node {$0+1$}  (S)
                edge              node {$0$} 		(1F)
                edge              node [sloped, anchor=center, above]{$1(1+0)$} 		(2T)
            (1F) edge              node [sloped, anchor=center, above]{$01$} 		(3T)
                 edge              node {$1$} 		(2T)
       %     (2F) edge              node {1} 		(3T)

        %    (1T) edge              node {1} 		(2T)
            (2T) edge              node {$1+0$} 		(3T);
    \end{tikzpicture}
    \caption{}\label{}
\end{subfigure}
\\
\begin{subfigure}[b]{\linewidth}
\centering
    \begin{tikzpicture}[initial text={},->,>=stealth',shorten >=1pt,auto,node distance=4.2cm, semithick]
      \tikzstyle{every state}=[fill=none,draw=black,text=black]

      \node[initial,state]    (S)              {$q_0$};
      %\node[state]            (1F) [right of=S] {$q_{1F}$};
      %\node[state]            (2F) [right of=1F] {$q_{2F}$};

      %\node[state]            (1T) [below of=1F] {$q_{1T}$};
      \node[state]            (2T) [below right of=S] {$q_{2T}$};
      \node[state,accepting]  (3T) [right of=2T] {$q_{3T}$};

      \path (S) edge [loop above]  node [left] {$0+1$}  (S)
                edge              node [sloped, anchor=center, above]{$001$} 		(3T)
                edge node [sloped, anchor=center, below]{$1(1+0)+01$} 		(2T)
        %    (1F) edge              node {01} 		(3T)
        %    	 edge              node {1} 		(2T)
       %     (2F) edge              node {1} 		(3T)

        %    (1T) edge              node {1} 		(2T)
            (2T) edge              node {$1+0$} 		(3T);
    \end{tikzpicture}
    \caption{}\label{}
\end{subfigure}%
\\
\begin{subfigure}[b]{\linewidth}
\centering
    \begin{tikzpicture}[initial text={},->,>=stealth',shorten >=1pt,auto,node distance=7.5cm, semithick]
      \tikzstyle{every state}=[fill=none,draw=black,text=black]

      \node[initial,state]    (S)              {$q_0$};
      %\node[state]            (1F) [right of=S] {$q_{1F}$};
      %\node[state]            (2F) [right of=1F] {$q_{2F}$};

      %\node[state]            (1T) [below of=1F] {$q_{1T}$};
      %\node[state]            (2T) [below of=S] {$q_{2T}$};
      \node[state,accepting]  (3T) [right of=S] {$q_{3T}$};

      \path (S) edge [loop above]  node [left] {$0+1$}  (S)
                edge              node [sloped, anchor=center, above]{$(1(1+0)+01)(1+0) + 001$} 		(3T)
                %edge [bend right=50]  node {11+01} 		(2T)
        %    (1F) edge              node {01} 		(3T)
        %    	 edge              node {1} 		(2T)
       %     (2F) edge              node {1} 		(3T)

        %    (1T) edge              node {1} 		(2T)
        %	(2T) edge              node {1} 		(3T)
            ;
    \end{tikzpicture}
    \caption{}\label{}
\end{subfigure}
\caption{}\label{app-ex-fig-5}
%
\end{figure}

    
    В итоге мы получаем регулярное выражение
    \begin{multline*}
    L(M^{ND}) =
            (0+1)^*((1(1+0)+01)(1+0) + 001) = \\
            (0+1)^*((11 + 10 + 01)(1+0) + 001) = \\
            (0+1)^*(001 + 010 + 011 + 100 + 101 + 110 + 111).
    \end{multline*}

    Обработка автомата $L(M^{D})$ приводит к более длительному процессу. Первые четыре шага показаны на рисунках \ref{app-ex-fig-6} (a)--(b) (с.~\pageref{app-ex-fig-6}) и \ref{app-ex-fig-7} (a)--(b) (с.~\pageref{app-ex-fig-7}).

\begin{figure}
\begin{subfigure}[b]{\linewidth}
\centering
\begin{tikzpicture}[initial text={}, ->,>=stealth',shorten >=1pt,auto,node distance=4.0cm, semithick]
  \tikzstyle{every state}=[fill=none,draw=black,text=black]

  \node[initial,state]    (0)              {$q_0'$};
  \node[state]            (1) [right of=0] {$q_1'$};
  \node[state]            (3) [right of=1] {$q_3'$};
  \node[state]            (6) [right of=3] {$q_6'$};

  \node[state]            (2) [below of=0] {$q_2'$};
  \node[state]            (5) [below right of=0] {$q_5'$};
  
  \node[state]            (4) [below of=2] {$q_4'$};

  \node[state,accepting]  (7) [below right of=2] {$q_7'$};

  \node[state,accepting]  (9) [below=6.6cm of 6] {$q_9'$};
  \node[state,accepting]  (8) [above left of=9] {$q_8'$};

  \path (0) edge              node[sloped, anchor=center, above] {0} 		(1)
        (0) edge              node {1} 		(2)

        (1) edge              node[sloped, anchor=center, above] {0} 		(3)
        (1) edge              node[sloped, anchor=center, above] {1} 		(5)

        (2) edge              node {0} 		(4)
        (2) edge              node[sloped, anchor=center, above] {1} 		(5)

        (3) edge              node[sloped, anchor=center, above] {0} 		(6)
        (3) edge              node[sloped, anchor=center, above] {1} 		(9)

        (4) edge              node[sloped, anchor=center, above] {0} 		(7)
        (4) edge              node[sloped, anchor=center, above] {1} 		(9)

        (5) edge              node[sloped, anchor=center, above] {0} 		(8)
        (5) edge              node[sloped, anchor=center, above] {1} 		(9)

        (6) edge [loop right] node {0} 		(6)
        (6) edge              node {1} 		(9)

        (7) edge              node[sloped, anchor=center, above] {0} 		(6)
        (7) edge              node[sloped, anchor=center, above] {1} 		(9)

        (8) edge              node[sloped, anchor=center, above] {0} 		(7)
        (8) edge [bend left=10] node[sloped, anchor=center, above] {1} 		(9)

        (9) edge [bend left=10] node {0} 		(8)
        (9) edge [loop right]  node {1} 		(9);
\end{tikzpicture}
    \caption{}\label{}
\end{subfigure}%
\\
\begin{subfigure}[b]{\linewidth}
\centering
\begin{tikzpicture}[initial text={}, ->,>=stealth',shorten >=1pt,auto,node distance=4.0cm, semithick]
  \tikzstyle{every state}=[fill=none,draw=black,text=black]

  \node[initial,state]    (0)              {$q_0'$};
  %\node[state]            (1) [right of=0] {$q_1'$};
  \node[state]            (3) [right of=1] {$q_3'$};
  \node[state]            (6) [right of=3] {$q_6'$};

  %\node[state]            (2) [below of=0] {$q_2'$};
  \node[state]            (5) [below right of=0] {$q_5'$};
  
  \node[state]            (4) [below of=2] {$q_4'$};

  \node[state,accepting]  (7) [below right of=2] {$q_7'$};

  \node[state,accepting]  (9) [below=6.6cm of 6] {$q_9'$};
  \node[state,accepting]  (8) [above left of=9] {$q_8'$};

  \path (0) edge              node[sloped, anchor=center, above] {$00$} 		(3)
        (0) edge              node[sloped, anchor=center, above] {$01+11$} 		(5)
        (0) edge              node {$10$} 		(4)

        %(1) edge              node[sloped, anchor=center, above] {0} 		(3)
        %(1) edge              node[sloped, anchor=center, above] {1} 		(5)

        %(2) edge              node[sloped, anchor=center, above] {0} 		(4)
        %(2) edge              node[sloped, anchor=center, above] {1} 		(5)

        (3) edge              node[sloped, anchor=center, above] {0} 		(6)
        (3) edge              node[sloped, anchor=center, above] {1} 		(9)

        (4) edge              node[sloped, anchor=center, above] {0} 		(7)
        (4) edge              node[sloped, anchor=center, above] {1} 		(9)

        (5) edge              node[sloped, anchor=center, above] {0} 		(8)
        (5) edge              node[sloped, anchor=center, above] {1} 		(9)

        (6) edge [loop right] node {0} 		(6)
        (6) edge              node {1} 		(9)

        (7) edge              node[sloped, anchor=center, above] {0} 		(6)
        (7) edge              node[sloped, anchor=center, above] {1} 		(9)

        (8) edge              node[sloped, anchor=center, above] {0} 		(7)
        (8) edge [bend left=10] node[sloped, anchor=center, above] {1} 		(9)

        (9) edge [bend left=10] node[sloped, anchor=center, above] {0} 		(8)
        (9) edge [loop right]  node {1} 		(9);
\end{tikzpicture} 
    \caption{}\label{}
\end{subfigure}
\caption{}\label{app-ex-fig-6}
%
\end{figure}

\begin{figure}
\begin{subfigure}[b]{\linewidth}
\centering
\begin{tikzpicture}[initial text={}, ->,>=stealth',shorten >=1pt,auto,node distance=4.0cm, semithick]
  \tikzstyle{every state}=[fill=none,draw=black,text=black]

  \node[initial,state]    (0)              {$q_0'$};
  %\node[state]            (1) [right of=0] {$q_1'$};
  %\node[state]            (3) [right of=1] {$q_3'$};
  \node[state]            (6) [right of=3] {$q_6'$};

  %\node[state]            (2) [below of=0] {$q_2'$};
  %\node[state]            (5) [below right of=0] {$q_5'$};
  
  \node[state]            (4) [below of=2] {$q_4'$};

  \node[state,accepting]  (7) [below right of=2] {$q_7'$};

  \node[state,accepting]  (9) [below=6.6cm of 6] {$q_9'$};
  \node[state,accepting]  (8) [above left of=9] {$q_8'$};

  \path (0) edge              node[sloped, anchor=center, above] {$000$} 		(6)
        (0) edge              node[sloped, anchor=center, above] {$(01+11)0$} 		(8)
        (0) edge              node[sloped, anchor=center, below] {$001 + (01+11)1$} 		(9)
        (0) edge              node {$10$} 		(4)

        %(1) edge              node[sloped, anchor=center, above] {0} 		(3)
        %(1) edge              node[sloped, anchor=center, above] {1} 		(5)

        %(2) edge              node[sloped, anchor=center, above] {0} 		(4)
        %(2) edge              node[sloped, anchor=center, above] {1} 		(5)

        %(3) edge              node[sloped, anchor=center, above] {0} 		(6)
        %(3) edge              node[sloped, anchor=center, above] {1} 		(9)

        (4) edge              node[sloped, anchor=center, above] {0} 		(7)
        (4) edge              node[sloped, anchor=center, above] {1} 		(9)

        %(5) edge              node[sloped, anchor=center, above] {0} 		(8)
        %(5) edge              node[sloped, anchor=center, above] {1} 		(9)

        (6) edge [loop right] node {0} 		(6)
        (6) edge              node {1} 		(9)

        (7) edge              node[sloped, anchor=center, above] {0} 		(6)
        (7) edge              node[sloped, anchor=center, above] {1} 		(9)

        (8) edge              node[sloped, anchor=center, above] {0} 		(7)
        (8) edge [bend left=10] node[sloped, anchor=center, above] {1} 		(9)

        (9) edge [bend left=10] node[sloped, anchor=center, above] {0} 		(8)
        (9) edge [loop right]  node {1} 		(9);
\end{tikzpicture} 
    \caption{}\label{}
\end{subfigure}%
\\
\begin{subfigure}[b]{\linewidth}
\centering
\begin{tikzpicture}[initial text={}, ->,>=stealth',shorten >=1pt,auto,node distance=6.0cm, semithick]
  \tikzstyle{every state}=[fill=none,draw=black,text=black]

  \node[initial,state]    (0)              {$q_0'$};
  %\node[state]            (1) [right of=0] {$q_1'$};
  %\node[state]            (3) [right of=1] {$q_3'$};
  %\node[state]            (6) [right of=3] {$q_6'$};

  %\node[state]            (2) [below of=0] {$q_2'$};
  %\node[state]            (5) [below right of=0] {$q_5'$};
  
  %\node[state]            (4) [below of=2] {$q_4'$};

  \node[state,accepting]  (7) [below of=0] {$q_7'$};

  \node[state,accepting]  (8) [right of=0] {$q_8'$};
  \node[state,accepting]  (9) [right of=7] {$q_9'$};

  \path %(0) edge              node {$000$} 		(6)
        (0) edge              node[sloped, anchor=center, above] {$(01+11)0$} 		(8)
        (0) edge              node [sloped, xshift=-0.2em,anchor=center, above]{$001 + (01+11)1 + 101 + 0000^*1$} 		(9)
        (0) edge              node[sloped, anchor=center, above] {$100$} 		(7)

        %(1) edge              node[sloped, anchor=center, above] {0} 		(3)
        %(1) edge              node[sloped, anchor=center, above] {1} 		(5)

        %(2) edge              node[sloped, anchor=center, above] {0} 		(4)
        %(2) edge              node[sloped, anchor=center, above] {1} 		(5)

        %(3) edge              node[sloped, anchor=center, above] {0} 		(6)
        %(3) edge              node[sloped, anchor=center, above] {1} 		(9)

        %(4) edge              node[sloped, anchor=center, above] {0} 		(7)
        %(4) edge              node[sloped, anchor=center, above] {1} 		(9)

        %(5) edge              node[sloped, anchor=center, above] {0} 		(8)
        %(5) edge              node[sloped, anchor=center, above] {1} 		(9)

        %(6) edge [loop below] node[sloped, anchor=center, above] {0} 		(6)
        %(6) edge              node[sloped, anchor=center, above] {1} 		(9)

        %(7) edge              node[sloped, anchor=center, above] {0} 		(6)
        (7) edge              node[sloped, anchor=center, above] {$1 + 00^*1$} 		(9)

        (8) edge              node  [sloped, xshift=-5.0em, anchor=west, above]{$0$} 		(7)
        (8) edge [bend left=10] node[sloped, anchor=center, above] {1} 		(9)

        (9) edge [bend left=10] node[sloped, anchor=center, above] {0} 		(8)
        (9) edge [loop below]  node[sloped, anchor=center, above] {1} 		(9);
\end{tikzpicture} 
    \caption{}\label{}
\end{subfigure}
\caption{}\label{app-ex-fig-7}
%
\end{figure}


    Далее, добавим новое состояние, которое сделаем финальным и в которое пустим спонтанные переходы из текущих финальных. Также текущие финальные сделаем нефинальными. Результат и последующие шаги показаны на рисунках 
    \ref{app-ex-fig-8} (a)--(b) (с.~\pageref{app-ex-fig-8}) и
    \ref{app-ex-fig-9} (a)--(b) (с.~\pageref{app-ex-fig-9}).
    На данных рисунках использованы следующие обозначения:
    \begin{align*}
    K_1 &= (01+11)0 + (001 + (01+11)1 + 101 + 0000^*1)1,\\
    K_2 &= (01+11)0 + (001 + (01+11)1 + 101 + 0000^*1)1 + 100(1+00^*1)1^*0,\\
    K_3 &= 001 + (01+11)1 + 101 + 0000^*1 + 100(e + (1+00^*1)1^*),\\
    K_4 &= e + 11^* + 0(e + (1+00^*1)1^*),\\
    K_5 &= 001 + (01+11)1 + 101 + 0000^*1 + 100(e + (1+00^*1)1^*) +\\
        & ((01+11)0 + (001 + (01+11)1 + 101 + 0000^*1)1
        + 100(1+00^*1)1^*0)\cdot\\
        & (10 + 0(1+00^*1)1^*0)^*(e + 11^* + 0(e + (1+00^*1)1^*)).
    \end{align*}

\begin{figure}
\begin{subfigure}[b]{\linewidth}
\centering
\begin{tikzpicture}[initial text={}, ->,>=stealth',shorten >=1pt,auto,node distance=6.0cm, semithick]
  \tikzstyle{every state}=[fill=none,draw=black,text=black]

  \node[initial,state]    (0)              {$q_0'$};
  %\node[state]            (1) [right of=0] {$q_1'$};
  %\node[state]            (3) [right of=1] {$q_3'$};
  %\node[state]            (6) [right of=3] {$q_6'$};

  %\node[state]            (2) [below of=0] {$q_2'$};
  %\node[state]            (5) [below right of=0] {$q_5'$};
  
  %\node[state]            (4) [below of=2] {$q_4'$};

  \node[state]  (7) [below of=0] {$q_7'$};

  \node[state]  (8) [right of=0] {$q_8'$};
  \node[state]  (9) [right of=7] {$q_9'$};
  \node[state,accepting]  (F) [right of=9] {$q_f'$};

  \path %(0) edge              node {$000$} 		(6)
        (0) edge              node[sloped, anchor=center, above] {$(01+11)0$} 		(8)
        (0) edge              node [sloped, xshift=-0.2em,anchor=center, above]{$001 + (01+11)1 + 101 + 0000^*1$} 		(9)
        (0) edge              node[sloped, anchor=center, above] {$100$} 		(7)

        %(1) edge              node {0} 		(3)
        %(1) edge              node {1} 		(5)

        %(2) edge              node {0} 		(4)
        %(2) edge              node {1} 		(5)

        %(3) edge              node {0} 		(6)
        %(3) edge              node {1} 		(9)

        %(4) edge              node {0} 		(7)
        %(4) edge              node {1} 		(9)

        %(5) edge              node {0} 		(8)
        %(5) edge              node {1} 		(9)

        %(6) edge [loop below] node {0} 		(6)
        %(6) edge              node {1} 		(9)

        %(7) edge              node {0} 		(6)
        (7) edge              node[sloped, anchor=center, above] {$1 + 00^*1$} 		(9)

        (8) edge              node  [sloped, xshift=-5.0em, anchor=west, above]{$0$} 		(7)
        (8) edge [bend left=10] node[sloped, anchor=center, above] {1} 		(9)

        (9) edge [bend left=10] node[sloped, anchor=center, above] {0} 		(8)
        (9) edge [loop below]  node[sloped, anchor=center, above] {1} 		(9)

        (7) edge [bend right=25] node [below] {$e$} 		(F)
        (8) edge node [sloped, anchor=center, above]{$e$} 		(F)
        (9) edge node [sloped, anchor=center, above]{$e$} 		(F);
\end{tikzpicture} 
    \caption{}\label{}
\end{subfigure}%
\\
\begin{subfigure}[b]{\linewidth}
\centering
\begin{tikzpicture}[initial text={}, ->,>=stealth',shorten >=1pt,auto,node distance=6.0cm, semithick]
  \tikzstyle{every state}=[fill=none,draw=black,text=black]

  \node[initial,state]    (0)              {$q_0'$};
  %\node[state]            (1) [right of=0] {$q_1'$};
  %\node[state]            (3) [right of=1] {$q_3'$};
  %\node[state]            (6) [right of=3] {$q_6'$};

  %\node[state]            (2) [below of=0] {$q_2'$};
  %\node[state]            (5) [below right of=0] {$q_5'$};
  
  %\node[state]            (4) [below of=2] {$q_4'$};

  \node[state]  (7) [below of=0] {$q_7'$};

  \node[state]  (8) [right of=0] {$q_8'$};
  %\node[state]  (9) [right of=7] {$q_9'$};
  \node[state,accepting]  (F) [right of=9 ] {$q_f'$};

  \path %(0) edge              node {$000$} 		(6)
        (0) edge              node {$K_1$} 		(8)
        (0) edge              node [sloped, xshift=0.4em,anchor=center, above]{$001 + (01+11)1 + 101 + 0000^*1$} 		(F)
        (0) edge              node [sloped, anchor=center, above]{$100$} 		(7)

        %(1) edge              node {0} 		(3)
        %(1) edge              node {1} 		(5)

        %(2) edge              node {0} 		(4)
        %(2) edge              node {1} 		(5)

        %(3) edge              node {0} 		(6)
        %(3) edge              node {1} 		(9)

        %(4) edge              node {0} 		(7)
        %(4) edge              node {1} 		(9)

        %(5) edge              node {0} 		(8)
        %(5) edge              node {1} 		(9)

        %(6) edge [loop below] node {0} 		(6)
        %(6) edge              node {1} 		(9)

        %(7) edge              node {0} 		(6)
        %(7) edge              node [sloped, anchor=center, above]{$1 + 00^*1$} 		(9)
        (7) edge [bend left]             node [sloped, anchor=center, above]{$(1 + 00^*1)1^*0$} 		(8)

        (8) edge [bend left]             node  {$0$} 		(7)
        (8) edge [loop above] node[sloped, anchor=center, above] {$10$} 		(8)

        %(9) edge [bend left=10] node {0} 		(8)
        %(9) edge [loop below]  node {1} 		(9)

        (7) edge node [below] {$e + (1+00^*1)1^*$} 		(F)
        (8) edge node [sloped, anchor=center, above]{$e + 11^*$} 		(F)
        %(9) edge node [sloped, anchor=center, above]{$e$} 		(F)
        ;
\end{tikzpicture} 
    \caption{}\label{}
\end{subfigure}
\caption{}\label{app-ex-fig-8}
%
\end{figure}

\begin{figure}
\centering
\begin{subfigure}[b]{0.4\linewidth}
\centering
\begin{tikzpicture}[initial text={}, ->,>=stealth',shorten >=1pt,auto,node distance=3cm, semithick]
  \tikzstyle{every state}=[fill=none,draw=black,text=black]

  \node[initial,state]    (0)              {$q_0'$};
  %\node[state]  (7) [below of=0] {$q_7'$};

  \node[state]  (8) [right of=0] {$q_8'$};
  \node[state,accepting]  (F) [below of=8 ] {$q_f'$};

  \path 
        (0) edge              node {$K_2$} 		(8)
        (0) edge              node [sloped, xshift=0.4em,anchor=center, above]{$K_3$} 		(F)
        (8) edge [loop above] node {$10 + 0(1+00^*1)1^*0$} 		(8)
        (8) edge node {$K_4$} 		(F)
        ;
\end{tikzpicture} 
    \caption{}\label{}
\end{subfigure}%
%
\begin{subfigure}[b]{0.4\linewidth}
\centering
\begin{tikzpicture}[initial text={}, ->,>=stealth',shorten >=1pt,auto,node distance=3cm, semithick]
  \tikzstyle{every state}=[fill=none,draw=black,text=black]

  \node[initial,state]    (0)              {$q_0'$};
  %\node[state]  (8) [right of=0] {$q_8'$};
  \node[state,accepting]  (F) [right of=0 ] {$q_f'$};

  \path 
        (0) edge              node {$K_5$} 		(F);
\end{tikzpicture}
    \caption{}\label{}
\end{subfigure}
\caption{}\label{app-ex-fig-9}
%
\end{figure}


    В результате мы получаем регулярное выражение 
    \begin{multline*}
    L(M^{D}) =
    001 + (01+11)1 + 101 + 0000^*1 + 100(e + (1+00^*1)1^*) +\\
    ((01+11)0 + (001 + (01+11)1 + 101 + 0000^*1)1 + 100(1+00^*1)1^*0) \cdot\\
    (10 + 0(1+00^*1)1^*0)^*(e + 11^* + 0(e + (1+00^*1)1^*)).
    \end{multline*}

	\item Автомат $\overline{M^D}$ приведен на рис.~\ref{app-ex-fig-15}.% (с.~\pageref{app-ex-fig-15}).
\begin{figure}[H]
\centering
		\begin{tikzpicture}[initial text={}, ->,>=stealth',shorten >=1pt,auto,node distance=4.0cm, semithick]
		  \tikzstyle{every state}=[fill=none,draw=black,text=black]

		  \node[initial,state,accepting]    (0)              {$q_0'$};
		  \node[state,accepting]            (1) [right of=0] {$q_1'$};
		  \node[state,accepting]            (3) [right of=1] {$q_3'$};

		  \node[state,accepting]            (2) [below of=0] {$q_2'$};
		  \node[state,accepting]            (5) [below right of=0] {$q_5'$};
		  
		  \node[state,accepting]            (4) [below of=2] {$q_4'$};

		  \node[state]  (7) [below right of=2] {$q_7'$};

		  \node[state]  (9) [below=6.6cm of 3] {$q_9'$};
		  \node[state]  (8) [above left=3.5cm and .7cm of 9] {$q_8'$};

		  \path (0) edge              node {0} 		(1)
		  		(0) edge              node {1} 		(2)

		  		(1) edge              node {0} 		(3)
		  		(1) edge              node {1} 		(5)

		  		(2) edge              node {0} 		(4)
		  		(2) edge              node {1} 		(5)

		  		(3) edge              node {1} 		(9)
		  		(3) edge [loop right] node {0} 		(3)

		  		(4) edge              node {0} 		(7)
		  		(4) edge              node {1} 		(9)

		  		(5) edge              node {0} 		(8)
		  		(5) edge              node {1} 		(9)

		  		(7) edge [bend left=20] node {0} 		(3)
		  		(7) edge              node {1} 		(9)

		  		(8) edge              node {0} 		(7)
		  		(8) edge [bend left=10] node {1} 		(9)

		  		(9) edge [bend left=10] node {0} 		(8)
				(9) edge [loop right]  node {1} 		(9);
		\end{tikzpicture}
\caption{}\label{app-ex-fig-15}
\end{figure}

    $\overline{L} = e + 0 + 1 + 00 + 01 + 10 + 11 + (0+1)^*000$.
\end{enumerate}

\clearpage
\section*{Задание 2}
\setcounter{section}{2}
\cohead{Задание 2}

$\Sigma = \{a, b, c\}, A = \{bbab, ccac, bacb, baabc\}.$
\begin{enumerate}[label=(\roman{*})]
	\item Для каждого слова $w_i \in A$ построить НКА $M^{ND}_i$, распознающий наличие в произвольной строке $s \in \Sigma^*$ подстроки $w_i$.

	\begin{enumerate}
		\item НКА, распознающий наличие в произвольной строке $s \in \Sigma^*$ подстроки $w_1$ = bbab:
%
		\[M^{ND}_1 = (\{q_0, q_1, q_2, q_3, q_4, q_f\}, \{a, b, c\}, \delta, q_0, \{q_f\}),\]
%
        функция $\delta$ представлена на рисунке ниже.
			\begin{center}
			\begin{tabular}{llll}
				\toprule
				\multicolumn{1}{c}{\multirow{2}{*}{\Large $\delta$}}
				& \multicolumn{3}{c}{Вход} \\
				\cmidrule(rl){2-4}
				& \multicolumn{1}{c}{$a$}
				& \multicolumn{1}{c}{$b$} 
				& \multicolumn{1}{c}{$c$} \\
				\midrule
				$\{q_0\}$       & $\{q_0\}$      		 & $\{q_1\}$     &$\{q_0\}$  \\
				$\{q_1\}$       & $\{q_0\}$    			 & $\{q_2\}$     &$\{q_0\}$ \\
				$\{q_2\}$       & $\{q_4\}$    			 & $\{q_3\}$     &$\{q_0\}$  \\
				$\{q_3\}$       & $\{q_4\}$    			 & $\{q_2, q_3\}$     &$\{q_0\}$  \\
				$\{q_4\}$       & $\{q_0\}$    			 & $\{q_f\}$     &$\{q_0\}$  \\
				$\{q_f\}$       & $\{\varnothing\}$    	 & $\{\varnothing\}$     &$\{\varnothing\}$  \\
				\bottomrule
			\end{tabular}
		\end{center}
		
		\item НКА, распознающий наличие в произвольной строке $s \in \Sigma^*$ подстроки $w_2$ = ссaс:
%
		\[M^{ND}_2 = (\{q_0, q_1, q_2, q_3, q_4, q_f\}, \{a, b, c\}, \delta, q_0, \{q_f\}),\]
%
        функция $\delta$ представлена на рис.~\ref{app-ex-task2-1} (с.~\pageref{app-ex-task2-1}).

\begin{figure}
\centering
\begin{subfigure}[b]{.4\linewidth}
\centering
			\begin{tabular}{llll}
				\toprule
				\multicolumn{1}{c}{\multirow{2}{*}{\Large $\delta$}}
				& \multicolumn{3}{c}{Вход} \\
				\cmidrule(rl){2-4}
				& \multicolumn{1}{c}{$a$}
				& \multicolumn{1}{c}{$b$} 
				& \multicolumn{1}{c}{$c$} \\
				\midrule
				$\{q_0\}$       & $\{q_0\}$      		 & $\{q_0\}$     &$\{q_1\}$  \\
				$\{q_1\}$       & $\{q_0\}$    			 & $\{q_0\}$     &$\{q_2\}$ \\
				$\{q_2\}$       & $\{q_4\}$    			 & $\{q_0\}$     &$\{q_3\}$  \\
				$\{q_3\}$       & $\{q_4\}$    			 & $\{q_0\}$     &$\{q_2, q_3\}$  \\
				$\{q_4\}$       & $\{q_0\}$    			 & $\{q_0\}$     &$\{q_f\}$  \\
				$\{q_f\}$       & $\{\varnothing\}$    	 & $\{\varnothing\}$     &$\{\varnothing\}$  \\
				\bottomrule
			\end{tabular}
\caption{}\label{app-ex-task2-1}
\end{subfigure}%
%
\begin{subfigure}[b]{.4\linewidth}
\centering
			\begin{tabular}{llll}
				\toprule
				\multicolumn{1}{c}{\multirow{2}{*}{\Large $\delta$}}
				& \multicolumn{3}{c}{Вход} \\
				\cmidrule(rl){2-4}
				& \multicolumn{1}{c}{$a$}
				& \multicolumn{1}{c}{$b$} 
				& \multicolumn{1}{c}{$c$} \\
				\midrule
				$\{q_0\}$       & $\{q_0\}$      		 & $\{q_1\}$     &$\{q_0\}$  \\
				$\{q_1\}$       & $\{q_3\}$    			 & $\{q_2\}$     &$\{q_0\}$ \\
				$\{q_2\}$       & $\{q_3\}$    			 & $\{q_1, q_2\}$     &$\{q_0\}$  \\
				$\{q_3\}$       & $\{q_0\}$    			 & $\{q_1, q_2\}$     &$\{q_4\}$  \\
				$\{q_4\}$       & $\{q_0\}$    			 & $\{q_f\}$     &$\{q_0\}$  \\
				$\{q_f\}$       & $\{\varnothing\}$    	 & $\{\varnothing\}$     &$\{\varnothing\}$  \\
				\bottomrule
			\end{tabular}
\caption{}\label{app-ex-task2-2}
\end{subfigure}
\caption{}\label{app-ex-task2-3}
\end{figure}
		
		
		\item НКА, распознающий наличие в произвольной строке $s \in \Sigma^*$ подстроки $w_3$ = bacb:
		\[
            M^{ND}_3 = (\{q_0, q_1, q_2, q_3, q_4, q_f\}, \{a, b, c\}, \delta, q_0, \{q_f\}),
        \]
		функция $\delta$ представлена на рис.~\ref{app-ex-task2-2}.
        
		\item НКА распознающий наличие в произвольной строке $s \in \Sigma^*$ подстроки $w_4$ = baabc:
		\[
            M^{ND}_4 = (\{q_0, q_1, q_2, q_3, q_4, q_5, q_f\}, \{a, b, c\}, \delta, q_0, \{q_f\}),
        \]
        функция $\delta$ представлена на рис.~\ref{app-ex-task2-4} (с.~\pageref{app-ex-task2-4}).

\begin{figure}
\centering
\begin{subfigure}[b]{.4\linewidth}
\centering
			\begin{tabular}{llll}
				\toprule
				\multicolumn{1}{c}{\multirow{2}{*}{\Large $\delta$}}
				& \multicolumn{3}{c}{Вход} \\
				\cmidrule(rl){2-4}
				& \multicolumn{1}{c}{$a$}
				& \multicolumn{1}{c}{$b$} 
				& \multicolumn{1}{c}{$c$} \\
				\midrule
				$\{q_0\}$       & $\{q_0\}$      		 & $\{q_1\}$     &$\{q_0\}$  \\
				$\{q_1\}$       & $\{q_3\}$    			 & $\{q_2\}$     &$\{q_0\}$ \\
				$\{q_2\}$       & $\{q_3\}$    			 & $\{q_2\}$     &$\{q_0\}$  \\
				$\{q_3\}$       & $\{q_4\}$    			 & $\{q_1, q_2\}$     &$\{q_0\}$  \\
				$\{q_4\}$       & $\{q_0\}$    			 & $\{q_5\}$     &$\{q_0\}$  \\
				$\{q_5\}$       & $\{q_3\}$    			 & $\{q_1, q_2\}$     &$\{q_f\}$  \\
				$\{q_f\}$       & $\{\varnothing\}$    	 & $\{\varnothing\}$     &$\{\varnothing\}$  \\
				\bottomrule
			\end{tabular}
\caption{}\label{app-ex-task2-4}
\end{subfigure}%
%
\begin{subfigure}[b]{.4\linewidth}
\centering
			\begin{tabular}{llll}
				\toprule
				\multicolumn{1}{c}{\multirow{2}{*}{\Large $\delta$}}
				& \multicolumn{3}{c}{Вход} \\
				\cmidrule(rl){2-4}
				& \multicolumn{1}{c}{$a$}
				& \multicolumn{1}{c}{$b$} 
				& \multicolumn{1}{c}{$c$} \\
				\midrule
				$\{q_0\}$       & $\{q_0\}$      		 & $\{q_1\}$     &$\{q_0\}$  \\
				$\{q_1\}$       & $\{q_0\}$    			 & $\{q_2\}$     &$\{q_0\}$ \\
				$\{q_2\}$       & $\{q_4\}$    			 & $\{q_3\}$     &$\{q_0\}$  \\
				$\{q_3\}$       & $\{q_4\}$    			 & $\{q_5\}$     &$\{q_0\}$  \\
				$\{q_4\}$       & $\{q_0\}$    			 & $\{q_f\}$     &$\{q_0\}$  \\
				$\{q_5\}$ 		& $\{q_4\}$    			 & $\{q_5\}$     &$\{q_0\}$  \\
				$\{q_f\}$       & $\{\varnothing\}$    	 & $\{\varnothing\}$     &$\{\varnothing\}$  \\
				\bottomrule
			\end{tabular}

\caption{}\label{app-ex-task2-5}
\end{subfigure}
\caption{}\label{app-ex-task2-6}
\end{figure}

	\end{enumerate}

	\item Для каждого НКА $M^{ND}_i$ построить соответствующий ДКА $M^D_i$.
		
	\begin{enumerate}
		
		\item Для НКА $M^{ND}_1$ соответствующий ДКА:
        \[
            M^{D}_1 = (\{q_0, q_1, q_2, q_3, q_4, q_5, q_f\}, \{a, b, c\}, \delta, q_0, \{q_f\}),
        \]
        с учетом переобозначения $\{q_2, q_3\} = \{q_5\}$, см. рис.~\ref{app-ex-task2-5}.
		
		\item Для НКА $M^{ND}_2$ соответствующий ДКА:
		\[
            M^{D}_2 = (\{q_0, q_1, q_2, q_3, q_4, q_5, q_f\}, \{a, b, c\}, \delta, q_0, \{q_f\}),
        \]
        с учетом переобозначения $\{q_2, q_3\} = \{q_5\}$, см. рис.~\ref{app-ex-task2-7} (с.~\pageref{app-ex-task2-7}).

\begin{figure}
\centering
\begin{subfigure}[b]{.4\linewidth}
\centering
			\begin{tabular}{llll}
				\toprule
				\multicolumn{1}{c}{\multirow{2}{*}{\Large $\delta$}}
				& \multicolumn{3}{c}{Вход} \\
				\cmidrule(rl){2-4}
				& \multicolumn{1}{c}{$a$}
				& \multicolumn{1}{c}{$b$} 
				& \multicolumn{1}{c}{$c$} \\
				\midrule
				$\{q_0\}$       & $\{q_0\}$      		 & $\{q_0\}$     &$\{q_1\}$  \\
				$\{q_1\}$       & $\{q_0\}$    			 & $\{q_0\}$     &$\{q_2\}$ \\
				$\{q_2\}$       & $\{q_4\}$    			 & $\{q_0\}$     &$\{q_3\}$  \\
				$\{q_3\}$       & $\{q_4\}$    			 & $\{q_0\}$     &$\{q_5\}$  \\
				$\{q_4\}$       & $\{q_0\}$    			 & $\{q_0\}$     &$\{q_f\}$  \\
				$\{q_5\}$       & $\{q_4\}$    			 & $\{q_0\}$     &$\{q_5\}$  \\
				$\{q_f\}$       & $\{\varnothing\}$    	 & $\{\varnothing\}$     &$\{\varnothing\}$  \\
				\bottomrule
			\end{tabular}
\caption{}\label{app-ex-task2-7}
\end{subfigure}%
%
\begin{subfigure}[b]{.4\linewidth}
\centering
			\begin{tabular}{llll}
				\toprule
				\multicolumn{1}{c}{\multirow{2}{*}{\Large $\delta$}}
				& \multicolumn{3}{c}{Вход} \\
				\cmidrule(rl){2-4}
				& \multicolumn{1}{c}{$a$}
				& \multicolumn{1}{c}{$b$} 
				& \multicolumn{1}{c}{$c$} \\
				\midrule
				$\{q_0\}$       & $\{q_0\}$      		 & $\{q_1\}$     &$\{q_0\}$  \\
				$\{q_1\}$       & $\{q_3\}$    			 & $\{q_2\}$     &$\{q_0\}$ \\
				$\{q_2\}$       & $\{q_3\}$    			 & $\{q_5\}$     &$\{q_0\}$  \\
				$\{q_3\}$       & $\{q_0\}$    			 & $\{q_5\}$     &$\{q_4\}$  \\
				$\{q_4\}$       & $\{q_0\}$    			 & $\{q_f\}$     &$\{q_0\}$  \\
				$\{q_5\}$       & $\{q_3\}$    			 & $\{q_5\}$     &$\{q_0\}$ \\
				$\{q_f\}$       & $\{\varnothing\}$    	 & $\{\varnothing\}$     &$\{\varnothing\}$  \\
				\bottomrule
			\end{tabular}
\caption{}\label{app-ex-task2-8}
\end{subfigure}
\caption{}\label{app-ex-task2-9}
\end{figure}
		
		\item Для НКА $M^{ND}_3$ соответствующий ДКА:
		\[M^{D}_3 = (\{q_0, q_1, q_2, q_3, q_4, q_5, q_f\}, \{a, b, c\}, \delta, q_0, \{q_f\})\]
		с учетом переобозначения $\{q_1, q_2\} = \{q_5\}$, см. рис.~\ref{app-ex-task2-8}.
		
		\item Для НКА $M^{ND}_4$ соответствующий ДКА:
		\[M^{D}_4 = (\{q_0, q_1, q_2, q_3, q_4, q_5, q_6, q_f\}, \{a, b, c\}, \delta, q_0, \{q_f\})\]
		с учетом переобозначения $\{q_1, q_2\} = \{q_6\}$, см. рис.~\ref{app-ex-task2-10} (с.~\pageref{app-ex-task2-10}).
			
\begin{figure}
\centering
			\begin{tabular}{llll}
				\toprule
				\multicolumn{1}{c}{\multirow{2}{*}{\Large $\delta$}}
				& \multicolumn{3}{c}{Вход} \\
				\cmidrule(rl){2-4}
				& \multicolumn{1}{c}{$a$}
				& \multicolumn{1}{c}{$b$} 
				& \multicolumn{1}{c}{$c$} \\
				\midrule
				$\{q_0\}$       & $\{q_0\}$      		 & $\{q_1\}$     &$\{q_0\}$  \\
				$\{q_1\}$       & $\{q_3\}$    			 & $\{q_2\}$     &$\{q_0\}$ \\
				$\{q_2\}$       & $\{q_3\}$    			 & $\{q_2\}$     &$\{q_0\}$  \\
				$\{q_3\}$       & $\{q_4\}$    			 & $\{q_6\}$     &$\{q_0\}$  \\
				$\{q_4\}$       & $\{q_0\}$    			 & $\{q_5\}$     &$\{q_0\}$  \\
				$\{q_5\}$       & $\{q_3\}$    			 & $\{q_6\}$     &$\{q_f\}$  \\
				$\{q_6\}$       & $\{q_3\}$    			 & $\{q_2\}$     &$\{q_0\}$ \\
				$\{q_f\}$       & $\{\varnothing\}$    	 & $\{\varnothing\}$     &$\{\varnothing\}$  \\
				\bottomrule
			\end{tabular}
\caption{}\label{app-ex-task2-10}
\end{figure}
		
	\end{enumerate}
\end{enumerate}

\clearpage
\section*{Задание 3}
\setcounter{section}{3}
\cohead{Задание 3}

\[L_1 = (aa + bb)^*a^*ab(a + ba)^*, \qquad L_2 = (aa)^*a(a + b)^*(bb)^*\]

Перед началом решения задания проведём \emph{вспомогательные построения}.

Граф переходов НКА для $L_1$:

	\begin{center}	
    \resizebox{\linewidth}{!}{%
		\begin{tikzpicture}[auto,>=stealth', node distance=3cm,auto,every state/.style={thick}]
		\node (init) {};
		\node[state] (1) [right=.7cm of init] {$1$};
		\node[state] (2) [above of=1] {$2$};
		\node[state] (3) [below of=1] {$3$};
		\node[state] (4) [right of=1] {$4$};	
		\node[state, accepting] (5) [right of=4] {$5$};	
		\node[state] (7) [right of=5, below of=5] {$7$};	
		\node[state, accepting] (6) [right of=7, above of=7] {$6$};	
		\node[state] (8) [right of=6] {$8$};
			
		\path[->]
		(init) edge (1)
		(1) edge[bend right] node[right] {$a$} (2)
		(2) edge[bend right] node[left] {$a$} (1)
		(1) edge[bend left] node[right] {$b$} (3)
		(3) edge[bend left] node[left] {$b$} (1)
		(1) edge node {$a$} (4)
		(4) edge [loop above] node {$a$} (4)
		(4) edge node {$b$} (5)
		(5) edge node {$a$} (6)
		(5) edge node {$b$} (7)
		(7) edge node {$a$} (6)
		(6) edge [loop above] node {$a$} (6)
		(6) edge[bend left] node {$b$} (8)
		(8) edge[bend left] node {$a$} (6)
		;
		\end{tikzpicture}
    }
	\end{center}

Граф переходов НКА для $L_2$:
	
		\begin{center}	
		\begin{tikzpicture}[auto,>=stealth', node distance=3cm,auto,every state/.style={thick}]
		\node (init) {};
		\node[state] (1) [right=.7cm of init] {$1$};
		\node[state] (2) [above of=1] {$2$};
		\node[state, accepting] (3) [right of=1] {$3$};	
		\node[state] (4) [right of=3] {$4$};	
		\node[state, accepting] (5) [right of=4] {$5$};	
		\node[state] (6) [right of=5] {$6$};	
		
		\path[->]
		(init) edge (1)
		(1) edge[bend right] node[right] {$a$} (2)
		(2) edge[bend right] node[left] {$a$} (1)
		(1) edge node {$a$} (3)
		(3) edge[loop above] node {$a, b$} (3)
		(3) edge node {$b$} (4)
		(4) edge node {$b$} (5)
		(5) edge[bend left] node {$b$} (6)
		(6) edge[bend left] node {$b$} (5)
		;
		\end{tikzpicture}
	\end{center}

Функции переходов $\delta$ для $L_1$ и $L_2$ показаны в табличном виде на рисунках \ref{app-ex-task3-1} и \ref{app-ex-task3-2} соответственно.

\begin{figure}
\centering
\begin{subfigure}[b]{.4\linewidth}
\centering
		\begin{tabular}{lll}
			\toprule
			\multicolumn{1}{c}{\multirow{2}{*}{\Large $\delta$}}
			& \multicolumn{2}{c}{Вход} \\
			\cmidrule(rl){2-3}
			& \multicolumn{1}{c}{$a$}
			& \multicolumn{1}{c}{$b$}  \\
			\midrule
			$\to \{1\}$       & $\{2\}, \{4\}$      		 & $\{3\}$      \\
			$\{2\}$       & $\{1\}$    			 & $\{\varnothing\}$      \\
			$\{3\}$       & $\{\varnothing\}$    			 & $\{1\}$       \\
			$\{4\}$       & $\{4\}$    			 & $\{5\}$       \\
			$\{\textbf{5}\}$       & $\{6\}$    			 & $\{7\}$       \\
			$\{\textbf{6}\}$       & $\{6\}$    			 & $\{8\}$     \\
			$\{7\}$       & $\{6\}$    			 & $\{\varnothing\}$     \\
			$\{8\}$       & $\{6\}$    	 & $\{\varnothing\}$  \\
			\bottomrule
		\end{tabular}
\caption{}\label{app-ex-task3-1}
\end{subfigure}%
%
\begin{subfigure}[b]{.4\linewidth}
\centering
		\begin{tabular}{lll}
			\toprule
			\multicolumn{1}{c}{\multirow{2}{*}{\Large $\delta$}}
			& \multicolumn{2}{c}{Вход} \\
			\cmidrule(rl){2-3}
			& \multicolumn{1}{c}{$a$}
			& \multicolumn{1}{c}{$b$}  \\
			\midrule
			$\to \{1\}$       & $\{2\}, \{3\}$      		 & $\{\varnothing\}$      \\
			$\{2\}$       & $\{1\}$    			 & $\{\varnothing\}$      \\
			$\{\textbf{3}\}$       & $\{3\}$    			 & $\{3\}, \{4\}$       \\
			$\{4\}$       & $\{\varnothing\}$    			 & $\{5\}$       \\
			$\{\textbf{5}\}$       & $\{\varnothing\}$    			 & $\{6\}$       \\
			$\{6\}$       & $\{\varnothing\}$    			 & $\{5\}$     \\
			\bottomrule
		\end{tabular}
\caption{}\label{app-ex-task3-2}
\end{subfigure}
\caption{}\label{app-ex-task3-3}
\end{figure}
	
		
\begin{enumerate}[label=(\roman{*})]
	\item Вычислить регулярное выражение, определяющее язык $L_1 \cap  L_2$. 
    
    Результат непосредственного применения известного алгоритма
    приводит к таблице и графу переходов показанных на рисунках
    \ref{app-ex-task3-1} (с.~\pageref{app-ex-task3-1}). и \ref{app-ex-task3-2} (с.~\pageref{app-ex-task3-2}).

	Состояния 84, 74, 54, 75, 14, 35, 16, 34, 15, 36 являются непродуктивными, поэтому их можно удалить~--- см. рис.~\ref{app-ex-task3-3} (с.~\pageref{app-ex-task3-3}).

\afterpage{%
   \clearpage % flush any accumulated floats
}
   \begin{figure}[ht!]  % or: \begin{table}[ht!]
\centering
		\begin{tabular}{lll}
			\toprule
			\multicolumn{1}{c}{\multirow{2}{*}{\Large $\delta$}}
			& \multicolumn{2}{c}{Вход} \\
			\cmidrule(rl){2-3}
			& \multicolumn{1}{c}{$a$}
			& \multicolumn{1}{c}{$b$}  \\
			\midrule
			$\to \{1, 1\}$       & $\{2, 2\}, \{2, 3\}, \{4, 2\}, \{4, 3\}$     		 & $\{\varnothing\}$      \\
			$\{2, 2\}$       & $\{1, 1\}$    			 & $\{\varnothing\}$      \\
			$\{2, 3\}$       & $\{1, 3\}$    			 & $\{\varnothing\}$      \\
			$\{1, 3\}$       & $\{2, 3\}, \{4, 3\}$     &  $\{3, 3\}, \{3, 4\}$  \\
			$\{3, 3\}$       & $\{\varnothing\}$     &  $\{1, 3\}, \{1, 4\}$  \\
			$\{3, 4\}$       & $\{\varnothing\}$     &  $\{1, 5\}$  \\
			$\{1, 4\}$       & $\{\varnothing\}$     &  $\{3, 5\}$  \\
			$\{1, 5\}$       & $\{\varnothing\}$     &  $\{3, 6\}$  \\
			$\{3, 5\}$       & $\{\varnothing\}$     &  $\{1, 6\}$  \\
			$\{3, 6\}$       & $\{\varnothing\}$     &  $\{1, 5\}$  \\
			$\{1, 6\}$       & $\{\varnothing\}$     &  $\{3, 5\}$  \\
			$\{4, 2\}$       & $\{4, 1\}$    			 & $\{\varnothing\}$      \\
			$\{4, 3\}$       & $\{4, 3\}$    			 & $\{5, 3\}, \{5, 4\}$      \\
			$\{4, 1\}$       & $\{4, 2\}, \{4, 3\}$    			 & $\{\varnothing\}$      \\
			$\textbf{\{5, 3\}}$       & $\{6, 3\}$    			 & $\{7, 3\}, \{7, 4\}$      \\
			$\{5, 4\}$       & $\{\varnothing\}$    			 & $\{7, 5\}$      \\
			$\textbf{\{6, 3\}}$       & $\{6, 3\}$    			 & $\{8, 3\}, \{8, 4\}$      \\
			$\{7, 3\}$       & $\{6, 3\}$    			 & $\{\varnothing\}$      \\
			$\{7, 4\}$       & $\{\varnothing\}$    			 & $\{\varnothing\}$      \\
			$\{7, 5\}$       & $\{\varnothing\}$    			 & $\{\varnothing\}$      \\
			$\{8, 3\}$       & $\{6, 3\}$    			 & $\{\varnothing\}$      \\
			$\{8, 4\}$       & $\{\varnothing\}$    			 & $\{\varnothing\}$      \\
			\bottomrule
		\end{tabular}
\caption{}\label{app-ex-task3-1}
\end{figure}
%}

\begin{figure}[t]
\centering
\resizebox{\linewidth}{!}{%
		\begin{tikzpicture}[auto,>=stealth', node distance=2.5cm,auto,every state/.style={thick}]
		\node (init) {};
		\node[state] (11) [right=.7cm of init] {$11$};
		\node[state] (22) [above of=11] {$22$};
		\node[state] (42) [below of=11] {$42$};
		\node[state] (23) [right of=11] {$23$};
		\node[state] (43) [below of=23] {$43$};
		\node[state] (13) [right of=23] {$13$};
		\node[state] (33) [right of=13] {$33$};
		\node[state] (34) [below of=33] {$34$};
		\node[state] (14) [right of=33] {$14$};	
		\node[state] (15) [right of=34] {$15$};
		\node[state] (35) [right of=14] {$35$};
		\node[state] (36) [right of=15] {$36$};
		\node[state] (16) [right of=35] {$16$};
		\node[state] (41) [below of=42] {$41$};
		\node[state] (54) [right of=43, below of=43] {$54$};
		\node[state, accepting] (53) [below of=54] {$53$};
		\node[state] (75) [right of=54] {$75$};
		\node[state, accepting] (63) [right of=53] {$63$};
		\node[state] (73) [right of=53, below of=53] {$73$};
		\node[state] (74) [below of=53] {$74$};
		\node[state] (84) [right of=63, above of=63] {$84$};
		\node[state] (83) [right of=84, below of=84] {$83$};
		
		\path[->]
		(init) edge (11)
		(11) edge node[right] {$a$} (22)
		(11) edge node {$a$} (23)
		(11) edge node[left] {$a$} (42)
		(11) edge node {$a$} (43)
		
		(22) edge[bend right] node[left] {$a$} (11)
		
		(23) edge node[below] {$a$} (13)
		
		(13) edge[bend right] node[above] {$a$} (23)
		(13) edge node {$a$} (43)
		(13) edge node[below]  {$b$}(33)
		(13) edge node[near end] {$b$} (34)
		
		(33) edge[bend right] node[above] {$b$} (13)
		(33) edge node {$b$} (14)
		(34) edge node {$b$} (15)
		
		(14) edge node {$b$} (35)
		(15) edge node[below] {$b$} (36)
		
		(35) edge node[below] {$b$} (16)
		(36) edge[bend right] node[above] {$b$} (15)
		(16) edge[bend right] node[above] {$b$} (35)
		
		(42) edge node {$a$} (41)
		(41) edge[bend left] node[left] {$a$} (42)
		(41) edge node[near end] {$a$} (43)
		
		(43) edge[loop below] node {$a$} (43)
		(43) edge[bend left] node[near end] {$b$} (54)
		(43) edge node[near end, left] {$b$} (53)
		
		(54) edge node {$b$} (75)
		(53) edge node {$a$} (63)
		(53) edge node[near end] {$b$} (73)
		(53) edge node[left] {$b$} (74)
		
		(63) edge[loop above] node {$a$} (63)
		(63) edge node {$b$} (83)
		(63) edge node {$b$} (84)
		
		(73) edge node[right] {$a$} (63)
		(83) edge[bend left] node {$a$} (63)
		;
		\end{tikzpicture}
}
\caption{}\label{app-ex-task3-2}
\end{figure}

\begin{figure}[!t]
\centering
		\begin{tikzpicture}[auto,>=stealth', node distance=2.5cm,auto,every state/.style={thick}]
		\node (init) {};
		\node[state] (11) [right=.7cm of init] {$11$};
		\node[state] (22) [above of=11] {$22$};
		\node[state] (42) [below of=11] {$42$};
		\node[state] (23) [right of=11] {$23$};
		\node[state] (43) [below of=23] {$43$};
		\node[state] (13) [right of=23] {$13$};
		\node[state] (33) [right of=13] {$33$};
		\node[state] (41) [below of=42] {$41$};
		\node[state, accepting] (53) [right of=43] {$53$};
		\node[state, accepting] (63) [right of=53] {$63$};
		\node[state] (73) [right of=53, below of=53] {$73$};
		\node[state] (83) [right of=63] {$83$};
		
		\path[->]
		(init) edge (11)
		(11) edge[bend right] node[right] {$a$} (22)
		(11) edge node {$a$} (23)
		(11) edge node[left] {$a$} (42)
		(11) edge node {$a$} (43)
		
		(22) edge[bend right] node[left] {$a$} (11)
		
		(23) edge node[below] {$a$} (13)
		
		(13) edge[bend right] node[above] {$a$} (23)
		(13) edge node {$a$} (43)
		(13) edge node[below]  {$b$}(33)
		
		(33) edge[bend right] node[above] {$b$} (13)
		
		(42) edge node {$a$} (41)
		(41) edge[bend left] node[left] {$a$} (42)
		(41) edge node[near end] {$a$} (43)
		
		(43) edge[loop below] node {$a$} (43)
		(43) edge[bend right] node {$b$} (53)
		
		(53) edge node {$a$} (63)
		(53) edge node[near end] {$b$} (73)
		
		(63) edge[loop above] node {$a$} (63)
		(63) edge node {$b$} (83)
		
		(73) edge node[right] {$a$} (63)
		(83) edge[bend left] node {$a$} (63)
		;
		\end{tikzpicture}
\caption{}\label{app-ex-task3-3}
\end{figure}

	Воспользуемся методом последовательного исключения состояний. Пустим $\varepsilon$-переходы из всех допускающих состояний в новое состояние $q_f$. Все допускающие состояния сделаем недопускающими, а новое состояние $q_f$ --- допускающим. Результат приведен на рис.~\ref{app-ex-task3-4}.
	
\begin{figure}[h]
\centering
%	\begin{center}	
		\begin{tikzpicture}[auto,>=stealth', node distance=2.5cm,auto,every state/.style={thick}]
		\node (init) {};
		\node[state] (11) [right=.7cm of init] {$11$};
		\node[state] (22) [above of=11] {$22$};
		\node[state] (42) [below of=11] {$42$};
		\node[state] (23) [right of=11] {$23$};
		\node[state] (43) [below of=23] {$43$};
		\node[state] (13) [right of=23] {$13$};
		\node[state] (33) [right of=13] {$33$};
		\node[state] (41) [below of=42] {$41$};
		\node[state] (53) [right of=43, below of=43] {$53$};
		\node[state] (63) [right of=53] {$63$};
		\node[state] (73) [right of=53, below of=53] {$73$};
		\node[state] (83) [right of=63] {$83$};
		\node[state, accepting] (qf) [right of=43] {$q_f$};
		
		\path[->]
		(init) edge (11)
		(11) edge[bend right] node[right] {$a$} (22)
		(11) edge node {$a$} (23)
		(11) edge node[left] {$a$} (42)
		(11) edge node {$a$} (43)
		
		(22) edge[bend right] node[left] {$a$} (11)
		
		(23) edge node[below] {$a$} (13)
		
		(13) edge[bend right] node[above] {$a$} (23)
		(13) edge node {$a$} (43)
		(13) edge node[below]  {$b$}(33)
		
		(33) edge[bend right] node[above] {$b$} (13)
		
		(42) edge node {$a$} (41)
		(41) edge[bend left] node[left] {$a$} (42)
		(41) edge node[near end] {$a$} (43)
		
		(43) edge[loop below] node {$a$} (43)
		(43) edge node {$b$} (53)
		
		(53) edge node {$a$} (63)
		(53) edge node[near end] {$b$} (73)
		
		(63) edge[loop above] node {$a$} (63)
		(63) edge node {$b$} (83)
		
		(73) edge node[right] {$a$} (63)
		(83) edge[bend left] node {$a$} (63)
		
		(53) edge node {$\varepsilon$} (qf)
		(63) edge node {$\varepsilon$} (qf)
		
		;
		\end{tikzpicture}
%	\end{center}
\caption{}\label{app-ex-task3-4}
\end{figure}

%\clearpage
\begin{minipage}{\linewidth}
	1) Исключим состояние $83$:	
		\begin{center}
\resizebox{!}{9.6cm}{%
		\begin{tikzpicture}[auto,>=stealth', node distance=2.5cm,auto,every state/.style={thick}]
		\node (init) {};
		\node[state] (11) [right=.7cm of init] {$11$};
		\node[state] (22) [above of=11] {$22$};
		\node[state] (42) [below of=11] {$42$};
		\node[state] (23) [right of=11] {$23$};
		\node[state] (43) [below of=23] {$43$};
		\node[state] (13) [right of=23] {$13$};
		\node[state] (33) [right of=13] {$33$};
		\node[state] (41) [below of=42] {$41$};
		\node[state] (53) [right of=43, below of=43] {$53$};
		\node[state] (63) [right of=53] {$63$};
		\node[state] (73) [right of=53, below of=53] {$73$};
		\node[state, accepting] (qf) [right of=43] {$q_f$};
		
		\path[->]
		(init) edge (11)
		(11) edge[bend right] node[right] {$a$} (22)
		(11) edge node {$a$} (23)
		(11) edge node[left] {$a$} (42)
		(11) edge node {$a$} (43)
		
		(22) edge[bend right] node[left] {$a$} (11)
		
		(23) edge node[below] {$a$} (13)
		
		(13) edge[bend right] node[above] {$a$} (23)
		(13) edge node {$a$} (43)
		(13) edge node[below]  {$b$}(33)
		
		(33) edge[bend right] node[above] {$b$} (13)
		
		(42) edge node {$a$} (41)
		(41) edge[bend left] node[left] {$a$} (42)
		(41) edge node[near end] {$a$} (43)
		
		(43) edge[loop below] node {$a$} (43)
		(43) edge node {$b$} (53)
		
		(53) edge node {$a$} (63)
		(53) edge node[near end] {$b$} (73)
		
		(63) edge[loop above] node {$a + ba$} (63)
		
		(73) edge node[right] {$a$} (63)
	
		(53) edge node {$\varepsilon$} (qf)
		(63) edge node {$\varepsilon$} (qf)
		
		;
		\end{tikzpicture}
}
	\end{center}
\end{minipage}
%

\begin{minipage}{\linewidth}
2) Исключим состояние $73$:
\begin{center}	
\resizebox{!}{9cm}{%
	\begin{tikzpicture}[auto,>=stealth', node distance=3cm,auto,every state/.style={thick}]
	\node (init) {};
	\node[state] (11) [right=.7cm of init] {$11$};
	\node[state] (22) [above of=11] {$22$};
	\node[state] (42) [below of=11] {$42$};
	\node[state] (23) [right of=11] {$23$};
	\node[state] (43) [below of=23] {$43$};
	\node[state] (13) [right of=23] {$13$};
	\node[state] (33) [right of=13] {$33$};
	\node[state] (41) [below of=42] {$41$};
	\node[state] (53) [right of=43, below of=43] {$53$};
	\node[state] (63) [right of=53] {$63$};
	\node[state, accepting] (qf) [right of=43] {$q_f$};
	
	\path[->]
	(init) edge (11)
	(11) edge[bend right] node[right] {$a$} (22)
	(11) edge node {$a$} (23)
	(11) edge node[left] {$a$} (42)
	(11) edge node {$a$} (43)
	
	(22) edge[bend right] node[left] {$a$} (11)
	
	(23) edge node[below] {$a$} (13)
	
	(13) edge[bend right] node[above] {$a$} (23)
	(13) edge node {$a$} (43)
	(13) edge node[below]  {$b$}(33)
	
	(33) edge[bend right] node[above] {$b$} (13)
	
	(42) edge node {$a$} (41)
	(41) edge[bend left] node[left] {$a$} (42)
	(41) edge node[near end] {$a$} (43)
	
	(43) edge[loop below] node {$a$} (43)
	(43) edge node {$b$} (53)
	
	(53) edge node {$a + ba$} (63)
	
	(63) edge[loop above] node {$a + ba$} (63)
	
	(53) edge node {$\varepsilon$} (qf)
	(63) edge node {$\varepsilon$} (qf)
	
	;
	\end{tikzpicture}
}
\end{center}
\end{minipage}

\begin{minipage}{\linewidth}
	3) Исключим состояние $22$:
\begin{center}	
	\begin{tikzpicture}[auto,>=stealth', node distance=3cm,auto,every state/.style={thick}]
	\node (init) {};
	\node[state] (11) [right=.7cm of init] {$11$};
	\node[state] (42) [below of=11] {$42$};
	\node[state] (23) [right of=11] {$23$};
	\node[state] (43) [below of=23] {$43$};
	\node[state] (13) [right of=23] {$13$};
	\node[state] (33) [right of=13] {$33$};
	\node[state] (41) [below of=42] {$41$};
	\node[state] (53) [right of=43, below of=43] {$53$};
	\node[state] (63) [right of=53] {$63$};
	\node[state, accepting] (qf) [right of=43] {$q_f$};
	
	\path[->]
	(init) edge (11)
	(11) edge[loop above] node {$aa$} (11)
	
	(11) edge node {$a$} (23)
	(11) edge node[left] {$a$} (42)
	(11) edge node {$a$} (43)
	
	(23) edge node[below] {$a$} (13)
	
	(13) edge[bend right] node[above] {$a$} (23)
	(13) edge node {$a$} (43)
	(13) edge node[below]  {$b$}(33)
	
	(33) edge[bend right] node[above] {$b$} (13)
	
	(42) edge node {$a$} (41)
	(41) edge[bend left] node[left] {$a$} (42)
	(41) edge node[near end] {$a$} (43)
	
	(43) edge[loop below] node {$a$} (43)
	(43) edge node {$b$} (53)
	
	(53) edge node {$a + ba$} (63)
	
	(63) edge[loop above] node {$a + ba$} (63)
	
	(53) edge node {$\varepsilon$} (qf)
	(63) edge node {$\varepsilon$} (qf)
	
	;
	\end{tikzpicture}
\end{center}
\end{minipage}

\begin{minipage}{\linewidth}
	4) Исключим состояние $33$:
	\begin{center}	
		\begin{tikzpicture}[auto,>=stealth', node distance=3cm,auto,every state/.style={thick}]
		\node (init) {};
		\node[state] (11) [right=.7cm of init] {$11$};
		\node[state] (42) [below of=11] {$42$};
		\node[state] (23) [right of=11] {$23$};
		\node[state] (43) [below of=23] {$43$};
		\node[state] (13) [right of=23] {$13$};
		\node[state] (41) [below of=42] {$41$};
		\node[state] (53) [right of=43, below of=43] {$53$};
		\node[state] (63) [right of=53] {$63$};
		\node[state, accepting] (qf) [right of=43] {$q_f$};
		
		\path[->]
		(init) edge (11)
		(11) edge[loop above] node {$aa$} (11)
		
		(11) edge node {$a$} (23)
		(11) edge node[left] {$a$} (42)
		(11) edge node {$a$} (43)
		
		(23) edge node[below] {$a$} (13)
		
		(13) edge[bend right] node[above] {$a$} (23)
		(13) edge node {$a$} (43)
		(13) edge[loop right] node {$bb$} (13)
		
		(42) edge node {$a$} (41)
		(41) edge[bend left] node[left] {$a$} (42)
		(41) edge node[near end] {$a$} (43)
		
		(43) edge[loop below] node {$a$} (43)
		(43) edge node {$b$} (53)
		
		(53) edge node {$a + ba$} (63)
		
		(63) edge[loop above] node {$a + ba$} (63)
		
		(53) edge node {$\varepsilon$} (qf)
		(63) edge node {$\varepsilon$} (qf)
		
		;
		\end{tikzpicture}
	\end{center}
\end{minipage}

\begin{minipage}{\linewidth}
	5) Исключим состояние $13$:
	\begin{center}	
		\begin{tikzpicture}[auto,>=stealth', node distance=3cm,auto,every state/.style={thick}]
		\node (init) {};
		\node[state] (11) [right=.7cm of init] {$11$};
		\node[state] (42) [below of=11] {$42$};
		\node[state] (23) [right of=11] {$23$};
		\node[state] (43) [below of=23] {$43$};
		\node[state] (41) [below of=42] {$41$};
		\node[state] (53) [right of=43, below of=43] {$53$};
		\node[state] (63) [right of=53] {$63$};
		\node[state, accepting] (qf) [right of=43] {$q_f$};
		
		\path[->]
		(init) edge (11)
		(11) edge[loop above] node {$aa$} (11)
		
		(11) edge node {$a$} (23)
		(11) edge node[left] {$a$} (42)
		(11) edge node {$a$} (43)
		
		(23) edge[loop above] node {$a(bb)^*a$} (23)
		
		(23) edge node[right] {$a(bb)^*a$} (43)
		
		(42) edge node {$a$} (41)
		(41) edge[bend left] node[left] {$a$} (42)
		(41) edge node[near end] {$a$} (43)
		
		(43) edge[loop below] node {$a$} (43)
		(43) edge node {$b$} (53)
		
		(53) edge node {$a + ba$} (63)
		
		(63) edge[loop above] node {$a + ba$} (63)
		
		(53) edge node {$\varepsilon$} (qf)
		(63) edge node {$\varepsilon$} (qf)
		
		;
		\end{tikzpicture}
	\end{center}
\end{minipage}

\begin{minipage}{\linewidth}
6) Исключим состояние $41$:
\begin{center}	
	\begin{tikzpicture}[auto,>=stealth', node distance=3cm,auto,every state/.style={thick}]
	\node (init) {};
	\node[state] (11) [right=.7cm of init] {$11$};
	\node[state] (42) [below of=11] {$42$};
	\node[state] (23) [right of=11] {$23$};
	\node[state] (43) [below of=23] {$43$};
	\node[state] (53) [right of=43, below of=43] {$53$};
	\node[state] (63) [right of=53] {$63$};
	\node[state, accepting] (qf) [right of=43] {$q_f$};
	
	\path[->]
	(init) edge (11)
	(11) edge[loop above] node {$aa$} (11)
	
	(11) edge node {$a$} (23)
	(11) edge node[left] {$a$} (42)
	(11) edge node {$a$} (43)
	
	(23) edge[loop above] node {$a(bb)^*a$} (23)
	
	(23) edge node[right] {$a(bb)^*a$} (43)
	
	(42) edge[loop left] node {$aa$} (42)
	(42) edge node {$aa$} (43)
	
	(43) edge[loop below] node {$a$} (43)
	(43) edge node {$b$} (53)
	
	(53) edge node {$a + ba$} (63)
	
	(63) edge[loop above] node {$a + ba$} (63)
	
	(53) edge node {$\varepsilon$} (qf)
	(63) edge node {$\varepsilon$} (qf)
	
	;
	\end{tikzpicture}
\end{center}
\end{minipage}

\begin{minipage}{\linewidth}
	7) Исключим состояние $42$:
\begin{center}	
	\begin{tikzpicture}[auto,>=stealth', node distance=3cm,auto,every state/.style={thick}]
	\node (init) {};
	\node[state] (11) [right=.7cm of init] {$11$};
	\node[state] (23) [right of=11] {$23$};
	\node[state] (43) [below of=23] {$43$};
	\node[state] (53) [right of=43, below of=43] {$53$};
	\node[state] (63) [right of=53] {$63$};
	\node[state, accepting] (qf) [right of=43] {$q_f$};
	
	\path[->]
	(init) edge (11)
	(11) edge[loop above] node {$aa$} (11)
	
	(11) edge node {$a$} (23)
	(11) edge node[left] {$a + a(aa)^*aa$} (43)
	
	(23) edge[loop above] node {$a(bb)^*a$} (23)
	
	(23) edge node[right] {$a(bb)^*a$} (43)
	
	
	(43) edge[loop below] node {$a$} (43)
	(43) edge node {$b$} (53)
	
	(53) edge node {$a + ba$} (63)
	
	(63) edge[loop above] node {$a + ba$} (63)
	
	(53) edge node {$\varepsilon$} (qf)
	(63) edge node {$\varepsilon$} (qf)
	
	;
	\end{tikzpicture}
\end{center}
\end{minipage}

\begin{minipage}{\linewidth}
8) Исключим состояние $23$:
\begin{center}	
	\begin{tikzpicture}[auto,>=stealth', node distance=3cm,auto,every state/.style={thick}]
	\node (init) {};
	\node[state] (11) [right=.7cm of init] {$11$};
	\node[state] (43) [below of=11] {$43$};
	\node[state] (53) [right of=43, below of=43] {$53$};
	\node[state] (63) [right of=53] {$63$};
	\node[state, accepting] (qf) [right of=43] {$q_f$};
	
	\path[->]
	(init) edge (11)
	(11) edge[loop above] node {$aa$} (11)
	
	(11) edge node[left] {$a + a(aa)^*aa + a(a(bb)^*a)^*a(bb)^*a$} (43)
	
	(43) edge[loop below] node {$a$} (43)
	(43) edge node {$b$} (53)
	
	(53) edge node {$a + ba$} (63)
	
	(63) edge[loop above] node {$a + ba$} (63)
	
	(53) edge node {$\varepsilon$} (qf)
	(63) edge node {$\varepsilon$} (qf)
	
	;
	\end{tikzpicture}
\end{center}
\end{minipage}

\begin{minipage}{\linewidth}
9) Исключим состояние $43$:
\begin{center}	
	\begin{tikzpicture}[auto,>=stealth', node distance=2.5cm,auto,every state/.style={thick}]
	\node (init) {};
	\node[state] (11) [right=.7cm of init] {$11$};
	\node[state] (53) [below of=11] {$53$};
	\node[state] (63) [right of=53] {$63$};
	\node[state, accepting] (qf) [below of=63] {$q_f$};
	
	\path[->]
	(init) edge (11)
	(11) edge[loop above] node {$aa$} (11)
	
	(11) edge node[left] {$(a + a(aa)^*aa + a(a(bb)^*a)^*a(bb)^*a)a^*b$} (53)
	
	
	(53) edge node {$a + ba$} (63)
	
	(63) edge[loop right] node {$a + ba$} (63)
	
	(53) edge node {$\varepsilon$} (qf)
	(63) edge node {$\varepsilon$} (qf)
	
	;
	\end{tikzpicture}
\end{center}
\end{minipage}

\begin{minipage}{\linewidth}
	10) Исключим состояние $63$:
    
\resizebox{\linewidth}{!}{%
		\begin{tikzpicture}[auto,>=stealth', node distance=7cm,auto,every state/.style={thick}]
		\node (init) {};
		\node[state] (11) [right=.7cm of init] {$11$};
		\node[state] (53) [right of=11] {$53$};
		\node[state, accepting] (qf) [right of=53] {$q_f$};
		
		\path[->]
		(init) edge (11)
		(11) edge[loop below] node {$aa$} (11)
		
		(11) edge[bend left] node {$(a + a(aa)^*aa + a(a(bb)^*a)^*a(bb)^*a)a^*b$} (53)
		
		(53) edge node {$\varepsilon + (a + ba)(a + ba)^*\varepsilon$} (qf)
		;
		\end{tikzpicture}
}
\end{minipage}

\begin{minipage}{\linewidth}
		11) Исключим состояние $53$:
        
\resizebox{\linewidth}{!}{%
		\begin{tikzpicture}[auto,>=stealth', node distance=15cm,auto,every state/.style={thick}]
		\node (init) {};
		\node[state] (11) [right=.7cm of init] {$11$};
		\node[state, accepting] (qf) [right of=11] {$q_f$};
		
		\path[->]
		(init) edge (11)
		(11) edge[loop above] node {$aa$} (11)
		(11) edge node {$(a + a(aa)^*aa + a(a(bb)^*a)^*a(bb)^*a)a^*b(\varepsilon + (a + ba)(a + ba)^*)$} (qf)
		;
		\end{tikzpicture}
}
\end{minipage}

Получаем:
\begin{multline*}
    Reg = (aa)^*(a + a(aa)^*aa + a(a(bb)^*a)^*a(bb)^*a)\cdot \\a^*b(\varepsilon + (a + ba)(a + ba)^*).
\end{multline*}
		
	\item Вычислить регулярное выражение, определяющее язык $L_1 \bigtriangleup L_2$.
    
    Напомним, что
	$
        L_1 \bigtriangleup L_2 = (L_1 \backslash L_2)\cup(L_2 \backslash L_1) = (L_1 \cap \overline{L_2}) \cup (L_2 \cap \overline{L_1})
    $.
    Результат непосредственного применения известных конструкций
    приводит к графу и таблице переходов показанных на рисунках
    \ref{app-ex-task3-5} и \ref{app-ex-task3-6} (с.~\pageref{app-ex-task3-6}).

\begin{figure}[h!]
\centering
		\begin{tikzpicture}[auto,>=stealth', node distance=2.3cm,auto,every state/.style={thick}]
		\node (init) {};
		\node[state] (11) [right=.7cm of init] {$11$};
		\node[state] (22) [above of=11] {$22$};
		\node[state] (42) [below of=11] {$42$};
		\node[state, accepting] (23) [right of=11] {$23$};
		\node[state, accepting] (43) [below of=23] {$43$};
		\node[state, accepting] (13) [right of=23] {$13$};
		\node[state, accepting] (33) [right of=13] {$33$};
		\node[state] (34) [below of=33] {$34$};
		\node[state] (14) [right of=33] {$14$};	
		\node[state, accepting] (15) [right of=34] {$15$};
		\node[state, accepting] (35) [right of=14] {$35$};
		\node[state] (36) [right of=15] {$36$};
		\node[state] (16) [right of=35] {$16$};
		\node[state] (41) [below of=42] {$41$};
		\node[state, accepting] (54) [right of=43, below of=43] {$54$};
		\node[state] (53) [below of=54] {$53$};
		\node[state, accepting] (75) [right of=54] {$75$};
		\node[state] (63) [right of=53] {$63$};
		\node[state, accepting] (73) [right of=53, below of=53] {$73$};
		\node[state] (74) [below of=53] {$74$};
		\node[state] (84) [right of=63, above of=63] {$84$};
		\node[state, accepting] (83) [right of=84, below of=84] {$83$};
		
		\path[->]
		(init) edge (11)
		(11) edge node[right] {$a$} (22)
		(11) edge node {$a$} (23)
		(11) edge node[left] {$a$} (42)
		(11) edge node {$a$} (43)
		
		(22) edge[bend right] node[left] {$a$} (11)
		
		(23) edge node[below] {$a$} (13)
		
		(13) edge[bend right] node[above] {$a$} (23)
		(13) edge node {$a$} (43)
		(13) edge node[below]  {$b$}(33)
		(13) edge node[near end] {$b$} (34)
		
		(33) edge[bend right] node[above] {$b$} (13)
		(33) edge node {$b$} (14)
		(34) edge node {$b$} (15)
		
		(14) edge node {$b$} (35)
		(15) edge node[below] {$b$} (36)
		
		(35) edge node[below] {$b$} (16)
		(36) edge[bend right] node[above] {$b$} (15)
		(16) edge[bend right] node[above] {$b$} (35)
		
		(42) edge node {$a$} (41)
		(41) edge[bend left] node[left] {$a$} (42)
		(41) edge node[near end] {$a$} (43)
		
		(43) edge[loop below] node {$a$} (43)
		(43) edge[bend left] node[near end] {$b$} (54)
		(43) edge node[near end, left] {$b$} (53)
		
		(54) edge node {$b$} (75)
		(53) edge node {$a$} (63)
		(53) edge node[near end] {$b$} (73)
		(53) edge node[left] {$b$} (74)
		
		(63) edge[loop above] node {$a$} (63)
		(63) edge node {$b$} (83)
		(63) edge node {$b$} (84)
		
		(73) edge node[right] {$a$} (63)
		(83) edge[bend left] node {$a$} (63)
		;
		\end{tikzpicture}
\caption{}\label{app-ex-task3-5}
\end{figure}

%\afterpage

\begin{figure}[h!]
\centering
		\begin{tabular}{lll}
			\toprule
			\multicolumn{1}{c}{\multirow{2}{*}{\Large $\delta$}}
			& \multicolumn{2}{c}{Вход} \\
			\cmidrule(rl){2-3}
			& \multicolumn{1}{c}{$a$}
			& \multicolumn{1}{c}{$b$}  \\
			\midrule
			$\to \{1, 1\}$       & $\{2, 2\}, \{2, 3\}, \{4, 2\}, \{4, 3\}$     		 & $\{\varnothing\}$      \\
			$\{2, 2\}$       & $\{1, 1\}$    			 & $\{\varnothing\}$      \\
			$\textbf{\{2, 3\}}$       & $\{1, 3\}$    			 & $\{\varnothing\}$      \\
			$\textbf{\{1, 3\}}$       & $\{2, 3\}, \{4, 3\}$     &  $\{3, 3\}, \{3, 4\}$  \\
			$\textbf{\{3, 3\}}$       & $\{\varnothing\}$     &  $\{1, 3\}, \{1, 4\}$  \\
			$\{3, 4\}$       & $\{\varnothing\}$     &  $\{1, 5\}$  \\
			$\{1, 4\}$       & $\{\varnothing\}$     &  $\{3, 5\}$  \\
			$\textbf{\{1, 5\}}$       & $\{\varnothing\}$     &  $\{3, 6\}$  \\
			$\textbf{\{3, 5\}}$       & $\{\varnothing\}$     &  $\{1, 6\}$  \\
			$\{3, 6\}$       & $\{\varnothing\}$     &  $\{1, 5\}$  \\
			$\{1, 6\}$       & $\{\varnothing\}$     &  $\{3, 5\}$  \\
			$\{4, 2\}$       & $\{4, 1\}$    			 & $\{\varnothing\}$      \\
			$\textbf{\{4, 3\}}$       & $\{4, 3\}$    			 & $\{5, 3\}, \{5, 4\}$      \\
			$\{4, 1\}$       & $\{4, 2\}, \{4, 3\}$    			 & $\{\varnothing\}$      \\
			$\{5, 3\}$       & $\{6, 3\}$    			 & $\{7, 3\}, \{7, 4\}$      \\
			$\textbf{\{5, 4\}}$       & $\{\varnothing\}$    			 & $\{7, 5\}$      \\
			$\{6, 3\}$       & $\{6, 3\}$    			 & $\{8, 3\}, \{8, 4\}$      \\
			$\textbf{\{7, 3\}}$       & $\{6, 3\}$    			 & $\{\varnothing\}$      \\
			$\{7, 4\}$       & $\{\varnothing\}$    			 & $\{\varnothing\}$      \\
			$\textbf{\{7, 5\}}$       & $\{\varnothing\}$    			 & $\{\varnothing\}$      \\
			$\textbf{\{8, 3\}}$       & $\{6, 3\}$    			 & $\{\varnothing\}$      \\
			$\{8, 4\}$       & $\{\varnothing\}$    			 & $\{\varnothing\}$      \\
			\bottomrule
		\end{tabular}
\caption{}\label{app-ex-task3-6}
\end{figure}
\clearpage

Состояния 74, 84 являются непродуктивными, поэтому их можно удалить~--- рис.~\ref{app-ex-task3-7}.

\begin{figure}[h!]
\centering
\resizebox{\linewidth}{!}{%
		\begin{tikzpicture}[auto,>=stealth', node distance=2.5cm,auto,every state/.style={thick}]
		\node (init) {};
		\node[state] (11) [right=.7cm of init] {$11$};
		\node[state] (22) [above of=11] {$22$};
		\node[state] (42) [below of=11] {$42$};
		\node[state, accepting] (23) [right of=11] {$23$};
		\node[state, accepting] (43) [below of=23] {$43$};
		\node[state, accepting] (13) [right of=23] {$13$};
		\node[state, accepting] (33) [right of=13] {$33$};
		\node[state] (34) [below of=33] {$34$};
		\node[state] (14) [right of=33] {$14$};	
		\node[state, accepting] (15) [right of=34] {$15$};
		\node[state, accepting] (35) [right of=14] {$35$};
		\node[state] (36) [right of=15] {$36$};
		\node[state] (16) [right of=35] {$16$};
		\node[state] (41) [below of=42] {$41$};
		\node[state, accepting] (54) [right of=43] {$54$};
		\node[state] (53) [below of=43] {$53$};
		\node[state, accepting] (75) [below of=34] {$75$};
		\node[state] (63) [right of=53] {$63$};
		\node[state, accepting] (73) [below of=53] {$73$};
		\node[state, accepting] (83) [below of=75] {$83$};
		
		\path[->]
		(init) edge (11)
		(11) edge node[right] {$a$} (22)
		(11) edge node {$a$} (23)
		(11) edge node[left] {$a$} (42)
		(11) edge node {$a$} (43)
		
		(22) edge[bend right] node[left] {$a$} (11)
		
		(23) edge node[below] {$a$} (13)
		
		(13) edge[bend right] node[above] {$a$} (23)
		(13) edge node {$a$} (43)
		(13) edge node[below]  {$b$}(33)
		(13) edge node[near end] {$b$} (34)
		
		(33) edge[bend right] node[above] {$b$} (13)
		(33) edge node {$b$} (14)
		(34) edge node {$b$} (15)
		
		(14) edge node {$b$} (35)
		(15) edge node[below] {$b$} (36)
		
		(35) edge node[below] {$b$} (16)
		(36) edge[bend right] node[above] {$b$} (15)
		(16) edge[bend right] node[above] {$b$} (35)
		
		(42) edge node {$a$} (41)
		(41) edge[bend left] node[left] {$a$} (42)
		(41) edge node[near end] {$a$} (43)
		
		(43) edge[loop left] node {$a$} (43)
		(43) edge node[near end] {$b$} (54)
		(43) edge node[near end, left] {$b$} (53)
		
		(54) edge node {$b$} (75)
		(53) edge node {$a$} (63)
		(53) edge node[near end] {$b$} (73)
		
		(63) edge[loop above] node {$a$} (63)
		(63) edge node {$b$} (83)
		
		(73) edge node[right] {$a$} (63)
		(83) edge[bend left] node {$a$} (63)
		;
		\end{tikzpicture}
}
\caption{}\label{app-ex-task3-7}
\end{figure}

	Воспользуемся методом последовательного исключения состояний. Пустим $\varepsilon$-переходы из всех допускающих состояний в новое состояние $q_f$. Все допускающие состояния сделаем недопускающими, а новое состояние $q_f$ --- допускающим. Результат приведен на рис.~~\ref{app-ex-task3-8} (с.~\pageref{app-ex-task3-8}).

\begin{figure}[h!]
\centering
		\begin{tikzpicture}[auto,>=stealth', node distance=2.5cm,auto,every state/.style={thick}]
		\node (init) {};
		\node[state] (11) [right=.7cm of init] {$11$};
		\node[state] (22) [above of=11] {$22$};
		\node[state] (42) [below of=11] {$42$};
		\node[state] (23) [right of=11] {$23$};
		\node[state] (43) [below of=23] {$43$};
		\node[state] (13) [right of=23] {$13$};
		\node[state] (33) [right of=13] {$33$};
		\node[state] (34) [below of=33] {$34$};
		\node[state] (14) [right of=33, above of=33] {$14$};	
		\node[state] (15) [right of=34] {$15$};
		\node[state] (35) [right of=14, below of=14] {$35$};
		\node[state] (36) [above of=15] {$36$};
		\node[state] (16) [above of=35] {$16$};
		\node[state] (41) [below of=42] {$41$};
		\node[state] (54) [below of=43] {$54$};
		\node[state] (53) [below of=41, left of=41] {$53$};
		\node[state] (75) [right of=54] {$75$};
		\node[state] (63) [right of=53, below of=53] {$63$};
		\node[state] (73) [above of=63, right of=63] {$73$};
		\node[state] (83) [right of=63] {$83$};
		\node[state, accepting] (qf) [right of=75] {$q_f$};
		
		\path[->]
		(init) edge (11)
		(11) edge node[right] {$a$} (22)
		(11) edge node {$a$} (23)
		(11) edge node[left] {$a$} (42)
		(11) edge node {$a$} (43)
		
		(22) edge[bend right] node[left] {$a$} (11)
		
		(23) edge node[below] {$a$} (13)
		
		(13) edge[bend right] node[above] {$a$} (23)
		(13) edge node[near end] {$a$} (43)
		(13) edge node[below]  {$b$}(33)
		(13) edge node[near end] {$b$} (34)
		
		(33) edge[bend right] node[above] {$b$} (13)
		(33) edge node {$b$} (14)
		(34) edge node {$b$} (15)
		
		(14) edge node[near start] {$b$} (35)
		(15) edge node[right] {$b$} (36)
		
		(35) edge node[right] {$b$} (16)
		(36) edge[bend right] node[left] {$b$} (15)
		(16) edge[bend right] node[left] {$b$} (35)
		
		(42) edge node {$a$} (41)
		(41) edge[bend left] node[left] {$a$} (42)
		(41) edge node {$a$} (43)
		
		(43) edge[loop left] node {$a$} (43)
		(43) edge node[near end] {$b$} (54)
		(43) edge[bend left] node[left] {$b$} (53)
		
		(54) edge node {$b$} (75)
		(53) edge node {$a$} (63)
		(53) edge node[near end] {$b$} (73)
		
		(63) edge[loop below] node {$a$} (63)
		(63) edge node {$b$} (83)
		
		(73) edge node[right] {$a$} (63)
		(83) edge[bend left] node {$a$} (63)
		
		(75) edge node {$\varepsilon$} (qf)
		(54) edge[bend right] node[below] {$\varepsilon$} (qf)
		(73) edge[bend right] node[below] {$\varepsilon$} (qf)
		(83) edge[bend right] node[below] {$\varepsilon$} (qf)
		(15) edge node[right] {$\varepsilon$} (qf)
		(43) edge node {$\varepsilon$} (qf)
		(13) edge node {$\varepsilon$} (qf)
		(33) edge[bend left] node[near start, left] {$\varepsilon$} (qf)
		(35) edge[bend left] node {$\varepsilon$} (qf)
		(23) edge node {$\varepsilon$} (qf)
		;
		\end{tikzpicture}
\caption{}\label{app-ex-task3-8}
\end{figure}
\clearpage

\begin{minipage}{\linewidth}
	1) Исключим состояния $22$, $36$ и $16$: 		
	\begin{center}	
		\begin{tikzpicture}[auto,>=stealth', node distance=2.3cm,auto,every state/.style={thick}]
		\node (init) {};
		\node[state] (11) [right=.7cm of init] {$11$};
		\node[state] (42) [below of=11] {$42$};
		\node[state] (23) [right of=11] {$23$};
		\node[state] (43) [below of=23] {$43$};
		\node[state] (13) [right of=23] {$13$};
		\node[state] (33) [right of=13] {$33$};
		\node[state] (34) [below of=33] {$34$};
		\node[state] (14) [right of=33] {$14$};	
		\node[state] (15) [right of=34] {$15$};
		\node[state] (35) [right of=14] {$35$};
		\node[state] (41) [below of=42] {$41$};
		\node[state] (54) [below of=43] {$54$};
		\node[state] (53) [below of=41, left of=41] {$53$};
		\node[state] (75) [right of=54] {$75$};
		\node[state] (63) [right of=53, below of=53] {$63$};
		\node[state] (73) [above of=63, right of=63] {$73$};
		\node[state] (83) [right of=63] {$83$};
		\node[state, accepting] (qf) [right of=75] {$q_f$};
		
		\path[->]
		(init) edge (11)
		(11) edge[loop above] node {$aa$} (11)
		(11) edge node {$a$} (23)
		(11) edge node[left] {$a$} (42)
		(11) edge node {$a$} (43)
		
		(23) edge node[below] {$a$} (13)
		
		(13) edge[bend right] node[above] {$a$} (23)
		(13) edge node[near end] {$a$} (43)
		(13) edge node[below]  {$b$}(33)
		(13) edge node[near end] {$b$} (34)
		
		(33) edge[bend right] node[above] {$b$} (13)
		(33) edge node {$b$} (14)
		(34) edge node {$b$} (15)
		
		(14) edge node[near start] {$b$} (35)
		(15) edge[loop above] node [right] {$bb$} (15)
		
		(35) edge[loop above] node {$bb$} (35)
		
		(42) edge node {$a$} (41)
		(41) edge[bend left] node[left] {$a$} (42)
		(41) edge node {$a$} (43)
		
		(43) edge[loop left] node [above] {$a$} (43)
		(43) edge node[near end] {$b$} (54)
		(43) edge[bend left] node[left] {$b$} (53)
		
		(54) edge node {$b$} (75)
		(53) edge node {$a$} (63)
		(53) edge node[near end] {$b$} (73)
		
		(63) edge[loop below] node {$a$} (63)
		(63) edge node {$b$} (83)
		
		(73) edge node[right] {$a$} (63)
		(83) edge[bend left] node {$a$} (63)
		
		(75) edge node {$\varepsilon$} (qf)
		(54) edge[bend right] node[below] {$\varepsilon$} (qf)
		(73) edge[bend right] node[below] {$\varepsilon$} (qf)
		(83) edge[bend right] node[below] {$\varepsilon$} (qf)
		(15) edge node[right] {$\varepsilon$} (qf)
		(43) edge node {$\varepsilon$} (qf)
		(13) edge node {$\varepsilon$} (qf)
		(33) edge[bend left] node[near start, left] {$\varepsilon$} (qf)
		(35) edge[bend left] node {$\varepsilon$} (qf)
		(23) edge node {$\varepsilon$} (qf)
		;
		\end{tikzpicture}
	\end{center}
\end{minipage}

\begin{minipage}{\linewidth}
	2) Исключим состояние $75$: 		
\begin{center}	
\resizebox{!}{9.3cm}{%
	\begin{tikzpicture}[auto,>=stealth', node distance=2.4cm,auto,every state/.style={thick}]
	\node (init) {};
	\node[state] (11) [right=.7cm of init] {$11$};
	\node[state] (42) [below of=11] {$42$};
	\node[state] (23) [right of=11] {$23$};
	\node[state] (43) [below of=23] {$43$};
	\node[state] (13) [right of=23] {$13$};
	\node[state] (33) [right of=13] {$33$};
	\node[state] (34) [below of=33] {$34$};
	\node[state] (14) [right of=33] {$14$};	
	\node[state] (15) [right of=34] {$15$};
	\node[state] (35) [right of=14] {$35$};
	\node[state] (41) [below of=42] {$41$};
	\node[state] (54) [below of=43] {$54$};
	\node[state] (53) [below of=41, left of=41] {$53$};
	\node[state] (73) [below of=54] {$73$};
	\node[state] (63) [right of=73] {$63$};
	\node[state] (83) [right of=63] {$83$};
	\node[state, accepting] (qf) [below of=34] {$q_f$};
	
	\path[->]
	(init) edge (11)
	(11) edge[loop above] node {$aa$} (11)
	(11) edge node {$a$} (23)
	(11) edge node[left] {$a$} (42)
	(11) edge node {$a$} (43)
	
	(23) edge node[below] {$a$} (13)
	
	(13) edge[bend right] node[above] {$a$} (23)
	(13) edge node[near end] {$a$} (43)
	(13) edge node[below]  {$b$}(33)
	(13) edge node[near end] {$b$} (34)
	
	(33) edge[bend right] node[above] {$b$} (13)
	(33) edge node {$b$} (14)
	(34) edge node {$b$} (15)
	
	(14) edge node[near start] {$b$} (35)
	(15) edge[loop above] node  [right] {$bb$} (15)
	
	(35) edge[loop above] node {$bb$} (35)
	
	(42) edge node {$a$} (41)
	(41) edge[bend left] node[left] {$a$} (42)
	(41) edge node {$a$} (43)
	
	(43) edge[loop left] node  [above] {$a$} (43)
	(43) edge node[near end] {$b$} (54)
	(43) edge[bend left] node[left] {$b$} (53)
	
	(53) edge[bend right] node {$a$} (63)
	(53) edge node[near end] {$b$} (73)
	
	(63) edge[loop below] node {$a$} (63)
	(63) edge node {$b$} (83)
	
	(73) edge node {$a$} (63)
	(83) edge[bend left] node {$a$} (63)
	
	(54) edge node[above] {$\varepsilon + b\varepsilon$} (qf)
	(73) edge node {$\varepsilon$} (qf)
	(83) edge node[right] {$\varepsilon$} (qf)
	(15) edge node[right] {$\varepsilon$} (qf)
	(43) edge node {$\varepsilon$} (qf)
	(13) edge node {$\varepsilon$} (qf)
	(33) edge[bend left] node[near start, left] {$\varepsilon$} (qf)
	(35) edge[bend left] node {$\varepsilon$} (qf)
	(23) edge node {$\varepsilon$} (qf)
	;
	\end{tikzpicture}
}
\end{center}
\end{minipage}

\begin{minipage}{\linewidth}
	3) Исключим состояние $83$: 		
\begin{center}	
\resizebox{!}{9.3cm}{%
	\begin{tikzpicture}[auto,>=stealth', node distance=2.4cm,auto,every state/.style={thick}]
	\node (init) {};
	\node[state] (11) [right=.7cm of init] {$11$};
	\node[state] (42) [below of=11] {$42$};
	\node[state] (23) [right of=11] {$23$};
	\node[state] (43) [below of=23] {$43$};
	\node[state] (13) [right of=23] {$13$};
	\node[state] (33) [right of=13] {$33$};
	\node[state] (34) [below of=33] {$34$};
	\node[state] (14) [right of=33] {$14$};	
	\node[state] (15) [right of=34] {$15$};
	\node[state] (35) [right of=14] {$35$};
	\node[state] (41) [below of=42] {$41$};
	\node[state] (54) [below of=43] {$54$};
	\node[state] (53) [below of=41, left of=41] {$53$};
	\node[state] (73) [below of=54] {$73$};
	\node[state] (63) [right of=73] {$63$};
	\node[state, accepting] (qf) [below of=34] {$q_f$};
	
	\path[->]
	(init) edge (11)
	(11) edge[loop above] node {$aa$} (11)
	(11) edge node {$a$} (23)
	(11) edge node[left] {$a$} (42)
	(11) edge node {$a$} (43)
	
	(23) edge node[below] {$a$} (13)
	
	(13) edge[bend right] node[above] {$a$} (23)
	(13) edge node[near end] {$a$} (43)
	(13) edge node[below]  {$b$}(33)
	(13) edge node[near end] {$b$} (34)
	
	(33) edge[bend right] node[above] {$b$} (13)
	(33) edge node {$b$} (14)
	(34) edge node {$b$} (15)
	
	(14) edge node[near start] {$b$} (35)
	(15) edge[loop above] node  [right] {$bb$} (15)
	
	(35) edge[loop above] node {$bb$} (35)
	
	(42) edge node {$a$} (41)
	(41) edge[bend left] node[left] {$a$} (42)
	(41) edge node {$a$} (43)
	
	(43) edge[loop left] node  [above] {$a$} (43)
	(43) edge node[near end] {$b$} (54)
	(43) edge[bend left] node[left] {$b$} (53)
	
	(53) edge[bend right] node {$a$} (63)
	(53) edge node[near end] {$b$} (73)
	
	(63) edge[loop below] node {$a + ba$} (63)
	
	(73) edge node {$a$} (63)
	
	(54) edge node[above] {$\varepsilon + b\varepsilon$} (qf)
	(73) edge node {$\varepsilon$} (qf)
	(63) edge[bend right] node[right] {$b\varepsilon$} (qf)
	(15) edge node[right] {$\varepsilon$} (qf)
	(43) edge node {$\varepsilon$} (qf)
	(13) edge node {$\varepsilon$} (qf)
	(33) edge[bend left] node[near start, left] {$\varepsilon$} (qf)
	(35) edge[bend left] node {$\varepsilon$} (qf)
	(23) edge node {$\varepsilon$} (qf)
	;
	\end{tikzpicture}
}
\end{center}
\end{minipage}

\begin{minipage}{\linewidth}
4) Исключим состояние $35$: 
	\begin{center}	
\resizebox{!}{9.3cm}{%
		\begin{tikzpicture}[auto,>=stealth', node distance=2.4cm,auto,every state/.style={thick}]
		\node (init) {};
		\node[state] (11) [right=.7cm of init] {$11$};
		\node[state] (42) [below of=11] {$42$};
		\node[state] (23) [right of=11] {$23$};
		\node[state] (43) [below of=23] {$43$};
		\node[state] (13) [right of=23] {$13$};
		\node[state] (33) [right of=13] {$33$};
		\node[state] (34) [below of=33] {$34$};
		\node[state] (15) [right of=34] {$15$};
		\node[state] (14) [right of=15, above of=15] {$14$};		
		\node[state] (41) [below of=42] {$41$};
		\node[state] (54) [below of=43] {$54$};
		\node[state] (53) [below of=41, left of=41] {$53$};
		\node[state] (73) [below of=54] {$73$};
		\node[state] (63) [right of=73] {$63$};
		\node[state, accepting] (qf) [below of=34] {$q_f$};
		
		\path[->]
		(init) edge (11)
		(11) edge[loop above] node {$aa$} (11)
		(11) edge node {$a$} (23)
		(11) edge node[left] {$a$} (42)
		(11) edge node {$a$} (43)
		
		(23) edge node[below] {$a$} (13)
		
		(13) edge[bend right] node[above] {$a$} (23)
		(13) edge node[near end] {$a$} (43)
		(13) edge node[below]  {$b$}(33)
		(13) edge node[near end] {$b$} (34)
		
		(33) edge[bend right] node[above] {$b$} (13)
		(33) edge node {$b$} (14)
		(34) edge node {$b$} (15)
		
		(15) edge[loop above] node {$bb$} (15)
		
		(42) edge node {$a$} (41)
		(41) edge[bend left] node[left] {$a$} (42)
		(41) edge node {$a$} (43)
		
		(43) edge[loop left] node [above] {$a$} (43)
		(43) edge node[near end] {$b$} (54)
		(43) edge[bend left] node[left] {$b$} (53)
		
		(53) edge[bend right] node {$a$} (63)
		(53) edge node[near end] {$b$} (73)
		
		(63) edge[loop below] node {$a + ba$} (63)
		
		(73) edge node {$a$} (63)
		
		(54) edge node[above] {$\varepsilon + b$} (qf)
		(73) edge node {$\varepsilon$} (qf)
		(63) edge[bend right] node[right] {$b$} (qf)
		(15) edge node[right] {$\varepsilon$} (qf)
		(43) edge node {$\varepsilon$} (qf)
		(13) edge node {$\varepsilon$} (qf)
		(33) edge[bend left] node[near start, left] {$\varepsilon$} (qf)
		(14) edge[bend left] node {$b(bb)^*\varepsilon$} (qf)
		(23) edge node {$\varepsilon$} (qf)
		;
		\end{tikzpicture}
}
	\end{center}
\end{minipage}

\begin{minipage}{\linewidth}
5) Исключим состояние $41$:
	\begin{center}	
\resizebox{!}{9.3cm}{%
		\begin{tikzpicture}[auto,>=stealth', node distance=2.4cm,auto,every state/.style={thick}]
		\node (init) {};
		\node[state] (11) [right=.7cm of init] {$11$};
		\node[state] (42) [below of=11] {$42$};
		\node[state] (23) [right of=11] {$23$};
		\node[state] (43) [below of=23] {$43$};
		\node[state] (13) [right of=23] {$13$};
		\node[state] (33) [right of=13] {$33$};
		\node[state] (34) [below of=33] {$34$};
		\node[state] (15) [right of=34] {$15$};
		\node[state] (14) [right of=15, above of=15] {$14$};		
		\node[state] (54) [below of=43] {$54$};
		\node[state] (73) [below of=54] {$73$};
		\node[state] (53) [left of=73] {$53$};
		\node[state] (63) [right of=73] {$63$};
		\node[state, accepting] (qf) [below of=34] {$q_f$};
		
		\path[->]
		(init) edge (11)
		(11) edge[loop above] node {$aa$} (11)
		(11) edge node {$a$} (23)
		(11) edge node[left] {$a$} (42)
		(11) edge node {$a$} (43)
		
		(23) edge node[below] {$a$} (13)
		
		(13) edge[bend right] node[above] {$a$} (23)
		(13) edge node[near end] {$a$} (43)
		(13) edge node[below]  {$b$}(33)
		(13) edge node[near end] {$b$} (34)
		
		(33) edge[bend right] node[above] {$b$} (13)
		(33) edge node {$b$} (14)
		(34) edge node {$b$} (15)
		
		(15) edge[loop above] node {$bb$} (15)
		
		(42) edge[loop left] node {$aa$} (42)
		(42) edge node[above] {$aa$} (43)
		
		(43) edge[loop above] node [right] {$a$} (43)
		(43) edge node[near end] {$b$} (54)
		(43) edge[bend right] node[left] {$b$} (53)
		
		(53) edge[bend right] node[below] {$a$} (63)
		(53) edge node[near end] {$b$} (73)
		
		(63) edge[loop below] node {$a + ba$} (63)
		
		(73) edge node {$a$} (63)
		
		(54) edge node[above] {$\varepsilon + b$} (qf)
		(73) edge node {$\varepsilon$} (qf)
		(63) edge[bend right] node[right] {$b$} (qf)
		(15) edge node[right] {$\varepsilon$} (qf)
		(43) edge node {$\varepsilon$} (qf)
		(13) edge node {$\varepsilon$} (qf)
		(33) edge[bend left] node[near start, left] {$\varepsilon$} (qf)
		(14) edge[bend left] node {$b(bb)^*\varepsilon$} (qf)
		(23) edge node {$\varepsilon$} (qf)
		;
		\end{tikzpicture}
}
	\end{center}
\end{minipage}

\begin{minipage}{\linewidth}
6) Исключим состояние $15$: 		
	\begin{center}	
\resizebox{!}{9.3cm}{%
		\begin{tikzpicture}[auto,>=stealth', node distance=2.4cm,auto,every state/.style={thick}]
		\node (init) {};
		\node[state] (11) [right=.7cm of init] {$11$};
		\node[state] (42) [below of=11] {$42$};
		\node[state] (23) [right of=11] {$23$};
		\node[state] (43) [below of=23] {$43$};
		\node[state] (13) [right of=23] {$13$};
		\node[state] (34) [right of=13, below of=13] {$34$};
		\node[state] (33) [right of=34, above of=34] {$33$};
		\node[state] (14) [right of=33] {$14$};		
		\node[state] (54) [below of=43] {$54$};
		\node[state] (73) [below of=54] {$73$};
		\node[state] (53) [left of=73] {$53$};
		\node[state] (63) [right of=73] {$63$};
		\node[state, accepting] (qf) [below of=34] {$q_f$};
		
		\path[->]
		(init) edge (11)
		(11) edge[loop above] node {$aa$} (11)
		(11) edge node {$a$} (23)
		(11) edge node[left] {$a$} (42)
		(11) edge node {$a$} (43)
		
		(23) edge node[below] {$a$} (13)
		
		(13) edge[bend right] node[above] {$a$} (23)
		(13) edge node[near end] {$a$} (43)
		(13) edge node[below]  {$b$}(33)
		(13) edge node[near end] {$b$} (34)
		
		(33) edge[bend right] node[above] {$b$} (13)
		(33) edge node {$b$} (14)
		
		(42) edge[loop left] node {$aa$} (42)
		(42) edge node[above] {$aa$} (43)
		
		(43) edge[loop above] node {$a$} (43)
		(43) edge node[near end] {$b$} (54)
		(43) edge[bend right] node[left] {$b$} (53)
		
		(53) edge[bend right] node[below] {$a$} (63)
		(53) edge node[near end] {$b$} (73)
		
		(63) edge[loop below] node {$a + ba$} (63)
		
		(73) edge node {$a$} (63)
		
		(54) edge node[above] {$\varepsilon + b$} (qf)
		(73) edge node {$\varepsilon$} (qf)
		(63) edge[bend right] node[right] {$b$} (qf)
		(34) edge node[right] {$b(bb)^*\varepsilon$} (qf)
		(43) edge node {$\varepsilon$} (qf)
		(13) edge node {$\varepsilon$} (qf)
		(33) edge[bend left] node[near start, right] {$\varepsilon$} (qf)
		(14) edge[bend left] node {$b(bb)^*$} (qf)
		(23) edge node {$\varepsilon$} (qf)
		;
		\end{tikzpicture}
}
	\end{center}
\end{minipage}

\begin{minipage}{\linewidth}
	7) Исключим состояние $34$: 		
\begin{center}	
\resizebox{!}{9.3cm}{%
	\begin{tikzpicture}[auto,>=stealth', node distance=2.4cm,auto,every state/.style={thick}]
	\node (init) {};
	\node[state] (11) [right=.7cm of init] {$11$};
	\node[state] (42) [below of=11] {$42$};
	\node[state] (23) [right of=11] {$23$};
	\node[state] (43) [below of=23] {$43$};
	\node[state] (13) [right of=23] {$13$};
	\node[state] (33) [right of=13] {$33$};
	\node[state] (14) [right of=33] {$14$};		
	\node[state] (54) [below of=43] {$54$};
	\node[state] (73) [below of=54] {$73$};
	\node[state] (53) [left of=73] {$53$};
	\node[state] (63) [right of=73] {$63$};
	\node[state, accepting] (qf) [right of=54] {$q_f$};
	
	\path[->]
	(init) edge (11)
	(11) edge[loop above] node {$aa$} (11)
	(11) edge node {$a$} (23)
	(11) edge node[left] {$a$} (42)
	(11) edge node {$a$} (43)
	
	(23) edge node[below] {$a$} (13)
	
	(13) edge[bend right] node[above] {$a$} (23)
	(13) edge node[near start] {$a$} (43)
	(13) edge node[below]  {$b$}(33)
	
	(33) edge[bend right] node[above] {$b$} (13)
	(33) edge node {$b$} (14)
	
	(42) edge[loop left] node {$aa$} (42)
	(42) edge node[above] {$aa$} (43)
	
	(43) edge[loop above] node {$a$} (43)
	(43) edge node[near end] {$b$} (54)
	(43) edge[bend right] node[left] {$b$} (53)
	
	(53) edge[bend right] node[below] {$a$} (63)
	(53) edge node[near end] {$b$} (73)
	
	(63) edge[loop below] node {$a + ba$} (63)
	
	(73) edge node {$a$} (63)
	
	(54) edge node[above] {$\varepsilon + b$} (qf)
	(73) edge node {$\varepsilon$} (qf)
	(63) edge[bend right] node[right] {$b$} (qf)
	(43) edge node {$\varepsilon$} (qf)
	(13) edge node {$\varepsilon + bb(bb)^*$} (qf)
	(33) edge[bend left] node[near start, right] {$\varepsilon$} (qf)
	(14) edge[bend left] node {$b(bb)^*$} (qf)
	(23) edge node {$\varepsilon$} (qf)
	;
	\end{tikzpicture}
}
\end{center}
\end{minipage}

\begin{minipage}{\linewidth}
	8) Исключим состояние $14$: 		
\begin{center}	
\resizebox{!}{9.3cm}{%
	\begin{tikzpicture}[auto,>=stealth', node distance=2.4cm,auto,every state/.style={thick}]
	\node (init) {};
	\node[state] (11) [right=.7cm of init] {$11$};
	\node[state] (42) [below of=11] {$42$};
	\node[state] (23) [right of=11] {$23$};
	\node[state] (43) [below of=23] {$43$};
	\node[state] (13) [right of=23] {$13$};
	\node[state] (33) [right of=13] {$33$};	
	\node[state] (54) [below of=43] {$54$};
	\node[state] (73) [below of=54] {$73$};
	\node[state] (53) [left of=73] {$53$};
	\node[state] (63) [right of=73] {$63$};
	\node[state, accepting] (qf) [right of=54] {$q_f$};
	
	\path[->]
	(init) edge (11)
	(11) edge[loop above] node {$aa$} (11)
	(11) edge node {$a$} (23)
	(11) edge node[left] {$a$} (42)
	(11) edge node {$a$} (43)
	
	(23) edge node[below] {$a$} (13)
	
	(13) edge[bend right] node[above] {$a$} (23)
	(13) edge node[near start] {$a$} (43)
	(13) edge node[below]  {$b$}(33)
	
	(33) edge[bend right] node[above] {$b$} (13)
	
	(42) edge[loop left] node {$aa$} (42)
	(42) edge node[above] {$aa$} (43)
	
	(43) edge[loop above] node {$a$} (43)
	(43) edge node[near end] {$b$} (54)
	(43) edge[bend right] node[left] {$b$} (53)
	
	(53) edge[bend right] node[below] {$a$} (63)
	(53) edge node[near end] {$b$} (73)
	
	(63) edge[loop below] node {$a + ba$} (63)
	
	(73) edge node {$a$} (63)
	
	(54) edge node[above] {$\varepsilon + b$} (qf)
	(73) edge node {$\varepsilon$} (qf)
	(63) edge[bend right] node[right] {$b$} (qf)
	(43) edge node {$\varepsilon$} (qf)
	(13) edge node {$\varepsilon + bb(bb)^*$} (qf)
	(33) edge[bend left] node[near start, right] {$\varepsilon + bb(bb)^*$} (qf)
	(23) edge node {$\varepsilon$} (qf)
	;
	\end{tikzpicture}
}
\end{center}
\end{minipage}

\begin{minipage}{\linewidth}
	9) Исключим состояние $33$: 		
	\begin{center}	
\resizebox{!}{9.3cm}{%
		\begin{tikzpicture}[auto,>=stealth', node distance=2.4cm,auto,every state/.style={thick}]
		\node (init) {};
		\node[state] (11) [right=.7cm of init] {$11$};
		\node[state] (42) [below of=11] {$42$};
		\node[state] (23) [right of=11] {$23$};
		\node[state] (43) [below of=23] {$43$};
		\node[state] (13) [right of=23] {$13$};
		\node[state] (54) [below of=43] {$54$};
		\node[state] (73) [below of=54] {$73$};
		\node[state] (53) [left of=73] {$53$};
		\node[state] (63) [right of=73] {$63$};
		\node[state, accepting] (qf) [above of=63, right of=63] {$q_f$};
		
		\path[->]
		(init) edge (11)
		(11) edge[loop above] node {$aa$} (11)
		(11) edge node {$a$} (23)
		(11) edge node[left] {$a$} (42)
		(11) edge node {$a$} (43)
		
		(23) edge node[below] {$a$} (13)
		
		(13) edge[bend right] node[above] {$a$} (23)
		(13) edge node[near end] {$a$} (43)
		(13) edge[loop above] node  {$bb$}(13)
		
		(42) edge[loop left] node {$aa$} (42)
		(42) edge node[above] {$aa$} (43)
		
		(43) edge[loop above] node {$a$} (43)
		(43) edge node[near end] {$b$} (54)
		(43) edge[bend right] node[left] {$b$} (53)
		
		(53) edge[bend right] node[below] {$a$} (63)
		(53) edge node[near end] {$b$} (73)
		
		(63) edge[loop below] node {$a + ba$} (63)
		
		(73) edge node {$a$} (63)
		
		(54) edge node[above] {$\varepsilon + b$} (qf)
		(73) edge node {$\varepsilon$} (qf)
		(63) edge[bend right] node[right] {$b$} (qf)
		(43) edge node {$\varepsilon$} (qf)
		(13) edge node {$\varepsilon + bb(bb)^* + b(\varepsilon + bb(bb)^*)$} (qf)
		(23) edge node {$\varepsilon$} (qf)
		;
		\end{tikzpicture}
}
	\end{center}
\end{minipage}

\begin{minipage}{\linewidth}
	10) Исключим состояние $54$: 		
	\begin{center}	
\resizebox{!}{9.3cm}{%
		\begin{tikzpicture}[auto,>=stealth', node distance=2.7cm,auto,every state/.style={thick}]
		\node (init) {};
		\node[state] (11) [right=.7cm of init] {$11$};
		\node[state] (42) [below of=11] {$42$};
		\node[state] (23) [right of=11] {$23$};
		\node[state] (43) [below of=23] {$43$};
		\node[state] (13) [right of=23] {$13$};
		\node[state] (73) [below of=43] {$73$};
		\node[state] (53) [left of=73] {$53$};
		\node[state] (63) [right of=73] {$63$};
		\node[state, accepting] (qf) [above of=63, right of=63] {$q_f$};
		
		\path[->]
		(init) edge (11)
		(11) edge[loop above] node {$aa$} (11)
		(11) edge node {$a$} (23)
		(11) edge node[left] {$a$} (42)
		(11) edge node {$a$} (43)
		
		(23) edge node[above] {$a$} (13)
		
		(13) edge[bend right] node[above] {$a$} (23)
		(13) edge node[left] {$a$} (43)
		(13) edge[loop above] node  {$bb$}(13)
		
		(42) edge[loop left] node {$aa$} (42)
		(42) edge node[above] {$aa$} (43)
		
		(43) edge[loop above] node {$a$} (43)
		(43) edge node[left] {$b$} (53)
		
		(53) edge[bend right] node[below] {$a$} (63)
		(53) edge node[near end] {$b$} (73)
		
		(63) edge[loop below] node {$a + ba$} (63)
		
		(73) edge node {$a$} (63)
		
		(73) edge node {$\varepsilon$} (qf)
		(63) edge[bend right] node[right] {$b$} (qf)
		(43) edge node {$\varepsilon + b(\varepsilon + b)$} (qf)
		(13) edge node {$\varepsilon + bb(bb)^* + b(\varepsilon + bb(bb)^*)$} (qf)
		(23) edge node {$\varepsilon$} (qf)
		;
		\end{tikzpicture}
}	
\end{center}
\end{minipage}
	
\begin{minipage}{\linewidth}
		11) Исключим состояние $73$: 		
	\begin{center}	
\resizebox{!}{9.3cm}{%
		\begin{tikzpicture}[auto,>=stealth', node distance=2.7cm,auto,every state/.style={thick}]
		\node (init) {};
		\node[state] (11) [right=.7cm of init] {$11$};
		\node[state] (42) [below of=11] {$42$};
		\node[state] (23) [right of=11] {$23$};
		\node[state] (43) [below of=23] {$43$};
		\node[state] (13) [right of=23] {$13$};
		\node[state] (53) [below of=42] {$53$};
		\node[state] (63) [right of=53] {$63$};
		\node[state, accepting] (qf) [below of=13, right of=13] {$q_f$};
		
		\path[->]
		(init) edge (11)
		(11) edge[loop above] node {$aa$} (11)
		(11) edge node {$a$} (23)
		(11) edge node[left] {$a$} (42)
		(11) edge node {$a$} (43)
		
		(23) edge node[above] {$a$} (13)
		
		(13) edge[bend right] node[above] {$a$} (23)
		(13) edge node[left] {$a$} (43)
		(13) edge[loop above] node  {$bb$}(13)
		
		(42) edge[loop left] node {$aa$} (42)
		(42) edge node[above] {$aa$} (43)
		
		(43) edge[loop above] node {$a$} (43)
		(43) edge node[left] {$b$} (53)
		
		(53) edge node[below] {$a + ba$} (63)
		(63) edge[loop below] node {$a + ba$} (63)
		
		(53) edge node {$b\varepsilon$} (qf)
		(63) edge[bend right] node {$b$} (qf)
		(43) edge node {$\varepsilon + b(\varepsilon + b)$} (qf)
		(13) edge node {$\varepsilon + bb(bb)^* + b(\varepsilon + bb(bb)^*)$} (qf)
		(23) edge node {$\varepsilon$} (qf)
		;
		\end{tikzpicture}
}
	\end{center}
\end{minipage}
	
	
\begin{minipage}{\linewidth}
	12) Исключим состояние $63$: 		
	\begin{center}	
		\begin{tikzpicture}[auto,>=stealth', node distance=2.7cm,auto,every state/.style={thick}]
		\node (init) {};
		\node[state] (11) [right=.7cm of init] {$11$};
		\node[state] (42) [below of=11] {$42$};
		\node[state] (23) [right of=11] {$23$};
		\node[state] (43) [below of=23] {$43$};
		\node[state] (13) [right of=23] {$13$};
		\node[state] (53) [below of=42] {$53$};
		\node[state, accepting] (qf) [below of=13, right of=13] {$q_f$};
		
		\path[->]
		(init) edge (11)
		(11) edge[loop above] node {$aa$} (11)
		(11) edge node {$a$} (23)
		(11) edge node[left] {$a$} (42)
		(11) edge node {$a$} (43)
		
		(23) edge node[above] {$a$} (13)
		
		(13) edge[bend right] node[above] {$a$} (23)
		(13) edge node[left] {$a$} (43)
		(13) edge[loop above] node  {$bb$}(13)
		
		(42) edge[loop left] node {$aa$} (42)
		(42) edge node[above] {$aa$} (43)
		
		(43) edge[loop above] node {$a$} (43)
		(43) edge node[left] {$b$} (53)
		
		(53) edge[bend right] node[right] {$b + (a + ba)(a + ba)^*b$} (qf)
		(43) edge node {$\varepsilon + b(\varepsilon + b)$} (qf)
		(13) edge node {$\varepsilon + bb(bb)^* + b(\varepsilon + bb(bb)^*)$} (qf)
		(23) edge node {$\varepsilon$} (qf)
		;
		\end{tikzpicture}
	\end{center}
\end{minipage}
	
\begin{minipage}{\linewidth}
	13) Исключим состояние $42$: 		
	\begin{center}	
		\begin{tikzpicture}[auto,>=stealth', node distance=2.7cm,auto,every state/.style={thick}]
		\node (init) {};
		\node[state] (11) [right=.7cm of init] {$11$};
		\node[state] (23) [right of=11] {$23$};
		\node[state] (43) [below of=23] {$43$};
		\node[state] (13) [right of=23] {$13$};
		\node[state] (53) [below of=43, left of=43] {$53$};
		\node[state, accepting] (qf) [below of=13, right of=13] {$q_f$};
		
		\path[->]
		(init) edge (11)
		(11) edge[loop above] node {$aa$} (11)
		(11) edge node {$a$} (23)
		(11) edge[bend right] node[left] {$a + a(aa)^*aa$} (43)
		
		(23) edge node[above] {$a$} (13)
		
		(13) edge[bend right] node[above] {$a$} (23)
		(13) edge node[left] {$a$} (43)
		(13) edge[loop above] node  {$bb$}(13)
		
		(43) edge[loop above] node {$a$} (43)
		(43) edge node[left] {$b$} (53)
		
		(53) edge node[right] {$b + (a + ba)(a + ba)^*b$} (qf)
		(43) edge node {$\varepsilon + b(\varepsilon + b)$} (qf)
		(13) edge node {$\varepsilon + bb(bb)^* + b(\varepsilon + bb(bb)^*)$} (qf)
		(23) edge node {$\varepsilon$} (qf)
		;
		\end{tikzpicture}
	\end{center}
\end{minipage}
	
	
\begin{minipage}{\linewidth}
	14) Исключим состояние $13$:
	\begin{center}	
\resizebox{!}{9.2cm}{%
		\begin{tikzpicture}[auto,>=stealth', node distance=4cm,auto,every state/.style={thick}]
		\node (init) {};
		\node[state] (11) [right=.7cm of init] {$11$};
		\node[state] (23) [right of=11] {$23$};
		\node[state] (43) [below of=23] {$43$};
		\node[state] (53) [below of=43, left of=43] {$53$};
		\node[state, accepting] (qf) [right of=43] {$q_f$};
		
		\path[->]
		(init) edge (11)
		(11) edge[loop above] node {$aa$} (11)
		(11) edge node {$a$} (23)
		(11) edge[bend right] node[left] {$a + a(aa)^*aa$} (43)
		
		(23) edge[loop above] node {$aa$} (23)
		(23) edge node {$aa$} (43)
		
		(43) edge[loop below] node {$a$} (43)
		(43) edge node[left,xshift=-2mm] {$b$} (53)
		
		(53) edge[bend right] node[right,xshift=3mm] {$b + (a + ba)(a + ba)^*b$} (qf)
		(43) edge node {$\varepsilon + b(\varepsilon + b)$} (qf)
		(23) edge node {$\varepsilon + a(bb)^*(\varepsilon + bb(bb)^* + b(\varepsilon + bb(bb)^*))$} (qf)
		;
		\end{tikzpicture}
}
	\end{center}
\end{minipage}

\begin{minipage}{\linewidth}
15) Исключим состояние $23$: 		
\begin{center}	
\resizebox{!}{9.2cm}{%
	\begin{tikzpicture}[auto,>=stealth', node distance=8cm,auto,every state/.style={thick}]
	\node (init) {};
	\node[state] (11) [right=.7cm of init] {$11$};
	\node[state] (43) [below of=11] {$43$};
	\node[state] (53) [right of=43] {$53$};
	\node[state, accepting] (qf) [right of=11] {$q_f$};
	
	\path[->]
	(init) edge (11)
	(11) edge[loop right] node {$aa$} (11)
	(11) edge node[left] {$a + a(aa)^*aa + a(aa)^*aa$} (43)

	(43) edge[loop left] node {$a$} (43)
	(43) edge node[above] {$b$} (53)
	
	(53) edge node[left] {$b + (a + ba)(a + ba)^*b$} (qf)
	(43) edge[bend left] node[xshift=6mm,yshift=3mm] {$\varepsilon + b(\varepsilon + b)$} (qf)
	(11) edge[bend left] node {$a(aa)^*(\varepsilon + a(bb)^*(\varepsilon + bb(bb)^* + b(\varepsilon + bb(bb)^*)))$} (qf)
	;
	\end{tikzpicture}
}
\end{center}
\end{minipage}
	

\begin{minipage}{\linewidth}
	16) Исключим состояние $53$: 		
	\begin{center}	
		\begin{tikzpicture}[auto,>=stealth', node distance=3cm,auto,every state/.style={thick}]
		\node (init) {};
		\node[state] (11) [right=.7cm of init] {$11$};
		\node[state, accepting] (qf) [right of=11] {$q_f$};
		\node[state] (43) [right of=qf] {$43$};
		
		\path[->]
		(init) edge (11)
		(11) edge[loop below] node {$aa$} (11)
		(11) edge[bend right] node[below]{$K_1$} (43)
		
		(43) edge[loop below] node {$a$} (43)
		
		(43) edge node[above] {$K_2$} (qf)
		(11) edge node {$K_3$} (qf)
		;
		\end{tikzpicture}
	\end{center}
    \begin{align*}
    K_1 &= a + a(aa)^*aa + a(aa)^*aa, \\
    K_2 &= \varepsilon + b(\varepsilon + b) + b(b + (a + ba)(a + ba)^*b), \\
    K_3 &= a(aa)^*(\varepsilon + a(bb)^*(\varepsilon + bb(bb)^* + b(\varepsilon + bb(bb)^*))).
    \end{align*}
\end{minipage}
	
	
	17) Исключим состояние $43$: 		

			\begin{center}	
			\begin{tikzpicture}[auto,>=stealth', node distance=8cm,auto,every state/.style={thick}]
			\node (init) {};
			\node[state] (11) [right=.7cm of init] {$11$};
			\node[state, accepting] (qf) [right of=11] {$q_f$};
			
			\path[->]
			(init) edge (11)
			(11) edge[loop right] node {$aa$} (11)
			
			(11) edge[bend left] node {$a(aa)^*(\varepsilon + a(bb)^*(\varepsilon + bb(bb)^* + b(\varepsilon + bb(bb)^*)))$} (qf)
			
			(11) edge[bend right] node[below] {$(a + a(aa)^*aa + a(aa)^*aa)a^*(\varepsilon + b(\varepsilon + b) + b(b + (a + ba)(a + ba)^*b))$} (qf)
			;
			\end{tikzpicture}
		\end{center}
	
	Получаем
    \begin{multline*}
        Reg = (aa)^*(a(aa)^*(\varepsilon + a(bb)^*(\varepsilon + bb(bb)^* + b(\varepsilon + bb(bb)^*)))\\
         + (a + a(aa)^*aa + + a(aa)^*aa)a^*(\varepsilon + b(\varepsilon + b) + b(b + (a + ba)(a + ba)^*b))).
    \end{multline*}

	\item Определить: совпадают ли языки $L_1$ и $L_2$, является ли $L_1$ дополнением $L_2$.
	
	1) Убедимся, что $L_1$ не совпадает с $L_2$, взяв в качестве контрпримера слово $ w= bbaba $. Очевидно, $w \in L_1$ и $ w \notin L_2$. Делаем вывод, что $L_1 \ne L_2$.
	
	2) Допустим, $L_1 = \overline{L_2}$. Приведем контрпример. Слово  $aaaab \in L_1$, 
	
	но при этом  $aaaab \in L_2$.
	 $L_1 \cup L_2 \ne \varnothing$, что противоречит определению дополнения языка.

	
	\item Построить $\varepsilon$-НКА, распознающий один из языков $L^R_1$ или $L^R_2$.
	
	НКА, распознающий язык $L_2$ имеет вид :
	
	\begin{center}	
		\begin{tikzpicture}[auto,>=stealth', node distance=3cm,auto,every state/.style={thick}]
		\node (init) {};
		\node[state] (1) [right=.7cm of init] {$1$};
		\node[state] (2) [above of=1] {$2$};
		\node[state, accepting] (3) [right of=1] {$3$};	
		\node[state] (4) [right of=3] {$4$};	
		\node[state, accepting] (5) [right of=4] {$5$};	
		\node[state] (6) [right of=5] {$6$};	
		
		\path[->]
		(init) edge (1)
		(1) edge[bend right] node[right] {$a$} (2)
		(2) edge[bend right] node[left] {$a$} (1)
		(1) edge node {$a$} (3)
		(3) edge[loop above] node {$a, b$} (3)
		(3) edge node {$b$} (4)
		(4) edge node {$b$} (5)
		(5) edge[bend left] node {$b$} (6)
		(6) edge[bend left] node {$b$} (5)
		;
		\end{tikzpicture}
	\end{center}
	
	Следуя алгоритму, преобразуем его к $\varepsilon$-НКА, распознающему язык $L_2^R$:
	
	1) Меняем направления дуг.
	
	2) Начальное состояние автомата делаем единственным конечным.
	
	3) Создаём новое начальное состояние $q_s$ с $\varepsilon$-переходами во все исходные допускающие состояния.
	
	$M = (\{q_s, 1, 2, 3, 4, 5, 6\}, \{a, b\}, \delta, q_s, \{1\})$

	\begin{center}	
		\begin{tikzpicture}[auto,>=stealth', node distance=3cm,auto,every state/.style={thick}]
		\node (init) {};
		\node[state] (qs) [right=.7cm of init] {$q_s$};
		\node[state] (4) [below of=qs] {$4$};
		\node[state] (3) [left of=4] {$3$};	
		\node[state, accepting] (1) [left of=3] {$1$};
		\node[state] (2) [above of=1] {$2$};
		\node[state] (5) [right of=4] {$5$};	
		\node[state] (6) [right of=5] {$6$};	
		
		\path[->]
		(init) edge (qs)
		(1) edge[bend left] node[left] {$a$} (2)
		(2) edge[bend left] node[right] {$a$} (1)
		(3) edge node {$a$} (1)
		(3) edge[loop below] node {$a, b$} (3)
		(4) edge node {$b$} (3)
		(5) edge node {$b$} (4)
		(5) edge[bend right] node[below] {$b$} (6)
		(6) edge[bend right] node[above] {$b$} (5)
		(qs) edge node[left] {$\varepsilon$} (3)
		(qs) edge node {$\varepsilon$} (5)
		;
		\end{tikzpicture}
	\end{center}
	
	\item Вычислить регулярное выражение по построенному $\varepsilon$-НКА.
	
	Воспользуемся методом исключения состояний:
	
\begin{minipage}{\linewidth}
	1) Исключим состояние $6$:
	\begin{center}	
		\begin{tikzpicture}[auto,>=stealth', node distance=3cm,auto,every state/.style={thick}]
		\node (init) {};
		\node[state] (qs) [right=.7cm of init] {$q_s$};
		\node[state] (4) [below of=qs] {$4$};
		\node[state] (3) [left of=4] {$3$};	
		\node[state, accepting] (1) [left of=3] {$1$};
		\node[state] (2) [above of=1] {$2$};
		\node[state] (5) [right of=4] {$5$};		
		
		\path[->]
		(init) edge (qs)
		(1) edge[bend left] node[left] {$a$} (2)
		(2) edge[bend left] node[right] {$a$} (1)
		(3) edge node {$a$} (1)
		(3) edge[loop below] node {$a, b$} (3)
		(4) edge node {$b$} (3)
		(5) edge node {$b$} (4)
		(5) edge[loop right] node {$bb$} (5)
		(qs) edge node[left] {$\varepsilon$} (3)
		(qs) edge node {$\varepsilon$} (5)
		;
		\end{tikzpicture}
	\end{center}
\end{minipage}

\begin{minipage}{\linewidth}
2) Исключим состояние $4$:
\begin{center}	
	\begin{tikzpicture}[auto,>=stealth', node distance=3cm,auto,every state/.style={thick}]
	\node (init) {};
	\node[state] (qs) [right=.7cm of init] {$q_s$};
	\node[state] (5) [right of=qs, below of=qs] {$5$};	
	\node[state] (3) [left of=qs, below of=qs] {$3$};	
	\node[state, accepting] (1) [left of=3] {$1$};
	\node[state] (2) [above of=1] {$2$};
		
	
	\path[->]
	(init) edge (qs)
	(1) edge[bend left] node[left] {$a$} (2)
	(2) edge[bend left] node[right] {$a$} (1)
	(3) edge node {$a$} (1)
	(3) edge[loop below] node {$a, b$} (3)
	(5) edge[loop right] node {$bb$} (5)
	(5) edge node {$bb$} (3)
	(qs) edge node[left] {$\varepsilon$} (3)
	(qs) edge node {$\varepsilon$} (5)
	;
	\end{tikzpicture}
\end{center}
\end{minipage}

\begin{minipage}{\linewidth}
3) Исключим состояние $2$:
\begin{center}	
	\begin{tikzpicture}[auto,>=stealth', node distance=3cm,auto,every state/.style={thick}]
	\node (init) {};
	\node[state] (qs) [right=.7cm of init] {$q_s$};
	\node[state] (5) [right of=qs, below of=qs] {$5$};	
	\node[state] (3) [left of=qs, below of=qs] {$3$};	
	\node[state, accepting] (1) [left of=3] {$1$};
		
	\path[->]
	(init) edge (qs)
	(1) edge[loop above] node {$aa$} (1)
	(3) edge node {$a$} (1)
	(3) edge[loop below] node {$a, b$} (3)
	(5) edge[loop right] node {$bb$} (5)
	(5) edge node {$bb$} (3)
	(qs) edge node[left] {$\varepsilon$} (3)
	(qs) edge node {$\varepsilon$} (5)
	;
	\end{tikzpicture}
\end{center}
\end{minipage}

\begin{minipage}{\linewidth}
4) Исключим состояние $5$:
\begin{center}	
	\begin{tikzpicture}[auto,>=stealth', node distance=3cm,auto,every state/.style={thick}]
	\node (init) {};
	\node[state] (qs) [right=.7cm of init] {$q_s$};
	\node[state] (3) [right=4cm of qs] {$3$};
	\node[state, accepting] (1) [right of=3] {$1$};
	
	\path[->]
	(init) edge (qs)
	(1) edge[loop above] node {$aa$} (1)
	(3) edge node {$a$} (1)
	(3) edge[loop above] node {$a, b$} (3)
	(qs) edge node {$\varepsilon + \varepsilon(bb)^*bb$} (3)
	;
	\end{tikzpicture}
\end{center}
\end{minipage}

\begin{minipage}{\linewidth}
5) Исключим состояние $3$:
\begin{center}	
	\begin{tikzpicture}[auto,>=stealth', node distance=8cm,auto,every state/.style={thick}]
	\node (init) {};
	\node[state] (qs) [right=.7cm of init] {$q_s$};	
	\node[state, accepting] (1) [right of=qs] {$1$};
	
	\path[->]
	(init) edge (qs)
	(1) edge[loop above] node {$aa$} (1)
	(qs) edge node {$(\varepsilon + (bb)^*bb)(a + b)^*a$} (1)
	;
	\end{tikzpicture}
\end{center}
\end{minipage}

Получаем: $Reg = (\varepsilon + (bb)^*bb)(a + b)^*a(aa)^*$.

	\item Построить $\varepsilon$-НКА, для одного из языков $L_1L_2$, $L_1 \cup L_2$ или $L^*_1$.

	На рис.~\ref{app-ex-task3-12} (с.~\pageref{app-ex-task3-12}) показан $\varepsilon$-НКА, распознающий язык $L_1^*$.

	\item Детерминизировать один из построенных $\varepsilon$-НКА.
    
    См. рис.~\ref{app-ex-task2-15}.
    
%    На рис.~\ref{app-ex-task2-15} показаны $\varepsilon$-НКА для языка $L_2^R$ и результат его детерминизации.

\begin{figure}[H]
\centering
\begin{subfigure}[b]{.6\linewidth}
\centering
	\begin{center}
		\begin{tabular}{llll}
			\toprule
			\multicolumn{1}{c}{\multirow{2}{*}{\Large $\delta$}}
			& \multicolumn{3}{c}{Вход} \\
			\cmidrule(rl){2-4}
			& \multicolumn{1}{c}{$a$}
			& \multicolumn{1}{c}{$b$} 
			& \multicolumn{1}{c}{$\varepsilon$} \\
			\midrule
			$\{q_s\}$       & $\{\varnothing\}$      		 & $\{\varnothing\}$     &$\{3\}, \{5\}$  \\
			$\{1\}$       & $\{2\}$    			 & $\{\varnothing\}$     &$\{\varnothing\}$ \\
			$\{2\}$       & $\{1\}$    			 & $\{\varnothing\}$     &$\{\varnothing\}$  \\
			$\{3\}$       & $\{1\}, \{3\}$    			 & $\{3\}$     &$\{\varnothing\}$  \\
			$\{4\}$       & $\{\varnothing\}$    			 & $\{3\}$     &$\{\varnothing\}$  \\
			$\{5\}$ 		& $\{\varnothing\}$    			 & $\{4\}, \{6\}$     &$\{\varnothing\}$  \\
			$\{6\}$       & $\{\varnothing\}$    	 & $\{5\}$     &$\{\varnothing\}$  \\
			\bottomrule
		\end{tabular}
	\end{center}
\caption{$\varepsilon$-НКА для языка $L_2^R$}\label{app-ex-task2-14}
\end{subfigure}%
%
\begin{subfigure}[b]{.4\linewidth}
\centering
		\begin{center}
		\begin{tabular}{lll}
			\toprule
			\multicolumn{1}{c}{\multirow{2}{*}{\Large $\delta$}}
			& \multicolumn{2}{c}{Вход} \\
			\cmidrule(rl){2-3}
			& \multicolumn{1}{c}{$a$}
			& \multicolumn{1}{c}{$b$}  \\
			\midrule
			$\{q_s, 3, 5\}$       & $\{1, 3\}$      		 & $\{3, 4, 6\}$ \\
			$\{1, 3\}$       & $\{1, 2, 3\}$    			 & $\{3\}$    \\
			$\{3, 4, 6\}$       & $\{1, 3\}$    			 & $\{3, 5\}$     \\
			$\{1, 2, 3\}$       & $\{1, 2, 3\}$    			 & $\{3\}$     \\
			$\{3\}$       & $\{1, 3\}$    			 & $\{3\}$     \\
			$\{3, 5\}$       & $\{1, 3\}$    	 & $\{3, 4, 6\}$   \\
			\bottomrule
		\end{tabular}
	\end{center}
\caption{Детерминизация}\label{app-ex-task2-13}
\end{subfigure}
\caption{}\label{app-ex-task2-15}
\end{figure}

\begin{figure}
\centering
\resizebox{\linewidth}{!}{%
		\begin{tikzpicture}[auto,>=stealth', node distance=2cm,auto,every state/.style={thick}]
		\node (init) {};
		\node[state] (30) [right=.7cm of init] {$30$};
		\node[state] (0) [right of=30] {$0$};
		\node[state] (1) [right of=0] {$1$};
		\node[state] (2) [right of=1] {$2$};
		\node[state] (3) [right of=1, below of=1] {$3$};
		\node[state] (4) [right of=2] {$4$};
		\node[state] (5) [right of=3] {$5$};
		\node[state] (6) [right of=4] {$6$};
		\node[state] (7) [right of=5] {$7$};
		\node[state] (8) [right of=6] {$8$};
		\node[state] (9) [right of=7] {$9$};
		\node[state] (10) [right of=8] {$10$};
		\node[state] (11) [right of=10] {$11$};
		\node[state] (12) [below of=11] {$12$};
		\node[state] (13) [below of=12] {$13$};
		\node[state] (14) [below of=13] {$14$};
		\node[state] (15) [below of=14] {$15$};
		\node[state] (16) [below of=15] {$16$};
		\node[state] (17) [below of=16] {$17$};
		\node[state] (18) [below of=17] {$18$};
		\node[state] (19) [below of=18] {$19$};
		\node[state] (20) [below of=19] {$20$};
		\node[state] (21) [left of=20] {$21$};
		\node[state] (22) [left of=21, above of=21] {$22$};
		\node[state] (23) [left of=21] {$23$};
		\node[state] (24) [left of=22] {$24$};
		\node[state] (25) [left of=23] {$25$};
		\node[state] (26) [left of=24] {$26$};
		\node[state] (27) [left of=26] {$27$};
		\node[state] (28) [left of=27, below of=27] {$28$};
		\node[state] (29) [left of=28] {$29$};
		\node[state, accepting] (31) [left of=29] {$31$};
		
		\path[->]
		(init) edge (30)
		(30) edge node {$\varepsilon$} (0)
		(0) edge node {$\varepsilon$} (1)
		(1) edge node {$\varepsilon$} (2)
		(1) edge node {$\varepsilon$} (3)
		(2) edge node {$a$} (4)
		(3) edge node {$b$} (5)
		(4) edge node {$\varepsilon$} (6)
		(5) edge node {$\varepsilon$} (7)
		(6) edge node {$a$} (8)
		(7) edge node {$b$} (9)
		(8) edge node {$\varepsilon$} (10)
		(9) edge node {$\varepsilon$} (10)
		(10) edge[bend right] node[above] {$\varepsilon$} (1)
		(10) edge node {$\varepsilon$} (11)
		(0) edge[bend left] node[above] {$\varepsilon$} (11)
		(11) edge node {$\varepsilon$} (12)
		(12) edge node {$\varepsilon$} (13)
		(12) edge[bend left] node {$\varepsilon$} (15)
		(13) edge node {$a$} (14)
		(14) edge[bend left] node {$\varepsilon$} (13)
		(14) edge node {$\varepsilon$} (15)
		(15) edge node {$\varepsilon$} (16)
		(16) edge node {$a$} (17)
		(17) edge node {$\varepsilon$} (18)
		(18) edge node {$b$} (19)
		(19) edge node {$\varepsilon$} (20)
		(20) edge node[above] {$\varepsilon$} (21)
		(21) edge node[right] {$\varepsilon$} (22)
		(21) edge node[above] {$\varepsilon$} (23)
		(22) edge node[above] {$b$} (24)
		(23) edge node[above] {$a$} (25)
		(24) edge node[above] {$\varepsilon$} (26)
		(25) edge node[above] {$\varepsilon$} (28)
		(26) edge node[above] {$a$} (27)
		(27) edge node[left] {$\varepsilon$} (28)
		(28) edge node[above] {$\varepsilon$} (29)
		(28) edge[bend right] node[above] {$\varepsilon$} (21)
		(20) edge[bend left] node {$\varepsilon$} (29)
		(29) edge node[above] {$\varepsilon$} (31)
		(29) edge node[left] {$\varepsilon$} (0)
		(30) edge node[left] {$\varepsilon$} (31)
		;
		\end{tikzpicture}
}
\caption{}\label{app-ex-task3-12}
\end{figure}
\clearpage

Введем обозначения:
	\begin{align*}
	\{q_s, 3, 5\} &= A, &
	\{1, 3\} &= B, &
	\{3, 4, 6\} &= C, \\
	\{1, 2, 3\} &= D, &
	\{3\} &= E, &
	\{3, 5\} &= F.
	\end{align*}
Построим граф переходов:
\begin{center}	
		\begin{tikzpicture}[auto,>=stealth', node distance=2.8cm,auto,every state/.style={thick}]
		\node (init) {};
		\node[state] (A) [right=.7cm of init] {$A$};
		\node[state, accepting] (B) [right of=A] {$B$};
		\node[state] (C) [right of=B] {$C$};
		\node[state, accepting] (D) [below of=A] {$D$};
		\node[state] (E) [below of=B] {$E$};
		\node[state] (F) [below of=C] {$F$};
		
		\path[->]
		(init) edge (A)
		(A) edge node {$a$} (B)
		(A) edge[bend left] node {$b$} (C)
		
		(B) edge[bend right] node[left] {$a$} (D)
		(B) edge node[left] {$b$} (E)
		
		(C) edge node[above] {$a$} (B)
		(C) edge node[left] {$b$} (F)
		
		(D) edge[loop left] node {$a$} (D)
		(D) edge node {$b$} (E)
		
		(E) edge[bend right] node[right] {$a$} (B)
		(E) edge[loop below] node {$b$} (E)
		
		(F) edge node[right] {$a$} (B)
		(F) edge[bend right] node[right] {$b$} (C)	
		;
		\end{tikzpicture}
	\end{center}
	
\end{enumerate}
