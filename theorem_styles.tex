% Оформление теорем (ntheorem)

\usepackage{MnSymbol}
\usepackage [thmmarks, amsmath] {ntheorem}
\theorempreskipamount 0.3cm

\theoremstyle {plain} %
\theoremheaderfont {\normalfont \bfseries} %
\theorembodyfont {\slshape} %
\theoremsymbol {\ensuremath {_\square}} %
\theoremseparator {.} %
\newtheorem {mystatement} {Утверждение} [section] %
\newtheorem {mylemma} {Лемма} [section] %
\newtheorem {mytheorem} {Теорема} [section] %
\newtheorem {mycorollary} {Следствие} [section] %

\theoremstyle {nonumberplain} %
\theoremseparator {.} %
\theoremsymbol {\ensuremath {_\diamondsuit}} %
\newtheorem {mydefinition} {Определение} %

\theoremstyle {plain} %
\theoremheaderfont {\normalfont \bfseries}
\theorembodyfont {\normalfont}
%\theoremsymbol {\ensuremath {_\Box}} %
\theoremseparator {.} %
\newtheorem {mytask} {Задача} [section]%
\renewcommand{\themytask}{\arabic{mytask}}

\theorembodyfont {\upshape} %
\theoremseparator {.} %

\theoremsymbol {\ensuremath{\blacksquare}} %
\newtheorem {myexample} {Пример}[section] %
\newtheorem {myexamples}[section] {Примеры} %

\theoremsymbol {\ensuremath{\square}} %
\newtheorem {myproblem} {Упражнение}[section] %

\theorempreskipamount 0cm

\theoremstyle {nonumberplain} %
\newtheorem {myremark} {Замечание} %

\theoremheaderfont {\itshape} %
\theorembodyfont {\upshape} %
\theoremsymbol {\rule {1ex} {1ex}} %
\newtheorem {myproof} {Доказательство} %

\theorempreskipamount 0.3cm

\theoremseparator {:} %
\theoremsymbol {\ensuremath {_\triangle}} %
\theoremstyle {nonumberbreak} %
\newtheorem {myremarks} {Замечания} %

%%%%%%%  Алгоритмы

\usepackage{clrscode}
\DeclareMathOperator{\Gen}{Gen}
\DeclareMathOperator{\Reach}{Reach}

% Линейка
\newcommand*\varhrulefill[1][0.4pt]{\leavevmode\leaders\hrule height#1\hfill\kern0pt}


% Список для описания алгоритма
\newlist{algoenum}{enumerate}{3}
\setlist[algoenum]{%
     label=\textit{Шаг~\arabic*.}%
    %,before=\raggedright%{}
    ,leftmargin=\parindent+\widthof{\textit{Шаг~9.}}+\labelsep%{}
    ,topsep=0pt
}

\usepackage{etoolbox}
\usepackage{float}
\floatstyle{ruled}

 \makeatletter
\patchcmd{\fs@ruled}
 {\def\@fs@post{\kern2pt\hrule\relax}}
 {\def\@fs@post{\kern2pt\hrule height 0pt depth .8pt\relax}}
 {}{}

\renewcommand\floatc@ruled[2]{{\@fs@cfont #1.} #2\par}
\makeatother

\floatname{AlgoEnv}{Алгоритм}
\newfloat{AlgoEnv}{htbp}{loa}[section]



\newcommand{\AlgoPre}[3]{%
\begin{description}
    \item [Вход:]
#1

    \item [Выход:]
#2

    \item [Метод:]
#3
\end{description}%
}

\newcommand{\AlgoPreNoMeth}[2]{%
\begin{description}
    \item [Вход:]
#1

    \item [Выход:]
#2
\end{description}%
}

\newcommand{\AlgoBody}[1]{%
\begin{algoenum}
#1
\end{algoenum}
}

% #1 - placement (optional), #2 - name, #3-5 - preamble (In/Out/Method), 6 - body
\newcommand{\Algo}[6][t]{%
\begin{AlgoEnv}[#1]
\caption{#2}
\AlgoPre{#3}{#4}{#5}
\vspace{-5mm}
\varhrulefill[.2pt]
\AlgoBody{#6}
\end{AlgoEnv}
}

% #1 - placement (optional), #2 - name, #3-4 - preamble (In/Out), 5 - body
\newcommand{\AlgoNoMeth}[5][t]{%
\begin{AlgoEnv}[#1]
\caption{#2}
\AlgoPreNoMeth{#3}{#4}
\vspace{-5mm}
\varhrulefill[.2pt]
\AlgoBody{#5}
\end{AlgoEnv}
}

% #1 - placement (optional), #2 - name, #3-4 - preamble (In/Out), 5 - body
\newcommand{\AlgoPseudoCode}[5][t]{%
\begin{AlgoEnv}[#1]
\caption{#2}
\AlgoPreNoMeth{#3}{#4}
\vspace{-5mm}
\varhrulefill[.2pt]
\begin{codebox}
#5
\end{codebox}
\end{AlgoEnv}
}

