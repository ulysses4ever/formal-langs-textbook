%\addchap{Литература}

\renewcaptionname{russian}{\bibname}{Список литературы}
\begin{thebibliography}{9}
\addcontentsline{toc}{chapter}{Список литературы}

\bibitem{AU} \emph{А.~Ахо, Дж.~Ульман.} Теория синтаксического анализа, перевода и компиляции. Т.~1. Синтаксический анализ. М.: Мир, 1978. \\

\bibitem{ASU} \emph{А.~Ахо, М.~Лам, Р.~Сети, Дж.~Ульман.} Компиляторы: принципы, технологии и инструментарий, 2-е изд.: Пер. с англ. М.: Вильямс, 2008. \\

\bibitem{Hop} \emph{Дж.~Хопкрофт, Р.~Мотвани, Дж.~Ульман.}  Введение в теорию автоматов, языков и
вычислений. — 2-е изд. М.: Вильямс, 2008.\\

%\bibitem{Glu} \emph{В.\,М.~Глушков, Г.\,Е.~Цейтлин, Е.\,Л.~Ющенко.} Алгебра, языки, программирование. Киев: Наукова думка, 1989. \\

\bibitem{Bau} \emph{Ф.\,Л.~Бауэр, Г.~Гооз.} Информатика. М.: Мир, 1990. \\

\bibitem{Sal} \emph{А.~Саломаа.} Жемчужины теории формальных языков. М.: Мир, 1986. \\

%\bibitem{Kot} \emph{В.\,Е.~Котов.} Сети Петри. М.: Наука, 1984. \\

\bibitem{Hoa} \emph{Ч.~Хоар.} Взаимодействующие последовательные процессы. М.: Мир, 1889.\\

\bibitem{BelTka} \emph{А.\,И.~Белоусов, С.\,Б.~Ткачёв} Дискретная математика. М.: МГТУ, 2010.\\

\bibitem{Hro} \emph{Ю.~Громкович.} Теоретическая информатика. Введение в теорию автоматов, теорию вычислимости, теорию сложности, теорию алгоритмов, рандомизацию, теорию связи и криптографию. СПб: БХВ-Петербург, 2010.
\end{thebibliography}
